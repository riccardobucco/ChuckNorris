% !TEX encoding = UTF-8 Unicode

\section{Visione generale della strategia}
	\subsection{Tecniche di controllo qualità di processo}
		Per qualità di processo si intende la quantificazione del valore di ciò che è stato fatto. Per ottenere la qualità di processo si utilizza il 
		principio di miglioramento continuo PDCA, da cui deriva un miglioramento di tutti i processi e quindi dei prodotti.\\
		In particolare i processi vengono definiti in modo da poterli controllare più facilmente e con obiettivi di efficacia, efficienza ed esperienza, 
		dove con obiettivo di esperienza si intende che si mira ad ottenere un miglioramento correttivo, derivante dalla comprensione degli errori.\\
		Il controllo dell'andamento della qualità dei processi avviene, in modo indiretto, attraverso il controllo, più pratico e veloce, della qualità dei prodotti, e in modo diretto attraverso le metriche indicate nella sottosezione ''Misure e metriche'' del presente documento.
	\subsection{Tecniche di controllo qualità di prodotto}
		Per ottenere un prodotto con le qualità indicate nella sezione 2 il gruppo \groupname{} effettua un controllo di qualità al fine di rimuovere ogni possibile causa di alterazione del prodotto. Tale metodo di natura preventiva è chiamato Quality Assurance.\\
		Una volta creati, i diversi componenti del prodotto finale verranno testati per garantire che il loro funzionamento risulti deterministico.\\
		A conclusione delle attività di controllo della qualità di prodotto verrà effettuata l'attività di validazione del framework, che confermerà se i requisiti saranno stati rispettati o meno.
	\subsection{Organizzazione}
		Come indicato nel \insdoc{Piano di Progetto v1.0}, all'interno del ciclo di vita possono essere individuate varie fasi. All'interno di ciascuna di 
		esse sono pianificati più momenti di verifica, riguardanti le attività (e dunque i processi) che sono state svolte. Tuttavia, in generale, attività 
		e processi diversi fanno ottenere prodotti differenti. Dunque, è necessario prevedere procedure di verifica differenti a seconda del processo e del 
		prodotto ottenuto.\\
		Fatta questa premessa, l'organizzazione della strategia di verifica viene suddivisa in base alle attività svolte durante specifiche fasi: per ognuna 
		di essere viene attuata un'attività di verifica (e le relative procedure) differente. Le attività di verifica riguardano tanto i processi tanto gli 
		eventuali prodotti ottenuti tramite essi.\\
		Il compito di effettuare le verifiche (utilizzando gli strumenti indicati nelle Norme di Progetto) spetta ai Verificatori in carica nei momenti 
		definiti dal Responsabile di Progetto. Le verifiche da essi svolte hanno il compito di consolidare i miglioramenti raggiunti.\\
		Di seguito viene riportato come è organizzata l'attività di verifica.
		\begin{description}
			\item[Fase A] In questa fase viene prodotta molta documentazione. Tutti i documenti che sono redatti durante tale fase devono essere 
			sottoposti a verifica. Durante questa fase, inoltre, vengono definiti i requisiti che il prodotto finale deve rispettare e i relativi casi 
			d'uso. Anche questi devono essere sottoposti a verifica, controllando le caratteristiche di essenzialità e completezza. È previsto inoltre 
			il controllo della corrispondenza tra fonti e requisiti obbligatori/desiderabili.\\
			Inoltre, si deve verificare e conseguentemente migliorare la qualità dei processo di documentazione messo in pratica durante tale fase. Il 
			principio da utilizzare è quello del PDCA (descritto in appendice NUMERO APPENDICE): essendo a inizio progetto, il modo in cui si documenta 
			scarseggia soprattutto in efficacia. È dunque assolutamente necessario trarre profitto dagli errori e imparare da essi, in modo tale che le 
			stesse attività vengano svolte in modo migliore durante le fasi successive.\\
			Le attività da svolgere, le procedure da eseguire e gli strumenti da utilizzare per fare quanto descritto in precedenza possono essere trovati 
			all'interno delle Norme di Progetto.\\
			I risultati di tale attività di verifica sono descritti nell'appendice INSERIRE NUMERO APPENDICE.
			\item[Fase AD] Durante questa fase vengono applicati miglioramenti alla documentazione. Anche i requisiti e i casi d'uso 
			sono soggetti a una revisione. Dunque, quando è prevista l'attività di verifica riguardante tale fase (vedi Piano di Progetto), i Verificatori 
			devono fare i controlli descritti al punto precedente, ponendo però particolare attenzione alle correzioni, alle segnalazioni e ai suggerimenti 
			dati dal committente durante la Revisione dei Requisiti.\\
			Le attività da svolgere, le procedure da eseguire e gli strumenti da utilizzare per fare quanto descritto in precedenza possono essere trovati 
			all'interno delle Norme di Progetto.
			\item[Fase PA] Durante tale fase si progetta l'architettura del sistema. Qui l'attività di verifica si deve dunque assicurare che ogni requisito 
			espresso durante le fase di Analisi sia in corrispondenza con ciascun modulo progettato durante questa fase.\\
			Inoltre, similmente a quanto deve essere fatto nelle fasi precedenti, i Verificatori devono assicurarsi che la documentazione (intesa sia 
			come processo sia come prodotto) venga fatta nel modo migliore possibile.\\
			Le attività da svolgere, le procedure da eseguire e gli strumenti da utilizzare per fare quanto descritto in precedenza possono essere trovati 
			all'interno delle Norme di Progetto.
			\item[Fase PROB - PRD - PROP] Durante tali fasi si progettano nel dettaglio e si codificano tutti requisiti obbligatori, desiderabili e 
			opzionali. Anche qui, come nella fase precedente, si deve verificare il corretto tracciamento dei requisiti: l'intera progettazione deve 
			riguardare tutti e soli i requisiti individuati nella fase di Analisi (in modo tale che non manchi nulla e non ci sia nulla in più).\\
			Per quanto riguarda l'attività di codifica, è compito del Verificatore (in collaborazione con il Programmatore) eseguire i test di unità, di 
			integrazione e di sistema: in questo modo viene testata la qualità del prodotto.\\
			Le attività da svolgere, le procedure da eseguire e gli strumenti da utilizzare per fare quanto descritto in precedenza possono essere trovati 
			all'interno delle Norme di Progetto.
		\end{description}
		\subsection{Pianificazione delle scadenze temporali}
			L'attività di controllo della qualità deve essere svolta per garantire la sufficienza dei risultati prodotti, in modo da dare la possibilità di rispettare le scadenze fissate dal committente.\\
			Tali scadenze si individuano nelle revisioni di seguito riportate:
			\begin{itemize}
				\item \insrev{Revisione dei Requisiti}: \insdate{16}{02}{2015} (revisione formale);
				\item \insrev{Revisione di Progettazione}: \insdate{24}{04}{2015} (revisione di progresso);
				\item \insrev{Revisione di Qualifica}: \insdate{29}{05}{2015} (revisione di progresso);
				\item \insrev{Revisione di Accettazione}: \insdate{18}{06}{2015} (revisione formale).
			\end{itemize}
			La pianificazione completa delle attività di progetto è descritta in modo dettagliato nel \insdoc{Piano di Progetto v1.0}
		\subsection{Responsabilità}
			\begin{itemize}
				\item Il processo di verifica deve essere efficiente. Dunque, si deve fare in modo, ove sia possibile, che esso sia quanto più automatizzabile 
				possibile. Questa è una responsabilità dell'Amministratore di Progetto.
				\item Il processo di verifica deve essere efficace. È compito dei Verificatori, i quali si attengono alle Norme di Progetto, fare in modo che 
				si riesca a raggiungere l'obiettivo atteso, migliorando allo stesso tempo la qualità del processo di cui si devono occupare.
				\item È responsabilità dei Programmatori fornire i test necessari per effettuare parte delle verifiche sulle unità software prodotte.
				\item Il Responsabile di Progetto assegna ai Verificatori determinati compiti, in base al Piano di Progetto da lui stesso stilato. Egli ha la 
				piena responsabilità delle attività di controllo del processo di verifica.
			\end{itemize}
			Per maggiori dettagli circa i compiti assegnati a ciascun ruolo si vedano le Norme di Progetto.
		\subsection{Risorse}
			\subsubsection{Risorse necessarie}
				Le risorse che allo stato attuale si ritiene essere necessarie per lo svolgimento di tutte le attività di verifica 
				sono di tre tipi:
				\begin{itemize}
					\item risorse umane;
					\item risorse software;
					\item risorse hardware.
				\end{itemize}
				\paragraph{Risorse umane}
					Le risorse umane, intese come figure professionali, che sono necessarie per applicare in modo corretto e completo il processo di 
					verifica sono le seguenti:
					\begin{itemize}
						\item Responsabile di Progetto;
						\item Amministratore di Progetto;
						\item Verificatori;
						\item Programmatori;
					\end{itemize}
					Una descrizione delle responsabilità di ciascun ruolo qui presentato è presente nella sezione INDICARE NUMERO del presente documento.\\
					Una descrizione dettagliata di ciascun ruolo è presente all'interno delle Norme di Progetto.
				\paragraph{Risorse software}
					Per risorse software si intendono tutti gli strumenti software che il team intende utilizzare per attuare le attività di verifica sui 
					prodotti e sui processi.\\
					Allo stato attuale sono necessari software che siano in grado di verificare in modo automatico il maggior numero di norme riguardanti 
					i documenti. Servono, per esempio, strumenti che eseguano i seguenti compiti:
					\begin{itemize}
						\item correggere eventuali errori ortografici;
						\item capire se un documento possiede la struttura adatta oppure no;
						\item trovare parti di testo che non rispettino alcune delle norme tipografiche.
					\end{itemize}
					Inoltre, per quanto possibile, servono software che aiutino a non commettere errori: infatti la correzione, in generale, costa 
					molto di più della prevenzione. Quindi, ad esempio, servono software con i seguenti compiti:
					\begin{itemize}
						\item rilevare (durante la scrittura) eventuali errori ortografici;
						\item costruire e visualizzare in real-time il documento scritto in Latex (in modo che sia facile accorgersi di errori 
						nell'utilizzo dei comandi);
						\item raccogliere tutti gli errori comuni, in modo da formare in automatico una lista alla quale possono guardare coloro che 
						devono redigere un documento. 
					\end{itemize}
				\paragraph{Risorse hardware}
					Sono necessarie macchine (computer) sufficientementi potenti da far girare tutto il software necessario durante le attività di verifica.
			\subsubsection{Risorse disponibili}
				\paragraph{Risorse umane}
					Le risorse umane disponibili per mettere in pratica le attività di verifica sono date dai componenti del team. Essi svolgono le 
					mansioni dei vari ruoli necessari (indicati nel paragrafo riguardante le risorse umane necessarie) in accordo con quanto descritto 
					nel Piano di Progetto e in generale in base alle decisioni del Responsabile di Progetto.
				\paragraph{Risorse software}
					Il team si è dotato di alcuni script che eseguono controlli automatici di diverso tipo. Alcuni di questi programmi (come il 
					correttore ortografico automatico) sono stati rinvenuti sulla rete. Quando ciò non è stato possibile, il team ha provveduto esso 
					stesso alla creazione di programmi di utilità.\\
					Una descrizione dettagliata di programmi e script a supporto delle attività di verifica può essere trovata nelle Norme di Progetto.
				\paragraph{Risorse hardware}
					Sono disponibili (e per ora sufficienti) le seguenti risorse hardware:
					\begin{itemize}
						\item un computer portatile per ogni membro del team (dotato del sistema operativo comune indicato nelle Norme di Progetto).
					\end{itemize}
					Sono inoltre disponibili le risorse hardware messe a disposizione dal Servizio Calcolo dell'Università degli Studi di Padova.
		\subsection{Misure e metriche}
			L'obiettivo è quello di misurare la qualità di "cosa" si realizza (il prodotto) e "come" lo si realizza (il processo). Il problema è 
			riuscire a stimare in modo preciso quanto si sta misurando. Per questo motivo sono necessarie delle metriche, dei criteri stabiliti a 
			priori su cui basare la misurazione della qualità.
			\subsubsection{Misure}
				Le misure che vengono fatte durante tutto l'arco del progetto possono essere molto varie. Di conseguenza, non si può far altro che 
				stabilire dei range di accettazione indipendenti dal tipo di stima che si sta effettuando. In particolare, una valore misurato può 
				essere:
				\begin{description}
					\item[negativo] Un valore appartiene a questo range non può essere accettato, qualunque sia l'ambito al quale appartiene.
					\item[accettabile] Un valore appartenente a questo range è un valore che ha raggiunto la soglia minima dell'accettazione. Può 
					quindi ovviamente essere accettato.
					\item[ottimale] Un valore appartenente a questo range ha raggiunto le massime aspettative del team. L'obiettivo dovrebbe essere 
					quello di avere tutti i valori all'interno di tale range.
				\end{description}
			\subsubsection{Metriche per i documenti}
				La qualità di un documento dipende prima di tutto dai suoi contenuti. La loro qualità, tuttavia, è difficilmente quantificabile allo 
				stato attuale del progetto a causa dell'esperienza pressoché nulla del team in quest'ambito. Si è deciso dunque di limitarsi a valutare 
				parametri maggiormente oggettivi e soprattutto misurabili automaticamente attraverso strumenti software.
				\paragraph{Indice di leggibilità}
					Una metrica che si è deciso di utilizzare per poter stimare la qualità di un documento è l'indice di leggibilità. In particolare, è 
					stato considerato l'indice Gulpease, studiato appositamente per la lingua italiana.\\
					Questo particolare indice si basa sulla lunghezza della parola e sulla lunghezza della frase rispetto al numero di lettere.\\
					La formula per il suo calcolo è la seguente:
					\begin{equation}
						\label{Indice Gulpease}
						Indice\ Gulpease = 89 + \frac{300*numero\_frasi-10*numero\_lettere}{numero\_parole}
					\end{equation}
					Il risultato di tale equazione tipicamente è compreso tra 0 e 100, dove valori alti indicano leggibilità elevata e viceversa.\\
					In generale, risulta che testi con un indice:
					\begin{itemize}
						\item inferiore a 80 risultano difficili da leggere per chi ha la licenza elementare;
						\item inferiore a 60 risultano difficili da leggere per chi ha la licenza media;
						\item inferiore a 40 risultano difficili da leggere per chi ha la licenza superiore;
					\end{itemize}
					Vengono di seguito riportati i range stabiliti per la metrica appena introdotta. Si noti che viene tenuto in considerazione il fatto 
					che la documentazione è destinata a persone sufficientemente preparate, competenti ed istruite.
					\begin{itemize}
						\item Valori minori di 35 sono considerati negativi.
						\item Valori compresi tra 35 e 50 sono considerati accettabili.
						\item Valori maggiori di 50 sono considerati ottimali.
					\end{itemize}
			\subsubsection{Metriche per i processi}
				\paragraph{Capability Maturity Model}
					Per valutare i processi si è deciso di fare riferimento al modello CMM. Se correttamente applicato esso ci fornisce una base 
					concettuale molto generale su cui appoggiarsi per valutare il livello dei processi.\\
					CMM ci consente di individuare la maturità di un processo: essa può assumere un valore da 1 (il peggiore) a 5 (il migliore). Mettendo 
					ora in relazione i risultati di tale modello con i range da noi stabiliti otteniamo quanto segue:
					\begin{itemize}
						\item i valori 1 e 2 sono considerati negativi;
						\item il valore 3 è considerato accettabile;
						\item i valori 4 e 5 sono considerati ottimali.
					\end{itemize}
					La scelta del team è ricaduta su questo modello in quanto ritenuto più adatto alla modesta entità del progetto e alla scarsa esperienza del 
					team di sviluppo. Altri modelli valutati sono stati CMMI, ritenuto inadatto per la sua complessità, e il modello SPY, ritenuto inadatto in 
					quanto, nella sua versione pura, non prevede il consolidamento delle best practice.
				\paragraph{Produttività}
					Si vuole cercare di calcolare la produttività media delle risorse impiegate, cioè delle persone coinvolte, nelle diverse fasi del 
					progetto. In generale, in una prima approssimazione, la si misura tramite la seguente formula:
					\begin{equation}
						Produttivita' = \frac{Quantita'\ di\ output\ ottenuto}{Quantita'\ di\ input\ utilizzato}
						\label{Produttività}
					\end{equation}
					Tale parametro genera metriche differenti in base a cosa si riferiscono i termini "input" e "output". Dunque, tale formula assume 
					aspetti diversi in base ai processi (e relative attività) ai quali viene applicata.
					\begin{description}
						\item[Documentazione] Il parametro "input" assume il significato di "numero di parole scritte", il parametro "output" assume il 
						significato di "tempo necessario alla scrittura della documentazione".
						\item[Codifica] Il parametro "input" assume il significato di "numero di linee di codice scritte", il parametro "output" assume 
						il significato di "tempo necessario alla scrittura del codice".
					\end{description}
					I valori che si ottengono in tale modo sono difficilmente valutabili in modo assoluto. Piuttosto, essi assumono importanza nel 
					momento in cui vengono confrontati fra loro in momenti diversi del progetto: in questo modo il Responsabile di Progetto può valutare 
					più facilmente i tempi e dunque i costi delle attività che devono essere svolte.
				\paragraph{Efficacia di una revisione}
					Si vuiole cercare di calcolare quanto è efficace una revisione. Possiamo pensare, in modo approssimativo, che tale valore sia dato 
					dalla seguente formula generale:
					\begin{equation}
						\label{Efficacia revisione}
						Efficacia\ Revisione = \frac{Numero\ di\ errori\ rilevati}{Numero\ di\ elementi\ ispezionati}
					\end{equation}
					A parità di elementi ispezionati si vuole che il numero di errori rilevati sia maggiore possibile.\\
					Anche tale indice è applicabile in modo diverso ai diversi processi: in ognuno di essi il termine "elementi" assume una connotazione 
					diversa. Alcuni esempi sono riportati di seguito.
					\begin{description}
						\item[Documentazione] Il termine "elementi" assume il significato di "pagine".
						\item[Codifica] Il termine "elementi" assume il significato di "righe di codice".
					\end{description}
					Anche in questo caso vale un discorso simile a quello fatto per la produttività: non siamo in grado di individuare con 
					precisione dei valori accettabili di riferimento. Di conseguenza tali misure avranno valore relativo, in quanto usate per essere 
					confrontate tra di loro in momenti diversi del progetto.\\
					Notare che una verifica effettuata tramite la presente metrica permette al Responsabile di Progetto di capire quando lo sforzo per 
					trovare gli errori diventa troppo grande (e quindi costoso). Si ricordi, infatti, che vale la legge del rendimento decrescente.
		\subsubsection{Metriche del codice}
			A questo punto dello sviluppo del progetto, il team non è ancora pronto per indicare delle metriche adeguate per il controllo della 
			qualità del codice, ma si riserva di indicare tali metriche in fase di progettazione.

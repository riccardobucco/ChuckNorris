% !TEX encoding = UTF-8 Unicode

\section{Visione generale della strategia}
\subsection{Tecniche di controllo qualità di processo}
	Per qualità di processo si intende la quantificazione del valore di ciò che è stato fatto. Per ottenere la qualità di processo si utilizza il 
	principio di miglioramento continuo PDCA, da cui deriva un miglioramento di tutti i processi e quindi dei prodotti.\\
	In particolare i processi vengono definiti in modo da poterli controllare più facilmente e con obiettivi di efficacia, efficienza ed esperienza, 
	dove con obiettivo di esperienza si intende che si mira ad ottenere un miglioramento correttivo, derivante dalla comprensione degli errori.\\
	Il controllo dell'andamento della qualità dei processi avviene, in modo indiretto, attraverso il controllo, più pratico e veloce, della qualità dei prodotti, e in modo diretto attraverso le metriche indicate nella sottosezione ''Misure e metriche'' del presente documento.
\subsection{Tecniche di controllo qualità di prodotto}
	Per ottenere un prodotto con le qualità indicate nella sezione 2 il gruppo \groupname{} effettua un controllo di qualità al fine di rimuovere ogni possibile causa di alterazione del prodotto. Tale metodo di natura preventiva è chiamato Quality Assurance.\\
	Una volta creati, i diversi componenti del prodotto finale verranno testati per garantire che il loro funzionamento risulti deterministico.\\
	A conclusione delle attività di controllo della qualità di prodotto verrà effettuata l'attività di validazione del framework, che confermerà se i requisiti saranno stati rispettati o meno.
\subsection{Organizzazione}
	Come indicato nel \insdoc{Piano di Progetto v1.0}, all'interno del ciclo di vita possono essere individuate varie fasi. All'interno di ciascuna di 
	esse sono pianificati più momenti di verifica, riguardanti le attività (e dunque i processi) che sono state svolte. Tuttavia, in generale, attività 
	e processi diversi fanno ottenere prodotti differenti. Dunque, è necessario prevedere procedure di verifica differenti a seconda del processo e delle 
	attuate e del prodotto ottenuto.\\
	Fatta questa premessa, l'organizzazione della strategia di verifica viene suddivisa in base alle attività svolte durante specifiche fasi: per ognuna 
	di essere viene attuata un'attività di verifica (e le relative procedure) differente. Le attività di verifica riguardano tanto i processi tanto gli 
	eventuali prodotti ottenuti tramite essi.\\
	Il compito di effettuare le verifiche (utilizzando gli strumenti indicati nelle Norme di Progetto) spetta ai Verificatori in carica nei momenti 
	definiti dal Responsabile di Progetto. Le verifiche da essi svolte hanno il compito di consolidare i miglioramenti raggiunti.\\
	Di seguito viene riportato come è organizzata l'attività di verifica.
	\begin{description}
		\item[Fase A] In questa fase viene prodotta molta documentazione. Tutti i documenti che sono redatti durante tale fase devono essere 
		sottoposti a verifica. Durante questa fase, inoltre, vengono definiti i requisiti che il prodotto finale deve rispettare e i relativi casi 
		d'uso. Anche questi devono essere sottoposti a verifica, controllando le caratteristiche di essenzialità e completezza. È previsto inoltre 
		il controllo della corrispondenza tra fonti e requisiti obbligatori/desiderabili.\\
		Inoltre, si deve verificare e conseguentemente migliorare la qualità dei processo di documentazione messo in pratica durante tale fase. Il 
		principio da utilizzare è quello del PDCA (descritto in appendice NUMERO APPENDICE): essendo a inizio progetto, il modo in cui si documenta 
		scarseggia soprattutto in efficacia. È dunque assolutamente necessario trarre profitto dagli errori e imparare da essi, in modo tale che le 
		stesse attività vengano svolte in modo migliore durante le fasi successive.\\
		Le attività da svolgere, le procedure da eseguire e gli strumenti da utilizzare per fare quanto descritto in precedenza possono essere trovati 
		all'interno delle Norme di Progetto.\\
		I risultati di tale attività di verifica sono descritti nell'appendice INSERIRE NUMERO APPENDICE.
		\item[Fase AD] Durante questa fase vengono applicati miglioramenti alla documentazione. Anche i requisiti e i casi d'uso 
		sono soggetti a una revisione. Dunque, quando è prevista l'attività di verifica riguardante tale fase (vedi Piano di Progetto), i Verificatori 
		devono fare i controlli descritti al punto precedente, ponendo però particolare attenzione alle correzioni, alle segnalazioni e ai asuggerimenti 
		dati dal committente durante la Revisione dei Requisiti.\\
		Le attività da svolgere, le procedure da eseguire e gli strumenti da utilizzare per fare quanto descritto in precedenza possono essere trovati 
		all'interno delle Norme di Progetto.\\
		\item[Fase PA - PROB - PRD] Durante tali fase si progetta il sistema in modo sempre più approfondito, partendo dall'architettura e arrivando 
		ai dettagli. Qui l'attività di verifica si deve dunque assicurare ogni requisito espresso durante le fase di Analisi sia in corrispondenza con 
		ciascun componente sviluppato durante la fase di Progettazione. Inoltre, similmente a quanto deve essere fatto nelle fasi precedenti, i Verificatori 
		devono assicurarsi che la documentazione (intesa sia come processo sia come prodotto) venga fatta nel modo migliore possibile.\\
		Le attività da svolgere, le procedure da eseguire e gli strumenti da utilizzare per fare quanto descritto in precedenza possono essere trovati 
		all'interno delle Norme di Progetto.\\
	\end{description}
	\subsection{Pianificazione delle scadenze temporali}
	L'attività di controllo della qualità deve essere svolta per garantire la sufficienza dei risultati prodotti, in modo da dare la possibilità di rispettare le scadenze fissate dal committente.\\
	Tali scadenze si individuano nelle revisioni di seguito riportate:
		\begin{itemize}
			\item \insrev{Revisione dei Requisiti}: \insdate{16}{02}{2015} (revisione formale);
			\item \insrev{Revisione di Progettazione}:\insdate{24}{04}{2015} (revisione di progresso);
			\item \insrev{Revisione di Qualifica}:\insdate{29}{05}{2015} (revisione di progresso);
			\item \insrev{Revisione di Accettazione}:\insdate{18}{06}{2015} (revisione formale).
		\end{itemize}
	La pianificazione completa delle attività di progetto è descritta in modo dettagliato nel \insdoc{Piano di Progetto v1.0}
	\subsection{Responsabilità}
	E' totalmente a carico del \insrole{Responsabile di Progetto} la responsabilità di accertare il corretto svolgimento delle attività di controllo della qualità e il loro esito.\\
	L'aggiornamento e il controllo dell'ambiente di sviluppo, nonchè del documento "\insdoc{Norme di Progetto}" è a carico degli \insrole{Amministratori di Sistema}.
	\subsection{Risorse}
	\subsubsection{Risorse necessarie}
	La realizzazione del progetto Norris necessita di risorse umane e tecnologiche.
	\paragraph{Risorse umane}
		Le risorse umane impiegate, intese come figure professionali, sono indicate e definite nel documento \insdoc{Norme di Progetto v1.0} alla sottosezione "Ruoli di progetto".
	\paragraph{Risorse tecnologiche}
		Le risorse tecnologiche, software e hardware, sono tutti gli strumenti necessari allo svolgimento efficace ed efficiente del progetto in tutte le sue componenti, dalla documentazione alla codifica, dalla verifica al coordinamento.
		\subparagraph{Risorse software}
		Di esse fanno parte:
			\begin{itemize}
				\item software per la documentazione nel linguaggio \LaTeX{};
				\item software per la creazione di diagrammi UML;
				\item IDE per lo sviluppo nei linguaggi di programmazione indicati nel capitolato;
				\item script per l'automazione di attività di verifica;
				\item software per l'analisi del codice;
				\item software per l'automazione dei test.
			\end{itemize}
		\subparagraph{Risorse hardware}
			\begin{itemize}
				\item Computer, fissi o portatili, sufficentemente potenti da poter supportare tutte le risorse software necessarie.
			\end{itemize}
	\subsubsection{Risorse disponibili}
	Le risorse umane derivano dalla disponibilità dei membri del gruppo.
	\paragraph{Risorse software}
		\begin{itemize}
			\item Gli script per l'automazione delle attività di verifica sono prodotti autonomamente dal team;
			\item I software necessari alla documentazione nelle diverse fasi di avanzamento  del progetto e alla fase di sviluppo sono di distribuzione open-source e in possesso di ogni membro del team.
		\end{itemize}
	\paragraph{Risorse hardware}
		\begin{itemize}
			\item Ogni membro del team possiede almeno un computer portatile con il sistema operativo comune individuato;
			\item Il team di sviluppo si appoggia, inoltre, alle risorse hardware del Servizio Calcolo del dipartimento di Matematica Pura e Applicata dell'Università degli studi di Padova, con sede in via Trieste, 63.
		\end{itemize}
	\subsection{Strumenti}
	Gli strumenti utilizzati dal team per rendere la realizzazione del progetto il più efficace ed efficiente possibile sono indicati nel documento \insdoc{Norme di progetto v1.0} alla sezione "Ambiente di sviluppo".
	\subsection{Tecniche di analisi}
	Per poter determinare il livello di qualità del prodotto, rispetto agli obiettivi sopra indicati, il team di sviluppo utilizza diversi tipi di tecniche di analisi dei dati.\\
	Le tecniche di analisi si possono suddividere in due grandi categorie:
	\begin{description}
		\item[Tecniche di analisi dinamica]: questo tipo di tecniche analizzano misurazioni effettuate durante l'esecuzione di un programma, e consentono di determinare attributi di qualità come l'efficienza e l'affidabilità. Nello specifico, di questa categoria faranno parte i test per la determinazione del tempo di latenza dei dati.
		\item [Tecniche di analisi statica]: questo tipo di tecniche analizzano misurazioni relative a caratteristiche del progetto, del codice o della documentazione, e consentono di determinare il livello di attributi di qualità come la comprensibilità, la manutenibilità e la complessità. I metodi di acquisizione degli indici da analizzare sono:
		\begin{itemize}
			\item \textbf{walkthrough}, cioè la lettura dell'oggetto della misurazione per cercare qualunque possibile errore;
		 	\item \textbf{inspection}, cioè la lettura mirata dell'oggetto della misurazione alla ricerca di errori ritenuti più probabili e frequenti.
		\end{itemize}
	\end{description}
	\subsection{Misure e metriche}
	Affinchè la qualità sia predittiva, è necessario che il processo di verifica sia quantificabile, e pertanto si definiscono delle metriche, cioè un sistema di misurazione formato da misure e criteri per valutarle.\\
	Ci sono due tipi di range all'interno dei quali si collocheranno i valori delle misurazioni:
	\begin{itemize}
		\item range di accettazione: questi valori saranno ammessi per l'accettazione del prodotto
		\item range ottimale: questi sono i valori ottimali in cui dovrebbe collocarsi la misurazione
	\end{itemize}
	\subsubsection{Metriche per i processi}
		Per il controllo della qualità dei processi è stato deciso di adottare il modello CMM. Questo modello ha lo scopo di migliorare i processi e, più in generale, di aiutare una organizzazione a migliorare le sue prestazioni.\\
		Per ogni processo vengono valutati i livelli di "\textit{Capability}" e "\textit{Maturity}" cioè, rispettivamente, la misura dell'adeguatezza del processo  rispetto agli scopi per cui è stato definito e il grado di consolidamento di un processo all'interno dell'organizzazione. I valori possono oscillare tra 1 e 5.\\
		La scelta del team è ricaduta su questo modello in quanto ritenuto più adatto alla modesta entità del progetto e alla scarsa esperienza del team di sviluppo. Altri modelli valutati sono stati CMMI, ritenuto inadatto per la sua complessità, e il modello SPY, ritenuto inadatto in quanto, nella sua versione pura, non prevede il consolidamento delle best practice.\\
		I range stabiliti per questa metrica sono:
		\begin{itemize}
			\item Range di accettazione: 3 - 5;
			\item Range ottimale: 4 - 5;
		\end{itemize}	
		La valutazione rispetto ai processi viene fornita solo dopo l'esito della revisione da parte del committente.
	\subsubsection{Metriche per i documenti}
		Per il controllo della qualità dei documenti è stato deciso di adottare l'indice Gulpease, un indice di leggibilità del testo tarato sulla lingua italiana.\\
		Questo particolare indice si basa sulla lunghezza della parola e sulla lunghezza della frase rispetto al numero di lettere.\\
		
		\begin{equation}\label{Indice Gulpease}
		Indice\_Gulpease = 89+ \frac{300*numero\_frasi-10*numero\_lettere}{numero\_parole}
		\end{equation}
			
		Il risultato è compreso tra 0 e 100, dove la leggibilità è direttamente proporzionale al valore.\\
		Tipicamente vengono associate le seguenti soglie dell'indice Gulpease al livello di istruzione:
		\begin{itemize}
			\item inferiore a 80: bassa leggibilità per chi ha la licenza elementare;
			\item inferiore a 60: bassa leggibilità per chi ha la licenza media;
			\item inferiore a 40: bassa leggibilità per chi ha la licenza superiore.
		\end{itemize}
		
		I range stabiliti per questa metrica, in considerazione del fatto che la documentazione si considera destinata a persone sufficientemente competenti, sono:
		\begin{itemize}
			\item range di accettazione: 35 - 100;
			\item range ottimale: 45 - 100.
		\end{itemize}
	\subsubsection{Metriche del codice}
		A questo punto dello sviluppo del progetto, il team non è ancora pronto per indicare delle metriche adeguate per il controllo della qualità del codice, ma si riserva di indicare tali metriche in fase di progettazione.

\level{1}{Esiti delle revisioni}
	Dopo ogni revisione, richiesta dal committente e sostenuta dal gruppo \groupname{}, segue una \insglo{fase} di assestamento. In questa \insglo{fase} vengono risolte le problematiche e le mancanze rilevate dal committente durante la revisione, prima di proseguire con le rimanenti attività indicate nel \insdoc{Piano di Progetto}.
	Vengono riportate di seguito le modifiche effettuate in seguito alle critiche e ai suggerimenti ricevuti in sede di revisione, assieme ad una valutazione sull'esito della revisione stessa.
	\level{2}{Revisione dei Requisiti}
		Il gruppo è rimasto soddisfatto dall'esito della revisione, in quanto ha dimostrato che è stato fatto un discreto lavoro fino alla scadenza.
		Si descrivono le modifiche apportate ai documenti presentati:
		\begin{description}
			\item[Analisi dei Requisiti:] Sono stati rivisti i casi d'uso. In particolare sono state effettuate le seguenti modifiche:
			\begin{itemize}
				\item correzione delle imprecisioni rilevate;
				\item aggiunta dei sottocasi ai casi d'uso poco dettagliati;
				\item aggiunta dei diagrammi che mostrano le relazioni tra gli utenti-attori;
				\item aggiunta di appendici;
				\item modifica e aggiunta di requisiti;
				\item nuovo tracciamento requisiti-fonti, fonti-requisiti.
			\end{itemize}
			\item[Norme di Progetto:] Si è cercato di riorganizzare le norme secondo i suggerimenti del committente in modo da accrescere l'usabilità del documento. A tal fine i contenuti del documento sono stati organizzati prima seguendo la classificazione data dallo standard 12 207 (processi primari, di supporto e organizzativi), poi per singoli processi e attività, e infine per procedure e strumenti di supporto.
			\item[Piano di Progetto:] Per correggere gli errori segnalati si è deciso di rinominare le fasi scegliendo un nome che le identificasse per obiettivo. Nella sezione “Analisi dei rischi” è stato inserito per ogni rischio il suo riscontro effettivo al termine di ogni \insglo{fase} portata a termine e modificato di conseguenza il grado di occorrenza e di pericolosità qualora questo sia diminuito o aumentato. La sezione “Resoconto” è stata riorganizzata spostando in una appendice le informazioni riguardanti l'investimento da parte del gruppo, non rilevanti per il committente. Infine la distribuzione delle ore è stata rivalutata per ruolo, riducendo progressivamente le ore al \insrole{Project Manager} e all'\insrole{Amministratore} in considerazione del fatto che il primo ha compiti prevalentemente organizzativi, che richiederanno una modesta percentuale di tempo, mentre i compiti del secondo si concentrano prevalentemente nella prima parte del progetto, in cui c'è la necessità di predisporre l'ambiente di lavoro e gli automatismi necessari, e pertanto si prevede che il lavoro svolto in precedenza permetterà una riduzione di tale ruolo. Il tempo tolto a questi ruoli è stato aggiunto all'attività di verifica, che occupava appena un terzo del monte ore totali, passando da 32\% a 37\%.
			\item[Piano di Qualifica:] Durante la revisione col committente sono state rilevate diverse irregolarità in merito ai contenuti di questo documento. Per porvi rimedio sono state corrette alcune parti fondamentali, in particolare:
			\begin{itemize}
				\item è stata riorganizzata la sezione Strategia relativa alla verifica durante le diverse fasi di progetto;
				\item è stato introdotto un paragrafo relativo all'applicazione del Ciclo di Deming;
				\item è stata riorganizzata la sezione Strategia relativa alle metriche per i processi e per il \insglo{prodotto}.
			\end{itemize}
		\end{description}
		In tutti i documenti si è fatto in modo di rivedere i punti in cui sono emersi fraintendimenti da parte del committente rispetto alle nostre argomentazioni.
	\level{2}{Revisione di Progettazione}
		Anche in seguito a questa revisione il gruppo può dirsi soddisfatto, in quanto le valutazioni da parte del committente sono state abbastanza positive.\\
		In base alle osservazioni che ci sono state fatte e ai consigli che ci sono stati dati, sono state apportate alcune modifiche ai documenti.
		\begin{description}
			\item[Analisi dei Requisiti] Non è stata fatta alcuna modifica sostanziale, in quanto tale documento è stato valutato positivamente.
			\item[Norme di Progetto] In sede di revisione è stata apprezzata la riorganizzazione del contenuto effettuata durante le fasi precedenti. Ci è stato fatto notare che lo stile era quasi esclusivamente narrativo. Inoltre l'usabilità del documento non era ancora ottimale. Abbiamo dunque agito nel seguente modo:
			\begin{itemize}
				\item abbiamo inserito numerose immagini e qualche diagramma per semplificare la comprensione durante la lettura del documento;
				\item abbiamo inserito un paragrafo che riassumesse i vari task che devono essere eseguiti durante le attività previste; tale paragrafo fa anche riferimento alle norme e alle procedure che si devono mano a mano seguire. In tal modo un lettore è in grado di sapere esattamente cosa deve fare e come lo deve fare nel momento in cui sta svolgendo una data attività.
			\end{itemize}
			Si noti che abbiamo apportato i cambiamenti sopra descritti concentrandoci soprattutto sulle nuove sezioni aggiunte (esempio: Progettazione). Questo perchè abbiamo ritenuto senza senso dover utilizzare una grande quantità di tempo per sistemare parti delle norme che a questo punto del progetto non vengono praticamente più usate.
			\item[Specifica Tecnica] L'esito della revisione di tale documento è stato negativo. Infatti, sono state fatte numerose osservazioni da parte del committente su cose che avrebbero potuto essere migliorate. In particolare, i problemi (e le soluzioni successivamente trovate) sono presentati in seguito.
			\begin{itemize}
				\item Innanzitutto, la struttura del documento non risultava adeguata. Abbiamo infatti impostato il documento esclusivbamente per livelli di astrazione. Questo modo di procedere fa si che la lettura non si di facile comprensione (si perde facilmente il filo). È stato risolto questo problema semplicemente rioganizzando la struttura della progettazione architetturale e dividendola per prodotti.
				\item In secondo luogo, le relazioni tra componenti e classi non sono state specificate molto bene, rimanendo perlopiù astratte. Per risolvere il problema, si è semplicemente fatto uno sforzo ulteriore per specificare meglio quanto richiesto.
				\item Infine, ci hanno fatto notare che il modo in cui avevamo presentato i pattern utilizzati era insufficiente: abbiamo risolto la cosa contestualizzando il loro uso mediante immagini rappresentanti le classi in cui sono presenti tali pattern.
			\end{itemize}
		\end{description}
\level{1}{Visione generale della strategia di gestione della qualità}
	\level{2}{Obiettivi di qualità} \label{sec:obiettivi}
		In questa sezione vengono riportati gli obiettivi di qualità che il gruppo \groupname{} si impegna a perseguire durante lo svolgimento dell'intero progetto.\\
		Per capire se un certo obiettivo è stato raggiunto oppure no, il \groupname{} fa uso di standard, modelli e metriche. Ognuno di questi fa uso di scale differenti e fissate a priori: qualunque sia il criterio utilizzato per misurare e dunque quantificare la vicinanza a un certo obiettivo, abbiamo fissato dei valori minimi che intendiamo raggiungere nell'arco dell'intero progetto. Oltre a ciò abbiamo anche fissato dei valori che riteniamo ottimali e che devono essere sperabilmente (ma non obbligatoriamente) raggiunti.\\
		Riassumendo:
		\begin{enumerate}
			\item vengono enunciati alcuni obiettivi;
			\item vengono scelti modelli, metriche, criteri, scale ecc. per poter misurare quanto si è vicini al raggiungimento di un obiettivo;
			\item in ognuno di questi modelli, metriche, criteri, scale ecc. si identifica un valore minimo che si intende raggiungere nell'arco del progetto (soglia di accettabilità) e un valore ottimale al quale si dovrebbe sperabilmente arrivare (soglia di ottimalità). Non possono essere acccettai valori negativi, ovvero valori al di sotto dell'accettabilità. 
		\end{enumerate}
		Si noti che tutte le metriche alle quali si fa riferimento in seguito sono presentate e descritte nel dettaglio all'interno della sezione \nameref{sec:metriche}.

		\level{3}{Qualità di processo}  \label{subsec:obiettiviprocesso}
			Assicurare la qualità dei processi è indispensabile durante lo svolgimento del progetto per le seguenti ragioni:
			\begin{itemize}
				\item aiuta ad ottimizzare l'uso delle risorse;
				\item fa in modo che i costi siano maggiormente contenuti;
				\item migliora la stima dei rischi e degli impegni.
			\end{itemize}
			Un altro fattore da tenere sempre in considerazione risiede nel fatto che molto spesso prodotti scadenti derivano da pessimi processi.\\
			Le caratteristiche ottimali che desideriamo che i processi posseggano vengono riportate in seguito:
			\begin{itemize}
				\item un processo dovrebbe essere in grado di migliorare continuamente le proprie performance
				\begin{itemize}
					\item le performance di un processo dovrebbero essere costantemente misurabili;
					\item un processo, durante il proprio svolgimento, dovrebbe cercare di perseguire sempre degli obiettivi quantitativi di miglioramento fissati a priori.
				\end{itemize}
				\item le attività di un processo dovrebbero proseguire nei tempi indicati nel \insdoc{Piano di Progetto};
				\item i costi effettivi di ogni processo dovrebbero essere in linea con quanto dichiarato nel \insdoc{Piano di Progetto};
				\item un processo dovrebbe essere produrre ottimi risultati in breve tempo;
			\end{itemize}
			Nelle prossime sezioni si enunciano gli obiettivi che si intendono raggiungere. Per ognuno di essi vengono specificati i criteri con i quali si effettuano le misurazioni sulla qualità (per capire quanto si è vicini all'obiettivo). Infine, per ogni criterio scelto vengono dichiarati i valori minimi che si intendono raggiungere, oltre a quelli ottimali.
			\level{4}{Miglioramento costante}
				Per misurare quanto si è vicini all'obiettivo di avere processi in grado di misurare le proprie performance e che sono quindi in grado di porsi obiettivi quantitativi di miglioramentom si è deciso di adottare il modello \insglo{CMM}. In particolare si vuole raggiungere almeno il livello 2 previsto da tale scala. Il livello ottimale che sperabilmente dovremmo raggiungere è il 4. Riassumendo:
				\begin{description}
					\item[Modello utilizzato per quantificare gli obiettivi]: \insglo{CMM};
					\item[Soglia di accettabilità]: livello 2 previsto da \insglo{CMM};
					\item[Soglia di ottimalità]: livello 4 previsto da \insglo{CMM}.
				\end{description}
				Per una migliore e più dettagliata descrizione del modello \insglo{CMM} qui adottato si faccia riferimento a \nameref{app:CMM}. Per approfondire la scelta dei range di accettazione e ottimalità si consulti invece la sezione \nameref{subsec:metricaCMM}.
			\level{4}{Rispetto della pianificazione}
				Per capire se le attività di un processo sono oppure no in ritardo rispetto a quanto è stato pianificato all'interno del \insdoc{Piano di Progetto} viene utilizzata la seguente metrica: Schedule Variance. Si desidera che il ritardo accumulato non sia maggiore del 5\% rispetto al totale pianificato. Sarebbe invece ottimale essere esattamente in linea con quanto prevede il \insdoc{Piano di Progetto}, o essere addirittura in anticipo. Riassumendo:
				\begin{description}
					\item[Metrica utilizzata per quantificare l'obiettivo]: Schedule Variance;
					\item[Soglia di accettabilità]: in ritardo al massimo del 5\% rispetto a quanto pianificato;
					\item[Soglia di ottimalità]: essere in linea o in anticipo con quanto pianificato (minore o uguale a 0\%).
				\end{description}
				Per una descrizione dettagliata della metrica qui utilizzata e per una maggiore comprensione degli indici di ottimalità e accettabilità presentati si faccia riferimento alla sezione \nameref{subsec:scheduleVariance}.
			\level{4}{Rispetto del budget}
				Per capire se i costi di un processo rientrano nel budget assegnato dal \insdoc{Piano di Progetto} oppure no viene utilizzata la seguente metrica: Budget Variance. L'obiettivo minimo è quello di avere dei costi che non superano il budget a disposizione di una percentuale maggiore al 10\%. Sarebbe ottimale, invece, che i costi fossero esattamente in linea con il preventivo o che addirittura si fosse speso meno. Riassumendo:
				\begin{description}
					\item[Metrica utilizzata per quantificare l'obiettivo]: Budget Variance;
					\item[Soglia di accettabilità]: costi al massimo maggiori del 10\% rispetto al preventivo;
					\item[Soglia di ottimalità]: costi in linea con il budget o addirittura minori (minore o uguale a 0\%).
				\end{description}
				Per una descrizione dettagliata della metrica qui utilizzata e per una maggiore comprensione degli indici di ottimalità e accettabilità presentati si faccia riferimento alla sezione \nameref{subsec:budgetVariance}.
			\level{4}{Efficacia ed efficienza}
				Per capire se un processo è davvero efficiente ed efficace si utilizza la seguente metrica: produttività. L'obiettivo minimo che ci si prefigge è di migliorare ogni volta che essa viene calcolata (a meno di casi particolari e giustificati). L'obiettivo ottimale è quello di migliorare ogni volta di almeno il 10\%. Riassumendo:
				\begin{description}
					\item[Metrica utilizzata per quantificare l'obiettivo]: produttività;
					\item[Soglia di accettabilità]: migliorare rispetto all'ultima misurazione (> 0\%);
					\item[Soglia di ottimalità]: migliorare in misura maggiore del 10\% rispetto all'ultima misurazione.
				\end{description}
				Particolare attenzione vuole essere posta sul processo di verifica, che è quello più dispendioso in termini di ore. È quindi stata introdotta una metrica apposta che viene utilizzata per misurare l'efficacia di un'attività di tale processo: efficacia di una revisione. L'obiettivo minimo è di migliorare ogni volta che tale metrica viene misurata (a meno di casi particolari e giustificati). L'obiettivo ottimale è quello di migliorare ogni volta di almeno il 5\%. Riassumendo:
				\begin{description}
					\item[Metrica utilizzata per quantificare l'obiettivo]: efficacia di una revisione;
					\item[Soglia di accettabilità]: migliorare rispetto all'ultima misurazione (> 0\%);
					\item[Soglia di ottimalità]: migliorare in misura maggiore del 5\% rispetto all'ultima misurazione.
				\end{description}
				Per una descrizione dettagliata delle metriche qui utilizzate e per una maggiore comprensione degli indici di ottimalità e accettabilità presentati si faccia riferimento alle sezioni
				\begin{itemize}
					\item \nameref{subsec:produttivita}
					\item \nameref{subsec:effRevisione}
				\end{itemize}						

		\level{3}{Qualità di prodotto}  \label{subsec:obiettiviprodotto}
			I prodotti che vengono realizzati durante l'intero progetto sono sostanzialmente di due tipi: documenti e software. Nelle prossime sezioni, si enunciano gli obiettivi che si intendono raggiungere suddivisi per tipologia di prodotto. Per ogni obiettivo, poi, vengono specificati i criteri con i quali si effettuano le misurazioni sulla qualità (per capire quanto si è vicini all'obiettivo). Infine, per ogni criterio scelto vengono dichiarati i valori minimi che si intendono raggiungere, oltre a quelli ottimali.
			\level{4}{Qualità dei documenti}
				Gli obiettivi di qualità riguardanti i documenti ai quali il gruppo \groupname{} desidera arrivare nell'arco dell'intero progetto sono i seguenti:
				\begin{itemize}
					\item i documenti devono essere comprensibili da individui dotati di un'istruzione media;
					\item i documenti devono essere corretti a livello ortografico;
					\item i documenti non devono contenere concetti errati.
				\end{itemize}
				Descriviamo ora quali sono le metriche o i criteri che si intendono utilizzare per quantificare la vicinanza a ognuno degli obiettivi sopra descritti. Individuiamo inoltre le soglie di accettabilità e ottimalità, per fissare quantitativemente i punti ai quali desideriamo arrivare.
				\level{5}{Leggibilità e comprensibilità}
					Per cercare di capire quanto i documenti siano effettivamente leggibili e comprensibili da persone dotate di un'istruzione media viene utilizzato l'indice Gulpease. Si desidera che i documenti posseggano costantemente un indice maggiore a 35 (soglia di accettabilità). Si dovrebbe tuttavia cercare di raggiungere un valore più alto, considerato ottimale, ovvero 50. Riassumendo:
					\begin{description}
						\item[Metrica utilizzata per quantificare l'obiettivo]: indice Gulpease;
						\item[Soglia di accettabilità]: valori almeno maggiori di 35;
						\item[Soglia di ottimalità]: valori almeno maggiori di 50.
					\end{description}
					Per una descrizione dettagliata della metrica qui utilizzata e per una maggiore comprensione degli indici di ottimalità e accettabilità presentati si faccia riferimento alla sezione \nameref{subsec:gulpease}.
				\level{5}{Correttezza ortografica}
					Per cercare di capire quanto i documenti siano effettivamente corretti a livello ortografico si utilizza la seguente metrica: percentuale di errori ortografici rinvenuti in modo automatico e non corretti. Si desidera che tutti gli errori ortografici che sono stati trovati siano corretti manualmente (se fosse fatto in modo automatico non sarebbe molto sicuro). In questo caso, dunque, si l'obiettivo minimo coincide con l'obiettivo ottimale. Riassumendo:
					\begin{description}
						\item[Metrica utilizzata per quantificare l'obiettivo]: percentuale di errori ortografici rinvenuti in modo automatico e non corretti;
						\item[Soglia di accettabilità]: tutti gli errori trovati sono corretti manualmente;
						\item[Soglia di ottimalità]: tutti gli errori trovati sono corretti manualmente.
					\end{description}
					Per una descrizione dettagliata della metrica qui utilizzata e per una maggiore comprensione degli indici di ottimalità e accettabilità presentati si faccia riferimento alla sezione \nameref{subsec:erroriOrtografici}.
				\level{5}{Correttezza concettuale}
					Per cercare di capire quanto i documenti siano effettivamente corretti a livello concettuale si utilizza la seguente metrica: percentuale di errori concettuali rinvenuti e non corretti. Si desidera che al massimo il 5\% degli errori concettuali rinvenuti non siano corretti. L'obiettivo ottimale sarebbe quello di avere documenti senza alcun errore di questo tipo. Riassumendo:
					\begin{description}
						\item[Metrica utilizzata per quantificare l'obiettivo]: percentuale di errori concettuali rinvenuti e non corretti;
						\item[Soglia di accettabilità]: almeno il 95\% degli errori trovati è stato corretto;
						\item[Soglia di ottimalità]: tutti gli errori trovati sono stati corretti.
					\end{description}
					Per una descrizione dettagliata della metrica qui utilizzata e per una maggiore comprensione degli indici di ottimalità e accettabilità presentati si faccia riferimento alla sezione \nameref{subsec:erroriConcettuali}.
			\level{4}{Qualità del software}
				Gli obiettivi di qualità del software ai quali il gruppo \groupname{} desidera arrivare nell'arco del progetto sono alcuni di quelli che sono enunciati all'interno delle norme [ISO/IEC 9126]. Vengono riassunti in seguito:
				\begin{itemize}
					\item il \insglo{prodotto} dispone di tutte le funzioni di cui gli utenti hanno bisogno;
					\item il \insglo{prodotto} permette agli utenti di utilizzare le funzioni in maniera semplice ed efficace;
					\item il codice risulta manutenibile e facilmente comprensibile;
					\item il \insglo{prodotto} è robusto e non collassa in seguito a situazioni anomale;
					\item il \insglo{prodotto} è testato in ogni sua parte e in ogni situazione nella quale si può trovare;
					\item il \insglo{prodotto} garantisce un funzionamento senza interruzioni;
					\item il \insglo{prodotto} è facilmente installabile.
				\end{itemize}
				Sebbene la sicurezza sia indicata come una delle caratteristiche fondamentali all'interno delle norme [ISO/IEC 9126], essa non deve essere tenuta particolarmente in considerazione, in quanto esula dagli obiettivi del presente progetto.\\
				Non intendiamo nemmeno garantire le qualità in uso (ovvero le caratteristiche che assumono rilevanza solo nel momento in cui il \insglo{prodotto} è effettivamente utilizzato in un certo contesto). Infatti, sebbene queste siano indicate come fondamentali all'interno delle norme [ISO/IEC 9126], il presente progetto termina prima del rilascio effettivo del \insglo{software}.\\
				Descriviamo ora quali sono le metriche o i criteri che si intendono utilizzare per quantificare la vicinanza a ognuno degli obiettivi sopra descritti. Individuiamo inoltre le soglie di accettabilità e ottimalità, per fissare quantitativemente i punti ai quali desideriamo arrivare.
				\level{5}{Funzionalità}
					Per capire se il \insglo{prodotto} dispone effettivamente delle funzioni di cui gli utenti hanno bisogno viene utilizzata la seguente metrica: numero di requisiti funzionali realizzati. Alla fine del progetto si desidera che siano stati realizzati almeno il 100\% dei requisiti obbligatori, il 98\% di quelli desiderabili e il 90\% di quelli opzionali (soglia di accettabilità). Ci si augura, invece, che si realizzino tutti i requisiti obbligatori e opzionali e almeno il 95\% di quelli opzionali (soglia di ottimalità). Riassumendo:
					\begin{description}
						\item[Metrica utilizzata per quantificare l'obiettivo]: numero di requisiti funzionali realizzati;
						\item[Soglia di accettabilità]: 100\% obbligatori, 98\% desiderabili, 90\% opzionali;
						\item[Soglia di ottimalità]: 100\% obbligatori, 100\% desiderabili, 95\% opzionali.
					\end{description}
					Per una descrizione dettagliata della metrica qui utilizzata e per una maggiore comprensione degli indici di ottimalità e accettabilità presentati si faccia riferimento alla sezione \nameref{subsec:numReqFunzionali}.
				\level{5}{Semplicità d'uso}
					Per capire se il \insglo{prodotto} è facilmente utilizzabile si utilizza la seguente metrica: numero di parametri necessari in un metodo a disposizione dell'utente finale. L'obiettivo minimo è di avere solo metodi con al massimo 4 parametri. L'obiettivo ideale è che non vi siano metodi con un numero di parametri maggiore a 2. Si noti che i metodi di cui si tiene conto sono solo quelli ai quali effettivamente l'utente finale ha accesso e tramite i quali utilizza il \insglo{prodotto}; qui non si parla di tutti i metodi di tutte le classi, ma solo di quelle dedicate all'utente. Riassumendo:
					\begin{description}
						\item[Metrica utilizzata per quantificare l'obiettivo]: numero di parametri necessari in un metodo a disposizione dell'utente finale;
						\item[Soglia di accettabilità]: solo metodi con al più 4 parametri;
						\item[Soglia di ottimalità]: solo metodi con al più 2 parametri.
					\end{description}
					Per una descrizione dettagliata della metrica qui utilizzata e per una maggiore comprensione degli indici di ottimalità e accettabilità presentati si faccia riferimento alla sezione \nameref{subsec:numParMet}.
				\level{5}{Manutenibilità e comprensibilità del codice}
					Per avere un'idea di quanto il codice scritto sia comprensibile e manutenibile si utilizzano le metriche riportate in seguito. Si noti che per ognuna di esse sono riportati i valori accettabili e ottimali, ovvero vengono fissati gli obiettivi che si intendono raggiungere nell'arco del progetto.
					\begin{itemize}
						\item numero di statement di un metodo:
						\begin{description}
							\item[Soglia di accettabilità]: solo metodi con al più 60 statement;
							\item[Soglia di ottimalità]: solo metodi con al più 30 statement.
						\end{description}
						\item numero di campi dati di una classe:
						\begin{description}
							\item[Soglia di accettabilità]: solo classi con al più 10 campi dati;
							\item[Soglia di ottimalità]: solo classi con al più 6 campi dati.
						\end{description}
						\item grado di accoppiamento:
						\begin{description}
							\item[Soglia di accettabilità]: massimo 7 classi dipendenti per ogni package;
							\item[Soglia di ottimalità]: massimo 5 classi dipendenti per ogni package.
						\end{description}
					\end{itemize}
					Per una descrizione dettagliata delle metriche qui utilizzate e per una maggiore comprensione degli indici di ottimalità e accettabilità presentati si faccia riferimento alle sezioni
					\begin{itemize}
						\item \nameref{subsec:numStatement}
						\item \nameref{subsec:numFields}
						\item \nameref{subsec:gradoAccoppiamento}
					\end{itemize}			
				\level{5}{Robustezza}
					Per misurare quanto il prodotto è robusto e per capire quanto bene si comporti nel caso in cui avvengano delle anomalie si utilizza la seguente metrica: percentuale di test di robustezza superati. L'obiettivo minimo che si intende raggiungere è quello di superare almeno il 90\% dei test di robustezza (nei quali si passano input diversi da quelli attesi). Tuttavia l'obiettivo reale dovrebbe essere quello di superarli tutti. Riassumendo:
					\begin{description}
						\item[Metrica utilizzata per quantificare l'obiettivo]: percentuale di test di robustezza superati;
						\item[Soglia di accettabilità]: superato almeno il 90\% dei test di robustezza;
						\item[Soglia di ottimalità]: superato il 100\% dei test di robustezza.
					\end{description}
					Per una descrizione dettagliata della metrica qui utilizzata e per una maggiore comprensione degli indici di ottimalità e accettabilità presentati si faccia riferimento alla sezione \nameref{subsec:percTestRobustezza}.
				\level{5}{Testabilità}
					Per far si che il prodotto sia facilmente testabile e per capire allo stesso tempo in che percentuale il prodotto è stato testato si fa uso delle metriche riportate in seguito. Si noti che per ognuna di esse sono riportati i valori accettabili e ottimali, ovvero vengono fissati gli obiettivi che si intendono raggiungere nell'arco del progetto.
					\begin{itemize}
						\item complessità ciclomatica:
						\begin{description}
							\item[Soglia di accettabilità]: solo valori inferiori a 10;
							\item[Soglia di ottimalità]: solo valori inferiori a 5.
						\end{description}
						\item livello di annidamento:
						\begin{description}
							\item[Soglia di accettabilità]: valori minori di 5;
							\item[Soglia di ottimalità]: valori minori di 3.
						\end{description}
						\item grado di accoppiamento:
						\begin{description}
							\item[Soglia di accettabilità]: valori minori di 7;
							\item[Soglia di ottimalità]: valori minori di 5.
						\end{description}
					\end{itemize}
					Per una descrizione dettagliata delle metriche qui utilizzate e per una maggiore comprensione degli indici di ottimalità e accettabilità presentati si faccia riferimento alle sezioni
					\begin{itemize}
						\item \nameref{subsec:complCiclomatica}
						\item \nameref{subsec:livAnnidamento}
						\item \nameref{subsec:gradoAccoppiamento}
					\end{itemize}	
	\level{2}{Scadenze temporali}
		L'organizzazione e la pianificazione delle attività di controllo della qualità che vengono presentate in seguito si basano sulle scadenze temporali che sono state fissate e descritte all'interno del documento \insdoc{Piano di Progetto v7.00}. Vengono qui riassunte:
		\begin{itemize}
			\item \insrev{Revisione dei Requisiti}: \insdate{16}{02}{2015} (revisione formale);
			\item \insrev{Revisione di Progettazione}: \insdate{24}{04}{2015} (revisione di progresso);
			\item \insrev{Revisione di Qualifica}: \insdate{29}{05}{2015} (revisione di progresso);
			\item \insrev{Revisione di Accettazione}: \insdate{18}{06}{2015} (revisione formale).
		\end{itemize}
	\level{2}{Responsabilità} \label{subsec:responsabilita}
		La qualità è responsabilità di tutti, nessuno escluso. Tutti i membri del \insglo{team} contribuiscono con il loro lavoro a costruire (o a non costruire) la qualità del \insglo{prodotto} finale e dei processi attraverso i quali ci si arriva. La qualità, infatti, viene costruita nel tempo, anche grazie alla cura e all'attenzione che viene posta nello svolgere i vari compiti. \\
		Di seguito vengono riportate le responsabilità riguardanti la qualità, catalogate in base al ruolo.
		\begin{itemize}
			\item Responsabile di Progetto:
			\begin{itemize}
				\item deve assicurarsi che i processi siano attentamente controllati e valutati in modo oggettivo (in modo tale che essi siano 
				migliorabili);
				\item deve assegnare i compiti relativi alla verifica di prodotti a persone per quanto possibile indipendenti dallo sviluppo di essi 
				(e secondo quanto descritto nel \insdoc{Piano di Progetto v7.00});
				\item deve pianificare attentamente controlli sul processo di qualità stesso.
			\end{itemize}
			\item Amministratore di Progetto:
			\begin{itemize}
				\item deve assicurarsi che siano sempre disponibili le risorse necessarie, sia realizzative che di verifica e validazione;
				\item deve fare in modo che il processo di verifica sia quanto più automatizzabile possibile (e quindi efficiente).
			\end{itemize}
			\item Analista:
			\begin{itemize}
				\item deve assicurarsi di documentare i requisiti qualitativi oltre a quelli funzionali;
				\item deve assicurarsi di aderire agli standard e alle norme riguardanti la documentazione da lui stesso prodotta.
			\end{itemize}
			\item Progettista:
			\begin{itemize}
				\item deve indirizzare nelle specifica tecnica i requisiti di qualità;
				\item deve realizzare la progettazione in modo da indirizzare completamente, correttamente ed efficacemente i requisiti di qualità;
				\item deve assicurarsi di aderire agli standard applicabili nella progettazione.
			\end{itemize}
			\item Programmatore:
			\begin{itemize}
				\item deve codificare secondo le norme imposte all'interno del progetto;
				\item deve codificare utilizzando gli standard applicabili;
				\item deve fornire i test necessari per effettuare parte delle verifiche sulle unità \insglo{software} prodotte.
			\end{itemize}
			\item Verificatore:
			\begin{itemize}
				\item deve eseguire le procedure di verifica previste dal presente documento e descritte nelle \insdoc{Norme di Progetto v7.00};
				\item deve tracciare gli errori rilevati durante ciascuna \insglo{fase} del progetto affinché possano essere risolti.
			\end{itemize}
		\end{itemize}
		Per maggiori dettagli circa i compiti assegnati a ciascun ruolo si vedano le \insdoc{Norme di Progetto v7.00}.

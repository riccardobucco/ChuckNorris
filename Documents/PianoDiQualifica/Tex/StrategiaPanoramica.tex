\level{1}{Visione generale della strategia di gestione della qualità}
	\level{2}{Obiettivi di qualità}
		In questa sezione vengono riportati gli obiettivi di qualità che il gruppo \groupname{} si impegna a perseguire durante lo svolgimento dell'intero progetto.\\
		Per capire se un certo obiettivo è stato raggiunto oppure no, il \groupname{} fa uso di standard, modelli e metriche. Ognuno di questi fa uso di scale differenti e fissate a priori: qualunque sia il criterio utilizzato per misurare e dunque quantificare la vicinanza a un certo obiettivo, abbiamo fissato dei valori minimi che intendiamo raggiungere nell'arco dell'intero progetto. Oltre a ciò abbiamo anche fissato dei valori che riteniamo ottimali e che devono essere sperabilmente (ma non obbligatoriamente) raggiunti.\\
		Riassumendo:
		\begin{enumerate}
			\item vengono enunciati alcuni obiettivi;
			\item vengono scelti modelli, metriche, criteri, scale ecc. per poter misurare quanto si è vicini al raggiungimento di un obiettivo;
			\item in ognuno di questi modelli, metriche, criteri, scale ecc. si identifica un valore minimo che si intende raggiungere nell'arco del progetto (soglia di accettabilità) e un valore ottimale al quale si dovrebbe sperabilmente arrivare (soglia di ottimalità). Non possono essere acccettai valori negativi, ovvero valori al di sotto dell'accettabilità. 
		\end{enumerate}

		\level{3}{Qualità di processo}  \label{subsec:obiettiviprocesso}
			Assicurare la qualità dei processi è indispensabile durante lo svolgimento del progetto per le seguenti ragioni:
			\begin{itemize}
				\item aiuta ad ottimizzare l'uso delle risorse;
				\item fa in modo che i costi siano maggiormente contenuti;
				\item migliora la stima dei rischi e degli impegni.
			\end{itemize}
			Un altro fattore da tenere sempre in considerazione risiede nel fatto che molto spesso prodotti scadenti derivano da pessimi processi.\\
			Le caratteristiche ottimali che desideriamo che i processi posseggano vengono riportate in seguito:
			\begin{itemize}
				\item un processo dovrebbe essere in grado di migliorare continuamente le proprie performance;
				\item le performance di un processo dovrebbero essere costantemente misurabili;
				\item un processo, durante il proprio svolgimento, dovrebbe cercare di perseguire sempre degli obiettivi quantitativi di miglioramento fissati a priori.
			\end{itemize}
			Per misurare quanto si è vicini agli obiettivi di qualità appena enunciati si è deciso di adottare il modello \insglo{CMM}. In particolare si vuole raggiungere almeno il livello 2 previsto da tale scala. Il livello ottimale che sperabilmente dovremmo raggiungere è il 4. Riassumendo:
			\begin{description}
				\item[Modello utilizzato per quantificare gli obiettivi]: \insglo{CMM};
				\item[Soglia di accettabilità]: livello 2 previsto da \insglo{CMM};
				\item[Soglia di ottimalità]: livello 4 previsto da \insglo{CMM}.
			\end{description}

		\level{3}{Qualità di prodotto}  \label{subsec:obiettiviprodotto}
			I prodotti che vengono realizzati durante l'intero progetto sono sostanzialmente di due tipi: documenti e software. Nelle prossime sezioni, per tipologia di prodotto si enunciano gli obiettivi che si intendono raggiungere. Per ogni obiettivo, poi, vengono specificati i criteri con i quali si effettuano le misurazioni sulla qualità (per capire quanto si è vicini all'obiettivo). Infine, per ogni criterio scelto vengono dichiarati i valori minimi che si intendono raggiungere, oltre a quelli ottimali.
			\level{4}{Qualità dei documenti}

			\level{4}{Qualità del software}
				Gli obiettivi di qualità del software ai quali il gruppo \groupname{} desidera arrivare nell'arco del progetto sono alcuni di quelli che sono enunciati all'interno delle norme [ISO/IEC 9126]. Vengono riassunti in seguito:
				\begin{itemize}
					\item il \insglo{prodotto} dispone di tutte le funzioni di cui gli utenti hanno bisogno;
					\item il \insglo{prodotto} permette agli utenti di utilizzare le funzioni in maniera semplice ed efficace;
					\item il \insglo{prodotto} fornisce prestazioni accettabili;
					\item il \insglo{prodotto} garantisce un funzionamento senza interruzioni;
					\item il \insglo{prodotto} è facilmente installabile.
				\end{itemize}
				Sebbene la sicurezza sia indicata come una delle caratteristiche fondamentali all'interno delle norme [ISO/IEC 9126], essa non deve essere tenuta particolarmente in considerazione, in quanto esula dagli obiettivi del presente progetto.\\
				Non intendiamo nemmeno garantire le qualità in uso (ovvero le caratteristiche che assumono rilevanza solo nel momento in cui il \insglo{prodotto} è effettivamente utilizzato in un certo contesto). Infatti, sebbene queste siano indicate come fondamentali all'interno delle norme [ISO/IEC 9126], il presente progetto termina prima del rilascio effettivo del \insglo{software}.\\
				Descriviamo ora quali sono le metriche o i criteri che si intendono utilizzare per quantificare la vicinanza a ognuno degli obiettivi sopra descritti. Individuiamo inoltre le soglie di accettabilità e ottimalità, per fissare quantitativemente i punti ai quali desideriamo arrivare.
				\level{5}{Funzionalità}
					Per capire se il \insglo{prodotto} dispone effettivamente delle funzioni di cui gli utenti hanno bisogno viene utilizzata la seguente metrica: numero di requisiti funzionali realizzati. Alla fine del progetto si desidera che siano stati realizzati almeno il 100\% dei requisiti obbligatori, il 98\% di quelli desiderabili e il 90\% di quelli opzionali (soglia di accettabilità). Ci si augura, invece, che si realizzino tutti i requisiti obbligatori e opzionali e almeno il 95\% di quelli opzionali (soglia di ottimalità). Riassumendo:
					\begin{description}
						\item[Metrica utilizzata per quantificare l'obiettivo]: numero di requisiti funzionali realizzati;
						\item[Soglia di accettabilità]: 100\% obbligatori, 98\% desiderabili, 90\% opzionali;
						\item[Soglia di ottimalità]: 100\% obbligatori, 100\% desiderabili, 95\% opzionali.
					\end{description}
				\level{5}{Semplicità d'uso}
					Per capire se il \insglo{prodotto} è facilmente utilizzabile si utilizza la seguente metrica: numero di parametri necessari in un metodo a disposizione dell'utente finale. L'obiettivo minimo è di avere solo metodi con al massimo 4 parametri. L'obiettivo ideale è che non vi siano metodi con un numero di parametri maggiore a 2. Si noti che i metodi di cui si tiene conto sono solo quelli ai quali effettivamente l'utente finale ha accesso e tramite i quali utilizza il \insglo{prodotto}. Riassumendo:
					\begin{description}
						\item[Metrica utilizzata per quantificare l'obiettivo]: numero di parametri necessari in un metodo a disposizione dell'utente finale;
						\item[Soglia di accettabilità]: solo metodi con al più 4 parametri;
						\item[Soglia di ottimalità]: solo metodi con al più 2 parametri.
					\end{description}
				\level{5}{Buone prestazioni}
					
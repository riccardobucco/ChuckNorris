% !TEX encoding = UTF-8 Unicode

\section{Obiettivi di qualità}
	In questa sezione vengono riportati gli obiettivi di qualità che il gruppo \groupname{} si impegna a perseguire durante lo svolgimento dell'intero 
	progetto.
	\subsection{Qualità di processo}
		Assicurare la qualità dei processi è indispensabile durante lo svolgimento del progetto per le seguenti ragioni:
		\begin{itemize}
			\item aiuta ad ottimizzare l'uso delle risorse;
			\item fa in modo che i costi siano maggiormente contenuti;
			\item migliora la stima dei rischi e degli impegni.
		\end{itemize}
		Un altro fattore da tenere sempre in considerazione risiede nel fatto che molto spesso prodotti scadenti derivano da pessimi processi.\\
		Per garantire la qualità dei processi impiegati si è quindi deciso di adottare il modello CMM. Il suo scopo è quello di valutare la maturità dei 
		processi e di fornire istruzioni su come migliorarli.\\
		Tale modello indica delle caratteristiche ottimali che i processi dovrebbero avere. Vengono riportate in seguito.
		\begin{itemize}
			\item Un processo dovrebbe essere in grado di migliorare continuamente le proprie performance. Per arrivare ad ottenere ciò è necessario 
			capire in modo approfondito le relazioni tra processi diversi. Inoltre, è necessario saper monitorare costantemente le performance del 
			processo in questione.
			\item Devono essere fissati degli obiettivi quantitativi di miglioramento per il processo in questione. Tali obiettivi devono essere dei veri 
			e propri requisiti nello svolgimento del progetto e si deve tener conto di essi all'interno del Piano di Progetto.
		\end{itemize}
	\subsection{Qualità di prodotto}
		Il modello che meglio si adatta a rappresentare caratteristiche differenti di qualità dei prodotti software è quello descritto nelle norme 
		[ISO/IEC 9126]. Il gruppo \groupname{} si impegna dunque a garantire quanto segue:
		\begin{itemize}
			\item il prodotto disponde di tutte le funzioni di cui gli utenti hanno bisogno;
			\item il prodotto permette agli utenti di utilizzare le funzioni in maniera semplice ed efficace;
			\item il prodotto fornisce prestazioni accettabili;
			\item il prodotto garantisce un funzionamento senza interruzioni;
			\item il prodotto è facilmente installabile.
		\end{itemize}
		Sebbene la sicurezza sia indicata come una delle caratteristiche fondamentali all'interno delle norme [ISO/IEC 9126], essa non verrà tenuta 
		particolarmente in considerazione, in quanto esula dagli obiettivi del progetto.\\
		Facendo sempre riferimento allo standard, il team si impegna a garantire tanto le qualità interne (ovvero proprietà intrinseche del software) 
		quanto quelle esterne (ovvero proprietà che hanno rilevanza solo per l'utente). Non garantiremo, invece, le qualità in uso (ovvero le 
		caratteristiche che assumono rilevanza solo nel momento in cui il prodotto è effettivamente utilizzato in un certo contesto), in quanto il 
		progetto termina prima del rilascio effettivo del software.

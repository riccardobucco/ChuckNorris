% !TEX encoding = UTF-8 Unicode

\section{Qualità di prodotto perseguite}
	In osservanza dello standard [ISO/IEC 9126] inerente le qualità di prodotto, il gruppo \groupname{} si impegna a garantire, nello sviluppo del \textit{framework} da produrre, una serie di qualità, per ognuna delle quali vengono definite, per quanto possibile allo stadio attuale del progetto, una misura, una metrica e lo strumento individuato per la loro misurazione.
	\subsection{Funzionalità}
		Per funzionalità di un prodotto software si intende la quantità di funzioni che esso mette a disposizione.\\
		Nello specifico, il \textit{framework} fornisce tutte funzionalità necessarie a garantire il rispetto dei requisiti indicati nel documento "\insdoc{Analisi dei Requisiti v1.0}", che vengono implementate nel modo ritenuto più efficace ed efficiente dal team di sviluppo.\\
		La misurazione della funzionalità si baserà sulla quantità di funzioni, individuate dai requisiti documentati, che si riuscirà a fornire.\\
		Questa qualità si ritiene soddisfatta se e solo se risultano soddisfatti tutti i requisiti obbligatori.\\
		Lo strumento individuato per la misurazione di questa qualità è il superamento dei test previsti nel documento \insdoc{Analisi dei requisiti v1.0}.
	\subsection{Affidabilità}
		Per affidabilità di un prodotto software si intende la sua capacità di dare i risultati attesi ogni volta che se ne ha bisogno.\\
		Nello specifico, il \textit{framework} prodotto deve risultare robusto e flessibile in caso di errori.\\
		La misurazione dell'affidabilità si basa sulla percentuale di test di sistema terminati nel modo atteso.
	\subsection{Usabilità}
		Per usabilità di un prodotto si intende la sua qualità di poter essere utilizzato indistintamente da più persone.\\
		Nello specifico, il \textit{framework} prodotto, proprio per la sua natura, deve essere di facile utilizzo per la categoria di utenti che si prevede lo utilizzi.\\
		A causa della natura fortemente soggettiva di questa qualità non è possibile definire una metrica.
	\subsection{Manutenibilità}
		Per manutenibilità di un prodotto si intende la facilità di apportarvi modifiche correttive, adattive e migliorative.\\
		Nello specifico, il \textit{framework} prodotto deve risultare ampiamente commentato per favorire la comprensione di manutentori esterni al team di sviluppo, e predisposto il più possibile al riuso e a un possibile ampliamento.\\
		Al momento il team di sviluppo non è in grado di fornire una metrica per la misurazione di questa qualità.\\
	\subsection{Efficienza}
		Per efficienza di un prodotto si intende la quantità di risorse necessarie a farlo funzionare.\\
		Nello specifico, il \textit{framework} prodotto deve fornire le funzionalità (garantite) impiegando la minore quantità di tempo.\\
		L'unità di misura per quantificare questa qualità verrà definita nella fase di progettazione.

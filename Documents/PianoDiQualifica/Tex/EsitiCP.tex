%DA VERIFICARE

\level{1}{Resoconto delle varie attività di verifica - Fase CP}
Sono riportati in questa appendice tutti i risultati ottenuti nei momenti di verifica, stabiliti nel \insdoc{Piano di Progetto v6.00} secondo la strategia di misurazione per il perseguimento della qualità individuata nel presente documento.\\
Durante questa \insglo{fase} sono usciti i risultati della Revisione di Progettazione, che hanno influito notevolmente sull'andamento del progetto.

\level{2}{Verifica dei prodotti}
\level{3}{Documenti}
In questa sezione vengono riportati i risultati delle attività di verifica svolte sui documenti. Esse sono si sue tipi:
\begin{itemize}
\item verifiche manuali;
\item verifiche automatizzate.
\end{itemize}
\level{4}{Verifiche manuali}
Le attività di verifica manuale della documentazione prodotta sono state svolte in base alla \insglo{procedura} riguardante la verifica dei documenti che è descritta nel documento \insdoc{Norme di Progetto v6.00}.
La verifica manuale ha permesso di individuare soprattutto errori riguardanti le seguenti tipologie:
\begin{itemize}
\item descrizioni imprecise di classi e metodi;
\item errori concettuali;
\item errori nell'utilizzo della lingua inglese.
\end{itemize}
Si riporta di seguito la quantità degli errori rilevati e risolti, per ciascuna tipologia, durante l'intera \insglo{fase}.
\begin{table}[H]
	\centering
		\begin{tabu}{| l | c |}
			\hline
				Descrizioni imprecise	&	12\\ \hline
				Errori concettuali	&	9\\ \hline
				Errori di inglese  &  35\\ \hline
		\end{tabu}
		\caption{Errori trovati tramite verifica manuale dei documenti durante la Fase CP}
\end{table}

Sono stati segnalati diversi errori nei diagrammi \insglo{UML} contenuti nella versione della \insdoc{Specifica Tecnica} presentata alla Revisione di Progettazione; tali errori sono stati quindi corretti in vista della Revisione di Qualifica.\\
Si può inoltre notare come gli errori concettuali che sono stati trovati siano ancora in leggera diminuzione, tale fatto ci porta a pensare che i documenti si stanno piano piano avvicinando al loro contenuto ottimale.\\
Si noti invece come vi sia ancora una notevole quantità di errori nell'uso della lingua inglese imputati alla stesura dei commenti del codice ed alla stesura del documento \insdoc{Manuale utente v3.00}.

\level{4}{Verifiche automatizzate}
Le attività di verifica automatizzate sono state effettuate secondo le procedure e attraverso gli strumenti descritti nel documento \insdoc{Norme di Progetto v6.00}. Esse hanno permesso di rilevare diversi errori riguardanti le seguenti tipologie:
\begin{itemize}
\item ortografia errata
\item utilizzo errato dei comandi \LaTeX{} indicati nelle \insdoc{Norme di Progetto v6.00};
\item norme tipografiche non rispettate.
\end{itemize}
Di seguito è presentato un riassunto della quantità di errori trovati (e successivamente risolti) utilizzando la verifica automatica.
\begin{table}[H]
	\centering
		\begin{tabu}{| l | c |}
			\hline
			Errori ortografici	& 93	\\ \hline
			Utilizzo errato \LaTeX{}	& 0	\\ \hline
			Errori riguardanti norme tipografiche	& 7	\\ \hline
		\end{tabu}
	\caption{Errori trovati tramite verifica automatica dei documenti durante la Fase CP}
\end{table}

Risulta sempre più evidente il miglioramento della qualità della documentazione prodotta.\\ \\
Si riportano i risultati delle misurazioni dell'indice di leggibilità Gulpease relative ai documenti modificati in questa \insglo{fase}.


\begin{table}[H]
	\centering
		\begin{tabu}{| l | c | c |}
			\hline
			Documenti 							& Gulpease	& Esito		\\ \hline \hline
			Piano di progetto v6.00				& 76 		& Superato  \\ \hline
			Norme di Progetto v6.00 			& 60		& Superato  \\ \hline
			Piano di Qualifica v6.00 			& 80		& Superato  \\ \hline
			Definizione di \insglo{prodotto} v3.00		& 63		& Superato  \\ \hline
			Specifica tecnica v3.00				& 68		& Superato \\ \hline
			Manuale utente v3.00				& 95		& Superato \\ \hline
			Glossario v6.00					 	& 70 		& Superato  \\ \hline
		\end{tabu}
	\caption{Esiti del calcolo dell'indice di leggibilità effettuato tramite strumenti automatici durante la Fase CP}
\end{table}

\level{3}{Codice}
In questa sezione sono riportati i risultati delle metriche, calcolati nei momenti di verifica, per l'Applicazione \insglo{Android}, sviluppata per ultima in quanto rispondente ad un requisito opzionale.

\level{4}{Numero di requisiti funzionali realizzati}
	\level{5}{Numero di requisiti funzionali obbligatori realizzati}

Si riportano di seguito le percentuali di requisiti funzionali obbligatori realizzati dalle componenti di \insglo{Norris}.
\begin{table}[H]
	\centering
		\begin{tabu}{| l | c | c |}
			\hline
			Componente	& Percentuale requisiti funzionali obbligatori soddisfatti	& Esito		\\ \hline \hline
			InternalAPIManager	& 100\% 	& Ottimale  \\ \hline
			ExternalAPIManager  & 	100\%	& Ottimale  \\ \hline
			DataModel  & 	100\%	& Ottimale  \\ \hline
		\end{tabu}
	\caption{Esiti del calcolo delle percentuali di requisiti funzionali obbligatori realizzati da Norris durante la Fase CP}
\end{table}
Come è possibile notare dalla tabella la percentuale dei requisiti funzionali obbligatori soddisfatti da \insglo{Norris} ha raggiunto un esito complessivamente ottimale. 
\\ \\
Si riportano di seguito le percentuali di requisiti funzionali obbligatori realizzati dalle componenti di \insglo{Chuck}.
\begin{table}[H]
	\centering
		\begin{tabu}{| l | c | c |}
			\hline
			Componente	& Numero requisiti funzionali obbligatori soddisfatti	& Esito		\\ \hline \hline
			Directive	& 100 \% 	& Ottimale  \\ \hline
			ChartView  & 	100\%	& Ottimale  \\ \hline
			ViewModel  & 	100\%	& Ottimale  \\ \hline
			DataModel  & 	100\%	& Ottimale  \\ \hline
		\end{tabu}
	\caption{Esiti del calcolo delle percentuali di requisiti funzionali obbligatori realizzati da Chuck durante la Fase CP}
\end{table}
Come è possibile notare dalla tabella la percentuale dei requisiti funzionali obbligatori soddisfatti da \insglo{Chuck} ha raggiunto un esito complessivamente ottimale.

\level{5}{Numero di requisiti funzionali desiderabili realizzati}

Si riportano di seguito le percentuali di requisiti funzionali desiderabilli realizzati dalle componenti di \insglo{Norris}.
\begin{table}[H]
	\centering
		\begin{tabu}{| l | c | c |}
			\hline
			Componente	& Percentuale requisiti funzionali desiderabilli soddisfatti	& Esito		\\ \hline \hline
			InternalAPIManager	& 95\% 	& Accettabile  \\ \hline
			ExternalAPIManager  & 	96\%	& Accettabile  \\ \hline
			DataModel  & 	100\%	& Ottimale  \\ \hline
		\end{tabu}
	\caption{Esiti del calcolo delle percentuali di requisiti funzionali desiderabilli realizzati da Norris durante la Fase CP}
\end{table}
Come è possibile notare dalla tabella la percentuale dei requisiti funzionali desiderabilli soddisfatti da \insglo{Norris} ha raggiunto un esito generalmente accettabile. 
\\ \\
Si riportano di seguito le percentuali di requisiti funzionali desiderabilli realizzati dalle componenti di \insglo{Chuck}.
\begin{table}[H]
	\centering
		\begin{tabu}{| l | c | c |}
			\hline
			Componente	& Numero requisiti funzionali desiderabilli soddisfatti	& Esito		\\ \hline \hline
			Directive	& 95\% 	& Accettabile  \\ \hline
			ChartView  & 	98\%	& Accettabile  \\ \hline
			ViewModel  & 	95\%	& Accettabile  \\ \hline
			DataModel  & 	100\%	& Ottimale  \\ \hline
		\end{tabu}
	\caption{Esiti del calcolo delle percentuali di requisiti funzionali desiderabilli realizzati da Chuck durante la Fase CP}
\end{table}
Come è possibile notare dalla tabella la percentuale dei requisiti funzionali desiderabilli soddisfatti da \insglo{Chuck} ha raggiunto un esito generalmente ottimale.

\level{5}{Numero di requisiti funzionali opzionali realizzati}

Si riportano di seguito le percentuali di requisiti funzionali opzionali realizzati dalle componenti di \insglo{Norris}.
\begin{table}[H]
	\centering
		\begin{tabu}{| l | c | c |}
			\hline
			Componente	& Percentuale requisiti funzionali opzionali soddisfatti	& Esito		\\ \hline \hline
			InternalAPIManager	& 95\% 	& Accettabile  \\ \hline
			ExternalAPIManager  & 	96\%	& Accettabile  \\ \hline
			DataModel  & 	100\%	& Ottimale  \\ \hline
		\end{tabu}
	\caption{Esiti del calcolo delle percentuali di requisiti funzionali opzionali realizzati da Norris durante la Fase CP}
\end{table}
Come è possibile notare dalla tabella la percentuale dei requisiti funzionali opzionali soddisfatti da \insglo{Norris} ha raggiunto un esito generalmente accettabile. 
\\ \\
Si riportano di seguito le percentuali di requisiti funzionali opzionali realizzati dalle componenti di \insglo{Chuck}.
\begin{table}[H]
	\centering
		\begin{tabu}{| l | c | c |}
			\hline
			Componente	& Numero requisiti funzionali opzionali soddisfatti	& Esito		\\ \hline \hline
			Directive	& 82\% 	& Accettabile  \\ \hline
			ChartView  & 	92\%	& Ottimale  \\ \hline
			ViewModel  & 	87\%	& Ottimale  \\ \hline
			DataModel  & 	96\%	& Ottimale  \\ \hline
		\end{tabu}
	\caption{Esiti del calcolo delle percentuali di requisiti funzionali opzionali realizzati da Chuck durante la Fase CP}
\end{table}
Come è possibile notare dalla tabella la percentuale dei requisiti funzionali opzionali soddisfatti da \insglo{Chuck} ha raggiunto un esito generalmente ottimale.
\\ \\
Si riportano di seguito le percentuali di requisiti funzionali realizzati dalle componenti dell'Applicazione.
\begin{table}[H]
	\centering
		\begin{tabu}{| l | c | c |}
			\hline
			Componente	& Numero requisiti funzionali soddisfatti	& Esito		\\ \hline \hline
			Model				&   80\% 	& Accettabile  \\ \hline
			Model:NorrisChart	&   74\% 	& Non accettabile  \\ \hline
			Model:Service 		& 	78\%	& Non accettabile   \\ \hline
			Presenter  			& 	76\%	& Non accettabile  \\ \hline
			View  				& 	95\%	& Ottimale  \\ \hline
		\end{tabu}
	\caption{Esiti del calcolo delle percentuali di requisiti funzionali realizzati dell'Applicazione durante la Fase CP}
\end{table}
Come è possibile notare dalla tabella la percentuale dei requisiti funzionali soddisfatti dell'Applicazione non ha raggiunto un esito generalmente accettabile.

\level{4}{Numero di parametri per metodo}

Si riportano di seguito, per ogni componente, le percentuali dei suoi metodi con un dato numero di parametri.
\begin{table}[H]
	\centering
		\begin{tabu}{| l | c | c | c | c | c | }
			\hline
			Componente	& 0 & 1 & 2 & 3 & Esito		\\ \hline \hline
			Norris::InternalAPIManager	& 44\% & 48\% & 8\% & 0\% & Ottimale  \\ \hline
			Norris::ExternalAPIManager  & 	47\% & 36\% & 14\% & 3\%	& Ottimale  \\ \hline
			Norris::DataModel  & 	53\%	&  37,5\% & 9,5\% & 0\% & Ottimale  \\ \hline
			Chuck::Model & 52\% & 20\% & 26\% & 2\% & Ottimale \\ \hline
			Chuck::ViewModel & 50\% & 50\% & 0\% & 0\% & Ottimale \\ \hline
			App::Model & 52\% & 25\% & 22\% & 0\% & Ottimale \\ \hline
			App::Presenter & 66\% & 18\% & 12\% & 4\% & Ottimale \\ \hline
			App::View & 9\% & 74\% & 12\% & 5\% & Ottimale \\ \hline
		\end{tabu}
	\caption{Esiti del calcolo delle percentuali del numero di parametri per metodo durante la Fase CP}
\end{table}
Come è possibile notare dalla tabella tutti i metodi hanno un numero di parametri accettabile. 
\\ \\

\level{4}{Numero di campi dati per classe}

Si riportano di seguito, per ogni componente, le percentuali delle sue classi con un dato numero di campi dati.
\begin{table}[H]
	\centering
		\begin{tabu}{| l | c | c | c | c | c | c | c | c | c | c | }
			\hline
			Componente	& 0 & 1 & 2 & 3 & 4 & 6 & 8 & 9 & 11 & Esito		\\ \hline \hline
			Norris::InternalAPIManager	& 0\% & 100\% & 0\% & 0\% & 0\% & 0\% & 0\% & 0\% & 0\% & Ottimale  \\ \hline
			Norris::ExternalAPIManager  & 0\% & 33\% & 47\% & 20\% & 0\% & 0\% & 0\% & 0\% & 0\% & Ottimale  \\ \hline
			Norris::DataModel  & 27\% & 53\% & 0\% & 13\% & 0\% & 7\% & 0\% & 0\% & 0\% & Ottimale  \\ \hline
			Chuck::Model & 23,5\% & 70,5\% & 0\% & 0\% & 0\% & 6\% & 0\% & 0\% & 0\% & Ottimale  \\ \hline
			Chuck::ViewModel & 0\% & 100\% & 0\% & 0\% & 0\% & 0\% & 0\% & 0\% & 0\% & Ottimale  \\ \hline
			Chuck::Directive & 0\% & 0\% & 50\% & 50\% & 0\% & 0\% & 0\% & 0\% & 0\% & Ottimale  \\ \hline
			App::Model & 22\% & 61,5\% & 5,5\% & 0\% & 5,5\% & 5,5\% & 0\% & 0\% & 0\% & Ottimale  \\ \hline
			App::Presenter & 37,5\% & 56,5\% & 0\% & 0\% & 0\% & 0\% & 6\% & 0\% & 0\% & Ottimale  \\ \hline
			App::View & 37,5\% & 25\% & 0\% & 0\% & 0\% & 0\% & 0\% & 25\% & 12,5\% & Accettabile  \\ \hline
		\end{tabu}
	\caption{Esiti del calcolo delle percentuali del numero di campi dati per classi durante la Fase CP}
\end{table}
Come è possibile notare dalla tabella la maggior parte delle tabelle possiede un numero ottimale di campi dati, ad eccezzione solamente di alcune classi della componente View dell'applicazione, le quali però hanno un numero comunque accettabile di campi dati.
\\ \\

\level{4}{Grado di accoppiamento dei componenti}

Si riportano di seguito, per ogni componente, le percentuali delle sue classi con un dato numero di campi dati.
\begin{table}[H]
	\centering
		\begin{tabu}{| l | c | c | c | c | c | c | }
			\hline
			Componente	& Afferente & Esiti Afferente & Efferente & Esiti Efferente 	\\ \hline \hline
			Norris::InternalAPIManager	& 1 & Ottimale & 3 & Ottimale  \\ \hline
			Norris::ExternalAPIManager  & 0 & Ottimale & 9 & Non Accettabile   \\ \hline
			Norris::DataModel  & 4 & Ottimale & 3 & Ottimale   \\ \hline
			Chuck::Model & 2 & Ottimale & 4 & Ottimale   \\ \hline
			Chuck::ViewModel & 8 & Non Accettabile & 6 & Accettabile   \\ \hline
			Chuck::Directive & 0 & Ottimale & 12 & Non Accettabile   \\ \hline
			Chuck::View & 4 & Ottimale & 4 & Ottimale   \\ \hline
			App::Model & 4 & Ottimale & 2 & Ottimale   \\ \hline
			App::Presenter & 3 & Ottimale & 6 & Accettabile   \\ \hline
			App::View & 1 & Ottimale & 6 & Accettabile   \\ \hline
		\end{tabu}
	\caption{Esiti del calcolo del grado di accoppiamento per le componenti durante la Fase CP}
\end{table}
Come è possibile notare dalla tabella, abbiamo ottenuto buoni risultati sia per Norris sia per l'applicazione Android, mentre i valori del grado di accoppiamento afferente della componente ViewModel di Chuck sono risultati molto alti, in quanto rappresenta la componente principale di Chuck.
\\ \\
 
\level{2}{Verifica dei processi}

\level{3}{Processo di documentazione}
\level{4}{Livello CMM}
Il livello del processo di documentazione si assesta al terzo gradino della scala \insglo{CMM}.
\level{4}{Schedule Variance}
Le attività del progetto in questa \insglo{fase} sono state sconvolte dall'uscita dei risultati della Revisione di Progetto che hanno richiesto la correzione dei documenti precedentemente redatti. Ciò ha comportato un ritardo generalizzato nelle attività pianificate.

Riportiamo di seguito i valori ottenuti calcolando la Schedule Variance sui tempi di stesura di ogni documento:
			\begin{table}[H]
					\centering
					\begin{tabu}{| l | c | c |}
							\hline
							Documenti 							& Schedule Variance	& Esito		\\ \hline \hline
							
							Piano di progetto v6.00				& 0\% 		& Ottimale  \\ \hline
							Norme di Progetto v6.00 			& -1\%		& Accettabile  \\ \hline
							Piano di Qualifica v6.00 			& -2\%		& Accettabile  \\ \hline
							Specifica Tecnica v6.00 			& 1\%		& Ottimali  \\ \hline
							Manuale Utente v3.00 			& 1\%		& Ottimale  \\ \hline
							Definizione di \insglo{Prodotto} v3.00 			& -1\%		& Accettabile  \\ \hline
							Glossario v6.00					 	& 0\% 		& Ottimale  \\ \hline
							Totale processo di documentazione & -2\% & Accettabile \\ \hline
						\end{tabu}
					\caption{Esiti del calcolo della Schedule Variance durante la Fase CP}
				\end{table}
				
\level{4}{Budget Variance}
Le risorse impiegate per la correzione degli errori e per l'avanzamento delle attività pianificate sono risultate inferiori rispetto a quelle pianificate, pertanto è stato necessario cambiare nel \insdoc{Piano di Progetto v6.00} la distribuzione delle ore pianificate, aumentando le ore di \insrole{progettisti} e \insrole{programmatori} e riducendo quelle di \insrole{Project Manager} e \insrole{amministratore}.
Riportiamo di seguito i valori ottenuti calcolando la Budget Variance sulle risorse utilizzate per la stesura di ogni documento:
			\begin{table}[H]
					\centering
					\begin{tabu}{| l | c | c |}
							\hline
							Documenti 							& Budget Variance	& Esito		\\ \hline \hline
							
							Piano di progetto v6.00				& 0\% 		& Ottimale  \\ \hline
							Norme di Progetto v6.00 			& -1\%		& Accettabile  \\ \hline
							Piano di Qualifica v6.00 			& 1\%		& Ottimale  \\ \hline
							Specifica Tecnica v6.00 			& 0\%		& Ottimale  \\ \hline
							Manuale Utente v3.00 			& 0\%		& Ottimale  \\ \hline
							Definizione di \insglo{Prodotto} v3.00 			& -2\%		& Accettabile  \\ \hline
							Glossario v6.00					 	& 3\% 		& Ottimale  \\ \hline
							Totale processo di documentazione & 1\% & Ottimale \\ \hline
						\end{tabu}
					\caption{Esiti del calcolo della Budget Variance durante la Fase CP}
				\end{table}
				
\level{4}{Produttività}
Utilizzando la formula descritta all'interno del presente documento (sezione \nameref{sec:metriche}) è stata calcolata la produttività del processo di documentazione. Questo indice è stato calcolato durante tutti i momenti di verifica previsti dal \insdoc{Piano di Progetto v6.00} per la \insphase{Fase CP}, e ciò è stato fatto per ogni documento che è stato redatto nel periodo antecedente la verifica. Ogni documento è stato sottoposto al processo di verifica al più cinque volte. Il calcolo fatto di volta in volta sullo stesso documento tiene conto solo delle nuove sezioni introdotte in esso.\\
Segue un riassunto di quanto è stato fatto.
\begin{table}[H]
      \centering
		\begin{tabu}{| l | c | c | c | c | c |}
		\hline
		&1	&2	&3	&4	&5	\\ \hline
		Norme di Progetto	& 24 &	&	&	& \\ \hline
		Piano di Progetto	& 170 &	&	&	& \\ \hline
		Piano di Qualifica	& 159	&	&	&	&\\ \hline
		Specifica Tecnica & 135 & & & & \\ \hline
		Definizione di \insglo{Prodotto} & 120 &	 	&	&  	&\\ \hline
		Manuale Utente & 118	&	&	&	& \\ \hline
		Glossario & 120 &  & & &\\ \hline
		\end{tabu}
		\caption{Produttività delle varie attività del processo di documentazione durante la fase CP}
\end{table}
Di seguito vengono riportati in un grafico i valori della produttività del processo di documentazione rilevati nei vari periodi della \insphase{Fase CP} nei quali è stato applicato il processo di verifica. Il grafico fa riferimento alla tabella precedente.\\

\begin{figure}[H]
	\centering
		\includegraphics[width=12cm]{PianoDiQualifica/Pics/ProduttivitaDocumentazioneFaseCP.jpg}
	\caption{Produttività del processo di documentazione durante la Fase CP}
\end{figure}
La bassa produttività del processo di documentazione relativamente alle \insdoc{Norme di Progetto} è dovuta al fatto che esse sono state modificate prevalentemente con l'aggiunta di grafici, al fine di renderne i contenuti meno testuali e quindi più fruibili. Nuovi diagrammi sono stati aggiunti anche alla \insdoc{Definizione di Prodotto}, causando un abbassamento del livello di produttività. Questo documento, invece, è stato modificato con l'aggiunta di una sezione, la cui stesura ha richiesto diverso tempo di riflessione da parte dei redattori.

\level{3}{Processo di verifica}
\level{4}{Livello CMM}
	Il livello del processo di verifica si assesta al terzo gradino della scala \insglo{CMM}. Il \groupname{} infatti ha mantenuto i livelli di documentazione e di qualità raggiunti già nella \insphase{Fase SD}.
\level{4}{Schedule Variance}
	Il processo di verifica è stato sempre svolto rispettando le scadenze temporali previste nel \insdoc{Piano di Progetto v6.00}. I valori della Schedule Variance calcolati per questo processo risultano quindi ottimi.\\
			Riportiamo di seguito il valore ottenuti:
			\begin{table}[H]
				\centering
				\begin{tabu}{| l | c | c |}
					\hline
						Processi 							& Schedule Variance	& Esito		\\ \hline \hline
						Processo di verifica & 0\% & Ottimale \\ \hline
				\end{tabu}
				\caption{Esiti del calcolo della Schedule Variance durante la Fase CP}
			\end{table}	

\level{4}{Budget Variance}
Le risorse utilizzate nel processo di verifica sono sovrabbondanti rispetto alle stime, a causa dell necessità di correggere gli errori rilevati nella Revisione di Qualifica. Ciò ha causato un valore di poco inferiore alla soglia di accettabilità della Budget Variance.
Riportiamo di seguito il valore ottenuto:
\begin{table}[H]
	\centering
	\begin{tabu}{| l | c | c |}
	\hline
	Processi 							& Budget Variance	& Esito		\\ \hline \hline
	Processo di verifica & -2\% & Non Accettabile \\ \hline
	\end{tabu}
	\caption{Esiti del calcolo della Budget Variance durante la Fase CP}
\end{table}	

\level{4}{Produttività}
Utilizzando la formula descritta all'interno del presente documento (sezione \nameref{sec:metriche}) è stata calcolata la produttività del processo di verifica. Tale indice è stato calcolato in seguito a tutti i momenti di verifica previsti dal \insdoc{Piano di Progetto v6.00} per la \insphase{Fase CP}. Di seguito vengono riportati i valori calcolati e una loro rappresentazione grafica.
\begin{table}[H]
	\centering
	\begin{tabu}{| c | c |}
		\hline
		Data verifica & Produttività\\ \hline \hline
		07-08/05 & 26 \\ \hline
		13-14/05 & 12 \\ \hline
		22-24/05 & 6 \\ \hline
		26/05 & 4 \\ \hline				
	\end{tabu}
	\caption{Produttività del processo di verifica durante la fase CP}
\end{table}


\begin{figure}[H]
	\centering
	\includegraphics[width=12cm]{PianoDiQualifica/Pics/ProduttivitaVerificaFaseCP.png}
	\caption{Produttività del processo di verifica durante la Fase CP}
\end{figure}

La produttività ha avuto un picco iniziale dovuto alla necessità di rivedere i documenti presentati alla Revisione di Progettazione, per poi scemare verso livelli più bassi una volta eseguiti gli assestamenti dei documenti.

\level{4}{Efficacia di una revisione}
Utilizzando la formula descritta all'interno del presente documento (sezione \nameref{sec:metriche}) è stata calcolata l'efficacia delle varie revisioni che sono state fatte durante la \insphase{Fase CP}. Di seguito vengono riportati i valori calcolati e una loro rappresentazione grafica.
\begin{table}[H]
	\centering
	\begin{tabu}{| c | c |}
	\hline
	Data verifica &Efficacia\\ \hline \hline
	07-08/05 & 13 \\ \hline
	13-14/05 & 10 \\ \hline
	22-24/05 & 11\\ \hline
	26/05 & 7 \\ \hline				
	\end{tabu}
	\caption{Efficacia delle revisioni durante la fase CP}
\end{table}\begin{figure}[H]
	\centering
	\includegraphics[width=12cm]{PianoDiQualifica/Pics/EfficaciaRevisioniFaseCP.pdf}
	\caption{Efficacia delle revisioni durante la Fase CP}
\end{figure}

I livelli di efficacia di revisione sono più bassi rispetto a quelli che ci si potrebbe aspettare vedendo i livelli di produttività, perchè la maggior parte degli errori è stata rilevata in sede di Revisione di Progettazione.

\level{3}{Processo di sviluppo}
		Si riportano in questa sezione gli esiti delle misurazioni effettuate rispetto all'attività di codifica.
		\level{4}{Livello CMM}
		Il \groupname{} conferma il livello \insglo{CMM} anche per la parte del processo di sviluppo relativa alla codifica. Il breve periodo intercorso tra la fine della \insglo{fase} pedente e la fine di questa \insglo{fase} non ha permesso di ricercare i miglioramenti necessari a far aumentare la qualità del processo; il gruppo di lavoro si ritiene tuttavia soddisfatto del livello raggiunto nel processo, complessivo anche dell'attività di progettazione.
		
		\level{4}{Schedule Variance}
		In questa \insglo{fase} l'attività di codifica ha subito un rallentamento dato dalla necessità di rivedere quanto già implementato, in considerazione delle correzioni fatte durante l'ultima revisione col committente. Tali rallentamenti sono risultati comunque accettabili rispetto alle metriche definite.\\
		Riportiamo di seguito i valori ottenuti calcolando la Schedule Variance sui tempi di stesura del codice sorgente.
					\begin{table}[H]
						\centering
						\begin{tabu}{| l | c | c |}
							\hline
								Processi 							& Schedule Variance	& Esito		\\ \hline \hline
								Processo di sviluppo (codifica) & -4\% & Accettabile \\ \hline
						\end{tabu}
						\caption{Esiti del calcolo della Schedule Variance durante la Fase CP}
					\end{table}	
					
		\level{4}{Budget Variance}
		Conseguentemente ai ritardi motivati nella sezione precedente, le risorse impiegate sono state maggiori del previsto, al punto da richiedere la redistribuzione del monte ore totale fra i ruoli, come indicato nel \insdoc{Piano di Progetto v6.00}, e da sforare i range di accettazione della metrica.\\
					\begin{table}[H]
						\centering
						\begin{tabu}{| l | c | c |}
							\hline
								Processi 							& Budget Variance	& Esito		\\ \hline \hline
								Processo di sviluppo (codifica) & -7\% & Non accettabile \\ \hline
						\end{tabu}
						\caption{Esiti del calcolo della Budgett Variance dell'attività di codifica durante la Fase CP}
					\end{table}	
					
		\level{4}{Produttività}
		Utilizzando la formula descritta all'interno del presente documento (sezione \nameref{sec:metriche}) è stata calcolata la produttività del processo di sviluppo, limitatamente all'attività di codifica. Questo indice è stato calcolato durante i due momenti di verifica previsti dal \insdoc{Piano di Progetto v6.00} per la \insphase{Fase CP}.\\
		Seguono i risultati delle misurazioni.
		\\ \begin{table}[H]
						\centering
						\begin{tabu}{| l | c | c |}
							\hline
								Processi 							& 1	& 2		\\ \hline \hline
								Processo di sviluppo (codifica) & 32 & 35  \\ \hline
						\end{tabu}
						\caption{Esiti del calcolo della produttività della codifica durante la Fase CP}
					\end{table}	
			
\level{3}{PDCA}
In questa sezione viene riportato il grafico \insglo{PDCA} della \insphase{Fase CP}. In ascissa è rappresentato il tempo, in ordinata le attività.

\begin{figure}[H]
	\centering
	\includegraphics[width=0.6\textwidth]{PianoDiQualifica/Pics/GraficoPDCAFaseCP.png}
	\caption{PDCA Fase CP}
\end{figure}
Si può facilmente notare come la pianificazione abbia avuto un brusco cambiamento mella prima metà del grafico. Si nota infatti un veloce aumento dei valori Plan dovuto alla consegna della valutazione della \insrev{Revisione di Progettazione} che ha causato una riprogettazione importante dell'apllicazione \insglo{Android}. Tale evento ha causato un rallentamento generale e ,bensì verso la fine i valori Act siano riusciti a decollare, non si è riusciti a concludere tutto il pianificato.
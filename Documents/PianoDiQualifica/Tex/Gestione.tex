\section{Gestione amministrativa della revisione}
	\subsection{Gestione anomalie}
		Per anomalia si intende una deviazione del prodotto dalle aspettative prefissate. //
		Il gruppo ha deciso di gestire le anomalie attraverso il meccanismo di sollevamento di issue offerto dalla piattaforma GitHub.//
		E' compito del verificatore sollevare una issue per ogni anomalia rilevata durante l'attività di controllo eseguita sul materiale presente nel repository, sia esso costituito da documenti o codice sorgente.
		Le issue sollevate dovranno contenere:
		\begin{itemize}
			\item l'oggetto della issue indicato sul campo "Title" dell'interfaccia fornita da GitHub;
			\item l'indicazione della priorità del problema della issue. Tale indicazione andrà scritta in maiuscolo vicino all'oggetto della issue e potrà avere i seguenti valori:
				\begin{itemize}
					\item BASSA;
					\item MEDIA;
					\item ALTA.
				\end{itemize}
			\item una descrizione esaustiva del problema individuato e, opzionalmente, una indicazione su come risolverlo;
			\item l'indicazione del membro del gruppo \groupname{} individuato come responsabile del problema o a cui la issue è rivolta;
			\item l'indicazione della categoria del problema rilevato, attraverso le labels messe a disposizione da GitHub.
		\end{itemize}
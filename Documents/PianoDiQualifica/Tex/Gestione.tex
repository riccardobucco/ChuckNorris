% !TEX encoding = UTF-8 Unicode

\section{Gestione amministrativa della revisione}
	\subsection{Gestione anomalie}
		Il gruppo ha deciso di gestire le anomalie attraverso il meccanismo di sollevamento di issue offerto dalla piattaforma GitHub.\\
		Un'anomalia può per esempio corrispondere a:
		\begin{itemize}
			\item un errore concettuale all'interno della documentazione di progetto;
			\item un errore ortografico;
			\item una violazione delle norme tipografiche riportate all'interno del documento Norme di Progetto;
			\item un'uscita dai range di accettazione descritti nella sezione "Misure e metriche" del presente documento;
			\item un'incongruenza nel prodotto software rispetto alle funzionalità descritte all'interno del documento Analisi dei Requisiti;
			\item un'incongruenza del codice rispetto a quanto è stato progettato.
		\end{itemize}
		È compito del verificatore sollevare una issue per ogni anomalia rilevata durante l'attività di controllo eseguita sul materiale presente 
		nel repository, sia esso costituito da documenti o codice sorgente.\\
		La procedura per sollevare una issue è descritta all'interno del documento Norme di Progetto.
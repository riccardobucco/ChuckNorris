% !TEX encoding = UTF-8 Unicode

\level{1}{Gestione amministrativa della revisione}
	\level{2}{Gestione anomalie}
		Il gruppo ha deciso di gestire le anomalie attraverso il meccanismo di sollevamento di \insglo{issue} offerto dalla piattaforma \insglo{GitHub}.\\
		Un'anomalia può per esempio corrispondere a:
		\begin{itemize}
			\item un errore concettuale all'interno della documentazione di progetto;
			\item un errore ortografico;
			\item una violazione delle norme tipografiche riportate all'interno del documento \insdoc{Norme di Progetto};
			\item un'uscita dai range di accettazione descritti nella sezione “Misure e metriche” del presente documento;
			\item un'incongruenza nel \insglo{prodotto} \insglo{software} rispetto alle funzionalità descritte all'interno del documento \insdoc{Analisi dei Requisiti};
			\item un'incongruenza del codice rispetto a quanto è stato progettato.
		\end{itemize}
		È compito del verificatore sollevare una \insglo{issue} per ogni anomalia rilevata durante l'attività di controllo eseguita sul materiale presente 
		nel \insglo{repository}, sia esso costituito da documenti o codice sorgente.\\
		La \insglo{procedura} per sollevare una \insglo{issue} è descritta all'interno del documento \insdoc{Norme di Progetto v6.00}.

\level{2}{Test di sistema}
Vengono di seguito riportati i test di sistema per la verifica del comportamento dinamico del sistema completo rispetto ai requisiti riportati nel documento \insdoc{Analisi dei Requisiti v3.00}.
\level{3}{TS1.1 Il framework deve permettere la creazione di bar chart}
Viene verificato che il framework permetta la creazione di bar chart.
\level{3}{TS1.2	Il framework deve permettere la creazione di line chart}
Viene verificato che il framework permetta la creazione di line chart.
\level{3}{TS1.3	Il framework deve permettere la creazione di map chart}
Viene verificato che il framework permetta la creazione di map chart.
\level{3}{TS1.4	Il framework deve permettere la creazione di table}
Viene verificato che il framework permetta la creazione di table.
\level{3}{TS1.5.1.1 Deve essere possibile scegliere se la legenda è visualizzata o nascosta}
iene verificato che il line chart permetta la scelta se visualizzare o nascondere la legenda.
\level{3}{TS1.5.1.2 Deve essere possibile scegliere la posizione in cui è visualizzata la legenda}
Viene verificato che il line chart permetta la scelta della posizione in cui è visualizzata la legenda.
\level{3}{TS1.5.2.1 Deve essere possibile scegliere se la legenda è visualizzata o nascosta}
Viene verificato che il bar chart permetta la scelta se visualizzare o nascondere la legenda.
\level{3}{TS1.5.2.2	Deve essere possibile scegliere la posizione in cui è visualizzata la legenda}
Viene verificato che il bar chart permetta la scelta della posizione in cui è visualizzata la legenda.
\level{3}{TS1.5.3.1 Deve essere possibile scegliere se la legenda è visualizzata o nascosta}
Viene verificato che il map chart permetta la scelta se visualizzare o nascondere la legenda.
\level{3}{TS1.5.3.2 Deve essere possibile scegliere la posizione in cui è visualizzata la legenda}
Viene verificato che il bar chart permetta la scelta della posizione in cui è visualizzata la legenda.
\level{3}{TS1.5.4 Ogni grafico deve permettere l'inserimento di una descrizione testuale}
Viene verificato che i grafici permettano l'inserimento di una descrizione testuale.
\level{3}{TS1.5.5.1	Deve essere possibile inserire il nome dei due assi cartesiani in un line chart}
Viene verificate che il line chart permetta la scelta del nome degli assi cartesiani.
\level{3}{TS1.5.5.2 Deve essere possibile scegliere se le linee della griglia sono visualizzate o nascoste in un line chart}
Viene verificato che il line chart consenta la scelta di visualizzare o nascondere le linee della griglia.
\level{3}{1.5.6.1 Deve essere possibile inserire il nome dei due assi cartesiani in un bar chart}
Viene verificate che il bar chart permetta la scelta del nome degli assi cartesiani.
\level{3}{1.5.6.2	Deve essere possibile scegliere se le linee della griglia sono visualizzate o nascoste in un bar chart}
Viene verificato che il bar chart consenta la scelta di visualizzare o nascondere le linee della griglia.
\level{3}{TS1.5.7.1 La table deve permettere la scelta del colore del testo di ogni cella}
Viene verificato che la table consenta la scelta del colore del testo di ogni cella.
\level{3}{TS1.5.7.2 La table deve consentire la scelta del colore di sfondo di ogni cella}
Viene verificato che la table consenta la scelta del colore di sfondo di ogni cella.
\level{3}{TS1.5.7.3 Il bar chart deve consentire di impostare la dimensione dello spazio tra due serie}
Viene verificato che il bar chart consenta la scelta della dimensione dello spazio tra due serie.
\level{3}{TS1.5.7.4 Il bar chart deve consentire di impostare la dimensione dello spazio tra due valori}
Viene verificato che il bar chart consenta la scelta della dimensione dello spazio tra due valori.
\level{3}{TS1.5.7.5 Deve esser possibile scegliere la forma dei marcatori del map chart}
Viene verificato che il map chart consenta la scelta della forma dei marcatori.
\level{3}{TS1.5.7.6 Deve essere possibile impostare la dimensione dei punti nel line chart}
Viene verificato che il line chart consenta la scelta della dimensione dei punti.
\level{3}{TS1.5.7.7	Deve esser possibile scegliere se la linea di un line chart è curva o segmentata}
Viene verificato che il line chart consenta la scelta della forma della linea tra curva e segmentata.
\level{3}{TS1.5.8 Il bar chart deve permettere la scelta dell'orientamento delle barre tra verticale e orizzontale}
Viene verificato che il bar chart consenta la scelta dell'orientamento delle barre tra verticale e orizzontale.
\level{3}{TS1.5.9 Il map chart deve permette la scelta delle dimensioni dell'area visualizzata}
Viene verificato che il map chart consenta la scelta delle dimensioni dell'area visualizzata
\level{3}{TS1.5.10	Il map chart deve permettere la scelta del punto centrale della mappa}
Viene verificato che il map chart consenta la scelta del punto centrale della mappa.
\level{3}{TS1.5.11	Line chart, bar chart e map chart devono permettere la scelta del massimo numero di dati visualizzati per ogni serie}
Viene verificato che il line chart consenta la scelta del massimo numero di dati visualizzati per ogni serie.
\level{3}{TS1.5.12	La table deve permettere la scelta del massimo numero di righe visualizzate}
Viene verificato che la table consenta la scelta del massimo numero di righe visualizzate.
\level{3}{TS1.5.13	La table deve permettere la scelta di rendere disponibile l'ordinamento per colonna}
Viene verificato che la table consenta la scelta di rendere dispobile l'ordinamento per colonna o meno.
\level{3}{TS1.5.14	La table deve consentire la scelta della posizione in cui vengono aggiunte nuove righe}
Viene verificato che la table consenta la scelta della posizione in cui vengono aggiunte nuove righe.
\level{3}{TS1.5.15 La table deve consentire la scelta dell'intestazione di ciascuna colonna}
Viene verificato che la table consenta la scelta dell'intestazione di ciascuna colonna.
\level{3}{TS1.5.16	La table deve permettere la scelta se le linee della tabella sono visualizzate o nascoste}
Viene verificato che la table consenta la scelta se le linee della tabella sono visualizzate o nascoste.
\level{3}{TS1.5.17	Ogni grafico deve permettere l'impostazione del titolo}
Viene verificato che ogni grafico consenta l'inserimento del titolo.
\level{3}{TS1.5.18	I grafici line chart devono permettere la scelta del colore per ciascun set di dati}
Viene verificato che line chart permetta la scelta del colore per ciascun set di dati.
\level{3}{TS1.5.19	 I grafici bar chart devono permettere la scelta del colore per ciascun set di dati}
Viene verificato che bar chart permetta la scelta del colore per ciascun set di dati.
\level{3}{TS1.5.20 I grafici map chart devono permettere la scelta del colore per ciascun set di dati}
Viene verificato che map chart permetta la scelta del colore per ciascun set di dati.
\level{3}{TS1.5.21	I grafici line chart devono permettere la scelta del nome per ciascun set di dati}
Viene verificato che line chart permetta la scelta del nome per ciascun set di dati.
\level{3}{TS1.5.22	I grafici bar chart devono permettere la scelta del nome per ciascun set di dati}
Viene verificato che bar chart permetta la scelta del nome per ciascun set di dati.
\level{3}{TS1.5.23	I grafici map chart devono permettere la scelta del nome per ciascun set di dati}
Viene verificato che map chart permetta la scelta del nome per ciascun set di dati.
\level{3}{TS1.6	Per ogni grafico deve essere permesso l'inserimento dei dati}
Viene verificato che per ogni tipo di grafico sia permesso l'inserimento dei dati.
\level{3}{TS1.7	Deve esser visualizzato un errore qualora i dati passati per l'aggiornamento di un grafico siano scorretti}
Viene verificato che il sistema lanci un errore quando i dati passati per l'aggiornamento di un grafico siano scorretti.
\level{3}{TS1.8 Deve esser visualizzato un errore qualora i dati passati per la creazione di un grafico siano scorretti}
Viene verificato che venga visualizzato un errore qualora i dati passati per la creazione di un grafico siano scorretti.
\level{3}{TS1.9	Deve esser visualizzato un errore qualora si cerchi di inserire un grafico oltre il limite consentito dalla pagina}
Viene verificato che venga visualizzato un errore qualora l'inserimento di un grafico in una pagina ecceda la sua dimensione massima.
\level{3}{TS1.10 Deve esser visualizzato un errore qualora si cerchi di aggiornare un grafico con un tipo di aggiornamento da lui non permesso}
Viene verificato che il sistema lanci un errore quando si cerca di aggiornare un grafico con un tipo di aggiornamento da lui non permesso.
\level{3}{TS2.1	Un grafico line chart deve consentire l'aggiornamento stream}
Viene verificato che un grafico line chart consenta l'aggiornamento in stream.
\level{3}{TS2.2	Un grafico table deve consentire l'aggiornamento stream}
Viene verificato che un grafico table consenta l'aggiornamento in stream.
\level{3}{TS2.3	Un grafico map chart deve consentire l'aggiornamento movie}
Viene verificato che un grafico map chart consenta l'aggiornamento in stream.
\level{3}{TS2.4	Un grafico bar chart deve consentire l'aggiornamento in place}
Viene verificato che un grafico bar chart consenta l'aggiornamento in place.
\level{3}{TS2.5	Un grafico line chart deve consentire l'aggiornamento in place}
Viene verificato che un grafico line chart consenta l'aggiornamento in place.
\level{3}{TS2.6	Un grafico map chart deve consentire l'aggiornamento in place}
Viene verificato che un grafico map chart consenta l'aggiornamento in place.
\level{3}{TS2.7	Un grafico table deve consentire l'aggiornamento in place}
Viene verificato che un grafico table consenta l'aggiornamento in place.
\level{3}{TS3 Il framework deve permettere la creazione di una pagina HTML contenente alcuni grafici tramite le API interne}
Viene verificato che il framework permetta la creazione di una pagina HTML contenente alcuni grafici tramite le API interne.
\level{3}{TS3.1	Ogni pagina deve permettere l'inserimento del titolo}
Viene verificato che ogni pagina HTML, creata dal framework, permetta l'inserimento del titolo.
\level{3}{TS3.2.1 Deve essere possibile impostare il massimo numero di grafici visualizzabile per riga}
Viene verificato che ogni pagine consenta l'impostazione del numero massimo di grafici visualizzabile per riga.
\level{3}{TS3.2.2 Deve essere possibile impostare il massimo numero di grafici visualizzabile per colonna}
Viene verificato che ogni pagine consenta l'impostazione del numero massimo di grafici visualizzabile per colonna.
\level{3}{TS3.3	Ogni pagina deve consentire l'aggiunta di grafici}
Viene verificato che ogni pagina contenta l'aggiunta di grafici.
\level{3}{TS4	Il framework deve fornire un middleware per Express.js tramite le API interne}
Viene verificato che il framework fornisca un middleware per Express.js tramite le API interne.
\level{3}{TS5.1.1 La lista dei grafici deve fornire ID, titolo, tipo e descrizione di ciascun grafico}
Viene verificato che le API esterne forniscano la lista dei grafici con ID, titolo, tipo e descrizione di ciascun grafico.
\level{3}{TS5.2	Le API esterne devono fornire i grafici con relativi aggiornamenti}
Viene verificato che le API esterne forniscano i grafici con relativi aggiornamenti.
\level{3}{TS5.3	Deve essere possibile accedere a un'istanza di Norris tramite username e password utilizzando le API esterne}
Viene verificato che sia possibile accedere ad un'istanza di Norris tramite username e password utilizzando le API esterne.
\level{3}{TS5.4	Deve essere possibile scollegarsi da un'istanza di Norris utilizzando le API esterne}
Viene verificato che sia possibile scollegarsi da un'istanza di Norris utilizzando le API esterne
\level{3}{TS6.1	La libreria deve consentire l'inserimento di un grafico all'interno di un determinato tag HTML}
Viene verificato che la libreria che permetta l'inserimento di grafici all'interno di un determinato tag HTML.
\level{3}{TS6.2	La libreria deve consentire la modifica delle principali impostazioni di un grafico}
Viene verificato che il framework fornisca una libreria che consenta la modifica delle principali impostazioni di un grafico (visualizzazione e posizione della legenda, visualizzazione della griglia del piano cartesiano, modifica dei colori per ciascun set di dati, modifica del colore di sfondo di una cella o del testo in essa contenuto).
\level{3}{TS6.3	Deve essere possibile accedere a un'istanza di Norris tramite username e password utilizzando le API fornite da CHUCK}
Viene verificato che sia possibile accedere a un'istanza di Norris tramite username e password utilizzando le API fornite da CHUCK.
\level{3}{TS6.4	Deve essere possibile scollegarsi da un'istanza di Norris utilizzando le API fornite da CHUCK}
Viene verificato che sia possibile scollegarsi da un'istanza di Norris utilizzando le API fornite da CHUCK.
\level{3}{TS7.1	Deve essere possibile la visualizzazione del numero di autobus attivi per ciascuna linea}
Viene verificato che la dashboard permetta la visualizzazione del numero di autobus attivi per ciascuna linea.
\level{3}{TS7.2	Deve essere possibile filtrare gli autobus per linea di appartenenza e visualizzarli di conseguenza}
Viene verificato che la dashboard permetta il filtraggio degli autobus per linea di appartenenza e visualizzarli di conseguenza.
\level{3}{TS8	Deve essere fornita un'applicazione Android per la visualizzazione dei grafici}
Viene verificato che sia fornita un'applicazione Android per la visualizzazione dei grafici.
\level{3}{TS8.1	Deve essere possibile accedere a un'istanza di Norris tramite username e password}
Viene verificato che l'applicazione effettui la chiamata al login mediante username e password specificati dall'utente.
\level{3}{TS8.2	Deve essere possibile visualizzare l'elenco dei grafici esistenti nell'istanza Norris con relativo ID, titolo, tipo e descrizione}
Viene verificato che l'applicazione visualizzi l'elenco dei grafici esistenti nell'istanza Norris con relativo ID, titolo, tipo e descrizione.
\level{3}{TS8.3	Deve essere possibile selezionare e visualizzare un singolo grafico dell'istanza di Norris}
Viene verificato che sia possibile selezionare e visualizzare un singolo grafico.
\level{3}{TS8.4	Deve essere visualizzato un errore quando viene immesso un indirizzo per un'istanza di Norris non valido}
Viene verificato che sia visualizzato un errore quando viene immesso un indirizzo non valido per un'istanza di Norris.
\level{3}{TS8.5	Deve essere visualizzato un errore quando vengono immessi dati di accesso (username e password) per un'istanza di Norris non validi}
Viene verificato che sia visualizzato un errore quando vengono immessi dati di accesso (username e password) per un'istanza di Norris non validi.
\level{3}{TS9	Il framework deve permettere al suo utilizzatore di fornire delle funzioni tramite le quali si può gestire l'autenticazione}
Viene verificato che il framework permetta allo sviluppatore di fornire delle funzioni tramite le quali si può gestire l'autenticazione.

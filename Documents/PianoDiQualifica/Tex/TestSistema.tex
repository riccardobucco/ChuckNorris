\level{2}{Test di sistema}
Nella presente sezione vengono riportati i test di sistema, ovvero quei test che verificano il comportamento dinamico del sistema completo rispetto ai requisiti riportati nel documento \insdoc{Analisi dei Requisiti v3.00}. Per una descrizione completa della sintassi utilizzata nella descrizione di tali test si consulti il documento \insdoc{Norme di Progetto v3.00}.


\begin{longtabu}{| X[0.5] | X | X[0.5] | X[0.4] |}

			\hline
			\rowfont{\bf}
			Test &
			Descrizione &
			Requisito &
			Esito \\
			\hline \endhead


TS1.1 & Viene verificato che il framework permetta la creazione di bar chart. & RRF1.1 & N.I.\\ \hline

TS1.2 & Viene verificato che il framework permetta la creazione di line chart. & RRF1.2 & N.I.\\ \hline

TS1.3 & Viene verificato che il framework permetta la creazione di map chart. & RRF1.3 & N.I.\\ \hline

TS1.4 & Viene verificato che il framework permetta la creazione di table. & RRF1.4 & N.I.\\ \hline

TS1.5.1.1 & Viene verificato che il line chart permetta la scelta se visualizzare o nascondere la legenda. & RRF1.5.1.1 & N.I.\\ \hline

TS1.5.1.2 & Viene verificato che il line chart permetta la scelta della posizione in cui è visualizzata la legenda. & RRF1.5.1.2 & N.I.\\ \hline

TS1.5.2.1 & Viene verificato che il bar chart permetta la scelta se visualizzare o nascondere la legenda. & RRF1.5.2.1 & N.I.\\ \hline

TS1.5.2.2 & Viene verificato che il bar chart permetta la scelta della posizione in cui è visualizzata la legenda. & RRF1.5.2.2 & N.I.\\ \hline

TS1.5.3.1 & Viene verificato che il map chart permetta la scelta se visualizzare o nascondere la legenda. & RRF1.5.3.1 & N.I.\\ \hline

TS1.5.3.2 & Viene verificato che il bar chart permetta la scelta della posizione in cui è visualizzata la legenda. & RRF1.5.3.2 & N.I.\\ \hline

TS1.5.4 & Viene verificato che i grafici permettano l'inserimento di una descrizione testuale. & RDF1.5.4 & N.I.\\ \hline

TS1.5.5.1 & Viene verificate che il line chart permetta la scelta del nome degli assi cartesiani. & RRF1.5.5.1 & N.I.\\ \hline

TS1.5.5.2 & Viene verificato che il line chart consenta la scelta di visualizzare o nascondere le linee della griglia.& RRF1.5.5.2 & N.I.\\ \hline

TS1.5.6.1 & Viene verificate che il bar chart permetta la scelta del nome degli assi cartesiani. & RRF1.5.6.1 & N.I.\\ \hline

TS1.5.6.2 &	Viene verificato che il bar chart consenta la scelta di visualizzare o nascondere le linee della griglia. & RRF1.5.6.2 & N.I.\\ \hline

TS1.5.7.1 & Viene verificato che la table consenta la scelta del colore del testo di ogni cella. & RRF1.5.7.1 & N.I.\\ \hline

TS1.5.7.2 & Viene verificato che la table consenta la scelta del colore di sfondo di ogni cella. & RRF1.5.7.2 & N.I.\\ \hline

TS1.5.7.3 & Viene verificato che il bar chart consenta la scelta della dimensione dello spazio tra due serie. & RRF1.5.7.3 & N.I.\\ \hline

TS1.5.7.4 & Viene verificato che il bar chart consenta la scelta della dimensione dello spazio tra due valori. & RRF1.5.7.4 & N.I.\\ \hline

TS1.5.7.5 & Viene verificato che il map chart consenta la scelta della forma dei marcatori. & RDF1.5.7.5 & N.I.\\ \hline

TS1.5.7.6 & Viene verificato che il line chart consenta la scelta della dimensione dei punti.&
RDF1.5.7.6 & N.I.\\ \hline

TS1.5.7.7 &	Viene verificato che il line chart consenta la scelta della forma della linea tra curva e segmentata. & RRF1.5.7.7 & N.I.\\ \hline

TS1.5.8 & Viene verificato che il bar chart consenta la scelta dell'orientamento delle barre tra verticale e orizzontale. & RRF1.5.8 & N.I.\\ \hline

TS1.5.9 & Viene verificato che il map chart consenta la scelta delle dimensioni dell'area visualizzata & RRF1.5.9 & N.I.\\ \hline

TS1.5.10 & Viene verificato che il map chart consenta la scelta del punto centrale della mappa. & RRF1.5.10 & N.I.\\ \hline

TS1.5.11 &	Viene verificato che il line chart consenta la scelta del massimo numero di dati visualizzati per ogni serie. & RRF1.5.11 & N.I.\\ \hline

TS1.5.12 & Viene verificato che la table consenta la scelta del massimo numero di righe visualizzate. & RRF1.5.12 & N.I.\\ \hline

TS1.5.13 &	Viene verificato che la table consenta la scelta di rendere disponile l'ordinamento per colonna o meno. & RRF1.5.13 & N.I.\\ \hline

TS1.5.14 & Viene verificato che la table consenta la scelta della posizione in cui vengono aggiunte nuove righe. & RRF1.5.14 & N.I.\\ \hline

TS1.5.15 & Viene verificato che la table consenta la scelta dell'intestazione di ciascuna colonna. & RRF1.5.14 & N.I.\\ \hline
 
TS1.5.16 & Viene verificato che la table consenta la scelta se le linee della tabella sono visualizzate o nascoste. & RRF1.5.16 & N.I.\\ \hline

TS1.5.17 & Viene verificato che ogni grafico consenta l'inserimento del titolo. & RRF1.5.18 & N.I.\\ \hline

TS1.5.18 & Viene verificato che line chart permetta la scelta del colore per ciascun set di dati. & RRF1.5.18 & N.I.\\ \hline

TS1.5.19 & Viene verificato che bar chart permetta la scelta del colore per ciascun set di dati. & RRF1.5.19 & N.I.\\ \hline

TS1.5.20 & Viene verificato che map chart permetta la scelta del colore per ciascun set di dati. & RRF1.5.20 & N.I.\\ \hline

TS1.5.21 &	Viene verificato che line chart permetta la scelta del nome per ciascun set di dati. & RRF1.5.21 & N.I.\\ \hline

TS1.5.22 &	Viene verificato che bar chart permetta la scelta del nome per ciascun set di dati. & RRF1.5.22 & N.I.\\ \hline

TS1.5.23 &	Viene verificato che map chart permetta la scelta del nome per ciascun set di dati. & RRF1.5.23 & N.I.\\ \hline

TS1.6 & Viene verificato che per ogni tipo di grafico sia permesso l'inserimento dei dati. & RRF1.6 & N.I.\\ \hline

TS1.7 & Viene verificato che il sistema lanci un errore quando i dati passati per l'aggiornamento di un grafico siano scorretti. & RRF1.7 & N.I.\\ \hline

TS1.8 & Viene verificato che venga visualizzato un errore qualora i dati passati per la creazione di un grafico siano scorretti. & RRF1.8 & N.I.\\ \hline

TS1.9 & Viene verificato che venga visualizzato un errore qualora l'inserimento di un grafico in una pagina ecceda la sua dimensione massima. & RRF1.9 & N.I.\\ \hline

TS1.10 & Viene verificato che il sistema lanci un errore quando si cerca di aggiornare un grafico con un tipo di aggiornamento da lui non permesso. & RRF1.10 & N.I.\\ \hline

TS2.1 & Viene verificato che un grafico line chart consenta l'aggiornamento in stream. & RRF2.1 & N.I.\\ \hline

TS2.2 & Viene verificato che un grafico table consenta l'aggiornamento in stream. & RRF2.2 & N.I.\\ \hline

TS2.3 & Viene verificato che un grafico map chart consenta l'aggiornamento in stream. & RRF2.3 & N.I.\\ \hline

TS2.4 & Viene verificato che un grafico bar chart consenta l'aggiornamento in place. & RRF2.4 & N.I.\\ \hline

TS2.5 & Viene verificato che un grafico line chart consenta l'aggiornamento in place. & RRF2.5 & N.I.\\ \hline

TS2.6 & Viene verificato che un grafico map chart consenta l'aggiornamento in place. & RRF2.6 & N.I.\\ \hline

TS2.7 & Viene verificato che un grafico table consenta l'aggiornamento in place. & RRF2.7 & N.I.\\ \hline

TS3 & Viene verificato che il framework permetta la creazione di una pagina HTML contenente alcuni grafici tramite le API interne. & RRF3 & N.I.\\ \hline

TS3.1 & Viene verificato che ogni pagina HTML, creata dal framework, permetta l'inserimento del titolo. & RRF3.1 & N.I.\\ \hline

TS3.2.1 & Viene verificato che ogni pagine consenta l'impostazione del numero massimo di grafici visualizzabile per riga. & RRF3.2.1 & N.I.\\ \hline

TS3.2.2 & Viene verificato che ogni pagine consenta l'impostazione del numero massimo di grafici visualizzabile per colonna. & RRF3.2.2 & N.I.\\ \hline

TS3.3 & Viene verificato che ogni pagina consenta l'aggiunta di grafici. & RRF3.3 & N.I.\\ \hline

TS4 & Viene verificato che il framework fornisca un middleware per Express.js tramite le API interne. & RRF4 & N.I.\\ \hline

TS5.1.1 & Viene verificato che le API esterne forniscano la lista dei grafici con ID, titolo, tipo e descrizione di ciascun grafico. & RRF5.1.1 & N.I.\\ \hline

TS5.2 & Viene verificato che le API esterne forniscano i grafici con relativi aggiornamenti. & RRF5.2 & N.I.\\ \hline

TS5.3 & Viene verificato che sia possibile accedere ad un'istanza di Norris tramite username e password utilizzando le API esterne. & RRF5.3 & N.I.\\ \hline

TS5.4 & Viene verificato che sia possibile scollegarsi da un'istanza di Norris utilizzando le API esterne & RRF5.4 & N.I.\\ \hline

TS6.1 & Viene verificato che la libreria che permetta l'inserimento di grafici all'interno di un determinato tag HTML. & RRF6.1 & N.I.\\ \hline

TS6.2 & Viene verificato che il framework fornisca una libreria che consenta la modifica delle principali impostazioni di un grafico (visualizzazione e posizione della legenda, visualizzazione della griglia del piano cartesiano, modifica dei colori per ciascun set di dati, modifica del colore di sfondo di una cella o del testo in essa contenuto). & RDF6.2 & N.I.\\ \hline

TS6.3 & Viene verificato che sia possibile accedere a un'istanza di Norris tramite username e password utilizzando le API fornite da Chuck. & RRF6.3 & N.I.\\ \hline

TS6.4 & Viene verificato che sia possibile scollegarsi da un'istanza di Norris utilizzando le API fornite da Chuck. & RRF6.4 & N.I.\\ \hline

TS7.1 & Viene verificato che la dashboard permetta la visualizzazione del numero di autobus attivi per ciascuna linea. & RFF7.1 & N.I.\\ \hline

TS7.2 & Viene verificato che la dashboard permetta il filtraggio degli autobus per linea di appartenenza e visualizzarli di conseguenza. & RFF7.2 & N.I.\\ \hline

TS8 & Viene verificato che sia fornita un'applicazione Android per la visualizzazione dei grafici. & ROF8 & N.I.\\ \hline

TS8.1 & Viene verificato che l'applicazione effettui la chiamata al login mediante username e password specificati dall'utente. & RFF8.1 & N.I.\\ \hline

TS8.2 & Viene verificato che l'applicazione visualizzi l'elenco dei grafici esistenti nell'istanza Norris con relativo ID, titolo, tipo e descrizione. & RFF8.2 & N.I.\\ \hline

TS8.3 & Viene verificato che sia possibile selezionare e visualizzare un singolo grafico. & RRF8.3 & N.I.\\ \hline

TS8.4 & Viene verificato che sia visualizzato un errore quando viene immesso un indirizzo non valido per un'istanza di Norris. & RRF8.4 & N.I.\\ \hline

TS8.5 & Viene verificato che sia visualizzato un errore quando vengono immessi dati di accesso (username e password) per un'istanza di Norris non validi. & RRF8.5 & N.I.\\ \hline

TS9 & Viene verificato che il framework permetta allo sviluppatore di fornire delle funzioni tramite le quali si può gestire l'autenticazione. & RRF9 & N.I.\\ \hline

\caption{Test di sistema}

\end{longtabu}

% !TEX encoding = UTF-8 Unicode

\section{Introduzione}
	\subsection{Scopo del documento}
		Col presente documento si intende definire le operazioni di verifica che il gruppo \groupname utilizzerà per raggiungere gli obiettivi di qualità, processo e prodotto. Il team di sviluppo si impone una rigida e costante attività di verifica in modo da poter individuare e rettificare tempestivamente eventuali errori, al fine di raggiungere i suddetti obiettivi. 
	\subsection{Scopo del prodotto}
		Lo scopo del prodotto è produrre un framework per lo stack tecnologico formato da Node.js, Express.js e Socket.io in grado di generare grafici i cui dati sono letti da sorgenti arbitrarie e che metta a disposizione funzioni di aggiornamento dei grafici lato server tramite tecnologia WebSocket.
	\subsection{Glossario}
		Al fine di evitare ogni genere di ambiguità relativa al linguaggio o ai termini utilizzati nei documenti, si allega il Glossario dei termini.
	\subsection{Riferimenti}
		\subsubsection{Normativi}
			\begin{itemize}
				\item Capitolato D'Appalto C3: \projectname: Real-time Business Intelligence. Reperibile all'indirizzo: \uri{http://www.math.unipd.it/~tullio/IS-1/2014/Progetto/C3.pdf}
				\item Norme Di Progetto: "Norme di Progetto \lastversion".
			\end{itemize}
		\subsubsection{Informativi}
			\begin{itemize}
				\item Analisi dei Requisiti: “Analisi dei Requisiti \lastversion”;
				\item Piano di Progetto: “Piano di Progetto \lastversion”;
				\itemStudio di Fattibilità: “Studio di Fattibilità \lastversion”;
				\item Software Engineering - Ian Sommerville - 9th Edition (2010): Part 4 - Software Management;
				\item ISO 9001: \uri{http://en.wikipedia.org/wiki/ISO\_9001};
				\item ISO/IEC 9126:2001: \uri{http//en.wikipedia.org/wiki/ISO/IEC\_9126};
				\item ISO/IEC 15504: \uri{http//en.wikipedia.org/wiki/ISO/IEC\_15504};
			\end{itemize}
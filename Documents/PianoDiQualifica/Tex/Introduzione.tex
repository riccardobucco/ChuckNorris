% !TEX encoding = UTF-8 Unicode

\section{Introduzione}
	
	\subsection{Scopo del documento}
		Col presente documento si intende definire le operazioni di verifica che il gruppo \groupname{} utilizzerà per raggiungere gli obiettivi di qualità, processo e prodotto. Il team di sviluppo si impone una rigida e costante attività di verifica in modo da poter individuare e rettificare tempestivamente eventuali errori, al fine di raggiungere i suddetti obiettivi. 
	
	\level{2}{Glossario}
	Allo scopo di rendere più semplice la comprensione dei documenti ed evitare eventuali ambiguità, viene allegato il \insdoc{Glossario v6.00}, che contiene la spiegazione della terminologia tecnica e degli acronimi utilizzati. Per facilitare la lettura, i termini presenti all'interno di tale documento saranno marcati da una “G” maiuscola a pedice.


	\subsection{Riferimenti utili}
		

		\subsubsection{Riferimenti normativi}
			\begin{itemize}
				\item \textbf{Capitolato d'appalto C3}: \projectname{}: Real-time Business Intelligence. Reperibile all'indirizzo: \insuri{http://www.math.unipd.it/~tullio/IS-1/2014/Progetto/C3.pdf};
				\item \textbf{Norme di Progetto}: \insdoc{Norme di Progetto v1.0}.
				\item \textbf{ISO 9001:} \insuri{http://en.wikipedia.org/wiki/ISO\_9001};
				\item \textbf{ISO/IEC 9126:2001:} \insuri{http//en.wikipedia.org/wiki/ISO/IEC\_9126};
				\item \textbf{ISO/IEC 15504:} \insuri{http//en.wikipedia.org/wiki/ISO/IEC\_15504}.
			\end{itemize}
		

		\subsubsection{Riferimenti informativi}
			\begin{itemize}
				\item \textbf{Analisi dei Requisiti:} \insdoc{Analisi dei Requisiti v1.00};
				\item \textbf{Piano di Progetto:} \insdoc{Piano di Progetto v1.00};
				\item \textbf{Studio di Fattibilità:} \insdoc{Studio di Fattibilità v1.00};
				\item \textbf{Software Engeneering di Ian Sommerville - 9th Edition (2010):} Part 6, chapter 27 - Quality management;
			\end{itemize}

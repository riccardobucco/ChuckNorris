% !TEX encoding = UTF-8 Unicode

\section{PDCA} \label{app:pdca}
Il PDCA, acronimo di Plan-Do-Check-Act, conosciuto anche come ``Ciclo di Deming'' o ``Ciclo di miglioramento continuo'', è un modello per la risoluzione dei problemi grazie all'applicazione di un approccio scentifico.\\
In particolare, nell'ambito dell'ingegneria, viene utilizzato per il miglioramento dei processi attraverso la pianificazione, l'esecuzione e il controllo delle attività e l'esecuzione di azioni conseguenti all'esito del controllo. Questa procedura produce miglioramenti continui nei processi eseguendo le quattro fasi in modo ciclico.\\
Le quattro fasi del modello sono quelle indicate nel nome, cioè:
\begin{itemize}
	\item \textbf{Plan - Pianificazione:} macroattività suddivisibile in:
		\begin{enumerate}
		\item l'identificazione e la descrizione del problema che si vuole affrontare, attraverso l'analisi dei suoi aspetti fondamentali;
		\item definizione degli obiettivi che si vuole raggiungere, degli effetti economici, e dei tempi e delle risorse che si prevede di utilizzare;
		\item l'analisi del problema alla ricerca di possibili rischi e punti di forza;
		\item la ricerca di tutte le possibili, e più probabili, cause del problema che si vuole risolvere, e l'analisi di come e se sia possibile rimuoverle;
		\item la progettazione di azioni correttive e la definizione di criteri di valutazione dei risultati.
		\end{enumerate}
	\item \textbf{Do - Eseguire:} cioè eseguire le attività pianificate, formare il personale per svolgere le attività nel modo corretto, ed eseguire le attività di controllo;
	\item \textbf{Check - Verificare:} cioè analizzare gli esiti delle attività allo scopo di trovare possibili punti su cui eseguire miglioramenti;
	\item \textbf{Act - Agire:} cioè agire in senso correttivo, qualora nella fase di verifica siano emersi punti deboli, e al fine di consolidare i miglioramenti ottenuti nei processi.
\end{itemize}

\subsection{Utilizzo sistematico del PDCA}
Come detto, si può utilizzare il PDCA per ottenere un contiuo miglioramento nei processi, attivando questi tre cicli:
\begin{enumerate}
	\item{\bf Ciclo del mantenimento:} questo ciclo inizia prima delle fasi \emph{plan} e \emph{do} e ha lo scopo di verificare se ciò che è stato pianificato ed eseguito \textbf{continua} a dare i risultati attesi; se la fase \emph{check} non rileva alcun aspetto negativo, durante la fase \emph{act} si cercherà di mantenere lo stato attuale ottenuto;
	\item{\bf Ciclo dell'azione correttiva:} questo ciclo si avvia se l'esito della \emph{check} è negativo, in tal caso vengono avviate azioni correttive e preventive;
	\item{\bf Ciclo del miglioramento:} questo ciclo viene avviato quando si arriva ad avere un esito positivo della \emph{check} del ciclo di mantenimento, a quel punto significa che il processo si è stabilizzato su un buon livello e che si può puntare a migliorarlo sotto diversi aspetti come la qualità e l'efficienza.
\end{enumerate}
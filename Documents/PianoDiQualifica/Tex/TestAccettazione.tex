\level{2}{Test di accettazione}
	Di seguito vengono riportati i test di accettazione, ovvero quei test che avvengono al momento del collaudo. Per una descrizione completa della sintassi utilizzata nella descrizione di tali test si consulti il documento \insdoc{Norme di Progetto v7.00}.

	\begin{longtabu}{| X[0.5] | X | X | X[0.5] |}

	\hline
	\rowfont{\bf}
	Test &
	Requisito &
	Descrizione &
	Esito \\
	\hline \endhead

	TA1 & Lo sviluppatore deve verificare che sia possibile creare grafici tramite le \insglo{API} interne.
	 
		& Allo sviluppatore è richiesto di:
		\begin{itemize}
			\item scegliere il tipo di grafico che vuole creare;
			\item inserire i valori del grafico;
			\item inserire le impostazioni del grafico;
			\item verificare che venga creato il tipo di grafico desiderato.
		\end{itemize}
& Superato \\ \hline

	TA1.1 & Lo sviluppatore deve verificare che sia possibile creare un \insglo{bar chart}.

		& Allo sviluppatore è richiesto di:
		\begin{itemize}
			\item scegliere come tipo di grafico il \insglo{bar chart};
			\item inserire i valori del grafico.
		\end{itemize}
& Superato \\ \hline

	TA1.2 & Lo sviluppatore deve verificare che sia possibile creare un \insglo{line chart}.

		& Allo sviluppatore è richiesto di:
		\begin{itemize}
			\item scegliere come tipo di grafico il \insglo{line chart};
			\item inserire i valori del grafico.
		\end{itemize}
& Superato \\ \hline

	TA1.3 & Lo sviluppatore deve verificare che sia possibile creare un \insglo{map chart}.

		& Allo sviluppatore è richiesto di:
		\begin{itemize}
			\item scegliere come tipo di grafico il \insglo{map chart};
			\item inserire i valori del grafico.
		\end{itemize}
& Superato \\ \hline

	TA1.4 & Lo sviluppatore deve verificare che sia possibile creare una \insglo{table}.

		& Allo sviluppatore è richiesto di:
		\begin{itemize}
			\item scegliere come tipo di grafico la \insglo{table};
			\item inserire i valori del grafico.
		\end{itemize}
& Superato \\ \hline

	TA1.5/1 & Lo sviluppatore deve verificare che sia possibile scegliere delle impostazioni di un \insglo{bar chart}.

		& Allo sviluppatore è richiesto di:
		\begin{itemize}
			\item scegliere come tipo di grafico il \insglo{bar chart};
			\item inserire delle impostazioni.
		\end{itemize}
& Superato \\ \hline

	TA1.5/2 & Lo sviluppatore deve verificare che sia possibile scegliere delle impostazioni di un \insglo{line chart}.

		& Allo sviluppatore è richiesto di:
		\begin{itemize}
			\item scegliere come tipo di grafico il \insglo{line chart};
			\item inserire delle impostazioni.
		\end{itemize}
& Superato \\ \hline

	TA1.5/3 & Lo sviluppatore deve verificare che sia possibile scegliere delle impostazioni di un \insglo{map chart}.

		& Allo sviluppatore è richiesto di:
		\begin{itemize}
			\item scegliere come tipo di grafico il \insglo{map chart};
			\item inserire delle impostazioni.
		\end{itemize}
& Superato \\ \hline

	TA1.5/4 & Lo sviluppatore deve verificare che sia possibile scegliere delle impostazioni di una \insglo{table}.

		& Allo sviluppatore è richiesto di:
		\begin{itemize}
			\item scegliere come tipo di grafico una \insglo{table};
			\item inserire delle impostazioni.
		\end{itemize}
& Superato \\ \hline

	TA1.5.1 & Lo sviluppatore deve verificare che sia possibile scegliere le impostazioni riguardanti la legenda di un \insglo{line chart}.

		& Allo sviluppatore è richiesto di:
		\begin{itemize}
			\item scegliere come tipo di grafico il \insglo{line chart};
			\item inserire le impostazioni riguardanti la legenda.
		\end{itemize}
& Superato \\ \hline

	TA1.5.1.1 & Lo sviluppatore deve verificare che sia possibile scegliere se visualizzare o nascondere la legenda di un \insglo{line chart}.
			
		& Allo sviluppatore è richiesto di:
		\begin{itemize}
			\item scegliere come tipo di grafico il \insglo{line chart};
			\item inserire i valori del grafico;
			\item verificare che, selezionando l'impostazione per la visualizzazione della legenda \insglo{line chart}, la legenda venga visualizzata;
			\item verificare che, selezionando l'impostazione per non visualizzare la legenda \insglo{line chart}, la legenda non venga visualizzata.
		\end{itemize}
& Superato \\ \hline

	TA1.5.1.2 & Lo sviluppatore deve verificare che sia possibile scegliere la posizione in cui visualizzare la legenda di un \insglo{line chart}.
			
		& Allo sviluppatore è richiesto di:
		\begin{itemize}
			\item scegliere come tipo di grafico il \insglo{line chart};
			\item inserire i valori del grafico;
			\item selezionare l'impostazione per la visualizzare il \insglo{line chart};
			\item impostare la posizione in cui la leggenda del \insglo{line chart} deve essere visualizzata.

		\end{itemize}
& Superato \\ \hline

	TA1.5.2 & Lo sviluppatore deve verificare che sia possibile scegliere le impostazioni riguardanti la legenda di un \insglo{bar chart}.

		& Allo sviluppatore è richiesto di:
		\begin{itemize}
			\item scegliere come tipo di grafico il \insglo{bar chart};
			\item inserire le impostazioni riguardanti la legenda.
		\end{itemize}
& Superato \\ \hline

	TA1.5.2.1 & Lo sviluppatore deve verificare che sia possibile scegliere se visualizzare o nascondere la legenda di un \insglo{bar chart}.
		
		& Allo sviluppatore è richiesto di:
		\begin{itemize}
			\item scegliere come tipo di grafico il \insglo{bar chart};
			\item inserire i valori del grafico;
			\item verificare che, selezionando l'impostazione per la visualizzazione della legenda \insglo{bar chart}, la legenda venga visualizzata;
			\item verificare che, selezionando l'impostazione per non visualizzare la legenda \insglo{bar chart}, la legenda non venga visualizzata.
		\end{itemize}
& Superato \\ \hline

	TA1.5.2.2 & Lo sviluppatore deve verificare che sia possibile scegliere la posizione in cui visualizzare la legenda di un \insglo{bar chart}.
			
		& Allo sviluppatore è richiesto di:
		\begin{itemize}
			\item scegliere come tipo di grafico il \insglo{bar chart};
			\item inserire i valori del grafico;
			\item selezionare l'impostazione per la visualizzare il \insglo{line chart};
			\item impostare la posizione in cui la leggenda del \insglo{line chart} deve essere visualizzata.
		\end{itemize}
& Superato \\ \hline

	TA1.5.3 & Lo sviluppatore deve verificare che sia possibile scegliere le impostazioni riguardanti la legenda di un \insglo{map chart}.

		& Allo sviluppatore è richiesto di:
		\begin{itemize}
			\item scegliere come tipo di grafico il \insglo{map chart};
			\item inserire le impostazioni riguardanti la legenda.
		\end{itemize}
& Superato \\ \hline

	TA1.5.3.1 & Lo sviluppatore deve verificare che sia possibile scegliere se visualizzare o nascondere la legenda di un \insglo{map chart}.
			
		& Allo sviluppatore è richiesto di:
		\begin{itemize}
			\item scegliere come tipo di grafico il \insglo{map chart};
			\item inserire i valori del grafico;
			\item verificare che, selezionando l'impostazione per la visualizzazione della legenda \insglo{map chart}, la legenda venga visualizzata;
			\item verificare che, selezionando l'impostazione per non visualizzare la legenda \insglo{map chart}, la legenda non venga visualizzata.
		\end{itemize}
& Superato \\ \hline

	TA1.5.3.2 & Lo sviluppatore deve verificare che sia possibile scegliere la posizione in cui visualizzare la legenda di un \insglo{map chart}.
			
		& Allo sviluppatore è richiesto di:
		\begin{itemize}
			\item scegliere come tipo di grafico il \insglo{map chart};
			\item inserire i valori del grafico;
			\item selezionare l'impostazione per la visualizzare il \insglo{line chart};
			\item impostare la posizione in cui la leggenda del \insglo{line chart} deve essere visualizzata.
		\end{itemize}
& Superato \\ \hline

	TA1.5.4/1 & Lo sviluppatore deve verificare se è possibile inserire una descrizione testuale per un un \insglo{bar chart}.

		& Allo sviluppatore è richiesto di:
		\begin{itemize}
			\item scegliere come tipo di grafico il \insglo{bar chart};
			\item inserire i valori del grafico; (forse non necessario perché dovrebbe esser visualizzata indipendentemente)
			\item inserire una descrizione testuale del grafico.
		\end{itemize}
& Superato \\ \hline

	TA1.5.4/2 & Lo sviluppatore deve verificare se è possibile inserire una descrizione testuale per un \insglo{line chart}.

		& Allo sviluppatore è richiesto di:
		\begin{itemize}
			\item scegliere come tipo di grafico il \insglo{line chart};
			\item inserire i valori del grafico; (forse non necessario perché dovrebbe esser visualizzata indipendentemente)
			\item inserire una descrizione testuale del grafico.
		\end{itemize}
& Superato \\ \hline

	TA1.5.4/3 & Lo sviluppatore deve verificare se è possibile inserire una descrizione testuale per un \insglo{map chart}.

		& Allo sviluppatore è richiesto di:
		\begin{itemize}
			\item scegliere come tipo di grafico il \insglo{map chart};
			\item inserire i valori del grafico; (forse non necessario perché dovrebbe esser visualizzata indipendentemente)
			\item inserire una descrizione testuale del grafico.
		\end{itemize}
& Superato \\ \hline

	TA1.5.4/4 & Lo sviluppatore deve verificare se è possibile inserire una descrizione testuale per in una \insglo{table}.

		& Allo sviluppatore è richiesto di:
		\begin{itemize}
			\item scegliere come tipo di grafico la \insglo{table};
			\item inserire i valori del grafico; (forse non necessario perché dovrebbe esser visualizzata indipendentemente)
			\item inserire una descrizione testuale del grafico.
		\end{itemize}
& Superato \\ \hline

	TA1.5.5 & Lo sviluppatore deve verificare che sia possibile scegliere le impostazioni riguardanti il piano cartesiano di un \insglo{line chart}.

		& Allo sviluppatore è richiesto di:
		\begin{itemize}
			\item scegliere come tipo di grafico il \insglo{line chart};
			\item inserire le impostazioni riguardanti il piano cartesiano.
		\end{itemize}
& Superato \\ \hline

	TA1.5.5.1 & Lo sviluppatore deve verificare se è possibile inserire il nome dei due assi cartesiani di un \insglo{line chart}.

		& Allo sviluppatore è richiesto di:
		\begin{itemize}
			\item scegliere come tipo di grafico il \insglo{line chart};
			\item inserire i valori del grafico;
			\item inserire il nome dei due assi cartesiani.
		\end{itemize}
& Superato \\ \hline

	TA1.5.5.2 & Lo sviluppatore deve verificare se è possibile scegliere se le linee della griglia di un \insglo{line chart} sono visualizzate o nascoste.

		& Allo sviluppatore è richiesto di::
		\begin{itemize}
			\item scegliere come tipo di grafico il \insglo{line chart};
			\item inserire i valori del grafico;
			\item scegliere se le linee della griglia sono visualizzate o nascoste.
		\end{itemize}
& Superato \\ \hline

	TA1.5.6 & Lo sviluppatore deve verificare che sia possibile scegliere le impostazioni riguardanti il piano cartesiano di un \insglo{bar chart}.

		& Allo sviluppatore è richiesto di:
		\begin{itemize}
			\item scegliere come tipo di grafico il \insglo{bar chart};
			\item inserire le impostazioni riguardanti il piano cartesiano.
		\end{itemize}
& Superato \\ \hline

	TA1.5.6.1 & Lo sviluppatore deve verificare se è possibile inserire il nome dei due assi cartesiani di un \insglo{bar chart}.

		& Allo sviluppatore è richiesto di:
		\begin{itemize}
			\item scegliere come tipo di grafico il \insglo{bar chart};
			\item inserire i valori del grafico;
			\item scegliere il massimo numero di dati visualizzati per ogni serie.
		\end{itemize}
& Superato \\ \hline

	TA1.5.6.2 & Lo sviluppatore deve verificare se è possibile scegliere se le linee della griglia di un \insglo{bar chart} sono visualizzate o nascoste.

		& Allo sviluppatore è richiesto di:
		\begin{itemize}
			\item scegliere come tipo di grafico il \insglo{bar chart};
			\item inserire i valori del grafico;
			\item scegliere se le linee della griglia sono visualizzate o nascoste.
		\end{itemize}
& Superato \\ \hline

	TA1.5.7/1 & Lo sviluppatore deve verificare che sia possibile scegliere le impostazioni per scegliere il formato di stampa dei dati di un \insglo{bar chart}.

		& Allo sviluppatore è richiesto di:
		\begin{itemize}
			\item scegliere come tipo di grafico il \insglo{bar chart};
			\item inserire le impostazioni per scegliere il formato di stampa dei dati.
		\end{itemize}
& Superato \\ \hline

	TA1.5.7/2 & Lo sviluppatore deve verificare che sia possibile scegliere le impostazioni per scegliere il formato di stampa dei dati di un \insglo{line chart}.

		& Allo sviluppatore è richiesto di:
		\begin{itemize}
			\item scegliere come tipo di grafico il \insglo{line chart};
			\item inserire le impostazioni per scegliere il formato di stampa dei dati.
		\end{itemize}
& Superato \\ \hline

	TA1.5.7/3 & Lo sviluppatore deve verificare che sia possibile scegliere le impostazioni per scegliere il formato di stampa dei dati di un \insglo{map chart}.

		& Allo sviluppatore è richiesto di:
		\begin{itemize}
			\item scegliere come tipo di grafico il \insglo{map chart};
			\item inserire le impostazioni per scegliere il formato di stampa dei dati.
		\end{itemize}
& Superato \\ \hline

	TA1.5.7/4 & Lo sviluppatore deve verificare che sia possibile scegliere le impostazioni per scegliere il formato di stampa dei dati di una \insglo{table}.

		& Allo sviluppatore è richiesto di:
		\begin{itemize}
			\item scegliere come tipo di grafico la \insglo{table};
			\item inserire le impostazioni per scegliere il formato di stampa dei dati.
		\end{itemize}
& Superato \\ \hline

	TA1.5.7.1 & Lo sviluppatore deve verificare se è possibile impostare il colore del testo di ogni cella in una \insglo{table}.

		& Allo sviluppatore è richiesto di:
		\begin{itemize}
			\item scegliere come tipo di grafico la \insglo{table};
			\item inserire i valori del grafico;
			\item impostare il colore del testo di ogni cella.
		\end{itemize}
& Superato \\ \hline

	TA1.5.7.2 & Lo sviluppatore deve verificare se è possibile impostare il colore dello sfondo di ogni cella in una \insglo{table}.

		& Allo sviluppatore è richiesto di:
		\begin{itemize}
			\item scegliere come tipo di grafico la \insglo{table};
			\item inserire i valori del grafico;
			\item impostare il colore dello sfondo di ogni cella.
		\end{itemize}
& Superato \\ \hline

	TA1.5.7.3 & Lo sviluppatore deve verificare se è possibile impostare la dimensione dello spazio tra due serie di un \insglo{bar chart}.

		& Allo sviluppatore è richiesto di:
		\begin{itemize}
			\item scegliere come tipo di grafico il \insglo{bar chart};
			\item inserire i valori del grafico;
			\item impostare la dimensione dello spazio tra due serie.
		\end{itemize}
& Superato \\ \hline

	TA1.5.7.4 & Lo sviluppatore deve verificare se è possibile impostare la dimensione dello spazio tra due valori di un \insglo{bar chart}.

		& Allo sviluppatore è richiesto di:
		\begin{itemize}
			\item scegliere come tipo di grafico il \insglo{bar chart};
			\item inserire i valori del grafico;
			\item impostare la dimensione dello spazio tra due serie.
		\end{itemize}
& Superato \\ \hline

	TA1.5.7.5 & Lo sviluppatore deve verificare se è possibile scegliere la forma dei marcatori del \insglo{map chart}.

		& Allo sviluppatore è richiesto di:
		\begin{itemize}
			\item scegliere come tipo di grafico il \insglo{map chart};
			\item inserire i valori del grafico;
			\item scegliere la forma dei marcatori.
		\end{itemize}
& Superato \\ \hline

	TA1.5.7.6 & Lo sviluppatore deve verificare se è possibile impostare la dimensione dei punti nel \insglo{line chart}.

		& Allo sviluppatore è richiesto di:
		\begin{itemize}
			\item scegliere come tipo di grafico il \insglo{line chart};
			\item inserire i valori del grafico;
			\item impostare la dimensione dei punti.
		\end{itemize}
& Superato \\ \hline

	TA1.5.7.7 & Lo sviluppatore deve verificare se è possibile scegliere se la linea di un \insglo{line chart} è curva o segmentata.

		& Allo sviluppatore è richiesto di:
		\begin{itemize}
			\item scegliere come tipo di grafico il \insglo{line chart};
			\item inserire i valori del grafico;
			\item scegliere se la linea di un \insglo{line chart} è curva o segmentata.
		\end{itemize}
& Superato \\ \hline

	TA1.5.8 & Lo sviluppatore deve verificare se è possibile scegliere l'orientamento delle barre tra verticale e orizzontale in un \insglo{bar chart}.

		& Allo sviluppatore è richiesto di:
		\begin{itemize}
			\item scegliere come tipo di grafico il \insglo{bar chart};
			\item inserire i valori del grafico;
			\item scegliere l'orientamento delle barre tra verticale e orizzontale.
		\end{itemize}
& Superato \\ \hline

	TA1.5.9 & Lo sviluppatore deve verificare se è possibile scegliere le dimensioni dell'area visualizzata in un \insglo{map chart}.

		& Allo sviluppatore è richiesto di:
		\begin{itemize}
			\item scegliere come tipo di grafico il \insglo{map chart};
			\item inserire i valori del grafico;
			\item scegliere l'orientamento delle barre tra verticale e orizzontale.
		\end{itemize}
& Superato \\ \hline

	TA1.5.10 & Lo sviluppatore deve verificare se è possibile scegliere il punto centrale della mappa in un \insglo{map chart}.

		& Allo sviluppatore è richiesto di:
		\begin{itemize}
			\item scegliere come tipo di grafico il \insglo{map chart};
			\item inserire i valori del grafico;
			\item scegliere il punto centrale della mappa.
		\end{itemize}
& Superato \\ \hline

	TA1.5.11/1 & Lo sviluppatore deve verificare se è possibile scegliere il massimo numero di dati visualizzati per ogni serie di un \insglo{line chart}.

		& Allo sviluppatore è richiesto di:
		\begin{itemize}
			\item scegliere come tipo di grafico il \insglo{line chart};
			\item inserire i valori del grafico;
			\item scegliere il massimo numero di dati visualizzati per ogni serie.
		\end{itemize}
& Superato \\ \hline

	TA1.5.11/2 & Lo sviluppatore deve verificare se è possibile scegliere il massimo numero di dati visualizzati per ogni serie di un \insglo{bar chart}.

		& Allo sviluppatore è richiesto di:
		\begin{itemize}
			\item scegliere come tipo di grafico il \insglo{bar chart};
			\item inserire i valori del grafico;
			\item scegliere il massimo numero di dati visualizzati per ogni serie.
		\end{itemize}
& Superato \\ \hline

	TA1.5.11/3 & Lo sviluppatore deve verificare se è possibile scegliere il massimo numero di dati visualizzati per ogni serie di un \insglo{map chart}.

		& Allo sviluppatore è richiesto di:
		\begin{itemize}
			\item scegliere come tipo di grafico il \insglo{map chart};
			\item inserire i valori del grafico;
			\item scegliere il massimo numero di dati visualizzati per ogni serie.
		\end{itemize}
& Superato \\ \hline

	TA1.5.12 & Lo sviluppatore deve verificare se è possibile scegliere il massimo numero di righe visualizzate di una \insglo{table}.

		& Allo sviluppatore è richiesto di:
		\begin{itemize}
			\item scegliere come tipo di grafico una \insglo{table};
			\item inserire i valori del grafico;
			\item scegliere il massimo numero di righe visualizzate.
		\end{itemize}
& Superato \\ \hline

	TA1.5.13 & Lo sviluppatore deve verificare se è possibile ordinare i dati per colonna in una \insglo{table}.

		& Allo sviluppatore è richiesto di:
		\begin{itemize}
			\item scegliere come tipo di grafico una \insglo{table};
			\item inserire i valori del grafico;
			\item ordinare i dati per colonna.
		\end{itemize}
& Superato \\ \hline

	TA1.5.14 & Lo sviluppatore deve verificare se è possibile scegliere la posizione in cui vengono aggiunte nuove righe in una \insglo{table}.

		& Allo sviluppatore è richiesto di:
		\begin{itemize}
			\item scegliere come tipo di grafico una \insglo{table};
			\item inserire i valori del grafico;
			\item scegliere la posizione in cui vengono aggiunte nuove righe.
		\end{itemize}
& Superato \\ \hline

	TA1.5.15 & Lo sviluppatore deve verificare se è possibile scegliere l'intestazione di ciascuna colonna in una \insglo{table}.

		& Allo sviluppatore è richiesto di:
		\begin{itemize}
			\item scegliere come tipo di grafico una \insglo{table};
			\item inserire i valori del grafico;
			\item scegliere l'intestazione di ciascuna colonna.
		\end{itemize}
& Superato \\ \hline

	TA1.5.16 & Lo sviluppatore deve verificare se è possibile scegliere se le linee di una \insglo{table} sono visualizzate o nascoste.

		& Allo sviluppatore è richiesto di:
		\begin{itemize}
			\item scegliere come tipo di grafico una \insglo{table};
			\item inserire i valori del grafico;
			\item scegliere se le linee sono visualizzate o nascoste.
		\end{itemize}
& Superato \\ \hline

	TA1.5.17/1 & Lo sviluppatore deve verificare se è possibile impostare il titolo in un \insglo{bar chart}.

		& Allo sviluppatore è richiesto di:
		\begin{itemize}
			\item scegliere come tipo di grafico il \insglo{bar chart};
			\item impostare il titolo del grafico.
		\end{itemize}
& Superato \\ \hline

	TA1.5.17/2 & Lo sviluppatore deve verificare se è possibile impostare il titolo in un \insglo{line chart}.

		& Allo sviluppatore è richiesto di:
		\begin{itemize}
			\item scegliere come tipo di grafico il \insglo{line chart};
			\item impostare il titolo del grafico.
		\end{itemize}
& Superato \\ \hline

	TA1.5.17/3 & Lo sviluppatore deve verificare se è possibile impostare il titolo in un \insglo{map chart}.

		& Allo sviluppatore è richiesto di:
		\begin{itemize}
			\item scegliere come tipo di grafico il \insglo{map chart};
			\item impostare il titolo del grafico.
		\end{itemize}
& Superato \\ \hline

	TA1.5.17/4 & Lo sviluppatore deve verificare se è possibile impostare il titolo in una \insglo{table}.

		& Allo sviluppatore è richiesto di:
		\begin{itemize}
			\item scegliere come tipo di grafico la \insglo{table};
			\item impostare il titolo del grafico.
		\end{itemize}
& Superato \\ \hline

	TA1.5.18 & Lo sviluppatore deve verificare se è possibile scegliere il colore per ciascun set di dati in un \insglo{line chart}.

		& Allo sviluppatore è richiesto di:
		\begin{itemize}
			\item scegliere come tipo di grafico il \insglo{line chart};
			\item inserire i valori del grafico;
			\item scegliere il colore per ciascun set di dati.
		\end{itemize}
& Superato \\ \hline

	TA1.5.19 & Lo sviluppatore deve verificare se è possibile scegliere il colore per ciascun set di dati in un \insglo{bar chart}.

		& Allo sviluppatore è richiesto di:
		\begin{itemize}
			\item scegliere come tipo di grafico il \insglo{bar chart};
			\item inserire i valori del grafico;
			\item scegliere il colore per ciascun set di dati.
		\end{itemize}
& Superato \\ \hline

	TA1.5.20 & Lo sviluppatore deve verificare se è possibile scegliere il colore per ciascun set di dati in un \insglo{map chart}.

		& Allo sviluppatore è richiesto di:
		\begin{itemize}
			\item scegliere come tipo di grafico il \insglo{map chart};
			\item inserire i valori del grafico;
			\item scegliere il colore per ciascun set di dati.
		\end{itemize}
& Superato \\ \hline

	TA1.5.21 & Lo sviluppatore deve verificare se è possibile scegliere il nome per ciascun set di dati di un \insglo{line chart}.

		& Allo sviluppatore è richiesto di:
		\begin{itemize}
			\item scegliere come tipo di grafico il \insglo{line chart};
			\item inserire i valori del grafico;
			\item impostare il titolo del grafico.
		\end{itemize}
& Superato \\ \hline

	TA1.5.22 & Lo sviluppatore deve verificare se è possibile scegliere il nome per ciascun set di dati di un \insglo{bar chart}.

		& Allo sviluppatore è richiesto di:
		\begin{itemize}
			\item scegliere come tipo di grafico il \insglo{bar chart};
			\item inserire i valori del grafico;
			\item impostare il titolo del grafico.
		\end{itemize}
& Superato \\ \hline

	TA1.5.23 & Lo sviluppatore deve verificare se è possibile scegliere il nome per ciascun set di dati di un \insglo{map chart}.

		& Allo sviluppatore è richiesto di:
		\begin{itemize}
			\item scegliere come tipo di grafico il \insglo{map chart};
			\item inserire i valori del grafico;
			\item impostare il titolo del grafico.
		\end{itemize}
& Superato \\ \hline

	TA1.6/1 & Lo sviluppatore deve verificare se è possibile inserire i dati in un \insglo{bar chart}.

		& Allo sviluppatore è richiesto di:
		\begin{itemize}
			\item scegliere come tipo di grafico il \insglo{bar chart};
			\item inserire i valori del grafico.
		\end{itemize}
& Superato \\ \hline	
			
	TA1.6/2 & Lo sviluppatore deve verificare se è possibile inserire i dati in un \insglo{line chart}.

		& Allo sviluppatore è richiesto di:
		\begin{itemize}
			\item scegliere come tipo di grafico il \insglo{line chart};
			\item inserire i valori del grafico.
		\end{itemize}
& Superato \\ \hline	

	TA1.6/3 & Lo sviluppatore deve verificare se è possibile inserire i dati in una \insglo{table}.

		& Allo sviluppatore è richiesto di:
		\begin{itemize}
			\item scegliere come tipo di grafico la \insglo{table};
			\item inserire i valori del grafico.
		\end{itemize}
& Superato \\ \hline	

	TA1.6/4 & Lo sviluppatore deve verificare se è possibile inserire i dati in un \insglo{map chart}.

		& Allo sviluppatore è richiesto di:
		\begin{itemize}
			\item scegliere come tipo di grafico il \insglo{map chart};
			\item inserire i valori del grafico.
		\end{itemize}
& Superato \\ \hline

	TA1.7 & Lo sviluppatore deve verificare se viene visualizzato un errore qualora i dati passati per l'aggiornamento di un grafico siano scorretti.

		& Allo sviluppatore è richiesto di:
		\begin{itemize}
			\item creare un grafico di tipo x (è necessario rifare questo test per ogni grafico di tipo x messo a disposizione dal \insglo{framework});
			\item inserire i dati;
			\item aggiornare il grafico con un metodo permesso dal grafico  di tipo x passando dei valori non presenti (valore della serie dei dati o valore da sostituire) nell'istanza del grafico;
			\item verificare che sia ritornato una segnalazione di errore.
		\end{itemize}
& Superato \\ \hline

	TA1.8 & Lo sviluppatore deve verificare se viene visualizzato un errore qualora i dati passati per la creazione di un grafico siano scorretti.

		& Allo sviluppatore è richiesto di:
		\begin{itemize}
			\item creare un grafico di tipo x (è necessario rifare questo test per ogni grafico di tipo x messo a disposizione dal \insglo{framework});
			\item inserire i dati delle impostazioni o dei valori scorretti;
			\item verificare che sia ritornato una segnalazione di errore.
		\end{itemize}
& Superato \\ \hline

	TA1.9 & Lo sviluppatore deve verificare se viene visualizzato un errore qualora si cerchi di inserire un grafico oltre il limite consentito dalla pagina.

		& Allo sviluppatore è richiesto di:
		\begin{itemize}
			\item creare una pagina con massimo x grafici;
			\item inserire x+1 grafici;
			\item verificare che sia ritornato una segnalazione di errore.
		\end{itemize}
& Superato \\ \hline

	TA1.10 & Lo sviluppatore deve verificare se viene visualizzato un errore qualora si cerchi di aggiornare un grafico con un tipo di aggiornamento da lui non permesso.

		& Allo sviluppatore è richiesto di:
		\begin{itemize}
			\item creare un grafico di tipo x (è necessario rifare questo test per ogni grafico di tipo x messo a disposizione dal \insglo{framework});
			\item inserire i dati;
			\item effettuare gli aggiornamenti non permessi;
			\item verificare che sia ritornato una segnalazione di errore.
		\end{itemize}
& Superato \\ \hline

	TA2 & Lo sviluppatore deve verificare se sia permesso l'aggiornamento di un grafico tramite le \insglo{API} interne.
		
		& Allo sviluppatore è richiesto di:
		\begin{itemize}
			\item creare un grafico inserendo dei dati;
			\item verificare che sia permesso l'aggiornamento di un grafico tramite le \insglo{API} interne.
		\end{itemize}
& Superato \\ \hline

	TA2.1 & Lo sviluppatore deve verificare se sia permesso l'aggiornamento \insglo{stream} di un grafico \insglo{line chart}
		
		& Allo sviluppatore è richiesto di:
		\begin{itemize}
			\item creare un grafico \insglo{line chart};
			\item verificare che sia permesso l'aggiornamento \insglo{stream} per il grafico creato.
		\end{itemize}
& Superato \\ \hline

	TA2.2 & Lo sviluppatore deve verificare se sia permesso l'aggiornamento di un grafico \insglo{table}.
		
		& Allo sviluppatore è richiesto di:
		\begin{itemize}
			\item creare un grafico \insglo{table};
			\item verificare che sia permesso l'aggiornamento \insglo{stream} per il grafico creato.
		\end{itemize}
& Superato \\ \hline

	TA2.3 & Lo sviluppatore deve verificare se sia permesso l'aggiornamento \insglo{movie} di un grafico \insglo{map chart}.
		
		& Allo sviluppatore è richiesto di:
		\begin{itemize}
			\item creare un grafico \insglo{map chart};
			\item verificare che sia permesso l'aggiornamento \insglo{movie} per il grafico creato.
		\end{itemize}
& Superato \\ \hline

	TA2.4 & Lo sviluppatore deve verificare se sia permesso l'aggiornamento \insglo{in place} di un grafico \insglo{bar chart}.
		
		& Allo sviluppatore è richiesto di:
		\begin{itemize}
			\item creare un grafico \insglo{bar chart};
			\item verificare che sia permesso l'aggiornamento \insglo{in place} per il grafico creato.
		\end{itemize}
& Superato \\ \hline

	TA2.5 & Lo sviluppatore deve verificare se sia permesso l'aggiornamento \insglo{in place} di un grafico \insglo{line chart}.
		
		& Allo sviluppatore è richiesto di:
		\begin{itemize}
			\item creare un grafico \insglo{line chart}
			\item verificare che sia permesso l'aggiornamento \insglo{in place} per il grafico creato.
		\end{itemize}
& Superato \\ \hline

	TA2.6 & Lo sviluppatore deve verificare se sia permesso l'aggiornamento \insglo{in place} di un grafico \insglo{map chart}.
		
		& Allo sviluppatore è richiesto di:
		\begin{itemize}
			\item creare un grafico \insglo{map chart}
			\item verificare che sia permesso l'aggiornamento \insglo{in place} per il grafico creato.
		\end{itemize}
& Superato \\ \hline

	TA2.7 & Lo sviluppatore deve verificare se sia permesso l'aggiornamento \insglo{in place} di un grafico \insglo{table}.
		
		& Allo sviluppatore è richiesto di:
		\begin{itemize}
			\item creare un grafico \insglo{table};
			\item verificare che sia permesso l'aggiornamento \insglo{in place} per il grafico creato.
		\end{itemize}
& Superato \\ \hline

	TA3 & Lo sviluppatore deve verificare se sia permessa la creazione di una pagina HTML contenente alcuni grafici tramite le \insglo{API} interne.
		
		& Allo sviluppatore è richiesto di:
		\begin{itemize}
			\item creare un'istanza di \projectname{};
			\item creare una pagina;
			\item creare un grafico da associare a quella pagina;
			\item aggiungere il grafico ala pagina.
		\end{itemize}
& Superato \\ \hline

	TA3.1 & Lo sviluppatore deve verificare se sia permesso l'inserimento del titolo nella pagina.
		
		& Allo sviluppatore è richiesto di:
		\begin{itemize}
			\item creare un'istanza di \projectname{};
			\item creare una pagina;
			\item impostare il titolo di quella pagina;
		\end{itemize}
& Superato \\ \hline

	TA3.2 & Lo sviluppatore deve verificare se sia permesso la scelta delle opzioni di visualizzazione.
		
		& Allo sviluppatore è richiesto di:
		\begin{itemize}
			\item creare un'istanza di \projectname{};
			\item creare una pagina;
			\item impostare l'opzione di visualizzazione desiderata per quella pagina.
		\end{itemize}
& Superato \\ \hline

	TA3.2.1 & Lo sviluppatore deve verificare se sia permessa l'impostazione del massimo numero di grafici nella pagina.
		
		& Allo sviluppatore è richiesto di:
		\begin{itemize}
			\item creare un'istanza di \projectname{};
			\item creare una pagina;
			\item impostare il massimo numero di grafici per quella pagina.
		\end{itemize}
& Superato \\ \hline

	TA3.2.2 & Lo sviluppatore deve verificare se sia permessa l'impostazione del massimo numero di grafici visualizzabile per colonna in una pagina.
		
		& Allo sviluppatore è richiesto di:
		\begin{itemize}
			\item creare un'istanza di \projectname{};
			\item creare una pagina;
			\item impostare l'opzione di visualizzazione;
			\item impostare il massimo numero di grafici visualizzabile per colonna in quella pagina;
		\end{itemize}
& Superato \\ \hline

	TA3.3 & Lo sviluppatore deve verificare se sia permesso l'aggiunta di grafici nella pagina.
		
		& Allo sviluppatore è richiesto di:
		\begin{itemize}
			\item creare un'istanza di \projectname{};
			\item creare una pagina;
			\item creare almeno un grafico;
			\item aggiungere i grafici creati alla pagina.
		\end{itemize}
& Superato \\ \hline

	TA4 & Lo sviluppatore deve verificare che il \insglo{framework} permetta di ottenere un \insglo{middleware} per \insglo{Express.js} tramite le \insglo{API} interne.
		
		& Allo sviluppatore è richiesto di:
		\begin{itemize}
			\item verificare che sia possibile ottenere un \insglo{middleware} per \insglo{Express.js} tramite le funzionalità offerte dalle \insglo{API} interne.
		\end{itemize}
& Superato \\ \hline

	TA5 & L'utente \insglo{client} deve verificare che il \insglo{framework} permetta di ottenere tramite le \insglo{API} esterne le informazioni sui grafici creati con le \insglo{API} interne.

		& All'utente \insglo{client} è richiesto di:
		\begin{itemize}
			\item autenticarsi ad una istanza di \projectname{} esistente;
			\item verificare che, tramite le funzionalità offerte dalle \insglo{API} esterne, sia possibile ottenere le informazioni sui grafici esistenti.
		\end{itemize}
& Superato \\ \hline

	TA5.1 & L'utente \insglo{client} deve verificare che le \insglo{API} esterne permettano di ottenere la lista dei grafici presenti.

		& All'utente \insglo{client} è richiesto di:
		\begin{itemize}
			\item autenticarsi ad una istanza di \projectname{} esistente;
			\item verificare che, tramite le funzionalità offerte dalle \insglo{API} esterne, sia possibile ottenere la lista dei grafici esistenti.
		\end{itemize}
& Superato \\ \hline

	TA5.1.1 & L'utente \insglo{client} deve verificare che la lista dei grafici fornisca ID, titolo, tipo e descrizione di ciascun grafico.
		
		& All'utente \insglo{client} è richiesto di:
		\begin{itemize}
			\item autenticarsi ad una istanza di \projectname{} esistente;
			\item verificare che la lista dei grafici fornisca ID, titolo, tipo e descrizione di ciascun grafico.
		\end{itemize}
& Superato \\ \hline

	TA5.2 & L'utente \insglo{client} deve verificare che le \insglo{API} esterne permettano di ottenere i grafici con relativi aggiornamenti.
		
		& All'utente \insglo{client} è richiesto di:
		\begin{itemize}
			\item autenticarsi ad una istanza di \projectname{} esistente;
			\item verificare che, tramite le funzionalità offerte dalle \insglo{API} esterne, sia possibile ottenere i grafici con relativi aggiornamenti.
		\end{itemize}
& Superato \\ \hline

	TA5.3 & L'utente \insglo{client} deve verificare che le \insglo{API} esterne permettano di accedere a un'istanza di \projectname{} tramite username e password.
		
		& All'utente \insglo{client} è richiesto di:
		\begin{itemize}
			\item verificare che, tramite le funzionalità offerte dalle \insglo{API} esterne, sia possibile accedere a un'istanza di \projectname{} tramite username e password.
		\end{itemize}
& Superato \\ \hline

	TA5.4 & L'utente \insglo{client} deve verificare che le \insglo{API} esterne permettano di scollegarsi  da un'istanza di \projectname{}.
		
		& All'utente \insglo{client} è richiesto di:
		\begin{itemize}
			\item autenticarsi ad una istanza di \projectname{} esistente;
			\item verificare che, tramite le funzionalità offerte dalle \insglo{API} esterne, sia possibile scollegarsi da un'istanza di \projectname{}.
		\end{itemize}
& Superato \\ \hline

	TA6 & L'utente \insglo{client} deve verificare che il \insglo{framework} permetta l'inserimento di grafici all'interno di pagine HTML tramite un'apposita libreria .
		
		& All'utente \insglo{client} è richiesto di:
		\begin{itemize}
			\item autenticarsi ad una istanza di \projectname{} esistente;
			\item verificare che, tramite le funzionalità offerte dalla libreria, sia possibile inserire grafici esistenti all'interno di pagine HTML.
		\end{itemize}
& Superato \\ \hline

	TA6.1 & L'utente \insglo{client} deve verificare che la libreria permetta l'inserimento di un grafico all'interno di un determinato tag HTML.
		
		& All'utente \insglo{client} è richiesto di:
		\begin{itemize}
			\item autenticarsi ad una istanza di \projectname{} esistente;
			\item scegliere un grafico dell'istanza di \projectname{};
			\item verificare che, tramite le funzionalità offerte dalla libreria, sia possibile inserire il grafico scelto all'interno di un determinato tag HTML;
			\item ripetere la verifica per ogni grafico.
		\end{itemize}
& Superato \\ \hline

	TA6.2 & L'utente \insglo{client} deve verificare che la libreria permetta di modificare le principali impostazioni di un grafico.
		
		& All'utente \insglo{client} è richiesto di:
		\begin{itemize}
			\item autenticarsi ad una istanza di \projectname{} esistente;
			\item scegliere un grafico dell'istanza di \projectname{};
			\item verificare che, tramite le funzionalità offerte dalla libreria, sia possibile modificare le principali impostazioni del grafico;
			\item ripetere la verifica per ogni grafico.
		\end{itemize}
& Superato \\ \hline

	TA6.2.1 & L'utente \insglo{client} deve verificare che la libreria permetta di modificare la scelta se visualizzare o meno la legenda di un \insglo{line chart}.
		
		& All'utente \insglo{client} è richiesto di:
		\begin{itemize}
			\item autenticarsi ad una istanza di \projectname{} esistente;
			\item scegliere un grafico di tipo \insglo{line chart} dell'istanza di \projectname{};
			\item verificare che, tramite le funzionalità offerte dalla libreria, sia possibile modificare la scelta se visualizzare o meno la legenda di un \insglo{line chart}.
			\item ripetere la verifica per ogni grafico di tipo \insglo{line chart}.
		\end{itemize}
& Superato \\ \hline

	TA6.2.2 & L'utente \insglo{client} deve verificare che la libreria permetta di modificare la scelta se visualizzare o meno la legenda di un \insglo{bar chart}.
		
		& All'utente \insglo{client} è richiesto di:
		\begin{itemize}
			\item autenticarsi ad una istanza di \projectname{} esistente;
			\item scegliere un grafico di tipo \insglo{bar chart} dell'istanza di \projectname{};
			\item verificare che, tramite le funzionalità offerte dalla libreria, sia possibile modificare la scelta se visualizzare o meno la legenda di un \insglo{bar chart}.
			\item ripetere la verifica per ogni grafico di tipo \insglo{bar chart}.
		\end{itemize}
& Superato \\ \hline

	TA6.2.3 & L'utente \insglo{client} deve verificare che la libreria permetta di modificare la scelta se visualizzare o meno la legenda di un \insglo{map chart}.
		
		& All'utente \insglo{client} è richiesto di:
		\begin{itemize}
			\item autenticarsi ad una istanza di \projectname{} esistente;
			\item scegliere un grafico di tipo \insglo{map chart} dell'istanza di \projectname{};
			\item verificare che, tramite le funzionalità offerte dalla libreria, sia possibile modificare la scelta se visualizzare o meno la legenda di un \insglo{map chart}.
			\item ripetere la verifica per ogni grafico di tipo \insglo{map chart}.
		\end{itemize}
& Superato \\ \hline

	TA6.2.4 & L'utente \insglo{client} deve verificare che la libreria permetta di modificare la posizione in cui è visualizzata la legenda di un \insglo{line chart}.
		
		& All'utente \insglo{client} è richiesto di:
		\begin{itemize}
			\item autenticarsi ad un'istanza di \projectname{} esistente;
			\item scegliere un grafico di tipo \insglo{line chart} dell'istanza di \projectname{};
			\item verificare che, tramite le funzionalità offerte dalla libreria, sia possibile modificare la posizione in cui è visualizzata la legenda di un \insglo{line chart}.
			\item ripetere la verifica per ogni grafico di tipo \insglo{line chart}.
		\end{itemize}
& Superato \\ \hline

	TA6.2.5 & L'utente \insglo{client} deve verificare che la libreria permetta di modificare la posizione in cui è visualizzata la legenda di un \insglo{bar chart}.
		
		& All'utente \insglo{client} è richiesto di:
		\begin{itemize}
			\item autenticarsi ad un'istanza di \projectname{} esistente;
			\item scegliere un grafico di tipo \insglo{bar chart} dell'istanza di \projectname{};
			\item verificare che, tramite le funzionalità offerte dalla libreria, sia possibile modificare la posizione in cui è visualizzata la legenda di un \insglo{bar chart}.
			\item ripetere la verifica per ogni grafico di tipo \insglo{bar chart}.
		\end{itemize}
& Superato \\ \hline

	TA6.2.6 & L'utente \insglo{client} deve verificare che la libreria permetta di modificare la posizione in cui è visualizzata la legenda di un \insglo{map chart}.
		
		& All'utente \insglo{client} è richiesto di:
		\begin{itemize}
			\item autenticarsi ad un'istanza di \projectname{} esistente;
			\item scegliere un grafico di tipo \insglo{map chart} dell'istanza di \projectname{};
			\item verificare che, tramite le funzionalità offerte dalla libreria, sia possibile modificare la posizione in cui è visualizzata la legenda di un \insglo{map chart}.
			\item ripetere la verifica per ogni grafico di tipo \insglo{map chart}.
		\end{itemize}
& Superato \\ \hline

	TA6.2.7 & L'utente \insglo{client} deve verificare che la libreria permetta di modificare la scelta se visualizzare o meno la griglia del piano cartesiano di un \insglo{line chart}.
		
		& All'utente \insglo{client} è richiesto di:
		\begin{itemize}
			\item autenticarsi ad un'istanza di \projectname{} esistente;
			\item scegliere un grafico di tipo \insglo{line chart} dell'istanza di \projectname{};
			\item verificare che, tramite le funzionalità offerte dalla libreria, sia possibile modificare la scelta se visualizzare o meno la griglia del piano cartesiano di un \insglo{line chart}.
			\item ripetere la verifica per ogni grafico di tipo \insglo{line chart}.
		\end{itemize}
& Superato \\ \hline

	TA6.2.8 & L'utente \insglo{client} deve verificare che la libreria permetta di modificare la scelta se visualizzare o meno la griglia del piano cartesiano di un \insglo{bar chart}.
		
		& All'utente \insglo{client} è richiesto di:
		\begin{itemize}
			\item autenticarsi ad un'istanza di \projectname{} esistente;
			\item scegliere un grafico di tipo \insglo{bar chart} dell'istanza di \projectname{};
			\item verificare che, tramite le funzionalità offerte dalla libreria, sia possibile modificare la scelta se visualizzare o meno la griglia del piano cartesiano di un \insglo{bar chart}.
			\item ripetere la verifica per ogni grafico di tipo \insglo{bar chart}.
		\end{itemize}
& Superato \\ \hline

	TA6.2.9 & L'utente \insglo{client} deve verificare che la libreria permetta di modificare il colore di ciascun set di dati di un \insglo{line chart}.
		
		& All'utente \insglo{client} è richiesto di:
		\begin{itemize}
			\item autenticarsi ad un'istanza di \projectname{} esistente;
			\item scegliere un grafico di tipo \insglo{line chart} dell'istanza di \projectname{};
			\item verificare che, tramite le funzionalità offerte dalla libreria, sia possibile modificare il colore di ciascun set di dati di un \insglo{line chart}.
			\item ripetere la verifica per ogni grafico di tipo \insglo{line chart}.
		\end{itemize}
& Superato \\ \hline

	TA6.2.10 & L'utente \insglo{client} deve verificare che la libreria permetta di modificare il colore di ciascun set di dati di un \insglo{bar chart}.
		
		& All'utente \insglo{client} è richiesto di:
		\begin{itemize}
			\item autenticarsi ad un'istanza di \projectname{} esistente;
			\item scegliere un grafico di tipo \insglo{bar chart} dell'istanza di \projectname{};
			\item verificare che, tramite le funzionalità offerte dalla libreria, sia possibile modificare il colore di ciascun set di dati di un \insglo{bar chart}.
			\item ripetere la verifica per ogni grafico di tipo \insglo{line chart}.
		\end{itemize}
& Superato \\ \hline

	TA6.2.11 & L'utente \insglo{client} deve verificare che la libreria permetta di modificare il colore di ciascun set di dati di un \insglo{map chart}.
		
		& All'utente \insglo{client} è richiesto di:
		\begin{itemize}
			\item autenticarsi ad un'istanza di \projectname{} esistente;
			\item scegliere un grafico di tipo \insglo{map chart} dell'istanza di \projectname{};
			\item verificare che, tramite le funzionalità offerte dalla libreria, sia possibile modificare il colore di ciascun set di dati di un \insglo{map chart}.
			\item ripetere la verifica per ogni grafico di tipo \insglo{map chart}.
		\end{itemize}
& Superato \\ \hline

	TA6.2.12 & L'utente \insglo{client} deve verificare che la libreria permetta di modificare il colore di sfondo di ogni cella per la \insglo{table}.
		
		& All'utente \insglo{client} è richiesto di:
		\begin{itemize}
			\item autenticarsi ad un'istanza di \projectname{} esistente;
			\item scegliere un grafico di tipo \insglo{table} dell'istanza di \projectname{};
			\item verificare che, tramite le funzionalità offerte dalla libreria, sia possibile modificare il colore di sfondo di ogni cella per la \insglo{table}.
			\item ripetere la verifica per ogni grafico di tipo \insglo{table}.
		\end{itemize}
& Superato \\ \hline

	TA6.2.13 & L'utente \insglo{client} deve verificare che la libreria permetta di modificare il colore del testo contenuto in ogni cella per la \insglo{table}.
		
		& All'utente \insglo{client} è richiesto di:
		\begin{itemize}
			\item autenticarsi ad un'istanza di \projectname{} esistente;
			\item scegliere un grafico di tipo \insglo{table} dell'istanza di \projectname{};
			\item verificare che, tramite le funzionalità offerte dalla libreria, sia possibile modificare il colore del testo contenuto in ogni cella per la \insglo{table}.
			\item ripetere la verifica per ogni grafico di tipo \insglo{table}.
		\end{itemize}
& Superato \\ \hline

	TA6.3 & L'utente \insglo{client} deve verificare che la libreria \insglo{Chuck} permetta di accedere a un'istanza di \projectname{} tramite username e password.
		
		& All'utente \insglo{client} è richiesto di:
		\begin{itemize}
			\item verificare che, tramite le funzionalità offerte dalla libreria \insglo{Chuck}, sia possibile raggiungere un'istanza di \projectname{} tramite username e password.
		\end{itemize}
& Superato \\ \hline

	TA6.4 & L'utente \insglo{client} deve verificare che la libreria \insglo{Chuck} permetta di scollegarsi da un'istanza di \projectname{}.

		& All'utente \insglo{client} è richiesto di:
		\begin{itemize}
			\item autenticarsi ad un'istanza di \projectname{} esistente;
			\item verificare che, tramite le funzionalità offerte dalla libreria \insglo{Chuck}, sia possibile scollegarsi dall'istanza di \projectname{}.
		\end{itemize}
& Superato \\ \hline

	TA7 & L'utente deve verificare che la \insglo{dashboard} permetta di visualizzare in tempo reali gli autobus dell'\insglo{APS}.

		& All'utente è richiesto di:
		\begin{itemize}
			\item andare all'indirizzo della \insglo{dashboard};
			\item verificare che vengano mostrati gli autobus dell'\insglo{APS};
			\item verificare che gli aggiornamenti avvengano in tempo reale, senza dover aggiornare la pagina.
		\end{itemize}
& Superato \\ \hline

	TA7.1 & L'utente deve verificare che la \insglo{dashboard} permetta di visualizzare il numero di autobus attivi per ciascuna linea.

		& All'utente è richiesto di:
		\begin{itemize}
			\item andare all'indirizzo della \insglo{dashboard};
			\item verificare che vengano mostrati il numero di autobus attivi per ciascuna linea.
		\end{itemize}
& Superato \\ \hline

	TA7.2 & L'utente deve verificare che la \insglo{dashboard} permetta di filtrare gli autobus per linea di appartenenza e visualizzarli di conseguenza.

		& All'utente è richiesto di:
		\begin{itemize}
			\item andare all'indirizzo della \insglo{dashboard};
			\item selezionare le linee che vogliono visualizzare;
			\item verificare che vengano mostrati solo gli autobus delle linee selezionate;
		\end{itemize}
& Superato \\ \hline

	TA8 & L'utente deve verificare che l'applicazione \insglo{Android} permetta di visualizzare dei grafici.

		& All'utente è richiesto di:
		\begin{itemize}
			\item installare l'apk;
			\item accedere ad una istanza di \projectname{};
			\item verificare che sia possibile visualizzare i grafici.
		\end{itemize}
& Superato \\ \hline

	TA8.1 & L'utente deve verificare che l'applicazione \insglo{Android} permetta di accedere a un'istanza di \projectname{} tramite username e password.

		& All'utente è richiesto di:
		\begin{itemize}
			\item avviare l'applicazione;
			\item inserire l'indirizzo di un'istanza di \projectname{};
			\item inserire le credenziali di autenticazione;
			\item verificare che sia visualizzata la lista dei grafici presenti nell'istanza di \projectname{} scelta.
		\end{itemize}
& Superato \\ \hline

	TA8.2 & L'utente deve verificare che l'applicazione \insglo{Android} permetta di visualizzare l'elenco dei grafici esistenti nell'istanza \projectname{} con relativo ID, titolo, tipo e descrizione.

		& All'utente è richiesto di:
		\begin{itemize}
			\item avviare l'applicazione;
			\item inserire un indirizzo valido di un'istanza di \projectname{} e se necessario autenticarsi;
			\item verificare che sia visualizzabile l'elenco dei grafici esistenti nell'istanza \projectname{} con relativo ID, titolo, tipo e descrizione.
		\end{itemize}
& Superato \\ \hline

	TA8.3 & L'utente deve verificare che l'applicazione \insglo{Android} permetta di selezionare e visualizzare un singolo grafico dell'istanza di \projectname{}.

		& All'utente è richiesto di:
		\begin{itemize}
			\item avviare l'applicazione;
			\item inserire un indirizzo valido di un'istanza di \projectname{} ed autenticarsi se necessario;
			\item selezionare un grafico dalla lista di grafici presenti;
			\item verificare visivamente la presenza di un singolo grafico e che quel grafico sia quello selezionato.
		\end{itemize}
& Superato \\ \hline

	TA8.4 & L'utente deve verificare che l'applicazione \insglo{Android} mostri un errore quando viene immesso un indirizzo per un'istanza di \projectname{} non valido.

		& All'utente è richiesto di:
		\begin{itemize}
			\item avviare l'applicazione;
			\item inserire un indirizzo per un'istanza di \projectname{} non valido;
			\item verificare che venga segnalato un errore.
		\end{itemize}
& Superato \\ \hline

	TA8.5 & L'utente deve verificare che l'applicazione \insglo{Android} mostri un errore quando vengono immessi dati di accesso (username e password) per un'istanza di \projectname{} non validi.

		& All'utente è richiesto di:
		\begin{itemize}
			\item avviare l'applicazione;
			\item inserire l'indirizzo di un'istanza di \projectname{};
			\item autenticarsi con credenziali errate;
			\item verificare che venga segnalato un errore di autenticazione.
		\end{itemize}
& Superato \\ \hline

\caption{Test di accettazione}

\end{longtabu}

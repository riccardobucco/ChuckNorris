\level{2}{Test di accettazione}
	\level{3}*{TA1}Lo sviluppatore deve verificare che sia possibile creare grafici tramite le API interne.
	 
		Allo sviluppatore è richiesto di
		\begin{itemize}
			\item scegliere il tipo di grafico che vuole creare;
			\item inserire i valori del grafico;
			\item inserire le impostazioni del grafico;
			\item verificare che venga creato il tipo di grafico desiderato.
		\end{itemize}

	\level{3}*{TA1.1}Lo sviluppatore deve verificare che sia possibile creare un bar chart.

		Allo sviluppatore è richiesto di
		\begin{itemize}
			\item scegliere come tipo di grafico il bar chart;
			\item inserire i valori del grafico.
		\end{itemize}

	\level{3}*{TA1.2}Lo sviluppatore deve verificare che sia possibile creare un line chart.

		Allo sviluppatore è richiesto di
		\begin{itemize}
			\item scegliere come tipo di grafico il line chart;
			\item inserire i valori del grafico.
		\end{itemize}

	\level{3}*{TA1.3}Lo sviluppatore deve verificare che sia possibile creare un map chart.

		Allo sviluppatore è richiesto di
		\begin{itemize}
			\item scegliere come tipo di grafico il map chart;
			\item inserire i valori del grafico.
		\end{itemize}

	\level{3}*{TA1.4}Lo sviluppatore deve verificare che sia possibile creare una table.

		Allo sviluppatore è richiesto di
		\begin{itemize}
			\item scegliere come tipo di grafico la table;
			\item inserire i valori del grafico.
		\end{itemize}

	\level{3}*{TA1.5/1}Lo sviluppatore deve verificare che sia possibile scegliere delle impostazioni di un bar chart.

		Allo sviluppatore è richiesto di
		\begin{itemize}
			\item scegliere come tipo di grafico il bar chart;
			\item inserire delle impostazioni.
		\end{itemize}

	\level{3}*{TA1.5/2}Lo sviluppatore deve verificare che sia possibile scegliere delle impostazioni di un line chart.

		Allo sviluppatore è richiesto di
		\begin{itemize}
			\item scegliere come tipo di grafico il line chart;
			\item inserire delle impostazioni.
		\end{itemize}

	\level{3}*{TA1.5/3}Lo sviluppatore deve verificare che sia possibile scegliere delle impostazioni di un map chart.

		Allo sviluppatore è richiesto di
		\begin{itemize}
			\item scegliere come tipo di grafico il map chart;
			\item inserire delle impostazioni.
		\end{itemize}

	\level{3}*{TA1.5/4}Lo sviluppatore deve verificare che sia possibile scegliere delle impostazioni di una table.

		Allo sviluppatore è richiesto di
		\begin{itemize}
			\item scegliere come tipo di grafico una table;
			\item inserire delle impostazioni.
		\end{itemize}

	\level{3}*{TA1.5.1}Lo sviluppatore deve verificare che sia possibile scegliere le impostazioni riguardanti la legenda di un line chart.

		Allo sviluppatore è richiesto di
		\begin{itemize}
			\item scegliere come tipo di grafico il line chart;
			\item inserire le impostazioni riguardanti la legenda.
		\end{itemize}

	\level{3}*{TA1.5.1.1}Lo sviluppatore deve verificare che sia possibile scegliere se visualizzare o nascondere la legenda di un line chart.
			
		Allo sviluppatore è richiesto di
		\begin{itemize}
			\item scegliere come tipo di grafico il line chart;
			\item inserire i valori del grafico;
			\item verificare che, selezionando l'impostazione per la visualizzazione della legenda line chart, la legenda venga visualizzata;
			\item verificare che, selezionando l'impostazione per la non visualizzare la legenda line chart, la legenda non venga visualizzata.
		\end{itemize}

	\level{3}*{TA1.5.1.2}Lo sviluppatore deve verificare che sia possibile scegliere la posizione in cui visualizzare la legenda di un line chart.
			
		Allo sviluppatore è richiesto di
		\begin{itemize}
			\item scegliere come tipo di grafico il line chart;
			\item inserire i valori del grafico;
			\item selezionare l'impostazione per la visualizzare il line chart;
			\item impostare la posizione in cui la leggenda del line chart deve essere visualizzata.

		\end{itemize}

	\level{3}*{TA1.5.2}Lo sviluppatore deve verificare che sia possibile scegliere le impostazioni riguardanti la legenda di un bar chart.

		Allo sviluppatore è richiesto di
		\begin{itemize}
			\item scegliere come tipo di grafico il bar chart;
			\item inserire le impostazioni riguardanti la legenda.
		\end{itemize}

	\level{3}*{TA1.5.2.1}Lo sviluppatore deve verificare che sia possibile scegliere se visualizzare o nascondere la legenda di un bar chart.
		
		Allo sviluppatore è richiesto di
		\begin{itemize}
			\item scegliere come tipo di grafico il bar chart;
			\item inserire i valori del grafico;
			\item verificare che, selezionando l'impostazione per la visualizzazione della legenda bar chart, la legenda venga visualizzata;
			\item verificare che, selezionando l'impostazione per la non visualizzare la legenda bar chart, la legenda non venga visualizzata.
		\end{itemize}

	\level{3}*{TA1.5.2.2}Lo sviluppatore deve verificare che sia possibile scegliere la posizione in cui visualizzare la legenda di un bar chart.
			
		Allo sviluppatore è richiesto di
		\begin{itemize}
			\item scegliere come tipo di grafico il bar chart;
			\item inserire i valori del grafico;
			\item selezionare l'impostazione per la visualizzare il line chart;
			\item impostare la posizione in cui la leggenda del line chart deve essere visualizzata.
		\end{itemize}

	\level{3}*{TA1.5.3}Lo sviluppatore deve verificare che sia possibile scegliere le impostazioni riguardanti la legenda di un map chart.

		Allo sviluppatore è richiesto di
		\begin{itemize}
			\item scegliere come tipo di grafico il map chart;
			\item inserire le impostazioni riguardanti la legenda.
		\end{itemize}

	\level{3}*{TA1.5.3.1}Lo sviluppatore deve verificare che sia possibile scegliere se visualizzare o nascondere la legenda di un map chart.
			
		Allo sviluppatore è richiesto di
		\begin{itemize}
			\item scegliere come tipo di grafico il map chart;
			\item inserire i valori del grafico;
			\item verificare che, selezionando l'impostazione per la visualizzazione della legenda map chart, la legenda venga visualizzata;
			\item verificare che, selezionando l'impostazione per la non visualizzare la legenda map chart, la legenda non venga visualizzata.
		\end{itemize}

	\level{3}*{TA1.5.3.2}Lo sviluppatore deve verificare che sia possibile scegliere la posizione in cui visualizzare la legenda di un map chart.
			
		Allo sviluppatore è richiesto di
		\begin{itemize}
			\item scegliere come tipo di grafico il map chart;
			\item inserire i valori del grafico;
			\item selezionare l'impostazione per la visualizzare il line chart;
			\item impostare la posizione in cui la leggenda del line chart deve essere visualizzata.
		\end{itemize}

	\level{3}*{TA1.5.4/1}Lo sviluppatore deve verificare se è possibile inserire una descrizione testuale per un un bar chart.

		Allo sviluppatore è richiesto di
		\begin{itemize}
			\item scegliere come tipo di grafico il bar chart;
			\item inserire i valori del grafico; (forse non necessario perché dovrebbe esser visualizzata indipendentemente)
			\item inserire una descrizione testuale del grafico.
		\end{itemize}

	\level{3}*{TA1.5.4/2}Lo sviluppatore deve verificare se è possibile inserire una descrizione testuale per un line chart.

		Allo sviluppatore è richiesto di
		\begin{itemize}
			\item scegliere come tipo di grafico il line chart;
			\item inserire i valori del grafico; (forse non necessario perché dovrebbe esser visualizzata indipendentemente)
			\item inserire una descrizione testuale del grafico.
		\end{itemize}

	\level{3}*{TA1.5.4/3}Lo sviluppatore deve verificare se è possibile inserire una descrizione testuale per un map chart.

		Allo sviluppatore è richiesto di
		\begin{itemize}
			\item scegliere come tipo di grafico il map chart;
			\item inserire i valori del grafico; (forse non necessario perché dovrebbe esser visualizzata indipendentemente)
			\item inserire una descrizione testuale del grafico.
		\end{itemize}

	\level{3}*{TA1.5.4/4}Lo sviluppatore deve verificare se è possibile inserire una descrizione testuale per in una table.

		Allo sviluppatore è richiesto di
		\begin{itemize}
			\item scegliere come tipo di grafico la table;
			\item inserire i valori del grafico; (forse non necessario perché dovrebbe esser visualizzata indipendentemente)
			\item inserire una descrizione testuale del grafico.
		\end{itemize}

	\level{3}*{TA1.5.5}Lo sviluppatore deve verificare che sia possibile scegliere le impostazioni riguardanti il piano cartesiano di un line chart.

		Allo sviluppatore è richiesto di
		\begin{itemize}
			\item scegliere come tipo di grafico il line chart;
			\item inserire le impostazioni riguardanti il piano cartesiano.
		\end{itemize}

	\level{3}*{TA1.5.5.1}Lo sviluppatore deve verificare se è possibile inserire il nome dei due assi cartesiani di un line chart.

		Allo sviluppatore è richiesto di
		\begin{itemize}
			\item scegliere come tipo di grafico il line chart;
			\item inserire i valori del grafico;
			\item inserire il nome dei due assi cartesiani.
		\end{itemize}

	\level{3}*{TA1.5.5.2}Lo sviluppatore deve verificare se è possibile scegliere se le linee della griglia di un line chart sono visualizzate o nascoste.

		Allo sviluppatore è richiesto di:
		\begin{itemize}
			\item scegliere come tipo di grafico il line chart;
			\item inserire i valori del grafico;
			\item scegliere se le linee della griglia sono visualizzate o nascoste.
		\end{itemize}

	\level{3}*{TA1.5.6}Lo sviluppatore deve verificare che sia possibile scegliere le impostazioni riguardanti il piano cartesiano di un bar chart.

		Allo sviluppatore è richiesto di
		\begin{itemize}
			\item scegliere come tipo di grafico il bar chart;
			\item inserire le impostazioni riguardanti il piano cartesiano.
		\end{itemize}

	\level{3}*{TA1.5.6.1}Lo sviluppatore deve verificare se è possibile inserire il nome dei due assi cartesiani di un bar chart.

		Allo sviluppatore è richiesto di
		\begin{itemize}
			\item scegliere come tipo di grafico il bar chart;
			\item inserire i valori del grafico;
			\item scegliere il massimo numero di dati visualizzati per ogni serie.
		\end{itemize}

	\level{3}*{TA1.5.6.2}Lo sviluppatore deve verificare se è possibile scegliere se le linee della griglia di un bar chart sono visualizzate o nascoste.

		Allo sviluppatore è richiesto di
		\begin{itemize}
			\item scegliere come tipo di grafico il bar chart;
			\item inserire i valori del grafico;
			\item scegliere se le linee della griglia sono visualizzate o nascoste.
		\end{itemize}

	\level{3}*{TA1.5.7/1}Lo sviluppatore deve verificare che sia possibile scegliere le impostazioni per scegliere il formato di stampa dei dati di un bar chart.

		Allo sviluppatore è richiesto di
		\begin{itemize}
			\item scegliere come tipo di grafico il bar chart;
			\item inserire le impostazioni per scegliere il formato di stampa dei dati.
		\end{itemize}

	\level{3}*{TA1.5.7/2}Lo sviluppatore deve verificare che sia possibile scegliere le impostazioni per scegliere il formato di stampa dei dati di un line chart.

		Allo sviluppatore è richiesto di
		\begin{itemize}
			\item scegliere come tipo di grafico il line chart;
			\item inserire le impostazioni per scegliere il formato di stampa dei dati.
		\end{itemize}

	\level{3}*{TA1.5.7/3}Lo sviluppatore deve verificare che sia possibile scegliere le impostazioni per scegliere il formato di stampa dei dati di un map chart.

		Allo sviluppatore è richiesto di
		\begin{itemize}
			\item scegliere come tipo di grafico il map chart;
			\item inserire le impostazioni per scegliere il formato di stampa dei dati.
		\end{itemize}

	\level{3}*{TA1.5.7/4}Lo sviluppatore deve verificare che sia possibile scegliere le impostazioni per scegliere il formato di stampa dei dati di una table.

		Allo sviluppatore è richiesto di
		\begin{itemize}
			\item scegliere come tipo di grafico la table;
			\item inserire le impostazioni per scegliere il formato di stampa dei dati.
		\end{itemize}

	\level{3}*{TA1.5.7.1}Lo sviluppatore deve verificare se è possibile impostare il colore del testo di ogni cella in una table.

		Allo sviluppatore è richiesto di
		\begin{itemize}
			\item scegliere come tipo di grafico la table;
			\item inserire i valori del grafico;
			\item impostare il colore del testo di ogni cella.
		\end{itemize}

	\level{3}*{TA1.5.7.2}Lo sviluppatore deve verificare se è possibile impostare il colore dello sfondo di ogni cella in una table.

		Allo sviluppatore è richiesto di
		\begin{itemize}
			\item scegliere come tipo di grafico la table;
			\item inserire i valori del grafico;
			\item impostare il colore dello sfondo di ogni cella.
		\end{itemize}

	\level{3}*{TA1.5.7.3}Lo sviluppatore deve verificare se è possibile impostare la dimensione dello spazio tra due serie di un bar chart.

		Allo sviluppatore è richiesto di
		\begin{itemize}
			\item scegliere come tipo di grafico il bar chart;
			\item inserire i valori del grafico;
			\item impostare la dimensione dello spazio tra due serie.
		\end{itemize}

	\level{3}*{TA1.5.7.4}Lo sviluppatore deve verificare se è possibile impostare la dimensione dello spazio tra due valori di un bar chart.

		Allo sviluppatore è richiesto di
		\begin{itemize}
			\item scegliere come tipo di grafico il bar chart;
			\item inserire i valori del grafico;
			\item impostare la dimensione dello spazio tra due serie.
		\end{itemize}

	\level{3}*{TA1.5.7.5}Lo sviluppatore deve verificare se è possibile scegliere la forma dei marcatori del map chart.

		Allo sviluppatore è richiesto di
		\begin{itemize}
			\item scegliere come tipo di grafico il map chart;
			\item inserire i valori del grafico;
			\item scegliere la forma dei marcatori.
		\end{itemize}

	\level{3}*{TA1.5.7.6}Lo sviluppatore deve verificare se è possibile impostare la dimensione dei punti nel line chart.

		Allo sviluppatore è richiesto di
		\begin{itemize}
			\item scegliere come tipo di grafico il line chart;
			\item inserire i valori del grafico;
			\item impostare la dimensione dei punti.
		\end{itemize}

	\level{3}*{TA1.5.7.7}Lo sviluppatore deve verificare se è possibile scegliere se la linea di un line chart è curva o segmentata.

		Allo sviluppatore è richiesto di
		\begin{itemize}
			\item scegliere come tipo di grafico il line chart;
			\item inserire i valori del grafico;
			\item scegliere se la linea di un line chart è curva o segmentata.
		\end{itemize}

	\level{3}*{TA1.5.8}Lo sviluppatore deve verificare se è possibile scegliere l'orientamento delle barre tra verticale e orizzontale in un bar chart.

		Allo sviluppatore è richiesto di
		\begin{itemize}
			\item scegliere come tipo di grafico il bar chart;
			\item inserire i valori del grafico;
			\item scegliere l'orientamento delle barre tra verticale e orizzontale.
		\end{itemize}

	\level{3}*{TA1.5.9}Lo sviluppatore deve verificare se è possibile scegliere le dimensioni dell'area visualizzata in un map chart.

		Allo sviluppatore è richiesto di
		\begin{itemize}
			\item scegliere come tipo di grafico il map chart;
			\item inserire i valori del grafico;
			\item scegliere l'orientamento delle barre tra verticale e orizzontale.
		\end{itemize}

	\level{3}*{TA1.5.10}Lo sviluppatore deve verificare se è possibile scegliere il punto centrale della mappa in un map chart.

		Allo sviluppatore è richiesto di
		\begin{itemize}
			\item scegliere come tipo di grafico il map chart;
			\item inserire i valori del grafico;
			\item scegliere il punto centrale della mappa.
		\end{itemize}

	\level{3}*{TA1.5.11/1}Lo sviluppatore deve verificare se è possibile scegliere il massimo numero di dati visualizzati per ogni serie di un line chart.

		Allo sviluppatore è richiesto di
		\begin{itemize}
			\item scegliere come tipo di grafico il line chart;
			\item inserire i valori del grafico;
			\item scegliere il massimo numero di dati visualizzati per ogni serie.
		\end{itemize}

	\level{3}*{TA1.5.11/2}Lo sviluppatore deve verificare se è possibile scegliere il massimo numero di dati visualizzati per ogni serie di un bar chart.

		Allo sviluppatore è richiesto di
		\begin{itemize}
			\item scegliere come tipo di grafico il bar chart;
			\item inserire i valori del grafico;
			\item scegliere il massimo numero di dati visualizzati per ogni serie.
		\end{itemize}

	\level{3}*{TA1.5.11/3}Lo sviluppatore deve verificare se è possibile scegliere il massimo numero di dati visualizzati per ogni serie di un map chart.

		Allo sviluppatore è richiesto di
		\begin{itemize}
			\item scegliere come tipo di grafico il map chart;
			\item inserire i valori del grafico;
			\item scegliere il massimo numero di dati visualizzati per ogni serie.
		\end{itemize}

	\level{3}*{TA1.5.12}Lo sviluppatore deve verificare se è possibile scegliere il massimo numero di righe visualizzate di una table.

		Allo sviluppatore è richiesto di
		\begin{itemize}
			\item scegliere come tipo di grafico una table;
			\item inserire i valori del grafico;
			\item scegliere il massimo numero di righe visualizzate.
		\end{itemize}

	\level{3}*{TA1.5.13}Lo sviluppatore deve verificare se è possibile ordinare i dati per colonna in una table.

		Allo sviluppatore è richiesto di
		\begin{itemize}
			\item scegliere come tipo di grafico una table;
			\item inserire i valori del grafico;
			\item ordinare i dati per colonna.
		\end{itemize}

	\level{3}*{TA1.5.14}Lo sviluppatore deve verificare se è possibile scegliere la posizione in cui vengono aggiunte nuove righe in una table.

		Allo sviluppatore è richiesto di
		\begin{itemize}
			\item scegliere come tipo di grafico una table;
			\item inserire i valori del grafico;
			\item scegliere la posizione in cui vengono aggiunte nuove righe.
		\end{itemize}

	\level{3}*{TA1.5.15}Lo sviluppatore deve verificare se è possibile scegliere l'intestazione di ciascuna colonna in una table.

		Allo sviluppatore è richiesto di
		\begin{itemize}
			\item scegliere come tipo di grafico una table;
			\item inserire i valori del grafico;
			\item scegliere l'intestazione di ciascuna colonna.
		\end{itemize}

	\level{3}*{TA1.5.16}Lo sviluppatore deve verificare se è possibile scegliere se le linee di una table sono visualizzate o nascoste.

		Allo sviluppatore è richiesto di
		\begin{itemize}
			\item scegliere come tipo di grafico una table;
			\item inserire i valori del grafico;
			\item scegliere se le linee sono visualizzate o nascoste.
		\end{itemize}

	\level{3}*{TA1.5.17/1}Lo sviluppatore deve verificare se è possibile impostare il titolo in un bar chart.

		Allo sviluppatore è richiesto di
		\begin{itemize}
			\item scegliere come tipo di grafico il bar chart;
			\item impostare il titolo del grafico.
		\end{itemize}

	\level{3}*{TA1.5.17/2}Lo sviluppatore deve verificare se è possibile impostare il titolo in un line chart.

		Allo sviluppatore è richiesto di
		\begin{itemize}
			\item scegliere come tipo di grafico il line chart;
			\item impostare il titolo del grafico.
		\end{itemize}

	\level{3}*{TA1.5.17/3}Lo sviluppatore deve verificare se è possibile impostare il titolo in un map chart.

		Allo sviluppatore è richiesto di
		\begin{itemize}
			\item scegliere come tipo di grafico il map chart;
			\item impostare il titolo del grafico.
		\end{itemize}

	\level{3}*{TA1.5.17/4}Lo sviluppatore deve verificare se è possibile impostare il titolo in una table.

		Allo sviluppatore è richiesto di
		\begin{itemize}
			\item scegliere come tipo di grafico la table;
			\item impostare il titolo del grafico.
		\end{itemize}

	\level{3}*{TA1.5.18}Lo sviluppatore deve verificare se è possibile scegliere il colore per ciascun set di dati in un line chart.

		Allo sviluppatore è richiesto di
		\begin{itemize}
			\item scegliere come tipo di grafico il line chart;
			\item inserire i valori del grafico;
			\item scegliere il colore per ciascun set di dati.
		\end{itemize}

	\level{3}*{TA1.5.19}Lo sviluppatore deve verificare se è possibile scegliere il colore per ciascun set di dati in un bar chart.

		Allo sviluppatore è richiesto di
		\begin{itemize}
			\item scegliere come tipo di grafico il bar chart;
			\item inserire i valori del grafico;
			\item scegliere il colore per ciascun set di dati.
		\end{itemize}

	\level{3}*{TA1.5.20}Lo sviluppatore deve verificare se è possibile scegliere il colore per ciascun set di dati in un map chart.

		Allo sviluppatore è richiesto di
		\begin{itemize}
			\item scegliere come tipo di grafico il map chart;
			\item inserire i valori del grafico;
			\item scegliere il colore per ciascun set di dati.
		\end{itemize}

	\level{3}*{TA1.5.21}Lo sviluppatore deve verificare se è possibile scegliere il nome per ciascun set di dati di un line chart.

		Allo sviluppatore è richiesto di
		\begin{itemize}
			\item scegliere come tipo di grafico il line chart;
			\item inserire i valori del grafico;
			\item impostare il titolo del grafico.
		\end{itemize}

	\level{3}*{TA1.5.22}Lo sviluppatore deve verificare se è possibile scegliere il nome per ciascun set di dati di un bar chart.

		Allo sviluppatore è richiesto di
		\begin{itemize}
			\item scegliere come tipo di grafico il bar chart;
			\item inserire i valori del grafico;
			\item impostare il titolo del grafico.
		\end{itemize}

	\level{3}*{TA1.5.23}Lo sviluppatore deve verificare se è possibile scegliere il nome per ciascun set di dati di un map chart.

		Allo sviluppatore è richiesto di
		\begin{itemize}
			\item scegliere come tipo di grafico il map chart;
			\item inserire i valori del grafico;
			\item impostare il titolo del grafico.
		\end{itemize}

	\level{3}*{TA1.6/1}Lo sviluppatore deve verificare se è possibile inserire i dati in un bar chart.

		Allo sviluppatore è richiesto di
		\begin{itemize}
			\item scegliere come tipo di grafico il bar chart;
			\item inserire i valori del grafico.
		\end{itemize}	
			
	\level{3}*{TA1.6/2}Lo sviluppatore deve verificare se è possibile inserire i dati in un line chart.

		Allo sviluppatore è richiesto di
		\begin{itemize}
			\item scegliere come tipo di grafico il line chart;
			\item inserire i valori del grafico.
		\end{itemize}	

	\level{3}*{TA1.6/3}Lo sviluppatore deve verificare se è possibile inserire i dati in una table.

		Allo sviluppatore è richiesto di
		\begin{itemize}
			\item scegliere come tipo di grafico la table;
			\item inserire i valori del grafico.
		\end{itemize}	

	\level{3}*{TA1.6/4}Lo sviluppatore deve verificare se è possibile inserire i dati in un map chart.

		Allo sviluppatore è richiesto di
		\begin{itemize}
			\item scegliere come tipo di grafico il map chart;
			\item inserire i valori del grafico.
		\end{itemize}

	\level{3}*{TA1.7}Lo sviluppatore deve verificare se viene visualizzato un errore qualora i dati passati per l'aggiornamento di un grafico siano scorretti.

		Allo sviluppatore è richiesto di
		\begin{itemize}
			\item creare un grafico di tipo x (è necessario rifare questo test per ogni grafico di tipo x messo a disposizione dal framework);
			\item inserire i dati;
			\item aggiornare il grafico con un metodo permesso dal grafico  di tipo x passando dei valori non presenti (valore della serie dei dati o valore da sostituire) nell'istanza del grafico;
			\item verificare che sia ritornato una segnalazione di errore.
		\end{itemize}

	\level{3}*{TA1.8}Lo sviluppatore deve verificare se viene visualizzato un errore qualora i dati passati per la creazione di un grafico siano scorretti.

		Allo sviluppatore è richiesto di
		\begin{itemize}
			\item creare un grafico di tipo x (è necessario rifare questo test per ogni grafico di tipo x messo a disposizione dal framework);
			\item inserire i dati delle impostazioni o dei valori scorretti;
			\item verificare che sia ritornato una segnalazione di errore.
		\end{itemize}

	\level{3}*{TA1.9}Lo sviluppatore deve verificare se viene visualizzato un errore qualora si cerchi di inserire un grafico oltre il limite consentito dalla pagina.

		Allo sviluppatore è richiesto di
		\begin{itemize}
			\item creare una pagina con massimo x grafici;
			\item inserire x+1 grafici;
			\item verificare che sia ritornato una segnalazione di errore.
		\end{itemize}

	\level{3}*{TA1.10}Lo sviluppatore deve verificare se viene visualizzato un errore qualora si cerchi di aggiornare un grafico con un tipo di aggiornamento da lui non permesso.

		Allo sviluppatore è richiesto di
		\begin{itemize}
			\item creare un grafico di tipo x (è necessario rifare questo test per ogni grafico di tipo x messo a disposizione dal framework);
			\item inserire i dati;
			\item effettuare gli aggiornamenti non permessi;
			\item verificare che sia ritornato una segnalazione di errore.
		\end{itemize}

	\level{3}*{TA2}Lo sviluppatore deve verificare se sia permesso l'aggiornamento di un grafico tramite le API interne.
		
		Allo sviluppatore è richiesto di
		\begin{itemize}
			\item creare un grafico inserendo dei dati;
			\item verificare che sia permesso l'aggiornamento di un grafico tramite le API interne.
		\end{itemize}

	\level{3}*{TA2.1}Lo sviluppatore deve verificare se sia permesso l'aggiornamento stream di un grafico line chart
		
		Allo sviluppatore è richiesto di
		\begin{itemize}
			\item creare un grafico line chart;
			\item verificare che sia permesso l'aggiornamento stream per il grafico creato.
		\end{itemize}

	\level{3}*{TA2.2}Lo sviluppatore deve verificare se sia permesso l'aggiornamento di un grafico table.
		
		Allo sviluppatore è richiesto di
		\begin{itemize}
			\item creare un grafico table;
			\item verificare che sia permesso l'aggiornamento stream per il grafico creato.
		\end{itemize}

	\level{3}*{TA2.3}Lo sviluppatore deve verificare se sia permesso l'aggiornamento movie di un grafico map chart.
		
		Allo sviluppatore è richiesto di
		\begin{itemize}
			\item creare un grafico map chart;
			\item verificare che sia permesso l'aggiornamento movie per il grafico creato.
		\end{itemize}

	\level{3}*{TA2.4}Lo sviluppatore deve verificare se sia permesso l'aggiornamento in place di un grafico bar chart.
		
		Allo sviluppatore è richiesto di
		\begin{itemize}
			\item creare un grafico bar chart;
			\item verificare che sia permesso l'aggiornamento in place per il grafico creato.
		\end{itemize}

	\level{3}*{TA2.5}Lo sviluppatore deve verificare se sia permesso l'aggiornamento in place di un grafico line chart.
		
		Allo sviluppatore è richiesto di
		\begin{itemize}
			\item creare un grafico line chart
			\item verificare che sia permesso l'aggiornamento in place per il grafico creato.
		\end{itemize}

	\level{3}*{TA2.6}Lo sviluppatore deve verificare se sia permesso l'aggiornamento in place di un grafico map chart.
		
		Allo sviluppatore è richiesto di
		\begin{itemize}
			\item creare un grafico map chart
			\item verificare che sia permesso l'aggiornamento in place per il grafico creato.
		\end{itemize}

	\level{3}*{TA2.7}Lo sviluppatore deve verificare se sia permesso l'aggiornamento in place di un grafico table.
		
		Allo sviluppatore è richiesto di
		\begin{itemize}
			\item creare un grafico table;
			\item verificare che sia permesso l'aggiornamento in place per il grafico creato.
		\end{itemize}

	\level{3}*{TA3}Lo sviluppatore deve verificare se sia permessa la creazione di una pagina HTML contenente alcuni grafici tramite le API interne.
		
		Allo sviluppatore è richiesto di
		\begin{itemize}
			\item creare un'istanza di \projectname{};
			\item creare una pagina;
			\item creare un grafico da associare a quella pagina;
			\item aggiungere il grafico ala pagina.
		\end{itemize}

	\level{3}*{TA3.1}Lo sviluppatore deve verificare se sia permesso l'inserimento del titolo nella pagina.
		
		Allo sviluppatore è richiesto di
		\begin{itemize}
			\item creare un'istanza di \projectname{};
			\item creare una pagina;
			\item impostare il titolo di quella pagina;
		\end{itemize}

	\level{3}*{TA3.2}Lo sviluppatore deve verificare se sia permesso la scelta delle opzioni di visualizzazione.
		
		Allo sviluppatore è richiesto di
		\begin{itemize}
			\item creare un'istanza di \projectname{};
			\item creare una pagina;
			\item impostare l'opzione di visualizzazione desiderata per quella pagina.
		\end{itemize}

	\level{3}*{TA3.2.1}Lo sviluppatore deve verificare se sia permessa l'impostazione del massimo numero di grafici nella pagina.
		
		Allo sviluppatore è richiesto di
		\begin{itemize}
			\item creare un'istanza di \projectname{};
			\item creare una pagina;
			\item impostare il massimo numero di grafici per quella pagina.
		\end{itemize}

	\level{3}*{TA3.2.2}Lo sviluppatore deve verificare se sia permessa l'impostazione del massimo numero di grafici visualizzabile per colonna in una pagina.
		
		Allo sviluppatore è richiesto di
		\begin{itemize}
			\item creare un'istanza di \projectname{};
			\item creare una pagina;
			\item impostare l'opzione di visualizzazione;
			\item impostare il massimo numero di grafici visualizzabile per colonna in quella pagina;
		\end{itemize}

	\level{3}*{TA3.3}Lo sviluppatore deve verificare se sia permesso l'aggiunta di grafici nella pagina.
		
		Allo sviluppatore è richiesto di
		\begin{itemize}
			\item creare un'istanza di \projectname{};
			\item creare una pagina;
			\item creare almeno un grafico;
			\item aggiungere i grafici creati alla pagina.
		\end{itemize}

	\level{3}*{TA4}Lo sviluppatore deve verificare che il framework permetta di ottenere un middleware per Express.js tramite le API interne.
		
		Allo sviluppatore è richiesto di
		\begin{itemize}
			\item verificare che sia possibile ottenere un middleware per Express.js tramite le funzionalità offerte dalle API interne.
		\end{itemize}

	\level{3}*{TA5}L'utente client deve verificare che il framework permetta di ottenere tramite le API esterne le informazioni sui grafici creati con le API interne.

		All'utente client è richiesto di
		\begin{itemize}
			\item autenticarsi ad una istanza di \projectname{} esistente;
			\item verificare che, tramite le funzionalità offerte dalle API esterne, sia possibile ottenere le informazioni sui grafici esistenti.
		\end{itemize}

	\level{3}*{TA5.1}L'utente client deve verificare che le API esterne permettano di ottenere la lista dei grafici presenti.

		All'utente client è richiesto di
		\begin{itemize}
			\item autenticarsi ad una istanza di \projectname{} esistente;
			\item verificare che, tramite le funzionalità offerte dalle API esterne, sia possibile ottenere la lista dei grafici esistenti.
		\end{itemize}

	\level{3}*{TA5.1.1}L'utente client deve verificare che la lista dei grafici fornisca ID, titolo, tipo e descrizione di ciascun grafico.
		
		All'utente client è richiesto di
		\begin{itemize}
			\item autenticarsi ad una istanza di \projectname{} esistente;
			\item verificare che la lista dei grafici fornisca ID, titolo, tipo e descrizione di ciascun grafico.
		\end{itemize}

	\level{3}*{TA5.2}L'utente client deve verificare che le API esterne permettano di ottenere i grafici con relativi aggiornamenti.
		
		All'utente client è richiesto di
		\begin{itemize}
			\item autenticarsi ad una istanza di \projectname{} esistente;
			\item verificare che, tramite le funzionalità offerte dalle API esterne, sia possibile ottenere i grafici con relativi aggiornamenti.
		\end{itemize}

	\level{3}*{TA5.3}L'utente client deve verificare che le API esterne permettano di accedere a un'istanza di \projectname{} tramite username e password.
		
		All'utente client è richiesto di
		\begin{itemize}
			\item verificare che, tramite le funzionalità offerte dalle API esterne, sia possibile accedere a un'istanza di \projectname{} tramite username e password.
		\end{itemize}

	\level{3}*{TA5.4} L'utente client deve verificare che le API esterne permettano di scollegarsi  da un'istanza di \projectname{}.
		
		All'utente client è richiesto di
		\begin{itemize}
			\item autenticarsi ad una istanza di \projectname{} esistente;
			\item verificare che, tramite le funzionalità offerte dalle API esterne, sia possibile scollegarsi da un'istanza di \projectname{}.
		\end{itemize}

	\level{3}*{TA6}L'utente client deve verificare che il framework permetta l'inserimento di grafici all'interno di pagine HTML tramite un'apposita libreria .
		
		All'utente client è richiesto di
		\begin{itemize}
			\item autenticarsi ad una istanza di \projectname{} esistente;
			\item verificare che, tramite le funzionalità offerte dalla libreria, sia possibile inserire grafici esistenti all'interno di pagine HTML.
		\end{itemize}

	\level{3}*{TA6.1}L'utente client deve verificare che la libreria permetta l'inserimento di un grafico all'interno di un determinato tag HTML.
		
		All'utente client è richiesto di
		\begin{itemize}
			\item autenticarsi ad una istanza di \projectname{} esistente;
			\item scegliere un grafico dell'istanza di \projectname{};
			\item verificare che, tramite le funzionalità offerte dalla libreria, sia possibile inserire il grafico scelto all'interno di un determinato tag HTML;
			\item ripetere la verifica per ogni grafico.
		\end{itemize}

	\level{3}*{TA6.2}L'utente client deve verificare che la libreria permetta di modificare le principali impostazioni di un grafico.
		
		All'utente client è richiesto di
		\begin{itemize}
			\item autenticarsi ad una istanza di \projectname{} esistente;
			\item scegliere un grafico dell'istanza di \projectname{};
			\item verificare che, tramite le funzionalità offerte dalla libreria, sia possibile modificare le principali impostazioni del grafico;
			\item ripetere la verifica per ogni grafico.
		\end{itemize}

	\level{3}*{TA6.2.1}L'utente client deve verificare che la libreria permetta di modificare la scelta se visualizzare o meno la legenda di un line chart.
		
		All'utente client è richiesto di
		\begin{itemize}
			\item autenticarsi ad una istanza di \projectname{} esistente;
			\item scegliere un grafico di tipo line chart dell'istanza di \projectname{};
			\item verificare che, tramite le funzionalità offerte dalla libreria, sia possibile modificare la scelta se visualizzare o meno la legenda di un line chart.
			\item ripetere la verifica per ogni grafico di tipo line chart.
		\end{itemize}

	\level{3}*{TA6.2.2}L'utente client deve verificare che la libreria permetta di modificare la scelta se visualizzare o meno la legenda di un bar chart.
		
		All'utente client è richiesto di
		\begin{itemize}
			\item autenticarsi ad una istanza di \projectname{} esistente;
			\item scegliere un grafico di tipo bar chart dell'istanza di \projectname{};
			\item verificare che, tramite le funzionalità offerte dalla libreria, sia possibile modificare la scelta se visualizzare o meno la legenda di un bar chart.
			\item ripetere la verifica per ogni grafico di tipo bar chart.
		\end{itemize}

	\level{3}*{TA6.2.3}L'utente client deve verificare che la libreria permetta di modificare la scelta se visualizzare o meno la legenda di un map chart.
		
		All'utente client è richiesto di
		\begin{itemize}
			\item autenticarsi ad una istanza di \projectname{} esistente;
			\item scegliere un grafico di tipo map chart dell'istanza di \projectname{};
			\item verificare che, tramite le funzionalità offerte dalla libreria, sia possibile modificare la scelta se visualizzare o meno la legenda di un map chart.
			\item ripetere la verifica per ogni grafico di tipo map chart.
		\end{itemize}

	\level{3}*{TA6.2.4}L'utente client deve verificare che la libreria permetta di modificare la posizione in cui è visualizzata la legenda di un line chart.
		
		All'utente client è richiesto di
		\begin{itemize}
			\item autenticarsi ad un'istanza di \projectname{} esistente;
			\item scegliere un grafico di tipo line chart dell'istanza di \projectname{};
			\item verificare che, tramite le funzionalità offerte dalla libreria, sia possibile modificare la posizione in cui è visualizzata la legenda di un line chart.
			\item ripetere la verifica per ogni grafico di tipo line chart.
		\end{itemize}

	\level{3}*{TA6.2.5}L'utente client deve verificare che la libreria permetta di modificare la posizione in cui è visualizzata la legenda di un bar chart.
		
		All'utente client è richiesto di
		\begin{itemize}
			\item autenticarsi ad un'istanza di \projectname{} esistente;
			\item scegliere un grafico di tipo bar chart dell'istanza di \projectname{};
			\item verificare che, tramite le funzionalità offerte dalla libreria, sia possibile modificare la posizione in cui è visualizzata la legenda di un bar chart.
			\item ripetere la verifica per ogni grafico di tipo bar chart.
		\end{itemize}

	\level{3}*{TA6.2.6}L'utente client deve verificare che la libreria permetta di modificare la posizione in cui è visualizzata la legenda di un map chart.
		
		All'utente client è richiesto di
		\begin{itemize}
			\item autenticarsi ad un'istanza di \projectname{} esistente;
			\item scegliere un grafico di tipo map chart dell'istanza di \projectname{};
			\item verificare che, tramite le funzionalità offerte dalla libreria, sia possibile modificare la posizione in cui è visualizzata la legenda di un map chart.
			\item ripetere la verifica per ogni grafico di tipo map chart.
		\end{itemize}

	\level{3}*{TA6.2.7}L'utente client deve verificare che la libreria permetta di modificare la scelta se visualizzare o meno la griglia del piano cartesiano di un line chart.
		
		All'utente client è richiesto di
		\begin{itemize}
			\item autenticarsi ad un'istanza di \projectname{} esistente;
			\item scegliere un grafico di tipo line chart dell'istanza di \projectname{};
			\item verificare che, tramite le funzionalità offerte dalla libreria, sia possibile modificare la scelta se visualizzare o meno la griglia del piano cartesiano di un line chart.
			\item ripetere la verifica per ogni grafico di tipo line chart.
		\end{itemize}

	\level{3}*{TA6.2.8}L'utente client deve verificare che la libreria permetta di modificare la scelta se visualizzare o meno la griglia del piano cartesiano di un bar chart.
		
		All'utente client è richiesto di
		\begin{itemize}
			\item autenticarsi ad un'istanza di \projectname{} esistente;
			\item scegliere un grafico di tipo bar chart dell'istanza di \projectname{};
			\item verificare che, tramite le funzionalità offerte dalla libreria, sia possibile modificare la scelta se visualizzare o meno la griglia del piano cartesiano di un bar chart.
			\item ripetere la verifica per ogni grafico di tipo bar chart.
		\end{itemize}

	\level{3}*{TA6.2.9}L'utente client deve verificare che la libreria permetta di modificare il colore di ciascun set di dati di un line chart.
		
		All'utente client è richiesto di
		\begin{itemize}
			\item autenticarsi ad un'istanza di \projectname{} esistente;
			\item scegliere un grafico di tipo line chart dell'istanza di \projectname{};
			\item verificare che, tramite le funzionalità offerte dalla libreria, sia possibile modificare il colore di ciascun set di dati di un line chart.
			\item ripetere la verifica per ogni grafico di tipo line chart.
		\end{itemize}

	\level{3}*{TA6.2.10}L'utente client deve verificare che la libreria permetta di modificare il colore di ciascun set di dati di un bar chart.
		
		All'utente client è richiesto di
		\begin{itemize}
			\item autenticarsi ad un'istanza di \projectname{} esistente;
			\item scegliere un grafico di tipo bar chart dell'istanza di \projectname{};
			\item verificare che, tramite le funzionalità offerte dalla libreria, sia possibile modificare il colore di ciascun set di dati di un bar chart.
			\item ripetere la verifica per ogni grafico di tipo line chart.
		\end{itemize}

	\level{3}*{TA6.2.11}L'utente client deve verificare che la libreria permetta di modificare il colore di ciascun set di dati di un map chart.
		
		All'utente client è richiesto di
		\begin{itemize}
			\item autenticarsi ad un'istanza di \projectname{} esistente;
			\item scegliere un grafico di tipo map chart dell'istanza di \projectname{};
			\item verificare che, tramite le funzionalità offerte dalla libreria, sia possibile modificare il colore di ciascun set di dati di un map chart.
			\item ripetere la verifica per ogni grafico di tipo map chart.
		\end{itemize}

	\level{3}*{TA6.2.12}L'utente client deve verificare che la libreria permetta di modificare il colore di sfondo di ogni cella per la table.
		
		All'utente client è richiesto di
		\begin{itemize}
			\item autenticarsi ad un'istanza di \projectname{} esistente;
			\item scegliere un grafico di tipo table dell'istanza di \projectname{};
			\item verificare che, tramite le funzionalità offerte dalla libreria, sia possibile modificare il colore di sfondo di ogni cella per la table.
			\item ripetere la verifica per ogni grafico di tipo table.
		\end{itemize}

	\level{3}*{TA6.2.13}L'utente client deve verificare che la libreria permetta di modificare il colore del testo contenuto in ogni cella per la table.
		
		All'utente client è richiesto di
		\begin{itemize}
			\item autenticarsi ad un'istanza di \projectname{} esistente;
			\item scegliere un grafico di tipo table dell'istanza di \projectname{};
			\item verificare che, tramite le funzionalità offerte dalla libreria, sia possibile modificare il colore del testo contenuto in ogni cella per la table.
			\item ripetere la verifica per ogni grafico di tipo table.
		\end{itemize}

	\level{3}*{TA6.3}L'utente client deve verificare che la libreria Chuck permetta di accedere a un'istanza di \projectname{} tramite username e password.
		
		All'utente client è richiesto di
		\begin{itemize}
			\item verificare che, tramite le funzionalità offerte dalla libreria Chuck, sia possibile raggiungere un'istanza di \projectname{} tramite username e password.
		\end{itemize}

	\level{3}*{TA6.4}L'utente client deve verificare che la libreria Chuck permetta di scollegarsi da un'istanza di \projectname{}.

		All'utente client è richiesto di
		\begin{itemize}
			\item autenticarsi ad un'istanza di \projectname{} esistente;
			\item verificare che, tramite le funzionalità offerte dalla libreria Chuck, sia possibile scollegarsi dall'istanza di \projectname{}.
		\end{itemize}

	\level{3}*{TA7}L'utente deve verificare che la dashboard permetta di visualizzare in tempo reali gli autobus dell'APS.

		All'utente è richiesto di
		\begin{itemize}
			\item andare all'indirizzo della dashboard;
			\item verificare che vengano mostrati gli autobus dell'APS;
			\item verificare che gli aggiornamenti avvengano in tempo reale, senza dover aggiornare la pagina.
		\end{itemize}

	\level{3}*{TA7.1}L'utente deve verificare che la dashboard permetta di visualizzare il numero di autobus attivi per ciascuna linea.

		All'utente è richiesto di
		\begin{itemize}
			\item andare all'indirizzo della dashboard;
			\item verificare che vengano mostrati il numero di autobus attivi per ciascuna linea.
		\end{itemize}

	\level{3}*{TA7.2}L'utente deve verificare che la dashboard permetta di filtrare gli autobus per linea di appartenenza e visualizzarli di conseguenza.

		All'utente è richiesto di
		\begin{itemize}
			\item andare all'indirizzo della dashboard;
			\item selezionare le linee che vogliono visualizzare;
			\item verificare che vengano mostrati solo gli autobus delle linee selezionate;
		\end{itemize}

	\level{3}*{TA8}L'utente deve verificare che l'applicazione Android permetta di visualizzare dei grafici.

		All'utente è richiesto di
		\begin{itemize}
			\item installare l'apk;
			\item accedere ad una istanza di \projectname{};
			\item verificare che sia possibile visualizzare i grafici.
		\end{itemize}

	\level{3}*{TA8.1}L'utente deve verificare che l'applicazione Android permetta di accedere a un'istanza di \projectname{} tramite username e password.

		All'utente è richiesto di
		\begin{itemize}
			\item avviare l'applicazione;
			\item inserire l'indirizzo di un'istanza di \projectname{};
			\item inserire le credenziali di autenticazione;
			\item verificare che sia visualizzata la lista dei grafici presenti nell'istanza di \projectname{} scelta.
		\end{itemize}

	\level{3}*{TA8.2}L'utente deve verificare che l'applicazione Android permetta di visualizzare l'elenco dei grafici esistenti nell'istanza \projectname{} con relativo ID, titolo, tipo e descrizione.

		All'utente è richiesto di
		\begin{itemize}
			\item avviare l'applicazione;
			\item inserire un indirizzo valido di un'istanza di \projectname{} e se necessario autenticarsi;
			\item verificare che sia visualizzabile l'elenco dei grafici esistenti nell'istanza \projectname{} con relativo ID, titolo, tipo e descrizione.
		\end{itemize}

	\level{3}*{TA8.3}L'utente deve verificare che l'applicazione Android permetta di selezionare e visualizzare un singolo grafico dell'istanza di \projectname{}.

		All'utente è richiesto di
		\begin{itemize}
			\item avviare l'applicazione;
			\item inserire un indirizzo valido di un'istanza di \projectname{} ed autenticarsi se necessario;
			\item selezionare un grafico dalla lista di grafici presenti;
			\item verificare visivamente la presenza di un singolo grafico e che quel grafico sia quello selezionato.
		\end{itemize}

	\level{3}*{TA8.4}L'utente deve verificare che l'applicazione Android mostri un errore quando viene immesso un indirizzo per un'istanza di \projectname{} non valido.

		All'utente è richiesto di
		\begin{itemize}
			\item avviare l'applicazione;
			\item inserire un indirizzo per un'istanza di \projectname{} non valido;
			\item verificare che venga segnalato un errore;
		\end{itemize}

	\level{3}*{TA8.5}L'utente deve verificare che l'applicazione Android mostri un errore quando vengono immessi dati di accesso (username e password) per un'istanza di \projectname{} non validi.

		All'utente è richiesto di
		\begin{itemize}
			\item avviare l'applicazione;
			\item inserire l'indirizzo di un'istanza di \projectname{};
			\item autenticarsi con credenziali errate;
			\item verificare che venga segnalato un errore di autenticazione.
		\end{itemize}

\level{1}{Resoconto delle varie attività di verifica - Fase P}
	Sono riportati in questa appendice tutti i risultati ottenuti nei momenti di verifica, stabiliti nel \insdoc{Piano di Progetto v4.00} secondo la strategia di misurazione per il perseguimento della qualità individuata nel presente documento. Ove fosse necessario, sono state tratte anche delle conclusioni sui risultati ottenuti e su come essi possano essere migliorati.
	\level{2}{Verifica dei prodotti}
		\level{3}{Documenti}
			In questa sezione vengono riportati i risultati delle attività di verifica svolte sui documenti. Esse sono di due tipi:
			\begin{itemize}
				\item verifiche manuali;
				\item verifiche automatizzate.
			\end{itemize}
			\level{4}{Verifiche manuali}
				Le attività di verifica manuale della documentazione prodotta sono state svolte in base alla procedura riguardante la verifica dei documenti che è descritta nel documento \insdoc{Norme di Progetto v4.00}.\\
				La verifica manuale ha permesso di individuare soprattutto errori riguardanti le seguenti tipologie:
				\begin{itemize}
					\item descrizioni imprecise di classi e metodi;
					\item errori concettuali;
					\item errori nell'utilizzo della lingua inglese.
				\end{itemize}
				Si riporta di seguito la quantità degli errori rilevati e risolti, per ciascuna tipologia, durante l'intera fase.
				\begin{table}[H]
					\centering
						\begin{tabu}{| l | c |}
							\hline
								Descrizioni imprecise	&	20\\ \hline
								Errori concettuali	&	13\\ \hline
								Errori di inglese  &  34\\ \hline
						\end{tabu}
						\caption{Errori trovati tramite verifica manuale dei documenti durante la Fase P}
				\end{table}
				La verifica manuale dei diagrammi UML, effettuata per controllare la correttezza dei metodi e dei campi dati inseriti per procedere con la stesura della \insdoc{Definizione di Prodotto v1.00}, è risultata piuttosto onerosa, ma poco produttiva, in quanto i \insrole{Progettisti} hanno posto molta attenzione nell'attività di progettazione di dettaglio.\\
				Si può inoltre notare come gli errori concettuali che sono stati trovati siano in leggera diminuzione, segno del fatto che i documenti si stanno piano piano avvicinando al loro contenuto ottimale.\\
				Si noti invece come vi sia una notevole quantità di errori nell'uso della lingua inglese, che nelle altre fasi non c'erano. Questo è dovuto al fatto che è stata cominciata la stesura del manuale, ed esso è scritto in lingua inglese.
			\level{4}{Verifiche automatizzate}
				Le attività di verifica automatizzate sono state effettuate secondo le procedure e attraverso gli strumenti descritti nel documento \insdoc{Norme di Progetto v4.00}. Esse hanno permesso di rilevare diversi errori riguardanti le seguenti tipologie:
				\begin{itemize}
					\item ortografia errata;
					\item utilizzo errato dei comandi \LaTeX{} indicati nelle \insdoc{Norme di Progetto v4.00};
					\item norme tipografiche non rispettate.
				\end{itemize}
				Di seguito è presentato un riassunto della quantità di errori trovati (e successivamente risolti) utilizzando la verifica automatica.
				\begin{table}[H]
					\centering
						\begin{tabu}{| l | c |}
							\hline
							Errori ortografici	& 78	\\ \hline
							Utilizzo errato \LaTeX{}	& 2	\\ \hline
							Errori riguardanti norme tipografiche	& 12	\\ \hline
						\end{tabu}
					\caption{Errori trovati tramite verifica automatica dei documenti durante la Fase P}
				\end{table}
				Come si può notare, gli errori ortografici sono diminuti rispetto alla fase precedente. Tuttavia, questo è anche imputabile alla minor mole di documentazione che è stata prodotta durante questa fase.\\
				Per quanto riguarda le altre due tipologie di errori, si può notare una certa stabilità rispetto alla fase precedente.\\
				Si riportano i risultati delle misurazioni dell'indice di leggibilità Gulpease relative ai documenti modificati in questa fase.
				\begin{table}[H]
					\centering
						\begin{tabu}{| l | c | c |}
							\hline
							Documenti 							& Gulpease	& Esito		\\ \hline \hline
							Piano di progetto v4.00				& 89 		& Superato  \\ \hline
							Norme di Progetto v4.00 			& 68		& Superato  \\ \hline
							Piano di Qualifica v4.00 			& 74		& Superato  \\ \hline
							Definizione di prodotto v1.00		& 84		& Superato \\ \hline
							Glossario v4.00					 	& 68 		& Superato  \\ \hline
						\end{tabu}
					\caption{Esiti del calcolo dell'indice di leggibilità effettuato tramite strumenti automatici durante la Fase P}
				\end{table}
				Si noti che non è stato calcolato l'indice Gulpease relativo al \insdoc{Manuale Utente v1.00} in quanto esso è scritto in lingua inglese e tale indice è relativo alla sola lingua italiana.
		\level{3}{Codice}
			In questa sezione sono riportate, per le parti del progetto implementate in questa prima fase di codifica, i risultati delle metriche calcolate nei momenti di verifica.
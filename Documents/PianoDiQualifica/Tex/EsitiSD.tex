% !TEX encoding = UTF-8 Unicode
\level{1}{Resoconto delle varie attività di verifica - Fase SD} \label{app:esiti}
Anche durante questa terza fase, secondo quanto riportato dal \insdoc{Piano di Progetto}, sono previsti più momenti in cui viene attivato il processo di verifica. Si riportano in questa sezione tutti i risultati ottenuti durante questi momenti. Ove fosse necessario, sono state tratte anche delle conclusioni sui risultati ottenuti e su come essi possano essere migliorati.
	\level{2}{Verifica dei prodotti}
		\level{3}{Documenti}
			In questa sezione vengono riportati i risultati delle attività di verifica svolte sui documenti. Esse sono di due tipi:
			\begin{itemize}
				\item verifiche manuali;
				\item verifiche automatizzate.
			\end{itemize}
			\level{4}{Verifiche manuali}
				Le attività di verifica manuale della documentazione prodotta sono state svolte in base alla procedura riguardante la verifica dei 
				documenti che è descritta del documento \insdoc{Norme di Progetto}.\\
				La verifica manuale ha permesso di individuare soprattutto errori che riguardano le seguenti tipologie:
				\begin{itemize}
					\item descrizioni troppo sommarie laddove sono richieste descrizioni accurate e dettagliate;
					\item incongruenze tra parti diverse dello stesso documento o appartenenti a documenti diversi;
					\item errori nei concetti esposti.
				\end{itemize}
				Di seguito è presentato un riassunto della quantità di errori trovati (e successivamente risolti) utilizzando la verifica manuale durante l'intera \insphase{Fase SD}.
				\begin{table}[H]
					\centering
					\begin{tabu}{| l | c |}
						\hline
						Descrizioni sommarie	&	26\\ \hline
						Incongruenze	&	21\\ \hline
						Errori concettuali	&	12\\ \hline
					\end{tabu}
					\caption{Errori trovati tramite verifica manuale dei documenti durante la Fase SD}
				\end{table}
				Rispetto alla \insphase{Fase DB}, è stato riscontrato un aumento delle incongruenze tra i vari documenti. Ciò è dovuto al fatto che i documenti sono stati corretti in seguito alla prima revisione. Inoltre, in seguito ad alcuni incontri con il proponente durante i quali si è discusso sulla progettazione, si è reso necessario apportare alcune modifiche ai casi d'uso e ai requisiti.\\
				Durante la verifica manuale sono stati individuati nuovi termini da aggiungere al \insdoc{Glossario}. Di seguito è presentato un 
				riassunto della quantità di nuovi termini da aggiungere al \insdoc{Glossario} che sono stati individuati.
				\begin{table}[H]
					\centering
					\begin{tabu}{| l | c |}
						\hline
						Termini candidati ad essere aggiunti	&	12\\ \hline
						Termini aggiunti al \insdoc{Glossario}	& 10\\ \hline
					\end{tabu}
					\caption{Nuovi termini da inserire nel Glossario individuati tramite verifica manuale dei documenti durante la Fase SD}
				\end{table}
				È stata infine verificata la correttezza dei diagrammi UML utilizzati all'interno dei vari documenti, sempre seguendo le procedure contenute nel documento \insdoc{Norme di Progetto}. Per quanto riguarda i diagrammi delle componenti non sono stati riscrontrati grossi problemi. Sì è rivelata più critica la stesura dei diagrammi di attività inerenti le modalità di interezione tra le varie componenti.
				
			\level{4}{Verifiche automatizzate}
			Le attività di verifica automatizzate, oltre a rispettare le procedure descritte all'interno delle \insdoc{Norme di Progetto}, fanno uso degli strumenti automatici previsti all'interno dello stesso documento. Questi hanno permesso di individuare numerosi errori che riguardano le seguenti tipologie:
			\begin{itemize}
				\item ortografia errata;
				\item utilizzo errato dei comandi \LaTeX{} previsti dalle \insdoc{Norme di Progetto};
				\item norme tipografiche non rispettate.
			\end{itemize}
			Di seguito è presentato un riassunto della quantità di errori trovati (e successivamente risolti) utilizzando la verifica automatica.
			\begin{table}[H]
					\centering
					\begin{tabu}{| l | c |}
						\hline
						Errori ortografici	&145	\\ \hline
						Utilizzo errato \LaTeX{}	&3	\\ \hline
						Errori riguardanti norme tipografiche	&11	\\ \hline
					\end{tabu}
					\caption{Errori trovati tramite verifica automatica dei documenti durante la Fase SD}
				\end{table}
				Come si può notare, gli errori di utilizzo errato dei comandi \LaTeX{} e gli errori relativi le norme tipografiche sono diminuiti rispetto alla \insphase{Fase DB}. Ciò è probabilmente dovuto ad una maggiore familiarità del team con le \insdoc{Norme di Progetto}.\\
				Di seguito riportiamo gli indici ottenuti dal calcolo dell'indice di leggibilità sui documenti completi.
				\begin{table}[H]
					\centering
					\begin{tabu}{| l | c | c |}
							\hline
							Documenti 							& Gulpease	& Esito		\\ \hline \hline
							
							Piano di progetto v1.00				& -- 		& Superato  \\ \hline
							Norme di Progetto v1.00 			& --		& Superato  \\ \hline
							Studio di Fattibilità v1.00 	& --		& Superato  \\ \hline
							Analisi dei Requisiti v1.00	 	& --		& Superato  \\ \hline
							Piano di Qualifica v1.00 			& --		& Superato  \\ \hline
							Glossario v1.00					 	& -- 		& Superato  \\ \hline
						\end{tabu}
					\caption{Esiti del calcolo dell'indice di leggibilità effettuato tramite strumenti automatici durante la Fase SD}
				\end{table}

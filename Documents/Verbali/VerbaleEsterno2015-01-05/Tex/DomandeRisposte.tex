% !TEX encoding = UTF-8 Unicode
\section{Domande e risposte}
	Vengono riportate le domande effettuate dal gruppo \groupname{} e le relative risposte date dal proponente \proponente\, nel corso della riunione avvenuta in data \insdate{05}{01}{2015}.\\

	\textbf{Dobbiamo fornire delle API tali per cui una volta che il client ha ricevuto i dati, in formato JSON, li rappresenta graficamente sul browser?}\\
	\textit{Si, vi consiglio di creare delle funzioni che trasformano il formato dei dati inviati dal server nel formato accettato dalle librerie grafiche che userete, per esempio chart.js. Ancora meglio: visto che il formato dovete sceglierlo voi, vi conviene utilizzare sempre lo stesso formato, sia per lo scambio che per la visualizzazione.}\\

	\textbf{Per quanto riguarda l'applicazione Android, bisogna restituire la lista di tutti i grafici presenti nel server o la lista di tutti i grafici presenti in una singola pagina?}\\
	\textit{Dipende da come vi siete immaginati l'applicazione Android. Si era pensato che immettendo l'indirizzo di un server o l'indirizzo di una istanza di \projectname{}, l'applicazione ricevesse tutte le informazioni che servono, in formato JSON.}\\

	\textbf{Ha senso permettere la modifica dei grafici, come per esempio la personalizzazione dello stile dei grafici, lato client?}\\
	\textit{Ha senso, ma non è richiesto. Il fatto di avere un widget, separa ulteriormente la parte sever dalla parte client, perchè se poi dovete modificare i dati nel client, occorre avere delle regole che funzionano per conto loro sul client, che non è detto vengano trasmesse dal server. \`{E} possibile implementare la prima chiamata html ad alto livello. Per esempio: un utente chiede una sola volta al server le regole di reindirizzazione, e quando queste sono state trasmesse e sono a regime client e server possono scambiarsi liberamente dati. \`{E} possibile che voi abbiate già le regole lato server e cominciate a trasmettere con il server di \projectname{}, in questo caso il client risulta molto più indipendente, quindi server e client riusltano separati. Una volta che si ha la parte di visualizzazione separata dalla parte dei dati, un modo di fare il widget è scrivere un codice JavaScript che carica in maniera sincrona dei file e, una volta caricati , modifica il dom mettendolo in un div. Questo div, configurato dall'utilizzatore di Norris, crea i grafici che l'utente finale potrà modificare come vuole. Il front-end classico, richiesto da specifica, fa questo in automatico, ma implementando un widget, permettete che venga usato anche da altri parti.}\\
% !TEX encoding = UTF-8 Unicode

\section{Domande e risposte}
Vengono riportate le domande effettuate dal gruppo \groupname{} e le relative risposte date dal proponente \proponente{}, nel corso della riunione avvenuta in data \insdate{22}{12}{2014}.\\

	\textbf{Per quanto riguarda la composizione dei grafici in pagine, dobbiamo fornire una pagina html completa o una parte di codice html che l'utilizzatore finale potrà inserire in un'altra pagina?}\\
	\textit{Si era pensato alla prima opzione: creare una pagina html in cui sono visualizzati i soli grafici. Tuttavia è una buona idea anche creare una sorta di ``widget'' di \projectname{}.}\\
	
	\textbf{\`{E} considerato utile poter permettere il salvataggio dei grafici?}\\
	\textit{Si. Può essere comodo avere a disposizione un bottone che scarica il pdf o l'immagine in formato jpg o png dei grafici.}\\

	\textbf{Nel caso i dati siano troppi come si dovrà comportare \projectname{}?}\\
	\textit{Per quanto riguarda la quantità dei dati, voi decidete i limiti e delegate all'utilizzatore di \projectname{} la responsabilità di decidere il comportamento nelle diverse situazioni che si possono presentare, mettendo a disposizione opzioni differenti.}\\
	
	\textbf{Nel capitolato, per ogni tipo di grafico, si parla di \textit{formato di stampa dei valori}: che cosa si intende?}\\
	\textit{Per \textit{formato di stampa dei valori} si intende come vengono visualizzati i dati dei grafici nel browser.}\\
	
	\textbf{Sarebbe utile impostare dei filtri, lato client, sui dati?}\\
	\textit{Si, sarebbe molto utile ma anche molto difficile, perché si dovrebbero fare assunzioni a priori sui dati.}\\
	
	\textbf{Per quanto riguarda l'applicazione Android, essa è un altro front-end?}\\
	\textit{Si, viene usata la stessa API. L'applicazione Android non nasce dal server, quindi può avere meno informazioni. In caso si usi l'interfaccia web per Agenda, bisogna far sapere dov'è l'istanza di \projectname{} e occorre prevedere una chiamata API che ritorni l'elenco di tutti i grafici. Inoltre deve servire per tutte le istanze di Norris, per cui la sorgente dei dati deve essere impostata dall'utilizzatore di \projectname{}.}\\

	

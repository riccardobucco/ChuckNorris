\level{1}{Schemi JSON} \label{app:schemi}
    In tale appendice vengono riportati i vari formati che sono assunti dagli oggetti JSON che sono presenti all'interno del progetto. Fare riferimento ad essi per creare e interpretare correttamente i vari pacchetti utilizzati dai metodi delle varie classi descritte all'interno del presente documento.\\
    Per ogni tipologia di grafico (bar chart, line chart, map chart e table) sono qui presentati i formati dei pacchetti necessari durante le seguenti situazioni:
    \begin{itemize}
        \item inserimento di dati durante l'inizializzazione del grafico;
        \item inserimento di impostazioni durante l'inizializzazione del grafico;
        \item inserimento di nuovi dati durante uno dei vari aggiornamenti a cui il grafico è soggetto.
    \end{itemize}
    \level{2}{BarChart}
        In tale sezione viene riportato il formato che devono rispettare i pacchetti JSON quando vengono utilizzati all'interno di un bar chart.
        \level{3}{Dati}
            Qui si fa riferimento al formato del pacchetto JSON che deve essere usato durante l'inizializzazione dei dati di un bar chart.
            \lstinputlisting[language=json]{DefinizioneDiProdotto/Schemi/BarChartData.json}
        \level{3}{Impostazioni}
            Qui si fa riferimento al formato del pacchetto JSON che deve essere usato durante l'inizializzazione dei delle impostazioni di un bar chart.
            \lstinputlisting[language=json]{DefinizioneDiProdotto/Schemi/BarChartSettings.json}
        \level{3}{Aggiornamenti}
            Qui si fa riferimento al formato del pacchetto JSON che deve essere usato durante uno dei vari aggiornamenti a cui può essere sottoposto un bar chart (inplace).
            \level{4}{InPlace}
                \lstinputlisting[language=json]{DefinizioneDiProdotto/Schemi/BarChartInPlaceUpdate.json}

    \level{2}{LineChart}
        In tale sezione viene riportato il formato che devono rispettare i pacchetti JSON quando vengono utilizzati all'interno di un line chart.
        \level{3}{Dati}
            Qui si fa riferimento al formato del pacchetto JSON che deve essere usato durante l'inizializzazione dei dati di un line chart.
            \lstinputlisting[language=json]{DefinizioneDiProdotto/Schemi/LineChartData.json}
        \level{3}{Impostazioni}
            Qui si fa riferimento al formato del pacchetto JSON che deve essere usato durante l'inizializzazione dei delle impostazioni di un line chart.
            \lstinputlisting[language=json]{DefinizioneDiProdotto/Schemi/LineChartSettings.json}
        \level{3}{Aggiornamenti}
            Qui si fa riferimento al formato del pacchetto JSON che deve essere usato durante uno dei vari aggiornamenti a cui può essere sottoposto un line chart (inplace e stream).
            \level{4}{InPlace}
                \lstinputlisting[language=json]{DefinizioneDiProdotto/Schemi/LineChartInPlaceUpdate.json}
            \level{4}{Stream}
                \lstinputlisting[language=json]{DefinizioneDiProdotto/Schemi/LineChartStreamUpdate.json}

    \level{2}{MapChart}
        In tale sezione viene riportato il formato che devono rispettare i pacchetti JSON quando vengono utilizzati all'interno di un map chart.
        \level{3}{Dati}
            Qui si fa riferimento al formato del pacchetto JSON che deve essere usato durante l'inizializzazione dei dati di un map chart.
            \lstinputlisting[language=json]{DefinizioneDiProdotto/Schemi/MapChartData.json}
        \level{3}{Impostazioni}
            Qui si fa riferimento al formato del pacchetto JSON che deve essere usato durante l'inizializzazione dei delle impostazioni di un map chart.
            \lstinputlisting[language=json]{DefinizioneDiProdotto/Schemi/MapChartSettings.json}
        \level{3}{Aggiornamenti}
            Qui si fa riferimento al formato del pacchetto JSON che deve essere usato durante uno dei vari aggiornamenti a cui può essere sottoposto un map chart (inplace e movie).
            \level{4}{InPlace}
                \lstinputlisting[language=json]{DefinizioneDiProdotto/Schemi/MapChartInPlaceUpdate.json}
            \level{4}{Movie}
                \lstinputlisting[language=json]{DefinizioneDiProdotto/Schemi/MapChartMovieUpdate.json}

    \level{2}{Table}
        In tale sezione viene riportato il formato che devono rispettare i pacchetti JSON quando vengono utilizzati all'interno di una table.
        \level{3}{Dati}
            Qui si fa riferimento al formato del pacchetto JSON che deve essere usato durante l'inizializzazione dei dati di una table.
            \lstinputlisting[language=json]{DefinizioneDiProdotto/Schemi/TableData.json}
        \level{3}{Impostazioni}
            Qui si fa riferimento al formato del pacchetto JSON che deve essere usato durante l'inizializzazione dei delle impostazioni di una table.
            \lstinputlisting[language=json]{DefinizioneDiProdotto/Schemi/TableSettings.json}
        \level{3}{Aggiornamenti}
            Qui si fa riferimento al formato del pacchetto JSON che deve essere usato durante uno dei vari aggiornamenti a cui può essere sottoposto una table (inplace e stream).
            \level{4}{InPlace}
                \lstinputlisting[language=json]{DefinizioneDiProdotto/Schemi/TableInPlaceUpdate.json}
            \level{4}{Stream}
                \lstinputlisting[language=json]{DefinizioneDiProdotto/Schemi/TableStreamUpdate.json}
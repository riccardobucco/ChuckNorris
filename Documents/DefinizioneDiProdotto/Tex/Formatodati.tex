% !TEX encoding = UTF-8 Unicode
\level{1}{Richiesta dei dati dall'APS e loro formato} \label{app:aps}

La richiesta dei dati all'APS viene effettuata tramite chiamata HTTP con metodo POST, passando come parametri nell'URL: option, view e format.
Nell'header della richiesta viene specificato che il contenuto è da interpretarsi in formato URL encoded.

\begin{table}[H]
	\centering
	\begin{tabu}{| X[0.5] | X |}
		\hline
		hostname & www.apsholding.it \\ \hline 
		port & 80 \\ \hline
		path & /index.php/informazioni/dov-e-il-mezzo-pubblico-in-tempo-reale?option=com\_mappeaps\&view=posmezzi\&format=raw  \\ \hline
		method & POST \\ \hline 
		content-type & application/x-www-form-urlencoded \\ \hline
	\end{tabu}
	\caption{Richiesta dei dati dall'APS}
\end{table}

La risposta dal server dell'APS si suppone sia un JSON con codifica utf8. I dati ricevuti contengono:

\begin{table}[H]
	\centering
	\begin{tabu}{| X | X | X | X |}
		\hline
		idMezzo & Coordinate GPS & Girometro & StatoPorte \\ \hline
		numero identificativo del mezzo & coordinate GPS della posizione del mezzo & velocità ancgolare & stato delle porte del mezzo \\ \hline
		\end{tabu}
	\caption{Dati ricevuti dall'APS}
\end{table}
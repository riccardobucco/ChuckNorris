\level{1}{Standard di progetto}
	\level{2}{Standard di progettazione architetturale}
	Per gli standard di progettazione architetturale, si fa riferimento al documento \insdoc{Specifica Tecnica v4.00}.

	\level{2}{Standard di documentazione del codice}
	Per le norme da rispettare nella scrittura del codice, si fa riferimento al documento \insdoc{Norme di Progetto v6.00}.

	\level{2}{Standard di denominazione di entità e relazioni}
	Per tutte le entità definite, come \textit{package}, classi, attributi e metodi è necessario fornire denominazioni chiare ed esplicative. Inoltre, è da preferire l'utilizzo di sostantivi per indicare le entità e dei verbi per le relazioni.\\
	Sono ammesse abbreviazioni, solo nei casi in cui esse non siano ambigue e siano immediatamente comprensibili. \\
	Per le regole tipografiche da seguire, relative ai nomi delle entità, si fa riferimento al documento \insdoc{Norme di Progetto v6.00}.

	\level{2}{Standard di programmazione}
	Per gli standard di programmazione, si fa riferimento al documento \insdoc{Norme di Progetto v6.00}.

	\level{2}{Strumenti di lavoro}
	Per gli strumenti di lavoro da utilizzare durante la codifica, si fa rferimento al documento \insdoc{Norme di progetto v6.00}.

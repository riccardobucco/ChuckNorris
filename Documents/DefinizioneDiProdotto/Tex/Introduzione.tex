\level{1}{Introduzione}

	\level{2}{Scopo del documento}
		Il seguente documento ha lo scopo di definire nel dettaglio la struttura del \insglo{prodotto} \groupname{}. \\ A tal proposito vengono descritte dettagliatamente i \insglo{package}, le classe e le interfacce utilizzate nella realizzazione dei vari sottoprodotti, seguiti dal tracciamento tra i requisiti individuati nel documento \insdoc{Analisi dei requisiti v6.00} e le classi individuate.

	\level{2}{Glossario}
	Allo scopo di rendere più semplice la comprensione dei documenti ed evitare eventuali ambiguità, viene allegato il \insdoc{Glossario v6.00}, che contiene la spiegazione della terminologia tecnica e degli acronimi utilizzati. Per facilitare la lettura, i termini presenti all'interno di tale documento saranno marcati da una “G” maiuscola a pedice.



	\level{2}{Riferimenti utili}
		\level{3}{Riferimenti normativi}
			\begin{itemize}
				\item \textbf{Analisi dei Requisiti:} \insdoc{Analisi dei Requisiti v6.00};
				\item \textbf{Specifica Tecnica:} \insdoc{Specifica Tecnica v4.00};
				\item \textbf{Norme di Progetto:} \insdoc{Norme di Progetto v6.00}.
			\end{itemize}
		\level{3}{Riferimenti informativi}
			\begin{itemize}
				\item \textbf{\insglo{UML} Distilled, Martin Fowler, 2004, Pearson (Addison Wesley)};
				\item \textbf{Learning \insglo{UML} 2.0, Kim Hamilton, Russell Miles, O'Reilly, 2006};
				\item \textbf{Diagramma di sequenza:} \\ 
				\insuri{http://www.math.unipd.it/~tullio/IS-1/2014/Dispense/E3a.pdf};
				\item \textbf{Documentazione di \insglo{Node.js}:} \\
				\insuri{https://nodejs.org/api/};
				\item \textbf{Documentazione di \insglo{Express.js}:} \\
				\insuri{http://expressjs.com/4x/api.html};
				\item \textbf{Documentazione di \insglo{Socket.io}:} \\
				\insuri{http://socket.io/docs/\#};
				\item \textbf{Documentazione di \insglo{Chart.js}:} \\
				\insuri{http://www.chartjs.org/docs/};
				\item \textbf{Documentazione di OpenStreetMap:} \\
				\insuri{http://wiki.openstreetmap.org/wiki/API_v0.6};
				\item \textbf{Documentazione di \insglo{Datatables}:} \\
				\insuri{http://datatables.net/manual/index}.
			\end{itemize}
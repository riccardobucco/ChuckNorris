\level{1}{Dashboard}
    Nel documento \insdoc{Specifica Tecnica v5.00} è stato definito in modo generale quali sono i componenti di cui è composta la \insglo{Dashboard} e ne sono stati definiti in modo preciso i ruoli. In esso, inoltre, è stato descritto il modo in cui l'utilizzatore di \insglo{Norris} deve procedere per creare e rendere disponibile la \insglo{Dashboard}.\\
    I task che lo sviluppatore deve eseguire durante la creazione della \insglo{Dashboard} sono talvolta complessi, nel senso che alcuni di essi implicano operazioni ulteriori rispetto al semplice uso dell' \insglo{API} interne fornite da \insglo{Norris}. Si è dunque deciso di suddividere l'intera logica di creazione e aggiornamento dei grafici e della pagina che andranno a formare la \insglo{Dashboard} in diverse funzioni. È stato fatto questo perché in tale modo il codice è più modulare e comprensibile.
    \level{2}{Funzionalità}
        Nel presente paragrafo vengono presentate le parti nelle quali è stato suddiviso il codice necessario alla creazione e all'aggiornamento della \insglo{Dashboard}. Tali parti fanno riferimento ai task che sono stati presentati e descritti nel documento \insdoc{Specifica Tecnica v5.00}.\\
        All'appendice \nameref{app:aps} si possono trovare le specifiche con le quali vengono effettuate le richieste al \insglo{server} \insglo{APS}. In tale appendice è riportato anche il formato della risposta che si ottiene, oltre ai criteri per interpretarlo.

        \begin{itemize}

            \item \textbf{Inizializzazione di \insglo{Norris}} \\
            Innanzitutto è necessario creare ed inizializzare l'istanza di \insglo{Norris}. Richiamando gli opportuni metodi, nel webserver viene creata l'istanza vera e propria e ne vengono settate le impostazioni. Ne viene infine montato il \insglo{middleware} con express.
            
            \item \textbf{Inizializzazione della mappa} \\
            Si procede alla creazione ed inizializzazione di un \insglo{map chart} nell'istanza di \insglo{Norris}. Il chart non conterrà dati (che saranno inseriti in seguito, tramite aggiornamenti). Vengono inoltre impostate alcune opzioni di default. 
            
           % \item \textbf{Inizializzazione del \insglo{bar chart}} \\
           % Si procede alla creazione ed inizializzazione di un \insglo{bar chart} nell'istanza di \insglo{Norris}. Il chart non conterrà dati (che saranno inseriti in seguito, tramite aggiornamenti). Vengono inoltre impostate alcune opzioni di default.
                        
            \item \textbf{Inizializzazione della tabella} \\
            Si procede alla creazione ed inizializzazione di una \insglo{table} nell'istanza di \insglo{Norris}. Il chart non conterrà dati (che saranno inseriti in seguito, tramite aggiornamenti). Vengono inoltre impostate alcune opzioni di default.
            
            \item \textbf{Creazione della pagina web} \\
            Una volta creati i grafici, si può procedere con la creazione ed inizializzazione di una pagina all'interno dell'istanza di \insglo{Norris}. I grafici vengono aggiunti alla pagina appena creata; vengono inoltre impostate alcune opzioni di default per la pagina.
            
            \item \textbf{Ottenere i dati degli autobus dell'\insglo{APS}} \\
            È ora necessario ottenere tutti i dati riguardanti le linee che il \insglo{server} \insglo{APS} mette a disposizione. Tali dati devono essere correttamente interpretati e convertiti prima di essere usati tramite le \insglo{API} di \insglo{Norris}. Il formato con il quale sono restituiti i dati da tale funzione dei dati provenienti dal sito dell'\insglo{APS} sono descritti nell'appendice \nameref{app:aps}.
            
            \item \textbf{Aggiornare la mappa} \\
            Per aggiornare il \insglo{map chart} si utilizza l'apposito metodo fornito dalle \insglo{API} di \insglo{Norris}, al quale vengono passati i dati nel formato corretto e il tipo di aggiornamento che si desidera effettuare. Scegliendo di aggiornare la mappa con il metodo \insglo{movie}, vengono aggiornate le posizioni di tutti i punti sulla mappa, vengono aggiunti eventuali nuovi punti e vengono tolti quelli non più presenti.
            
            %\item \textbf{Aggiornare il \insglo{bar chart}} \\
            %Per aggiornare il \insglo{bar chart} si utilizza l'apposito metodo fornito dalle \insglo{API} di \insglo{Norris}, al quale vengono passati il tipo di aggiornamento che si desidera effettuare e i dati nel formato corretto. In particolare, si deve calcolare quanti mezzi sono attivi per ciascuna linea. A partire poi dal risultato ottenuto viene creato il pacchetto di aggiornamento tramite il quale le \insglo{API} interne possono effettuare l'aggiornamento vero e proprio del grafico.
            
            \item \ignoreglo{\textbf{Aggiornare la tabella}} \\
            Per aggiornare il \insglo{bar chart} si utilizza l'apposito metodo fornito dalle \insglo{API} di \insglo{Norris}, al quale vengono passati il tipo di aggiornamento che si desidera effettuare e i dati nel formato corretto. In particolare, si deve calcolare quanti mezzi sono attivi per ciascuna linea. A partire poi dal risultato ottenuto viene creato il pacchetto di aggiornamento tramite il quale le \insglo{API} interne possono effettuare l'aggiornamento vero e proprio del grafico.

        \end{itemize}

\level{1}{Dashboard}

    \level{2}{Funzioni individuate}
        La dashboard utilizza una programmazione di tipo imperativo. Per rendere il codice più modulare, si è deciso di suddividere la logica in diverse funzioni. Quì di seguito è possibile trovare le funzioni individuate, insieme ad una descrizione.

        \begin{description}
            \item \textbf{init\_norris() : Norris}
            \item \textbf{init\_map(norris : Norris) : Chart}
            \item \textbf{init\_bar(norris : Norris) : Chart}
            \item \textbf{init\_table(norris : Norris) : Chart}
            \item \textbf{init\_page(charts : Chart[]) : Page}
            \item \textbf{getLines() : Dataset}
            \item \textbf{update\_map(chart : Chart, old\_values : Dataset, new\_values : Dataset) : void}
            \item \textbf{update\_bar(chart : Chart, old\_values : Dataset, new\_values : Dataset) : void}
            \item \textbf{update\_table(chart : Chart, old\_values : Dataset, new\_values : Dataset) : void}
        \end{description}
\level{1}{Description of Norris' External APIs} \label{sec:ExternalAPI}
    This section describes the functionality provided by the External APIs of Norris. Through these APIs it's possible to request a chart to an instance of Norris, that will be automatically updated. You can render the chart with a library of your choice.\\
    In this section we refer to \texttt{[endpoint]} as the Internal API given endpoint.

    \level{2}{Chart Request API}
        The Chart Request API provides the functionalities to get a chart. The API is available over WebSocket. Once opened a WebSocket, you will get both the chart and its own updates until you close the connection. \\
        Request details:
        \begin{enumerate}
            \item Open the WebSocket at \texttt{[endpoint]/chart} using \texttt{/[chartid]} as namespace;
            \item You will get a \texttt{chart} event with chart type, chart settings and chart data as parameters;
            \item For each update you will get an \texttt{update} event with update type and update data as parameters.
        \end{enumerate}

    \level{2}{List Request API}
        The List Request API provides the list of all charts contained in a Norris instance. The API is available over HTTP and you will get the list of the charts with the following details:
        \begin{itemize}
            \item Chart id;
            \item Chart type;
            \item Chart description.
        \end{itemize}
        Request details:
        \begin{table}[H]
            \centering
            \begin{tabu}{|l|l|}
                \hline
                \textbf{Host} & your host \\ \hline
                \textbf{Path} & [endpoint]/list \\ \hline
                \textbf{Method} & GET \\ \hline
                \textbf{Accept} & application/json \\ \hline
            \end{tabu}
            \caption{External API - list request}
        \end{table}

    \level{2}{Authentication API}
        The Authentication API provides the functionalities to perform authentication to a Norris instance. The API is available over HTTP request and may sets some cookies.
        \level{3}{Login}
            The login starts a new session for the given user. \\
            \begin{table}[H]
                \centering
                \begin{tabu}{|l|l|}
                    \hline
                    \textbf{Host} & your host \\ \hline
                    \textbf{Path} & [endpoint]/auth/login \\ \hline
                    \textbf{Method} & POST \\ \hline
                    \textbf{Content-Type} & application/x-www-form-urlencoded \\ \hline
                    \textbf{Content} & username=[username]\&password=[password] \\ \hline
                \end{tabu}
                \caption{External API - login request}
            \end{table}
        \level{3}{Keep alive}
            The keep alive prevents the user session to expire. \\
            \begin{table}[H]
                \centering
                \begin{tabu}{|l|l|}
                    \hline
                    \textbf{Host} & your host \\ \hline
                    \textbf{Path} & [endpoint]/auth/keepalive \\ \hline
                    \textbf{Method} & POST \\ \hline
                \end{tabu}
                \caption{External API - keep alive request}
            \end{table}
        \level{3}{Logout}
            The logout ends the user session. \\
            \begin{table}[H]
                \centering
                \begin{tabu}{|l|l|}
                    \hline
                    \textbf{Host} & your host \\ \hline
                    \textbf{Path} & [endpoint]/auth/logout \\ \hline
                    \textbf{Method} & POST \\ \hline
                \end{tabu}
                \caption{External API - logout request}
            \end{table}
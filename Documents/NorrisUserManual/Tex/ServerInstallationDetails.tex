\level{2}{Installation details}
	\level{3}{Prerequisites}
		Before you begin, make sure
		\begin{description}
			\item[you have \insglo{Node.js} and npm installed] Get the latest version from \insuri{http://nodejs.org}. If you're using Mac or Windows, the best way to install \insglo{Node.js} is to use one of the pre-built installers. If you're using Linux, you can use the installer or you can download the source code.\\
			\insglo{Node.js} comes with npm installed so you should have a version of npm. However, npm gets updated more frequently than Node does, so you'll want to make sure it's the latest version.
			\item[you have a \texttt{package.json} file in the root directory of your application] It holds various metadata relevant to your project. This file is used to give information to npm that allows it to identify the project as well as handle the project's dependencies.\\
			If \texttt{package.json} file does not exist already, you can create it with the \texttt{npm init} command. A second option is to create it manually (specifics of npm's \texttt{package.json} handling can be found at \insuri{https://docs.npmjs.com/files/package.json}).
			\item[you have \insglo{Express.js} installed in the app directory and saved in the dependencies list] This \insglo{package} can be downloaded and added to the dependencies list with the \texttt{npm install express --save} command (detailed instructions can be found at \insuri{http://expressjs.com/starter/installing.html}).
		\end{description}
	\level{3}{Performing the installation}
		First, make the app directory your working directory. Then use the following npm command:
		\begin{lstlisting}
			$ npm install \insglo{norris} --save
		\end{lstlisting}
		This will create the \texttt{node\_modules} directory (if one doesn't exist yet), and will download the \insglo{package} to that directory. It will also add the \insglo{package} to the dependencies in \texttt{package.json} (unless it was already there).\\
		You might want to add \insglo{Norris} temporarily, just to it out, without adding it to the dependencies list in the \texttt{package.json} file. Just avoid using \texttt{--save} flag:
		\begin{lstlisting}
			$ npm install \insglo{norris}
		\end{lstlisting}
\level{2}{Usage}
	\level{3}{Initialization}
		\begin{itemize}
			\item You have to create a HTTP server.
			\begin{enumerate}
				\item Include the library:
				\begin{lstlisting}
					var http = require('http');
				\end{lstlisting}
				\item Create a new instance of \texttt{http.Server}:
				\begin{lstlisting}
					var server = http.createServer();
				\end{lstlisting}
				\texttt{server} will emit events you can listen. For example the event \texttt{'request'} is emitted each time there is a request.
				\item Make the HTTP server accept connections on the specified \texttt{port}:
				\begin{lstlisting}
					server.listen(port);
				\end{lstlisting}
			\end{enumerate}
			\item You have to create an Express web server.
			\begin{enumerate}
				\item Include the library:
				\begin{lstlisting}
					var express = require('express');
				\end{lstlisting}
				\item Create a new instance of Express:
				\begin{lstlisting}
					var app = new express();
				\end{lstlisting}
			\end{enumerate}
			\item You have to add a listener (\texttt{app}) to the end of the listeners array for the \texttt{'request'} event:
			\begin{lstlisting}
				server.on('request',app);
			\end{lstlisting}
			\item You have to create an instance of Norris.
			\begin{enumerate}
				\item Include the library:
				\begin{lstlisting}
					var norris = require('norris');
				\end{lstlisting}
				\item Create a new instance of Norris, and make it accessible by a client application at the given endpoint (e.g. \texttt{'/norris'}):
				\begin{lstlisting}
					var nrr = norris(server, endpoint);
				\end{lstlisting}
				\item (optional) You can choose some settings (properties) that affect how Norris behaves:
				\begin{lstlisting}
					nrr.settings(opts);
				\end{lstlisting}
				For more information about the opts object, see NORRIS SETTINGS
			\end{enumerate}
		\end{itemize}
	\level{3}{Example usage}
		This section contains some code that show you a common scenario. It is explained how it works and what it does. For more detailed information about the internal API, you can check out API REFERENCE.\\
		You can create one of the charts Norris has available. In the example below, we will build a Bar Chart based on some data and settings.
		\begin{lstlisting}
			// create a new Bar Chart whose id is 'cp3'
			var myChart = new nrr.Chart('barchart', 'cp3');
			// enter data for the chart
			myChart.data({...});
			// choose options for the chart
			myChart.settings({...});
		\end{lstlisting}
		\level{4}{Data structure}
			The Bar Chart requires an array of labels for each of the data points (independent values). This is shown on the X-axis or on the Y-axis, depending on the settings relating to the orientation of the bars. The data for Bar Charts is broken up into an array of datasets. Each dataset has a name and an array of dependent values, whose length is equal to that of the array of labels.
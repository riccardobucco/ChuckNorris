\level{2}{Usage}
	\level{3}{Initialization}
		\begin{itemize}
			\item You have to create a \insglo{HTTP} \insglo{server}.
			\begin{enumerate}
				\item Include the library:
				\begin{lstlisting}
					var \insglo{http} = require('\insglo{http}');
				\end{lstlisting}
				\item Create a new instance of \texttt{http.Server}:
				\begin{lstlisting}
					var \insglo{server} = \insglo{http}.createServer();
				\end{lstlisting}
				\texttt{server} will emit events you can listen. For example the event \texttt{'request'} is emitted each time there is a request.
				\item Make the \insglo{HTTP} \insglo{server} accept connections on the specified \texttt{port}:
				\begin{lstlisting}
					\insglo{server}.listen(port);
				\end{lstlisting}
			\end{enumerate}
			\item You have to create an Express web \insglo{server}.
			\begin{enumerate}
				\item Include the library:
				\begin{lstlisting}
					var express = require('express');
				\end{lstlisting}
				\item Create a new instance of Express:
				\begin{lstlisting}
					var app = new express();
				\end{lstlisting}
			\end{enumerate}
			\item You have to add a listener (\texttt{app}) to the end of the listeners array for the \texttt{'request'} event:
			\begin{lstlisting}
				\insglo{server}.on('request',app);
			\end{lstlisting}
			\item You have to create an instance of \insglo{Norris}.
			\begin{enumerate}
				\item Include the library:
				\begin{lstlisting}
					var \insglo{norris} = require('\insglo{norris}');
				\end{lstlisting}
				\item Create a new instance of \insglo{Norris}, and make it accessible by a \insglo{client} application at the given endpoint (e.g. \texttt{'/norris'}):
				\begin{lstlisting}
					var nrr = \insglo{norris}(\insglo{server}, endpoint);
				\end{lstlisting}
				\item (optional) You can choose some settings (properties) that affect how \insglo{Norris} behaves:
				\begin{lstlisting}
					nrr.settings(opts);
				\end{lstlisting}
				For more information about the opts object, see \insglo{NORRIS} SETTINGS
			\end{enumerate}
		\end{itemize}
	\level{3}{Example usage}
		This section contains some code that show you a common scenario. It is explained how it works and what it does. For more detailed information about the internal \insglo{API}, you can check out \insglo{API} REFERENCE.\\
		You can create one of the charts \insglo{Norris} has available. In the example below, we will build a \insglo{Bar Chart} based on some data and settings. Then we will add it to a page, and we keep it updated.
		\begin{lstlisting}
			// create a new empty \insglo{Bar Chart} whose id is 'cp3'
			var myChart = new nrr.Chart('barchart', 'cp3');
			// enter data for the chart
			myChart.data({...});
			// choose options for the chart [optional]
			myChart.settings({...});
			// create a new page
			var page = new nrr.Page('myFirstPage');
			// add the myChart to the new page
			page.add(myChart);
			// myPage will be available at '/norris' path
			app.use('/norris', nrr.getMiddleware());
			// keep myChart updated using inplace method (some values will be replaced)
			myChart.update('inplace', {...});
		\end{lstlisting}
		\level{4}{Data object structure}
			The \insglo{Bar Chart} requires an array of labels for each of the data points (independent values). This is shown on the X-axis or on the Y-axis, depending on the settings relating to the orientation of the bars. The data for Bar Charts is broken up into an array of datasets. Each dataset has a name and an array of dependent values, whose length is equal to that of the array of labels. Each dataset represents a set of bars.\\
			More details about tha data object of a Bar Chart can be found at REFERENCE.
		\level{4}{Options object structure}
			It is possible to set some options for the chart. For example, you can choose the bars orientation, legend position, distance between two bars... However, this step isn't mandatory, you can avoid it: in this case some default options will be set.\\
			More details about options object of a Bar Chart can be found at REFERENCE
		\level{5}{Update object structure}
			When you want to update with the inplace method a Bar Chart you have to choose which bars (of which series) you desire to modify. To modify a bar means that its height (or length) will change. In the update object you can insert more then one single value: in this way you are able to modify multiple bars at the same time.\\
			More details about tha update object of a Bar Chart can be found at REFERENCE.
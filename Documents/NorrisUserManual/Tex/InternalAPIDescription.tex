\level{1}{Description of Norris' Internal API}
	In this section it's described the functions allowed by the Internal API of Norris. Through those classes it's possible creating four types of chart, bar chart, line chart, map chart and table, setting their settings and the type of update.\\ The methods of the different classes are explained in the following paragraph.
	\begin{description}
	\item[NorrisBridge] NorrisBridge implements the interface of Norris. It transmits the requests of change of settings and of the authentication functions to the DataModel. Furthermore it manages the addition of charts and pages. \\It has only the field \texttt{model}, which is an instance of the class \texttt{NorrisImpl}. Its methods are:
		\begin{itemize}
		\item \texttt{NorrisBridge(model)} :  It's the constructor method of the \texttt{NorrisBridge} class. It receives a \texttt{NorrisImpl} object as parameter, that is used to set the Norris field model.
		\item \texttt{setSettings(settings): void} : This method is used to set the settings of the NorrisImpl's instance contained into the invocation object. The method doesn't return any value.
		\item \texttt{getSettings(): NorrisSettings} : This method is used to get the settings of the NorrisImpl's instance contained into the invocation object. The method returns a JSON object containing all settings.
		\item \texttt{createChart(chartType, chartId): Chart} : This method is used to create an instance of a chart. It takes two String object as parameter, \texttt{chartType} and \texttt{chartId}, and returns a chart of the indicated type with the indicated id in its settings.
		\item \texttt{getChart(chartId): Chart} : This method is used to get the instance of a \texttt{ChartBridge} contained in the invocation method. The id of the chart is given as parameter of the method.
		\item \texttt{getCharts(): Chart[]} : This method return an array with all instances of \texttt{Chart} which are into the invocation object.
		\item \texttt{createPage(pageId): Page} : This method is used to create an instance of \texttt{Page} setting the id of the created object with the id received as parameter.
		\item \texttt{getPage(pageId): Page} : This method is used to get the instance of a \texttt{Page} contained in the invocation method. The id of the page is given as parameter of the method.
		\item \texttt{getPages(): Page[]} : This method return an array with all instances of \texttt{Page} which are into the invocation object.
		
		\end{itemize}
	\item[PageBridge] \texttt{PageBridge} implements the interface of a Page. It transmits the request of data and it manages the requests of change of settings and the requests of update to the DataModel.\\
	Its field is \texttt{page}, a \texttt{PageImpl} object.
	Its methods are:
	\begin{itemize}
		\item \texttt{PageBridge(page: PageImpl)} : It's the constructor method of \texttt{PageBridge}. It uses a \texttt{PageImple} object, that is given as parameter, to set the field.
		\item \texttt{add(chart: Chart): Page} : This method is used to add the chart in input to the represented page. It returns the instance of the page.
		\item \texttt{getId(): String} : This method is used to get the id of the represented page. The id is returned as a String object.
		\item \texttt{setSettings(settings: PageSettings): void} : This method is used to set the settings of the PageImpl's instance contained into the invocation object. The method doesn't return any value.
		\item \texttt{getSettings(): PageSettings} : This method is used to get the settings of the PageImpl's instance contained into the invocation object. It returns the settings in form of JSON object.
		\item \texttt{setCharts(charts: Chart[]): void} : This method is used to set the charts of the represented page. Using this method, all charts previously inserted will be deleted and changed with those into the array in input. It doesn't return any object.
		\item \texttt{getCharts(): Chart[]} : This method is used to get the charts contained into the invocation object. It returns the charts in form of array.
	\end{itemize}
	
	\item[ChartBridge] \texttt{ChartBridge} implements the interface of a chart. It transmits the requests of data and it manages the requests of change of settings and the requests of update DataModel.\\
	Its field is \texttt{chart}, a \texttt{ChartImpl} object. Its methods are:
	\begin{itemize}
		\item \texttt{ChartBridge(chart)} : It's the constructor method of \texttt{ChartBridge}. It receives a \texttt{ChartImpl} object as parameter, that is used to set the chart's field model.
		\item \texttt{getChartModel(): Chart} : This method is used to get the instance of the represented chart. The the type of the returned object is Chart.
		\item \texttt{getId(): String} : This method is used to get the id of the represented chart. It returns the id in form of String object.
		\item \texttt{getType(): String} : This method is used to get the type of the represented chart. It returns the type in form of String object.
		\item \texttt{setData(data): void} : This method takes in input the data to represent and it sets in the settings of the chart.
		\item \texttt{getData(): ChartData} : This method is used to get the data of the represented chart. It returns the data in form of JSON object.
		\item \texttt{setSettings(settings: ChartSettings): void} : This method is used to set the settings of the ChartImpl'instance contained into the invocation object. The method doesn't return any value.
		\item \texttt{getSettings(): ChartSettings} : This method is used to get the settings of the ChartImpl'instance contained into the invocation object. It returns the settings in form of JSON object.
		\item \texttt{update(updateType: String, updateData: ChartUpdate): void} : This method is used to update the represented chart with the indicated type and data. 
	\end{itemize}
	\end{description}

\section{Introduzione}
	\subsection{Scopo del documento}
		Il presente documento intende stabilire le varie norme che il gruppo Kaizen Team segue nello sviluppo del progetto Norris. Esso, inoltre, indica gli strumenti e le procedure da utilizzare. Infine, vengono descritte le attività che il gruppo deve affrontare durante lo sviluppo.\\
		Ciascun componente è obbligato a prendere visione di tale documento e a rispettare le norme e le convenzioni in esso descritte. Qualora nel documento fossero presenti delle convenzioni, non è strettamente obbligatorio seguirle, sebbene sia caldamente consigliato.\\
		Tutti i contenuti di tale documento permettono di migliorare l’efficienza del lavoro svolto. Essi, inoltre, servono a facilitare la fase di verifica e a garantire una maggiore coerenza tra i documenti prodotti.\\
		Segue un riassunto di quanto contenuto all'interno di tale documento:
		\begin{itemize}
			\item organizzazione della comunicazione tra i vari membri del team;
			\item metodologia di stesura dei documenti;
			\item descrizione del modo in cui il lavoro viene organizzato durante lo sviluppo del progetto;
			\item strumenti utilizzati per gestire l'ambiente di lavoro, il repository ed il ticketing.
		\end{itemize}
	\subsection{Scopo del prodotto}
		Durante tale progetto si deve produrre un framework per lo stack tecnologico formato da Node.js, Express.js e Socket.io in grado di generare grafici i cui dati sono letti da sorgenti arbitrarie.\\
		Il framework creato deve fornire delle API per la configurazione programmatica della rappresentazione grafica e dei dati che si vogliono utilizzare. Esso inoltre deve fornire funzioni di aggiornamento dei grafici lato server tramite tecnologia WebSocket.
	\subsection{Glossario}
		Allo scopo di rendere più semplice la comprensione dei documenti ed evitare eventuali ambiguità, viene allegato il "Glossario vX.Y", che contiene la spiegazione della terminologia tecnica e degli acronimi utilizzati. Per facilitare la lettura, i termini presenti all'interno del Glossario saranno scritti in corsivo e marcati da una “G” maiuscola a pedice.
	\subsection{Riferimenti}
		\subsubsection{Riferimenti normativi}
		\subsubsection{Riferimenti informativi}
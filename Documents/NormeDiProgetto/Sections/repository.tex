\section{Repository}


Per la gestione del repository si è scelto di utilizzare il sistema di versionamento distribuito Git. Viene preferita questa soluzione perché consigliata dal capitolato d’appalto.
Visibilità repository
Per la condivisione e il versionamento dei configuration item è stato creato un repository privato su GitHub,  raggiungibile all’indirizzo \texttt{https://github.com/marcorubin/kaizen-team/}. L’accesso è consentito solamente agli utenti approvati dal Project Manager.
\subsection{Struttura del repository}
I file all’interno del repository sono organizzati secondo la seguente struttura: \\
ROOT
\begin{itemize}
	\item Documents
		\begin{itemize}
			\item Commons
			\item NormeDiProgetto
			\item StudioDiFattibilita
			\item AnalisiDeiRequisiti
			\item PianoDiProgetto
			\item PianoDiQualifica
			\item SpecificaTecnica
			\item Glossario
		\end{itemize}
	\item Source
		\begin{itemize}
			\item Da definire non prima della progettazione architetturale
		\end{itemize}
\end{itemize}

	
\subsection{Norme sui nomi dei file}
I nomi dei file all’interno del repository sono soggetti alle seguenti norme:
\begin{itemize}
	\item devono contenere solamente lettere, numeri, il carattere underscore U+005F, il segno meno U+2212 e il punto U+002E
	\item devono avere una lunghezza minima di 3 caratteri
	\item devono contenere le informazioni sufficiente per distinguere il file in modo non ambiguo
	\item le informazioni devono essere riportate dal generale al particolare
	\item le date devono essere specificate nel formato YYMMDD
\end{itemize}

Si consiglia inoltre:
\begin{itemize}
\item di utilizzare il meno possibile i caratteri underscore e segno meno, utilizzando al loro posto la notazione camel case
\item di utilizzare nomi con una lunghezza compresa tra i 5 e i 25 caratteri
\item di specificare sempre l’estensione quando possibile
\end{itemize}

\subsection{Norme per il commit}
\begin{itemize}
\item Quando si effettua un commit è necessario specificare nel messaggio una descrizione sintetica e non ambigua delle modifiche apportate
\item Le modifiche apportate con un commit devono essere complete e testate con successo
\item Le modifiche apportate con un commit devono essere logicamente correlate tra di loro
\end{itemize}

\subsection{Procedure di aggiornamento del repository}
Per aggiornare il repository è prevista una procedura ben definita:
\begin{enumerate}
\item Pull del repository tramite comando “git pull”
\item Eventuale merge tramite comando “git merge”
\item Stage dei file da aggiornare tramite comando “git add [files]”
\item Commit tramite comando “git commit -m ‘messaggio’”
\item Push del repository tramite comando “git push”
\end{enumerate}

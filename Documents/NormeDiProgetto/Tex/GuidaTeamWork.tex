% !TEX encoding = UTF-8 Unicode
\section{Guida per utilizzare TeamWork}
		Ogni componente del gruppo ha il dovere/obbligo di consultare TeamWork per sapere quali compiti deve svolgere ed entro che date e segnalare lo stato di completamento di un task. L'inserimento, modifica, eliminazione di un ticket di pianificazione, una lista di task o di una milestone spetta solamente al \insrole{Responsabile di Progetto}.
		\subsection{Accedere alla pagina del progetto}
			Per accedere alla pagina del progetto è sufficiente rispettare i seguenti passaggi:
			\begin{enumerate}
				\item autenticarsi su TeamWork all'indirizzo \insuri{https://www.teamwork.com/};
				\item selezionare il team di sviluppo \groupname{};
				\item selezionare il progetto \projectname{} dal pannello sinistro.
			\end{enumerate}
			
		\subsection{Visualizzazione dei task assegnati}
			Come detto ogni membro del gruppo \groupname{} avrà accesso al sistema di ticketing.\\
			Per visualizzare i propri task attraverso applicazione web è sufficiente autenticarsi e attendere il caricamento della dashboard. Apertasi abbiamo dunque la possibilità di visualizzare i task in ritardo, quelli odierni, quelli futuri e le varie scadenze.\\
			
Utilizzando l'applicazione Android è possibile vedere le stesse cose seguendo i passaggi seguenti:
			\begin{enumerate}
				\item autenticarsi alla prima apertura dell'applicazione ;
				\item aprire la barra laterale sinistra e selezionare il progetto \projectname{};
				\item selezionare dunque \textit{Tasks}.
			\end{enumerate}
			E ora possibile impostare alcuni filtri sui risultati da visualizzare cliccando in alto a destra il bottone  \textit{Filters}.

		\subsection{Creare una milestone}
			Il \insrole{Responsabile di Progetto} avrà dunque il compito di creare le varie milestone e per farlo dovrà seguire i seguenti passaggi:
			\begin{enumerate}
				\item accedere alla pagina del progetto;
				\item cliccare sulla scheda \textit{MILESTONE};
				\item premere il bottone \textit{Add a milestone};
				\item compilare la scheda con nome della milestone, data, responsabile e opzioni di notifica;
				\item finalizzare l'inserimento cliccando su \textit{Add this milestone}.
			\end{enumerate}
			
		\subsection{Modificare una milestone}
			Per modificare una milestone il \insrole{Responsabile di Progetto} dovrà:
			\begin{enumerate}
				\item accedere alla pagina del progetto;
				\item cliccare sulla scheda \textit{MILESTONE};
				\item cliccare \textit{Edit} sotto il nome della milestone che si vuole modificare;
				\item effettuare le modifiche desiderate attraverso la maschera di modifica apparsa;
				\item finalizzare la modifica cliccando su \textit{Update milestone}.
			\end{enumerate}
			
		\subsection{Eliminare una milestone}
			Per eliminare una milestone il \insrole{Responsabile di Progetto} dovrà:
			\begin{enumerate}
				\item accedere alla pagina del progetto;
				\item cliccare sulla scheda \textit{MILESTONE};
				\item cliccare \textit{Delete} sotto il nome della milestone che si vuole eliminare;
				\item confermare la scelta.
			\end{enumerate}
			
		\subsection{Creare una task list}
			Per inserire una lista di task il \insrole{Responsabile di Progetto} dovrà:
			\begin{enumerate}
				\item accedere alla pagina del progetto;
				\item cliccare sulla scheda \textit{TASKS};
				\item premere il bottone \textit{Add task list};
				\item compilare la scheda con nome della lista e note aggiuntive;
				\item finalizzare l'inserimento cliccando su \textit{Add this task list}.
			\end{enumerate}
			
		\subsection{Modificare una task list}
			Per modificare una lista di task il \insrole{Responsabile di Progetto} dovrà:
			\begin{enumerate}
				\item accedere alla pagina del progetto;
				\item cliccare sulla scheda \textit{TASKS};
				\item premere il bottone triangolare accanto alla lista e cliccare \textit{Edit};
				\item effettuare le modifiche desiderate attraverso la maschera di modifica apparsa;
				\item finalizzare la modifica cliccando su \textit{Update task list}.
			\end{enumerate}
			
		\subsection{Eliminare una task list}
			Per eliminare una lista di task il \insrole{Responsabile di Progetto} dovrà:
			\begin{enumerate}
				\item accedere alla pagina del progetto;
				\item cliccare sulla scheda \textit{TASKS};
				\item premere il bottone \textit{Add task list};
				\item compilare la scheda con nome della lista e note aggiuntive;
				\item finalizzare l'inserimento cliccando su \textit{Add this task list}.
			\end{enumerate}
			
		\subsection{Inserire un task}
			Per inserire un task il \insrole{Responsabile di Progetto} dovrà:
			\begin{enumerate}
				\item accedere alla pagina del progetto;
				\item cliccare sulla scheda \textit{TASKS};
				\item premere il bottone \textit{Add task} sotto la lista sulla quale vuole fare l'inserimento;
				\item compilare la scheda con nome task, data di inizio e fine, componente che dovrà svolgere quel compito e delle note aggiuntive chiare per specificare lo scopo;
				\item finalizzare l'inserimento cliccando su \textit{Add this task list}.
			\end{enumerate}
			
		\subsection{Modificare un task}
			Per modificare un task il \insrole{Responsabile di Progetto} dovrà:
			\begin{enumerate}
				\item accedere alla pagina del progetto;
				\item cliccare sulla scheda \textit{TASKS};
				\item premere sul task che si vuole modificare;
				\item cliccare il bottone \textit{Edit};
				\item effettuare le modifiche desiderate attraverso la maschera di modifica apparsa;
				\item finalizzare la modifica cliccando su \textit{Update task}.
			\end{enumerate}
			
		\subsection{Eliminare un task}
			Per eliminare un task il \insrole{Responsabile di Progetto} dovrà:
			\begin{enumerate}
				\item accedere alla pagina del progetto;
				\item cliccare sulla scheda \textit{TASKS};
				\item premere sul task che si vuole eliminare;
				\item cliccare il bottone \textit{Options};
				\item cliccare quindi il bottone \textit{Delete};
				\item confermare la cancellazione.
			\end{enumerate}
			
		\subsection{Impostare lo stato di un task}
			Per impostare li stato di un task basta  seguire la procedura di modifica di un task e modificare la sezione \textit{progress and time} della maschera di modifica aggiungendo la percentuale di completamento.
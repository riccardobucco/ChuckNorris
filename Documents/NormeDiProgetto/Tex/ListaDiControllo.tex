\level{1}{Lista di controllo}
Si riporta la lista di controllo contenente gli errori più frequenti che sono stati riscontrati durante la fase di verifica dei documenti:
\begin{itemize}
	\item \textbf{Norme stilistiche:}
	\begin{itemize}
		\item nome dei ruoli di progetto: non viene utilizzata l'apposita macro;
		\item nome dei documenti: non viene utilizzata l'apposita macro;
		\item nomi propri: non vengono rispettate le norme, secondo le quali il cognome precede il nome;
		\item date: non viene usato il comando apposito;
		\item elenchi: quando l'elenco è preceduto da ‘:’, le prime lettere di ogni elemento non sono minuscole;
		\item elenchi: gli elementi terminano con ‘.’ anche se non contengono al loro interno un ‘.’;
		\item numeri: non seguono lo standard [SI/ISO 31-0];
		\item apici: utilizzo non corretto degli apici singoli e doppi.
	\end{itemize}
	\item \textbf{Sintassi:}
	\begin{itemize}
		\item periodi: proposizioni troppo lunghe;
		\item virgola tra soggetto e verbo: errore di sintassi;
		\item doppie negazioni: rendono la frase di difficile comprensione.
	\end{itemize}
	\item \textbf{\LaTeX{}:}
	\begin{itemize}
		\item macro \LaTeX{}: non viene utilizzato l'apposito comando;
		\item caratteri speciali: per alcuni caratteri, come ‘\'{E}’, non vengono utilizzati gli appositi comandi.
	\end{itemize}
	\item \textbf{UML:}
	\begin{itemize}
		\item frecce: direzione delle frecce non corrette;
		\item descrizioni: a volte troppo sintetiche o astratte;
		\item eccessiva frammentazione;
		\item dipendenze: eccessivo numero di dipendenze tra le componenti.
	\end{itemize}
	\item \textbf{Tracciamento:}
	\begin{itemize}
		\item ad ogni caso d'uso deve corrispondere almeno un requisito;
		\item ad ogni requisito deve corrispondere almeno una fonte;
		\item ad ogni componente deve corrispondere almeno un requisito;
		\item ogni requisito deve essere associato ad almeno una componente;
		\item ad ogni test deve corrispondere almeno una componente.
	\end{itemize}
	
\end{itemize}


\level{1}{Unified Modelling Language} \label{app:UML}
L'UML (Linguaggio di Modellazione Unificato) è un linguaggio di modellazione e specifica basato sul concetto object-oriented, ed è ampiamente diffuso nell'ambito dell'ingegneria del software.\\
L'obiettivo per il quale è stato creato questo linguaggio è quello di avere un linguaggio comune che permetta di descrivere soluzioni analitiche e progettuali in modo sintetico e comprensibile.

\level{2}{Caratteristiche}
UML si presenta come un linguaggio visuale, facile da imparare e semanticamente molto ricco; ha una propria sintassi, che regola il modo in cui gli elementi sono assemblati in espressioni, e una propria semantica per interpretare correttamente quanto è stato modellato.
Le caratteristiche principali di questo linguaggio di modellazione sono:
\begin{itemize}
	\item l'incorporazione della migliore esperienza sviluppata a livello industriale;
	\item la flessibilità verso la maggior parte dei sistemi produttivi;
	\item la scalabilità;
	\item il supporto delle architetture distribuite;
	\item l'indipendenza da qualsiasi linguaggio di sviluppo e programmazione;
	\item il supporto di concetti di sviluppo ad alto livello (come framework e pattern);
	\item il supporto all'intero ciclo di vita del software (i diagrammi UML sono impiegati nell'esplicazione di concetti di molte attività svolte durante il ciclo di vita del software, come l'analisi e la progettazione).
\end{itemize}

\level{2}{Diagrammi UML}
Attraverso l'UML è possibile realizzare i seguenti diversi tipi di diagrammi, utili, se non talvolta fondamentali, durante la progettazione di un software.
\begin{description}
\item[Diagrammi dei casi d'uso:] i diagrammi dei casi d'uso modellano il comportamento di un sistema definendo le funzioni messe a disposizione degli utenti esterni, siano essi utenti fisici o altri sistemi software.
\item[Diagrammi delle classi:] i diagrammi delle classi modellano delle entità, degli oggetti, definendone caratteristiche, metodi, proprietà ed eventuali relazioni.
\item[Diagrammi delle attività:] i diagrammi delle attività definiscono le attività da svolgere per realizzare una certa funzione; un loro classico utilizzo è la modellazione di algoritmi.
\item[Diagramma di sequenza:] i diagrammi di sequenza sono usati per modellare uno scenario, ovvero una sequenza di azioni dovute a scelte prestabilite.
\item[Diagramma degli oggetti:] i diagrammi degli oggetti sono usati per descrivere un sistema in termini di oggetti e relative relazioni.
\item[Diagramma degli stati:] i diagrammi degli stati sono usati per descrivere il comportamento delle classi come se fossero macchine a stati.
\item[Diagramma delle comunicazioni:] i diagrammi delle comunicazioni sono usati per descrivere l'interazione fra più partecipanti alla realizzazione di una certa funzionalità.
\item[Diagramma dei componenti:] i diagrammi dei componenti rappresentano la struttura interna del sistema software modellato in termini dei suoi componenti principali (cioè unità software dotate di precise identità e responsabilità).
\item[Diagramma di dispiegamento:] i diagrammi di dispiegamento descrivono un sistema in termini di risorse hardware, dette nodi, e di relazioni fra di esse.
\end{description}
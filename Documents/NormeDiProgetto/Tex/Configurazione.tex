\section{Configurazione}
	\subsection{Attività}
		\subsubsection{Identificazione di configurazione}
		\subsubsection{Controllo di baseline}
		\subsubsection{Gestione modifiche}
		Per effettuare una modifica ad un qualsiasi documento è necessario inoltrare una richiesta formale di modifica al \insrole{Responsabile di Progetto}, il quale si preoccuperà di avviare le procedure necessarie alla gestione di tale modifica.
		\subsubsection{Controllo di versione}
	
	\subsection{Norme}
		\subsubsection{Struttura di una richiesta di modifica}
			Ogni richiesta di modifica è costituita dai seguenti campi:
			\begin{enumerate}
				\item Autore: contiene nome e cognome di colui che richiede la modifica;
				\item Documento: contiene il nome del documento di cui si richiede la modifica;
				\item Motivo: contiene la spiegazione chiara e concisa delle motivazioni per cui si sta richiedendo una modifica;
				\item Urgenza: può assumere uno dei seguenti valori:
					\begin{itemize}
						\item [-] Alta: si tratta di una modifica importante, con forti conseguenze nell'organizzazione e nello svolgimento delle attività future del progetto;
						\item [-] Media: si tratta di una modifica importante, ma che non comporta grandi modifiche alle attività del progetto;
						\item [-] Bassa: si tratta di una modifica di importanza secondaria, che può essere rimandata ad un secondo momento senza che ciò influisca in alcun modo sullo svoglimento delle attività.
					\end{itemize}
			\end{enumerate}
			In seguito alla decisione del \insrole{Responsabile} di approvare o meno la richiesta, va aggiunto il campo:
			\begin{itemize}
				\item [5.] Decisione del \insrole{Responsabile di Progetto}: contiene uno dei seguenti valori:
					\begin{itemize}
						\item [-] Approvata
						\item [-] Respinta
					\end{itemize}
			\end{itemize}
	
		\subsubsection{Struttura del repository}
				I file all’interno del repository sono organizzati secondo la seguente struttura:
				\begin{itemize}
					\item /Documents
					\begin{itemize}
						\item Commons
						\item NormeDiProgetto
						\item StudioDiFattibilità
						\item AnalisiDeiRequisiti
						\item PianoDiProgetto
						\item PianoDiQualifica
						\item SpecificaTecnica
						\item Glossario
					\end{itemize}
					\item /Source
				\end{itemize}
				La struttura di /Source viene definita prima della progettazione architetturale.

		\subsubsection{Nomi dei file}
				I nomi dei file all’interno del repository sono soggetti alle seguenti norme:
				\begin{itemize}
					\item devono contenere solamente lettere, numeri, il carattere underscore U+005F, il segno meno U+2212 e il punto U+002E;
					\item devono avere una lunghezza minima di 3 caratteri;
					\item devono contenere le informazioni sufficienti per distinguere il file in modo non ambiguo;
					\item le informazioni devono essere riportate dal generale al particolare;
					\item le date devono essere specificate nel formato YYYYMMDD.
				\end{itemize}
				Si consiglia inoltre:
				\begin{itemize}
					\item di utilizzare il meno possibile i caratteri underscore e segno meno, utilizzando al loro posto la notazione camel case;
					\item di utilizzare nomi con una lunghezza compresa tra i 5 e i 25 caratteri;
					\item di specificare sempre l’estensione quando possibile.
				\end{itemize}
			\subsubsection{Commit}
				L'esecuzione di un comando di commit implica il rispetto delle seguenti norme:
				\begin{itemize}
					\item quando si effettua un commit è necessario specificare nel messaggio una descrizione sintetica e non ambigua delle modifiche apportate;
					\item le modifiche apportate con un commit devono essere complete e testate con successo;
					\item le modifiche apportate con un commit devono essere logicamente correlate tra di loro.
				\end{itemize}
	
	\subsection{Procedure}
		\subsubsection{Richiesta di modifica}
		Ogni richiesta di modifica va inoltrata in modo formale al \insrole{Responsabile di Progetto}, affichè venga sottoposta ad un rigoroso processo di analisi. Dopodichè il \insrole{Responsabile di Progetto} decide se approvarla o meno. Nel primo caso assegna la realizzazione della modifica ad un membro del gruppo. Una volta effettuata la modifica, questa deve essere sottoposta a verifica. Tutto ciò deve avvenire mantenendo traccia dello stato precedente.
		
		\subsubsection{Aggiornamento del repository}
				Per aggiornare il repository è prevista una procedura ben definita:
				\begin{enumerate}
					\item pull del repository tramite comando git pull;
					\item eventuale merge tramite comando git merge;
					\item stage dei file da aggiornare tramite comando git add [files];
					\item commit tramite comando git commit -m ‘messaggio’;
					\item push del repository tramite comando git push.
				\end{enumerate}
		
	\subsection{Strumenti}	
		\subsubsection{Repository}
			Per la gestione del repository si è scelto di utilizzare il sistema di versionamento distribuito Git. AGGIUNGERE MOTIVAZIONI TECNICHE CHE HANNO PORTATO A QUESTA SCELTA
				\paragraph{Visibilità del repository}
				Per la condivisione e il versionamento dei configuration item è stato creato un repository privato su GitHub, raggiungibile all’indirizzo https://github.com/marcorubin/kaizen-team/. L’accesso è consentito solamente agli utenti approvati dal Project Manager.
						

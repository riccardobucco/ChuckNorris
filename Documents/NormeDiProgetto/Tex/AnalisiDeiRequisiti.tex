\section{Analisi dei Requisiti}
	\subsection{Attività}
		\subsubsection{Studio di Fattibilità}
			Lo studio di fattibilità deve essere rapido e accurato.\\
			Il documento di Studio di Fattibilità viene redatto dagli analisti, sulla base di ciò che emerge dalle prime due riunioni. Qui i membri del gruppo si confrontano su tematiche riguardanti i capitolati quali:
			\begin{itemize}
				\item il dominio applicativo e tecnologico;
				\item il rapporto costi/benefici :
				\begin{itemize}
					\item confronto tra mercato attuale e futuro;
					\item costo di produzione e redditività dell'investimento;
				\end{itemize}
			\item i rischi ai quali essi sono soggetti.
			\end{itemize}
			In tale studio si tiene anche conto delle preferenze e delle conoscenze dei vari membri del gruppo.
		\subsubsection{Analisi dei Requisiti}
			DA FARE - DIRE COSA VIENE FATTO DURANTE ANALISI DEI REQUISITI, CHI FA COSA, CON CHE RISORSE ECC.
	\subsection{Norme}
		\subsubsection{Classificazione dei requisiti}
			I requisiti dovranno essere classificati per tipo e importanza, utilizzando la seguente codifica: R[X][Y][Z].\\
			La codifica appena presentata deve essere interpretata nel modo seguente:
			\begin{enumerate}
				\item X indica l'importanza strategica del requisito e può assumere uno dei seguenti valori:
				\begin{itemize}
					\item 1: requisiti obbligatori;
					\item 2: requisiti desiderabili;
					\item 3: requisiti opzionali.
				\end{itemize}
				\item Y indica la tipologia del requisito e può assumere i seguenti valori:
				\begin{itemize}
					\item F: indica che si tratta di un requisito funzionale;
					\item P: indica che si tratta di un requisito prestazionale;
					\item Q: indica che si tratta di un requisito di qualità;
					\item V: indica che si tratta di un requisito di vincolo.
				\end{itemize}
				\item Z indica il codice univoco che identifica un requisito.
			\end{enumerate}
		\subsubsection{Classificazione dei casi d'uso}
			I casi d'uso verranno classificati in base alla seguente codifica: DA FARE.\\
			I casi d'uso sono identificati e trattati dagli analisti. Essi dovranno preoccuparsi di correlare a ogni caso d'uso un diagramma che lo rappresenti. Inoltre dovranno includere le seguenti informazioni:
			\begin{itemize}
				\item titolo;
				\item attori (principali e secondari);
				\item descrizione testuale;
				\item precondizione;
				\item postcondizione;
				\item flusso principale degli eventi, dove si descrive il flusso dei casi d'uso figli; per ogni evento va specificato quanto segue:
				\begin{itemize}
					\item titolo;
					\item descrizione testuale;
					\item attori coinvolti;
					\item se l’evento è descritto nel dettaglio da un altro caso d’uso;
				\end{itemize}
				\item scenari alternativi; per ognuno di essi va specificato quanto segue:
				\begin{itemize}
					\item titolo;
					\item descrizione testuale;
					\item attori coinvolti;
					\item se lo scenario è descritto nel dettaglio da un altro caso d’uso.
				\end{itemize}
			\end{itemize}
	\subsection{Strumenti}
		INSERIRE STRUMENTI CON I QUALI SI FA TRACCIAMENTO DEI REQUISITI
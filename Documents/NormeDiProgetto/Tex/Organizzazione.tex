\section{Organizzazione}
	\subsection{Attività}
		\subsubsection{Comunicazioni}
			\paragraph{Comunicazioni interne}
				Per  le comunicazioni interne si utilizza una mailing list appositamente creata: mailinglist@kaizenteam.it.\\
				I componenti del gruppo utilizzano tale indirizzo per comunicare tra di loro, risultando così sempre aggiornati su ogni scambio di informazioni. Tale sistema può essere utilizzato esclusivamente per questioni riguardanti il progetto.\\
				Per facilitare la comunicazione il gruppo si avvale anche di sistemi di instant messaging quali:
				\begin{itemize}
					\item WhatsApp;
					\item chat di Facebook.
				\end{itemize}
				Quando si utilizzano i sistemi sopra citati si ha l'obbligo di stilare un verbale qualora siano state prese decisioni di importanza rilevante. Tale verbale deve essere caricato nel repository e i membri del gruppo devono essere avvisati delle decisioni tramite mail.
			\paragraph{Comunicazioni esterne}
				Il responsabile di progetto ha l’incarico di mantenere i contatti tra il team e le componenti esterne, utilizzando una casella di posta elettronica dedicata: info@kaizenteam.it.\\
				E' inoltre suo compito informare tutti i membri del gruppo delle discussioni da lui avute con componenti esterni al gruppo. Questo viene fatto inviando una mail che riassume la conversazione per mezzo della mailing list ufficiale.
		\subsubsection{Riunioni}
			\paragraph{Gestione delle riunioni}
				All'inizio degli incontri viene nominato un Segretario. Esso ha il compito di verificare la presenza dei membri del team e di annotare gli argomenti affrontati. Una volta terminato l'incontro, esso deve redigere il verbale, il quale viene archiviato nel repository per la libera consultazione da parte del gruppo.\\
				I partecipanti alle riunioni devono tenere un atteggiamento che favorisca la discussione di tutti i punti dell'ordine del giorno individuati.
			\paragraph{Convocazione di una riunione}
				Sia per la convocazione delle riunioni ordinarie che di quelle straordinarie il Responsabile di Progetto deve comunicare la data, l'ora e il luogo ai membri del team con almeno tre giorni di preavviso utilizzando la mailing list.\\
				La mail inviata dovrà essere così strutturata:
				\begin{itemize}
					\item oggetto: Convocazione della riunione [ordinaria/straorinaria] numero N per il giorno aaaa-mm-gg
					\item corpo:
					\begin{itemize}
						\item data: data e ora prevista;
						\item luogo: luogo previsto;
						\item tipo:	ordinaria/straordinaria;
						\item ordine del giorno.
					\end{itemize}
				\end{itemize}
	\subsection{Procedure}
		\subsubsection{Aggiornamento del repository}
			Per aggiornare il repository è prevista una procedura ben definita:
			\begin{enumerate}
				\item pull del repository tramite comando git pull;
				\item eventuale merge tramite comando git merge;
				\item stage dei file da aggiornare tramite comando git add [files];
				\item commit tramite comando git commit -m ‘messaggio’;
				\item push del repository tramite comando git push.
			\end{enumerate}
		\subsubsection{Creazione di un ticket}
			Per la realizzazione di un ticket è prevista la seguente procedura:
			\begin{enumerate}
				\item scelta del titolo;
				\item assegnazione del destinatario;
				\item selezione dell'etichetta;
				\item aggiunta di una descrizione testuale;
				\item invio del ticket.
			\end{enumerate}
		\subsubsection{Aggiornamento di un ticket}
			Una volta preso in carico un ticket, la procedura per l’aggiornamento è la seguente:
			\begin{enumerate}
				\item eventuale modifica dello stato nel titolo;
				\item inserimento di un commento contenente il lavoro svolto.
			\end{enumerate}
	\subsection{Norme}
		\subsubsection{Riunioni}
			Il compito di indire le riunioni spetta unicamente al Responsabile di Progetto, che può esercitare tale diritto ogniqualvolta lo ritiene opportuno. Le riunioni indette su iniziativa del Responsabile di Progetto sono considerate "ordinarie".\\
			Gli altri membri del team possono richiedere la convocazione di riunioni straordinarie solo previa comunicazione e autorizzazione del Responsabile di Progetto. Quest'ultimo può negare lo svolgimento di tali riunioni, qualora non le ritenesse importanti per tutto il team o per lo stato di avanzamento dei lavori. Eventualmente può anche decidere di posticiparle o	suggerire l'utilizzo di altri mezzi di comunicazione.\\
			I membri del team devono comunicare tempestivamente la loro disponibilità a partecipare alla riunione allo scopo di consentire al Responsabile di rimandare eventualmente l'incontro. I membri del team possono manifestare le loro preferenze in merito all'ora e al luogo dell'incontro, ma non riguardo la data, tranne nel caso straordinario in cui il Responsabile di Progetto convochi una riunione dando meno di tre giorni di preavviso.\\
			E' prevista la possibilità di riunioni riguardanti solo alcune persone del team per agevolare la transizione da una fase all'altra del processo di sviluppo. Tali incontri sono ritenuti di carattere informale e pertanto non sono regolamentati nel presente documento.
		\subsubsection{Comunicazione tramite email}
			In questa sezione viene definita la struttura di ogni email, sia per comunicazioni interne che esterne.
			\paragraph{Destinatario}
				Di seguito si indicano i possibili destinatari:
				\begin{itemize}
					\item interno: l’indirizzo utilizzabile è mailinglist@kaizenteam.it;
					\item esterno: varia se ci si deve riferire al Prof. Vardanega Tullio, Prof. Cardin Riccardo o al Proponente.
				\end{itemize}
			\paragraph{Mittente}
				Di seguito si indicano i possibili mittenti:
				\begin{itemize}
					\item interno: l’indirizzo e-mail di chi scrive il messaggio;
					\item esterno: l’unico indirizzo utilizzabile è info@kaizenteam.it.
				\end{itemize}
			\paragraph{Oggetto}
				L’oggetto deve essere chiaro e conciso e diverso da altri già utilizzati.\\
				Per comporre un messaggio di risposta è necessario anteporre all’oggetto il prefisso “Re:”, per inoltrare è obbligatorio anteporre “I:”, in modo da facilitare le varie comunicazioni.\\
				Inoltre, l’oggetto del messaggio non deve mai essere modificato.
			\paragraph{Corpo}
				Il corpo di un messaggio deve essere esaustivo e facilmente comprensibile a tutti i componenti del team.\\
				In caso di risposta o di inoltro, la parte aggiunta deve essere posta all’inizio, mantenendo anche il messaggio precedente, per non obbligare i membri a dover scorrere l’e-mail completa e per permettere la visione intera della conversazione.
			\paragraph{Allegati}
				In nessun caso è previsto l'uso di allegati, nemmeno nelle comunicazioni esterne. Qualora vi sia la necessità di inviare documenti e/o file, questi devono essere caricati su Dropbox. Nella mail si deve solo indicare il link al quale è possibile trovare il file.
		\subsubsection{Ticket}
			\paragraph{Tipologie di ticket}
				I ticket possono essere di vari tipi, ciascuno dei quali è rappresentato da un’etichetta nel sistema di ticketing. Le etichette possono essere una delle seguenti:
				\begin{itemize}
					\item Realizzazione: i ticket provvisti di tale etichetta vengono creati dal Responsabile di Progetto e servono per assegnare attività a persone;
					\item Verifica: i ticket provvisti di tale etichetta vengono creati dai verificatori e servono per indicare gli errori rilevati durante l’attività di verifica.
				\end{itemize}
			\paragraph{Sintassi}
				Il titolo dei ticket deve avere la seguente sintassi: [X][Y]Z.\\
				Di seguito le norme per interpretare tale sintassi.
				\begin{itemize}
					\item X rappresenta lo stato del ticket. Esso può assumere uno dei seguenti valori:
					\begin{itemize}
						\item Proposto: il ticket è in attesa di essere accettato dal Project Manager (solo ticket di verifica);
						\item Accettato: il ticket è stato accettato e assegnato dal Project Manager ad un componente del gruppo, ma deve ancora essere preso in carico;
						\item InCorso: il ticket è in fase di lavorazione;
						\item Completato: il ticket è stato completato;
						\item Verificato: il ticket è stato verificato;
						\item Sospeso: il ticket è stato sospeso temporaneamente e il motivo dovrà essere riportato tra le note del ticket;
					\end{itemize}
					\item Y rappresenta la priorità del ticket. Esso può assumere uno dei seguenti valori:
					\begin{itemize}
						\item Bassa: il ticket non è di particolare importanza;
						\item Normale: il ticket è di media importanza;
						\item Alta: il ticket è di importanza rilevante;
						\item Urgente: il ticket è di importanza strategica;
					\end{itemize}
					\item Z rappresenta il titolo del ticket. Esso non deve permettere alcuna ambiguità.
				\end{itemize}
				Il testo del ticket deve essere conciso, ma allo stesso tempo fornire tutte le informazioni necessarie per gestire il ticket.
		\subsubsection{Ruoli di progetto}
			Per lo svolgimento di un progetto è necessaria la presenza di figure professionali che rivestono ruoli ben precisi. I componenti del gruppo ricoprono, in momenti diversi, ruoli diversi, in modo da permettere a ciascuno di ricoprire ogni ruolo almeno un volta. E' necessario garantire che il ruolo di ciascun membro del gruppo non sia in conflitto con il ruolo che ha ricoperto in passato. Segue l'elenco dei ruoli:
			\paragraph{Responsabile}
				Il Responsabile rappresenta il progetto presso il fornitore e presso il committente. Egli accentra su di sè le responsabilità di scelta e approvazione. In particolare ha responsabilità su:
				\begin{itemize}
					\item pianificazione, controllo e coordinamento delle attività;
					\item assegnazione delle attività a persone;
					\item gestione delle risorse;
					\item relazioni esterne, in quanto rappresenta il gruppo e sa, in ogni momento, in quale stato di avanzamento ci si trova;
					\item convocazione di riunioni interne;
					\item verifica dei risultati;
					\item approvazione dei documenti;
					\item analisi e gestione dei fattori di rischio.
				\end{itemize}
				Il Responsabile si occupa inoltre di redigere il Piano di Progetto e collabora alla stesura del Piano di Qualifica, in particolare nella sezione relativa alla pianificazione.
			\paragraph{Amministratore}
				L'Amministratore si occupa del controllo e della gestione dell'ambiente di lavoro. Nel dettaglio si preoccupa di:
				\begin{itemize}
					\item equipaggiare l'ambiente di lavoro con strumenti, procedure, infrastrutture e servizi a supporto dei processi che permettano di automatizzare il più possibile le attività o una parte di esse;
					\item garantire che l’ambiente di lavoro sia sempre completo, ovvero dotato di tutti gli strumenti necessari, ordinato e aggiornato;
					\item controllare versioni e configurazioni del prodotto;
					\item gestire la documentazione di progetto;
					\item garantire e controllare la disponibilità e la diffusione della documentazione di progetto;
					\item fornire procedure e strumenti per il monitoraggio e segnalazione per il controllo qualità;
					\item risolvere problemi legati alla gestione dei processi tramite l'adozione di strumenti adatti.
				\end{itemize}
				L'Amministratore si occupa inoltre di redigere le Norme di Progetto e la sezione del Piano di Qualifica dove vengono descritti strumenti e metodi di verifica. Nel fare tutto ciò, l'amministratore non compie scelte gestionali, ma attua scelte tecnologiche concordate con il responsabile di progetto.
			\paragraph{Analista}
				L'Analista è responsabile delle attività di analisi. Si occupa di:
				\begin{itemize}
					\item capire appieno il problema tramite l'analisi dei bisogni e delle fonti;
					\item classificare i requisiti;
					\item redigere i diagrammi dei casi d'uso;
					\item assegnare i requisiti a parti distinte del sistema;
					\item assicurarsi che i requisiti trovati siano conformi alle richieste del committente.
				\end{itemize}
				L'analista non porta la soluzione, ma definisce il problema redigendo lo Studio di Fattibilità e l'Analisi dei Requisiti. Partecipa alla redazione del Piano di Qualifica in quanto conosce l’ambito del progetto ed ha chiari i livelli di qualità richiesta e le procedure da applicare per ottenerla.
			\paragraph{Progettista}
				Il Progettista si occupa della progettazione. Ha responsabilità di scelta e decisione nei seguenti ambiti:
				\begin{itemize}
					\item produzione della descrizione di una soluzione soddisfacente per tutti gli stakeholder;
					\item definizione dell'architettura del prodotto in modo tale che sia realizzabile con risorse date e costi prefissati;
					\item applicazione al prodotto di soluzioni note ed ottimizzate;
					\item applicazione al prodotto di soluzioni che lo rendano facilmente manutenibile;
					\item soddisfacimento dei requisiti con un sistema di qualità.
				\end{itemize}
				Il Progettista si occupa della redazione della Specifica Tecnica, della Definizione di Prodotto e delle sezioni del Piano di Qualifica inerenti le metriche di verifica della programmazione.
			\paragraph{Programmatore}
				Il Programmatore si occupa delle attività di codifica. In particolare ha i seguenti compiti:
				\begin{itemize}
					\item deve seguire in modo rigoroso le soluzioni descritte dal progettista;
					\item deve provvedere a documentare il codice scritto;
					\item deve realizzare i test per la verifica e la validazione del codice stesso.
				\end{itemize}
				Il Programmatore, infine, si occupa della redazione del Manuale Utente.
			\paragraph{Verificatore}
				Il Verificatore si occupa delle attività di verifica. Nello specifico le sue funzioni sono:
				\begin{itemize}
					\item assicurare che ogni attività venga svolta secondo le norme stabilite;
					\item verificare che ogni stadio del ciclo di vita del prodotto sia conforme.
				\end{itemize}
				Il verificatore redige la sezione del Piano di Qualifica che illustra l’esito e la completezza delle verifiche effettuate.
		\subsubsection{Repository}
			Per la gestione del repository si è scelto di utilizzare il sistema di versionamento distribuito Git. AGGIUNGERE MOTIVAZIONI TECNICHE CHE HANNO PORTATO A QUESTA SCELTA
			\paragraph{Visibilità del repository}
				Per la condivisione e il versionamento dei configuration item è stato creato un repository privato su GitHub, raggiungibile all’indirizzo https://github.com/marcorubin/kaizen-team/. L’accesso è consentito solamente agli utenti approvati dal Project Manager.
			\paragraph{Struttura del repository}
				I file all’interno del repository sono organizzati secondo la seguente struttura:
				\begin{itemize}
					\item /Documents
					\begin{itemize}
						\item Commons
						\item NormeDiProgetto
						\item StudioDiFattibilità
						\item AnalisiDeiRequisiti
						\item PianoDiProgetto
						\item PianoDiQualifica
						\item SpecificaTecnica
						\item Glossario
					\end{itemize}
					\item /Source
				\end{itemize}
				La struttura di /Source viene definita prima della progettazione architetturale.
			\paragraph{Nomi dei file}
				I nomi dei file all’interno del repository sono soggetti alle seguenti norme:
				\begin{itemize}
					\item devono contenere solamente lettere, numeri, il carattere underscore U+005F, il segno meno U+2212 e il punto U+002E;
					\item devono avere una lunghezza minima di 3 caratteri;
					\item devono contenere le informazioni sufficienti per distinguere il file in modo non ambiguo;
					\item le informazioni devono essere riportate dal generale al particolare;
					\item le date devono essere specificate nel formato YYYYMMDD.
				\end{itemize}
				Si consiglia inoltre:
				\begin{itemize}
					\item di utilizzare il meno possibile i caratteri underscore e segno meno, utilizzando al loro posto la notazione camel case;
					\item di utilizzare nomi con una lunghezza compresa tra i 5 e i 25 caratteri;
					\item di specificare sempre l’estensione quando possibile.
				\end{itemize}
			\paragraph{Commit}
				L'esecuzione di un comando di commit implica il rispetto delle seguenti norme:
				\begin{itemize}
					\item quando si effettua un commit è necessario specificare nel messaggio una descrizione sintetica e non ambigua delle modifiche apportate;
					\item le modifiche apportate con un commit devono essere complete e testate con successo;
					\item le modifiche apportate con un commit devono essere logicamente correlate tra di loro.
				\end{itemize}
	\subsection{Strumenti}
		
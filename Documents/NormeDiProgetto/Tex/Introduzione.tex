\section{Introduzione}
	\subsection{Scopo del documento}
		Il presente documento intende stabilire le varie norme che il gruppo \groupname{} segue nello sviluppo del progetto \projectname{}. Esso, inoltre, indica gli strumenti e le procedure da utilizzare. Infine, vengono descritte le attività che il gruppo deve affrontare durante lo sviluppo.\\
		Ciascun componente è obbligato a prendere visione di tale documento e a rispettare le norme in esso descritte. Qualora nel documento fossero presenti delle convenzioni, non è strettamente obbligatorio seguirle, sebbene sia caldamente consigliato.\\
		Tutti i contenuti di tale documento permettono di migliorare l’efficienza del lavoro svolto. Essi, inoltre, servono a facilitare la fase di verifica e a garantire una maggiore coerenza tra i documenti prodotti.\\
		Segue un riassunto di quanto contenuto all'interno di tale documento:
		\begin{itemize}
			\item organizzazione della comunicazione tra i vari membri del team;
			\item metodologia di stesura dei documenti;
			\item descrizione del modo in cui il lavoro viene organizzato durante lo sviluppo del progetto;
			\item strumenti utilizzati per gestire l'ambiente di lavoro, il repository ed il \textit{ticketing}.
		\end{itemize}

	\level{2}{Glossario}
	Allo scopo di rendere più semplice la comprensione dei documenti ed evitare eventuali ambiguità, viene allegato il \insdoc{Glossario v6.00}, che contiene la spiegazione della terminologia tecnica e degli acronimi utilizzati. Per facilitare la lettura, i termini presenti all'interno di tale documento saranno marcati da una “G” maiuscola a pedice.

	

	\subsection{Riferimenti utili}
		\subsubsection{Riferimenti normativi}
		\subsubsection{Riferimenti informativi}
			\begin{itemize}
				\item \textbf{Git:} \insuri{http://git-scm.com/};
				\item \textbf{Portable Document Format:} \insuri{http://en.wikipedia.org/wiki/Portable_Document_Format};
				\item \textbf{Portable Network Graphics:} \insuri{http://en.wikipedia.org/wiki/Portable_Network_Graphics};
				\item \textbf{Unicode:} \insuri{http://en.wikipedia.org/wiki/Unicode}.
			\end{itemize}
\section{Verifica}
	\subsection{Attività}
		\subsubsection{Analisi}
			Per poter determinare il livello di qualità del prodotto il team di sviluppo utilizza diversi tipi di tecniche di analisi dei dati.\\
			Queste possono essere suddivise in due grandi categorie:
			\begin{itemize}
				\item analisi statica;
				\item analisi dinamica.
			\end{itemize}
			\paragraph{Analisi statica}
				L'analisi statica è una forma di valutazione di un sistema o di un suo componente basato sulla sua forma, struttura, contenuto, 
				documentazione senza che esso sia eseguito. Tali tecniche, dunque, sono applicabili tanto al codice quanto alla documentazione.
				Verranno usate le seguenti tecniche di analisi statica:
				\begin{description}
					\item[walkthrough] Tale tecnica viene utilizzata quando non si sa realmente cosa si sta cercando. Essa, infatti, consiste in una 
					lettura da cima a fondo del documento/codice. Tale lettura ha lo scopo di trovare anomalie di tipo qualsiasi.
					\item[inspection] Tale tecnica viene utilizzata quando si ha un'idea di cosa si sta cercando e di cosa si potrebbe trovare. Essa 
					consiste in una lettura mirata del documento/codice, sulla base di una lista degli errori comuni stilata in precedenza.
				\end{description}
				Applicare in modo fruttuoso la tecnica del walkthrough è molto oneroso e dispendioso. Tuttavia, quando non si possiede una lista degli
				errori comuni (per esempio a inizio progetto), essa è l'unica soluzione. L'obiettivo è quello di formare una lista quanto più completa 
				possibile di errori comuni in modo tale che possa essere eseguita quasi sempre la tecnica dell'inspection.
			\paragraph{Analisi dinamica}
				L'analisi dinamica è una forma di valutazione di un sistema software o di un suo componente basato sulla osservazione del suo 
				comportamento in esecuzione. Tali tecniche, dunque, sono applicabili solo a componenti software e vengono svolte tramite l'esecuzione 
				di test su essi.\\
				Deve essere preoccupazione di chi scrive i test fare in modo che essi siano di grande valore dimostrativo, in quanto il numero di test 
				che devono essere effettuati è per forza di cose in numero finito e relativamente piccolo.
	\subsection{Procedure}
		\subsubsection{Verifica dei documenti}
			Dare inizio all'attività di verifica di un documento è compito del Responsabile di Progetto: esso assegna il compito a uno o più Verificatori. 
			Questi, con l'aiuto del diario delle modifiche, focalizzano la loro attenzione sulle sezioni del documento che sono state modificate.\\
			Per eseguire una verifica quanto più accurate e completa possibile è necessario controllare che sia stato rispettato quanto segue:
			\begin{enumerate}
				\item sintassi semplice e corretta;
				\item periodi brevi e leggibili;
				\item struttura del documento;
				\item norme tipografiche;
				\item proprietà di glossario.
			\end{enumerate}
			Durante l'esecuzione dell'intera procedura è importante che i Verificatori tengano traccia di tutti gli errori più comuni. In questo modo, 
			essi potranno essere riportati sulla lista di controlla necessaria per applicare in modo fruttuoso la tecnica dell'inspection.
			\paragraph{Sintassi}
				I Verificatori devono preoccuparsi di individuare gli errori sintattici all'interno dei testi del documento. Tale compito può essere in 
				parte attuato tramite strumenti automatici. È tuttavia necessario che un documento sia sempre sottoposto a walkthrough, in quanto gli 
				strumenti automatici non sono in grado di individuare parole corrette che sono utilizzate al di fuori del loro contesto.\\
				Inoltre, i Verificatori devono cercare tutte quelle parole che sono poco frequenti o comunque complesse (possono generare fraintendimenti 
				o incomprensioni): va applicata la tecnica del walkthrough.
			\paragraph{Periodi lunghi}
				I Verificatori devono calcolare l'indice Gulpease del documento (utilizzando l'appostio strumento automatico). Qualora si ottenga un 
				risultato inferiore alle aspettative si deve applicare il walkthrough all'interno del documento: l'obiettivo deve essere quello di 
				individuare frasi troppo lunghe (e che dunque possono essere di difficile comprensione e/o scarsa leggibilità).
			\paragraph{Struttura del documento}
				I Verificatori controllano con l'ausilio di strumenti automatici che la struttura del documento rispetti quanto descritto nel presente 
				documento.
			\paragraph{Norme tipografiche}
				Nel presente documento sono state definite delle norme tipografiche di carattere generale. Molte di esse sono verificabili tramite 
				strumenti automatici. Per le restanti è necessario che i Verificatori applichino una fra le tecniche di inspection e walkthrough 
				(si preferisce la prima nel momento in cui tramite la lista di controllo si riesce a trovare la grande maggioranza degli errori).
			\paragraph{Proprietà di glossario}
				È necessario che tutte le parole presenti all'interno del Glossario siano marcate appropriatamente (come descritto nel presente 
				documento). Ciò viene fatto con l'ausilio di strumenti automatici. Inoltre, è compito dei Verificatori trovare tutti quei termini che 
				dovrebbero essere inclusi nel Glossario ma che ancora non lo sono. Per tale compito si deve utilizzare walthrough delle sezioni che sono 
				state soggette a modifica prima dell'ultima verifica.
		\subsubsection{Verifica dei diagrammi UML}
			I Verificatori hanno il compito di controllare tutti i diagrammi UML che sono stati prodotti. I diagrammi utilizzati fino a questo momento 
			sono i seguenti:
			\begin{itemize}
				\item Diagrammi dei casi d'uso.
			\end{itemize}
			\paragraph{Diagrammi dei casi d'uso}
				Le verifiche che devono essere effettuate riguardano sostanzialmente i casi d'uso e gli attori.\\
				Innanzitutto si deve controllare se un attore è davvero tale. Per fare questo può essere utile porsi domande descritte in seguito.
				\begin{enumerate}
					\item Il presunto attore in questione è una persona che interagisce con il sistema? Se la risposta è si allora esso probabilmente 
					è un attore, se la risposta è no passare al punto due.
					\item Sono in grado di modificare il presunto attore all'interno del mio sistema? Se la risposta è no allora probabilmente è un attore, 
					se la risposta è si allora probabilmente non si tratta di un attore.
				\end{enumerate}
				Per quanto riguarda i casi d'uso, si controlla innanzitutto che siano utilizzate secondo lo standard UML le seguenti relazioni:
				\begin{itemize}
					\item inclusioni;
					\item generalizzazioni;
					\item estensioni
				\end{itemize}
				Infine, assicurarsi che i casi d'uso descrivano cosa il sistema fa e non cosa non fa.
	\subsection{Strumenti}
		
\level{2}{Processo di sviluppo}
		\level{3}{Studio di Fattibilità}
			Lo \insdoc{Studio di Fattibilità} deve essere rapido e accurato.\\
			Il documento di \insdoc{Studio di Fattibilità} viene redatto dagli analisti, sulla base di ciò che emerge dalle prime due riunioni. Qui i membri del gruppo si confrontano su tematiche riguardanti i capitolati quali:
			\begin{itemize}
				\item il dominio applicativo e tecnologico;
				\item il rapporto costi/benefici:
				\begin{itemize}
					\item confronto tra mercato attuale e futuro;
					\item costo di produzione e redditività dell'investimento;
				\end{itemize}
				\item i rischi ai quali essi sono soggetti.
			\end{itemize}

		\level{3}{Analisi dei Requisiti}
			Durante l'\insdoc{Analisi dei Requisiti} vengono catalogati e descritti tutti i requisiti che il prodotto dovrà possedere. Tali requisiti sono ottenuti da una delle seguenti fonti (presentate in ordine decrescente di importanza):
			\begin{enumerate}
				\item capitolato d’appalto;
				\item incontri con il proponente;
				\item incontri con il committente;
				\item valutazioni interne al gruppo di lavoro.
			\end{enumerate}
			Tali requisiti sono raccolti all'interno del documento \insdoc{Analisi dei Requisiti}. All'interno di tale documento è anche indicato un modo per verificare i requisiti in esso raccolti.
			\level{4}{Norme}
				\level{5}{Classificazione dei casi d'uso} \label{sec:classificazioneUC}
					I casi d'uso vengono identificati univocamente da una sigla nella forma UCYX.
					\begin{itemize}
						\item Y può assumere tre valori: N, A o D. N rappresenta tutti i casi d'uso inerenti a Norris, A rappresenta quelli inerenti l'applicazione Android e D rappresenta quelli che riguardano la Dashboard.
						\item X è un codice gerarchico numerico.
					\end{itemize}
					Sarà compito degli \insrole{analisti} identificare i casi d'uso e essi dovranno preoccuparsi di fornire un diagramma conforme allo standard UML per ogni caso d'uso non foglia. Qui di seguito sono riportate le informazioni da inserire nella parte testuale:
					\begin{itemize}
						\item titolo;
						\item attori (principali e secondari);
						\item precondizione;
						\item postcondizione;
						\item flusso principale degli eventi, dove si descrive il flusso dei casi d'uso figli; per ogni evento va specificato quanto segue:
						\begin{itemize}
							\item descrizione testuale;
							\item attori coinvolti;
							\item se l’evento è descritto nel dettaglio da un altro caso d’uso;
						\end{itemize}
						\item scenari alternativi; per ognuno di essi va specificato quanto segue:
						\begin{itemize}
							\item descrizione testuale;
							\item attori coinvolti;
							\item se lo scenario è descritto nel dettaglio da un altro caso d’uso.
						\end{itemize}
					\end{itemize}
				\level{5}{Classificazione dei requisiti}
					I requisiti dovranno essere classificati per tipo e importanza, utilizzando la seguente codifica: R[X][Y][Z].\\
					La codifica appena presentata deve essere interpretata nel modo seguente:
					\begin{enumerate}
						\item X indica l'importanza strategica del requisito e può assumere uno dei seguenti valori:
						\begin{itemize}
							\item R: requisiti obbligatori (required);
							\item D: requisiti desiderabili (desiderable);
							\item O: requisiti opzionali (optional).
						\end{itemize}
						\item Y indica la tipologia del requisito e può assumere i seguenti valori:
						\begin{itemize}
							\item F: indica che si tratta di un requisito funzionale (functional);
							\item P: indica che si tratta di un requisito prestazionale (performance);
							\item Q: indica che si tratta di un requisito di qualità (quality);
							\item C: indica che si tratta di un requisito di vincolo (constraint).
						\end{itemize}
						\item Z indica il codice gerarchico che identifica un requisito e dev'essere univoco.
					\end{enumerate}
	\level{4}{Strumenti}
		\level{5}{Software per la creazione di diagrammi UML}
			Per l'attività di Analisi dei Requisiti sono stati utilizzati i seguenti strumenti:
			\begin{itemize}
				\item per la creazione dei diagrammi UML è stato utilizzato il software Astah;
				\item per la classificazione dei recuisiti è stata utilizzata un'applicazione web creata dagli \insrole{Amministratori}.
			\end{itemize}
			Il software Astah è descritto nell'appendice \nameref{app:strumenti}, mentre l'applicazione per la classificzione dei requisiti è descritta nell'appendice \nameref{sec:tracker}.
			
	\level{3}{Progettazione}
Scopo dell'attività di progettazione è produrre una soluzione soddisfacente per tutti i soggetti coinvolti nel progetto, una volta compreso esattamente quali sono i requisiti del problema. \\
Approfondendo la progettazione del prodotto fino ad arrivare a moduli abbastanza semplici da essere capiti da una sola persona, si otterranno le istruzioni necessarie ai \insrole{Programmatori} per ottenere il prodotto finito.
			
\level{4}{Progettazione architetturale}
La progettazione ad alto livello dell'architettura e dei singoli componenti del framework \projectname{} viene descritta dai \insrole{Progettisti} nel documento \insdoc{Specifica Tecnica v1.00}, avendo cura di rispettare i seguenti aspetti.
\level{5}{Norme}
\level{6}{Design Pattern}
Al fine di consentire una migliore comprensibilità del documento, nonchè delle scelte progettuali e della progettazione stessa, i \insrole{Progettisti} dovranno indicare i pattern architetturali impiegati, fornendo per ciascuno di essi:
\begin{itemize}
\item una descrizione testuale e grafica;
\item la motivazione che ha portato alla sua scelta;
\item una breve descrizione dell'applicazione del pattern al progetto.
\end{itemize}
\level{6}{Diagrammi UML}
Per formalizzare ed esplicitare alcuni aspetti specifici del sistema è necessario utilizzare i seguenti diagrammi UML:
\begin{itemize}
\item Diagrammi dei package, per rappresentare l'aspetto generale del sistema.
\item Diagrammi delle classi, per rappresentare le componenti fondamentali dell'architettura.
\item Diagrammi delle attività, per rappresentare il flusso delle operazioni e dei componenti del sistema, qualora la sola descrizione testuale risulti complessa.
\item Diagrammi di sequenza, per rappresentare le interazioni tra le componenti del sistema, qualora la sola descrizione testuale risulti complessa.
\end{itemize}
Tali diagrammi dovranno seguire le indicazioni fornite nella sezione \nameref{sec:UML}.
				
\level{6}{Tracciamento delle componenti}
Per garantire la qualità del prodotto è fondamentale che ogni requisito sia soddisfatto da uno e un solo componente. Tale tracciamento viene generato automaticamente dall'applicazione per la gestione dei requisiti sotto forma di tabelle.
\level{6}{Test di integrazione}
I \insrole{Progettisti} devono definire delle classi di verifica per verificare che il funzionamento dei componenti sia quello previsto.
\level{5}{Strumenti}
Per la progettazione architetturale sono stati utilizzati i seguenti strumenti:
\begin{itemize}
\item per la creazione dei diagrammi UML è stato utilizzato Astah;
\item per il tracciamento delle componenti è stata utilizzata l'applicazione web sviluppata dagli \insrole{Amministratori}.
\end{itemize}
Il software Astah è descritto in modo dettagliato nell'appendice \nameref{app:strumenti}, metre l'applicazione web per il tracciamento viene descritta nell'appendice \nameref{sec:tracker}.


\level{4}{Progettazione di dettaglio}
La progettazione di dettaglio dei singoli componenti del framework \projectname{} viene descritta dai \insrole{Progettisti} nel documento \insdoc{Definizione di Prodotto v1.00}, avendo cura di rispettare i seguenti aspetti.
\level{5}{Norme}
\level{6}{Diagrammi UML}
Durante questa progettazione devono essere aggiornati con l'aggiunta di dettagli i seguenti diagrammi:
\begin{itemize}
\item diagrammi delle classi;
\item diagrammi di sequenza;
\item diagrammi delle attività.
\end{itemize}
Tali diagrammi dovranno seguire le indicazioni fornite nella sezione\nameref{sec:UML}.
\level{6}{Definizione di classe}
Ogni classe individuata dai \insrole{Progettisti} deve avere, oltre ad un proprio diagramma, una descrizione testuale contenente:
\begin{itemize}
\item il nome della classe completo dei package di cui fa parte;
\item la descrizione della classe con l'indicazione della sua funzione nel contesto del package a cui appartiene;
\item le sue relazioni con altre classi;
\item l'elenco dei metodi accompagnati da una breve ma significativa descrizione;
\item l'elenco degli attributi.
\end{itemize}
\level{6}{Test di unità}
I \insrole{Progettisti} devono definire i test di unità necessari per verificare che il funzionamento delle classi sia quello previsto.
\level{5}{Strumenti}
Per la progettazione di dettaglio è stato utilizzato il software Astah, descritto nell'appendice \nameref{app:strumenti}, per la creazione dei diagrammi UML.
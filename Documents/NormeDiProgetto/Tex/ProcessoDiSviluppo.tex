\level{2}{Processo di sviluppo}
		\level{3}{Studio di Fattibilità}
			Lo \insdoc{Studio di Fattibilità} deve essere rapido e accurato.\\
			Il documento di \insdoc{Studio di Fattibilità} viene redatto dagli analisti, sulla base di ciò che emerge dalle prime due riunioni. Qui i membri del gruppo si confrontano su tematiche riguardanti i capitolati quali:
			\begin{itemize}
				\item il dominio applicativo e tecnologico;
				\item il rapporto costi/benefici:
				\begin{itemize}
					\item confronto tra mercato attuale e futuro;
					\item costo di produzione e redditività dell'investimento;
				\end{itemize}
				\item i rischi ai quali essi sono soggetti.
			\end{itemize}

		\level{3}{Analisi dei Requisiti}
			Durante l'\insdoc{Analisi dei Requisiti} vengono catalogati e descritti tutti i requisiti che il prodotto dovrà possedere. Tali requisiti sono ottenuti da una delle seguenti fonti (presentate in ordine decrescente di importanza):
			\begin{enumerate}
				\item capitolato d’appalto;
				\item incontri con il proponente;
				\item incontri con il committente;
				\item valutazioni interne al gruppo di lavoro.
			\end{enumerate}
			Tali requisiti sono raccolti all'interno del documento \insdoc{Analisi dei Requisiti}. All'interno di tale documento è anche indicato un modo per verificare i requisiti in esso raccolti.
			\level{4}{Norme}
				\level{5}{Classificazione dei casi d'uso}
					I casi d'uso vengono identificati univocamente da una sigla nella forma UCYX.
					\begin{itemize}
						\item Y può assumere tre valori: N, A o D. N rappresenta tutti i casi d'uso inerenti a Norris, A rappresenta quelli inerenti l'applicazione Android e D rappresenta quelli che riguardano la Dashboard.
						\item X è un codice gerarchico numerico.
					\end{itemize}
					Sarà compito degli \insrole{analisti} identificare i casi d'uso e essi dovranno preoccuparsi di fornire un diagramma conforme allo standard UML per ogni caso d'uso non foglia. Qui di seguito sono riportate le informazioni da inserire nella parte testuale:
					\begin{itemize}
						\item titolo;
						\item attori (principali e secondari);
						\item precondizione;
						\item postcondizione;
						\item flusso principale degli eventi, dove si descrive il flusso dei casi d'uso figli; per ogni evento va specificato quanto segue:
						\begin{itemize}
							\item descrizione testuale;
							\item attori coinvolti;
							\item se l’evento è descritto nel dettaglio da un altro caso d’uso;
						\end{itemize}
						\item scenari alternativi; per ognuno di essi va specificato quanto segue:
						\begin{itemize}
							\item descrizione testuale;
							\item attori coinvolti;
							\item se lo scenario è descritto nel dettaglio da un altro caso d’uso.
						\end{itemize}
					\end{itemize}
				\level{5}{Classificazione dei requisiti}
					I requisiti dovranno essere classificati per tipo e importanza, utilizzando la seguente codifica: R[X][Y][Z].\\
					La codifica appena presentata deve essere interpretata nel modo seguente:
					\begin{enumerate}
						\item X indica l'importanza strategica del requisito e può assumere uno dei seguenti valori:
						\begin{itemize}
							\item R: requisiti obbligatori (required);
							\item D: requisiti desiderabili (desiderable);
							\item O: requisiti opzionali (optional).
						\end{itemize}
						\item Y indica la tipologia del requisito e può assumere i seguenti valori:
						\begin{itemize}
							\item F: indica che si tratta di un requisito funzionale (functional);
							\item P: indica che si tratta di un requisito prestazionale (performance);
							\item Q: indica che si tratta di un requisito di qualità (quality);
							\item C: indica che si tratta di un requisito di vincolo (constraint).
						\end{itemize}
						\item Z indica il codice gerarchico che identifica un requisito e dev'essere univoco.
					\end{enumerate}
	\level{4}{Strumenti}
		\level{5}{Software per creazione di diagrammi UML}
			Per l'attività di Analisi dei Requisiti sono stati utilizzati i seguenti strumenti:
			\begin{itemize}
				\item per la creazione dei diagrammi UML è stato utilizzato il software Astah;
				\item per la classificazione dei recuisiti è stata utilizzata un'applicazione web creata dagli \insrole{Amministratori}.
			\end{itemize}
			Il software Astah è descritto nell'appendice \nameref{app:strumenti}, mentre l'applicazione per la classificzione dei requisiti è descritta nella sezione \nameref{sec:tracker}.
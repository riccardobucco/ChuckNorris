\level{1}{Strumenti} \label{app:strumenti}
	\level{2}{Astah community}
		Per la modellazione di diagrammi \insglo{UML} (ad esempio diagrammi per use case) sono stati presi in considerazione soprattutto tre editor: Dia, Microsoft Visio, \insglo{Astah} community. Infine i membri del gruppo \groupname{} ha optato per adottare \insglo{Astah} community in quanto si tratta di un \insglo{software} \insglo{open source}, con supporto di \insglo{UML} 2.x e secondo una veloce analisi del \insglo{team} si è notato che esso ha un interfaccia più responsiva ed intuitiva degli altri \insglo{software}.
	\level{2}{\LaTeX{}}
		Per la stesura dei documenti si è scelto di utilizzare il \insglo{linguaggio di markup} \LaTeX{}.  Il motivo principale che ha portato a questa scelta è la facilità di separazione tra contenuto e formattazione. Con \LaTeX{} è possibile scrivere testo senza doversi preoccupare di dare a esso una formattazione coerente con le norme: è infatti possibile definire un unico template, valido per qualsiasi documento, che applica le regole tipografiche in automatico.
		\level{3}{Template}
			Ogni documento dovrà essere realizzato a partire da un template \LaTeX{} appropriato reperibile al percorso Commons/Template.\\
			Grazie ai template di \LaTeX{}, non ci si deve preoccupare della formattazione del testo durante la sua stesura: infatti, basta impostare le regole all’interno di un unico file e queste verranno adottate in automatico all’interno di tutti i documenti che fanno uso del template. Inoltre, se le regole dovessero cambiare in corso d’opera (cambiamenti alle \insdoc{Norme di Progetto}), basta modificare il singolo file contenente le regole invece che tutti i documenti scritti fino a quel determinato momento.
		\level{3}{Comandi personalizzati}
			Per permettere un’adeguazione automatica alle regole imposte all’interno delle norme tipografiche si è deciso di creare dei comandi personalizzati utilizzabili in \LaTeX{}. Ogni membro del gruppo dovrà saperli usare in modo appropriato ogniqualvolta ve ne sia il bisogno.\\
			L’elenco completo di tutti i comandi lo si può trovare al punto \nameref{sec:formatiricorrenti};
	\level{2}{Strumenti per la rilevazione di errori ortografici}
		Per la rilevazione e la correzione di errori ortografici viene usato uno \insglo{script} creato dal gruppo \groupname{} che si appoggia al \insglo{software} \insglo{Aspell}, opportunamente configurato col dizionario di lingua italiana.
	\level{2}{Repository}\label{sec:SceltaRepository}
		Per la gestione del \insglo{repository} si è scelto di utilizzare il sistema di versionamento distribuito \insglo{Git}. Questa scelta è stata presa in considerazione del fatto che questa piattaforma permette, oltre al servizio di \insglo{repository}, il controllo di versione e la gestione di anomalie. \\
		\insglo{GitHub} è anche stato suggerito dal proponente del \insglo{capitolato}, in quanto permette un'interazione agevolata con \insglo{Heroku}, altro strumento consigliato per lo svolgimento del progetto. 
		\level{3}{Visibilità del repository}
			Per la condivisione e il versionamento dei \textit{configuration item} è stato creato un \insglo{repository} privato su \insglo{GitHub}, raggiungibile all’indirizzo \insuri{https://github.com/marcorubin/kaizen-team}. L’accesso è consentito solamente agli utenti approvati dal \insrole{Project Manager}.
	\level{2}{Script di verifica}
		In questa sezione vengono descritti gli aspetti essenziali degli strumenti che i \insrole{Verificatori} hanno a disposizione.
		\begin{description}
			\item[OrtographicCheck] Questo \insglo{script} sfrutta il \insglo{software} GNU \insglo{Aspell} per eseguire il controllo ortografico di un documento.
			\item[Gulpease] Questo \insglo{script} estrae il testo da un documento e ne calcola l'indice di leggibilità Gulpease.
			\item[NonBreakingSpaceCheck] Questo \insglo{script} controlla che un numero formattato secondo lo standard [SI/ISO 31-0] utilizzi lo spazio unificatore e non quello normale (errore in cui è facile incorrere).
		\end{description}
	\level{2}{Android Studio}
		\insglo{Android} Studio è un \insglo{IDE}, basato su IntelliJ IDEA di JetBrains, nato appositamente per agevolare il programmatore nello sviluppo di applicazioni sulla piattaforma \insglo{Android} attraverso tool per la realizzazione della grafica e template per la definizione di funzioni ricorrenti; grazie al designer, ad esempio, il programmatore è in grado di creare le \insglo{Activity} con estrema facilità,  senza doverle definire manualmente in \insglo{XML}.\\
		Attualmente è il più popolare ambiente di sviluppo integrato utilizzato per realizzare applicazioni, insieme all'\insglo{IDE} Eclipse, che però richiede l'installazione di appositi plug-in.\\ 
		Il linguaggio utilizzato è \insglo{Java}, ampliato con le librerie \insglo{Android}.\\
		Questo strumento fornisce:
		\begin{itemize}
			\item un sistema flessibile per l'automazione dello sviluppo basato su Gradle;
			\item un editor WYSIWYG utile per la realizzazione delle interfacce grafiche;
			\item un editor di \insglo{layout} con funzioni drag-and-drop, che permette di creare facilmente il \insglo{layout} delle applicazioni;
			\item template per facilitare la creazione di caratteristiche comuni nelle applicazioni;
			\item strumenti per monitorare le performance e risolvere problemi di usabilità e compatibilità di versione;
			\item supporto per la creazione di applicazione per la piattaforma \insglo{Android} Wear;
			\item supporto integrato a Google Cloud Platform, Google Cloud Messaging e App Engine.
		\end{itemize}
	\level{2}{Heroku}	
		\insglo{Heroku} è una piattaforma cloud, basata sul sistemi operativi Debian o Ubuntu, progettata per realizzare e distribuire applicazioni online. \\
		In particolare \insglo{Heroku} implementa il modello cloud Platform as a service, pertanto viene messo a disposizione degli utenti un sistema informatico, completo di sistema operativo, ambiente di esecuzione per diversi linguaggi di programmazione, spazio per il \insglo{database} e un web \insglo{server}, sul quale è possibile sviluppare una qualunque applicazione che sfrutti il cloud computing.\\
		Fra i significativi vantaggi di \insglo{Heroku} rientrano il supporto dei linguaggi di programmazione più diffusi in ambito web (\insglo{Java}, \insglo{Node.js}, Scala, Ruby, Clojure, Python e PHP) e la possibilità di importare caricare progetti risiedenti su \insglo{Git} all'interno di \insglo{Heroku}.
	\level{2}{Software per i test}
		Per l'esecuzione dei test di Norris è stato utilizzato \textbf{Mocha}, un framework JavaScript che può essere utilizzato come modulo di node.js. L'esecuzione dei test è accompagnata da accurati report. Mocha permette l'utilizzo di una qualsiasi libreria di asserzioni\footnote{Una libreria che permette di fare delle vere e proprie asserzioni durante la programmazione, utilile durante la creazione dei test}: nel nostro caso si è scelto di utilizzare il modulo \texttt{assert} presente in node.js.\\
		Per quanto riguarda l'eseccuzione dei test di Chuck si è optato per \textbf{Jasmine}. Anch'esso è un framework JavaScript utile all'attività di testing. La scelta è ricaduta su di esso in quanto fortemente consigliato dai creatori di Angular.js.\\
		Infine, per testare l'Applicazione Android si è scelto l'ambiente \textbf{JUnit}. Esso è un framework ben integrato con l'SDK Android, e fornito direttamente dagli sviluppatori stessi. Esso permette la scrittura semiautomatica di buona parte del codice di cui è composto un test.
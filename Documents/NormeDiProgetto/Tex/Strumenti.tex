\level{1}{Strumenti} \label{app:strumenti}
\level{2}{Astah community}
Per la modellazione di diagrammi UML (ad esempio diagrammi per use case) sono stati presi in considerazione soprattutto tre editor: Dia, Microsoft Visio, Astah community. Infine i membri del gruppo \groupname{} ha optato per adottare Astah community in quanto si tratta di un software open source, con supporto di UML 2.x e secondo una veloce analisi del team si è notato che esso ha un interfaccia più responsiva ed intuitiva degli altri software.
\level{2}{\LaTeX{}}
	Per la stesura dei documenti si è scelto di utilizzare il linguaggio di markup \LaTeX{}.  Il motivo principale che ha portato a questa scelta è la facilità di separazione tra contenuto e formattazione. Con \LaTeX{} è possibile scrivere testo senza doversi preoccupare di dare a esso una formattazione coerente con le norme: è infatti possibile definire un unico template, valido per qualsiasi documento, che applica le regole tipografiche in automatico.
		\level{3}{Template}
				Ogni documento dovrà essere realizzato a partire da un template \LaTeX{} appropriato reperibile al percorso Commons/Template.\\
				Grazie ai template di \LaTeX{}, non ci si deve preoccupare della formattazione del testo durante la sua stesura: infatti, basta impostare le regole all’interno di un unico file e queste verranno adottate in automatico all’interno di tutti i documenti che fanno uso del template. Inoltre, se le regole dovessero cambiare in corso d’opera (cambiamenti alle \insdoc{Norme di Progetto}), basta modificare il singolo file contenente le regole invece che tutti i documenti scritti fino a quel determinato momento.
		\level{3}{Comandi personalizzati}
				Per permettere un’adeguazione automatica alle regole imposte all’interno delle norme tipografiche si è deciso di creare dei comandi personalizzati utilizzabili in \LaTeX{}. Ogni membro del gruppo dovrà saperli usare in modo appropriato ogniqualvolta ve ne sia il bisogno.\\
				L’elenco completo di tutti i comandi lo si può trovare al punto \nameref{sec:formatiricorrenti};
\level{2}{Strumenti per la rilevazione di errori ortografici}
			Per la rilevazione e la correzione di errori ortografici viene usato uno script creato dal gruppo \groupname{} che si appoggia al software Aspell, opportunamente configurato col dizionario di lingua italiana.
\level{2}{Repository}\label{sec:SceltaRepository}
				Per la gestione del repository si è scelto di utilizzare il sistema di versionamento distribuito Git. Questa scelta è stata presa in considerazione del fatto che questa piattaforma permette, oltre al servizio di repository, il controllo di versione e la gestione di anomalie. \\
				GitHub è anche stato suggerito dal proponente del capitolato, in quanto permette un'interazione agevolata con Heroku, altro strumento consigliato per lo svolgimento del progetto. 
			\level{3}{Visibilità del repository}
					Per la condivisione e il versionamento dei \textit{configuration item} è stato creato un repository privato su GitHub, raggiungibile all’indirizzo \insuri{https://github.com/marcorubin/kaizen-team}. L’accesso è consentito solamente agli utenti approvati dal \insrole{Project Manager}.
					
\level{2}{Script di verifica}
			In questa sezione vengono descritti gli aspetti essenziali degli strumenti che i \insrole{Verificatori} hanno a disposizione.
			\begin{description}
				\item[OrtographicCheck] Questo script sfrutta il software GNU Aspell per eseguire il controllo ortografico di un documento.
				\item[Gulpease] Questo script estrae il testo da un documento e ne calcola l'indice di leggibilità Gulpease.
				\item[NonBreakingSpaceCheck] Questo script controlla che un numero formattato secondo lo standard [SI/ISO 31-0] utilizzi lo spazio unificatore e non quello normale (errore in cui è facile incorrere).
			\end{description}

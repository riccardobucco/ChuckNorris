\level{2}{Processo di Gestione}
	\level{3}{Norme}
		\level{4}{Comunicazione}
			\level{5}{Comunicazioni interne}
				Per  le comunicazioni interne si utilizza una \insglo{mailing list} appositamente creata: mailinglist@kaizenteam.it.\\
				I componenti del gruppo utilizzano tale indirizzo per comunicare tra di loro, risultando così sempre aggiornati su ogni scambio di informazioni. Tale sistema può essere utilizzato esclusivamente per questioni riguardanti il progetto.\\
				Per facilitare la comunicazione il gruppo si avvale anche di sistemi di \textit{instant messaging} quali:
				\begin{itemize}
					\item Telegram;
					\item Chat di \insglo{Facebook}.
				\end{itemize}
				Quando si utilizzano i sistemi sopra citati si ha l'obbligo di stilare un verbale qualora siano state prese decisioni di importanza rilevante. Tale verbale deve essere caricato nel \insglo{repository} e i membri del gruppo devono essere avvisati delle decisioni tramite mail.
			\level{5}{Comunicazioni esterne}
				Il \insrole{Responsabile di progetto} ha l’incarico di mantenere i contatti tra il \insglo{team} e le componenti esterne, utilizzando una casella di posta elettronica dedicata: info@kaizenteam.it.\\
				E' inoltre suo compito informare tutti i membri del gruppo delle discussioni da lui avute con componenti esterni al gruppo. Questo viene fatto inviando una mail che riassume la conversazione per mezzo della \insglo{mailing list} ufficiale.
			\level{5}{Comunicazione tramite e-mail}
				In questa sezione viene definito il modo in cui le \textit{e-mail} devono essere scritte sia nelle comunicazioni interne sia in quelle esterne.
				\level{6}{Oggetto}
					L’oggetto di una \textit{e-mail} deve essere chiaro e conciso. Possibilmente, inoltre, esso deve essere diverso da oggetti utilizzati in precedenza.\\
					Per comporre un messaggio di risposta è necessario anteporre all’oggetto il prefisso “Re:”, per inoltrare è obbligatorio anteporre “I:”. Non va invece mai modificata la rimanente parte del testo dell'oggetto, sia in caso di risposta che in caso di inoltro.
				\level{6}{Corpo}
					Il corpo di un messaggio deve essere esaustivo e facilmente comprensibile a tutti i componenti del \insglo{team}.\\
					In caso di risposta o di inoltro, la parte aggiunta deve essere posta all’inizio, mantenendo anche il messaggio precedente, per non obbligare i membri a dover scorrere l’\textit{e-mail} completa e per permettere la visione intera della conversazione.
				\level{6}{Allegati}
					È preferibile non fare uso di allegati. Qualora vi sia la necessità di inviare documenti e/o file, questi dovrebbero essere caricati su \insglo{Dropbox}. Nella \textit{e-mail} si dovrebbe solo indicare il \textit{link} al quale è possibile trovare il file desiderato.
		\level{4}{Riunioni}
			\level{5}{Tipologia di riunioni}
				Le riunioni che si svolgono possono essere di vario tipo:
				\begin{itemize}
					\item interne (sono presenti solo membri del gruppo). Esse, a loro volta, si suddividono in:
					\begin{itemize}
						\item ordinarie, ovvero indette dal Responsabile di Progetto;
						\item straordinarie, ovvero richieste da un componente del \insglo{team} diverso dal Responsabile di Progetto;
					\end{itemize}
					\item esterne (sono presenti anche il proponente o il committente).
				\end{itemize}
				Le riunioni che non compaiono nella lista precedente non vengono regolamentate dal presente documento in quanto sono di carattere informale.
			\level{5}{Convocazione di una riunione}
				\level{6}{Riunione interna}
					In generale, il compito di indire le riunioni spetta unicamente al Responsabile di Progetto (riunioni ordinarie), che può esercitare tale diritto ogniqualvolta lo ritiene opportuno.\\
					Gli altri membri del \insglo{team} possono richiedere la convocazione di riunioni straordinarie. Il Responsabile di Progetto può:
					\begin{itemize}
						\item autorizzare lo svolgimento della riunione richiesta;
						\item negare lo svolgimento della riunione richiesta qualora non la ritenga importante per tutto il \insglo{team} o per lo stato di avanzamento dei lavori;
						\item suggerire una data e mezzi di comunicazione diversi da quelli proposti.
					\end{itemize}
					Per qualsiasi tipo di riunione, i membri del \insglo{team} devono comunicare tempestivamente la loro disponibilità a partecipare allo scopo di consentire al Responsabile di rimandare eventualmente l'incontro. I membri del \insglo{team} possono manifestare le loro preferenze in merito alla data, all'ora e al luogo dell'incontro: la decisione finale, in ogni caso, spetta al Responsabile di Progetto\\
					Sia per la convocazione delle riunioni ordinarie che di quelle straordinarie il Responsabile di Progetto deve comunicare la data, l'ora e il luogo ai membri del \insglo{team} con almeno tre giorni di preavviso utilizzando la \insglo{mailing list}.\\
					La mail inviata dovrà essere così strutturata:
					\begin{itemize}
						\item oggetto: Convocazione della riunione [ordinaria/straordinaria] numero N per il giorno AAAA-mm-GG
						\item corpo:
						\begin{itemize}
							\item data: data e ora prevista;
							\item luogo: luogo previsto;
							\item tipo: ordinaria/straordinaria;
							\item ordine del giorno.
						\end{itemize}
					\end{itemize}
				\level{6}{Riunione esterna}
					Spetta al \insrole{Responsabile di Progetto} accordare una data, un orario e un luogo con il proponente e/o il committente per una riunione. Egli deve fare in modo, per quanto possibile, che possano essere presenti alla riunione la maggior parte dei membri del \insglo{team}.
			\level{5}{Gestione delle riunioni}
				\level{6}{Riunioni interne}
					All'inizio degli incontri viene nominato un Segretario. Esso ha il compito di verificare la presenza dei membri del \insglo{team} e di annotare gli argomenti affrontati. Una volta terminato l'incontro, esso deve redigere il verbale, il quale viene archiviato nel \insglo{repository} per la libera consultazione da parte del gruppo.\\
					I partecipanti alle riunioni devono tenere un atteggiamento che favorisca la discussione di tutti i punti dell'ordine del giorno individuati.
				\level{6}{Riunioni esterne}
					I contenuti delle riunioni esterne sono di grande importanza. E' quindi necessaria una trascrizione quanto più fedele possibile di quanto viene detto. A causa di ciò si deve chiedere ai membri esterni al gruppo la possibilità di registrare la conversazione, in modo tale da conservare quanto viene detto. Alla fine della riunione si affida il compito di trascrivere i contenuti salienti a una persona interna al gruppo.\\
					Sia che la registrazione venga permessa dai membri esterni sia in caso contrario, è auspicabile che i componenti del \insglo{team} adottino dei mezzi per poter avere a fine riunione una trascrizione quanto più fedele possibile dei contenuti più importanti della conversazione (esempio: prendere appunti su quanto viene detto di volta in volta).
		\level{4}{Ruoli di progetto}
			Per lo svolgimento di un progetto è necessaria la presenza di figure professionali che rivestono ruoli ben precisi. I componenti del gruppo ricoprono, in momenti diversi, ruoli diversi, in modo da permettere a ciascuno di ricoprire ogni ruolo almeno un volta. E' necessario garantire che il ruolo di ciascun membro del gruppo non sia in conflitto con il ruolo che ha ricoperto in passato.
			\level{5}{Responsabile di Progetto}
				Il Responsabile rappresenta il progetto presso il fornitore e presso il committente. Egli accentra su di sé le responsabilità di scelta e approvazione. In particolare ha responsabilità su:
				\begin{itemize}
					\item pianificazione, controllo e coordinamento delle attività;
					\item assegnazione delle attività a persone;
					\item gestione delle risorse;
					\item relazioni esterne, in quanto rappresenta il gruppo e sa, in ogni momento, in quale stato di avanzamento ci si trova;
					\item convocazione di riunioni interne;
					\item verifica dei risultati;
					\item approvazione dei documenti;
					\item analisi e gestione dei fattori di rischio.
				\end{itemize}
				Il \insrole{Responsabile} si occupa inoltre di redigere il \insdoc{Piano di Progetto} e collabora alla stesura del \insdoc{Piano di Qualifica}, in particolare nella sezione relativa alla pianificazione.
			\level{5}{Amministratore di Sistema}
				L'\insrole{Amministratore} si occupa del controllo e della gestione dell'ambiente di lavoro. Nel dettaglio si preoccupa di:
				\begin{itemize}
					\item equipaggiare l'ambiente di lavoro con strumenti, procedure, infrastrutture e servizi a supporto dei processi che permettano di automatizzare il più possibile le attività o una parte di esse;
					\item garantire che l’ambiente di lavoro sia sempre completo, ovvero dotato di tutti gli strumenti necessari, ordinato e aggiornato;
					\item controllare versioni e configurazioni del \insglo{prodotto};
					\item gestire la documentazione di progetto;
					\item garantire e controllare la disponibilità e la diffusione della documentazione di progetto;
					\item fornire procedure e strumenti per il monitoraggio e segnalazione per il controllo qualità;
					\item risolvere problemi legati alla gestione dei processi tramite l'adozione di strumenti adatti.
				\end{itemize}
				L'\insrole{Amministratore} si occupa inoltre di redigere le \insdoc{Norme di Progetto} e la sezione del \insdoc{Piano di Qualifica} dove vengono descritti strumenti e metodi di verifica. Nel fare tutto ciò, l'\insrole{Amministratore} non compie scelte gestionali, ma attua scelte tecnologiche concordate con il \insrole{Responsabile di Progetto}.
			\level{5}{Analista}
				L'\insrole{Analista} è responsabile delle attività di analisi. Si occupa di:
				\begin{itemize}
					\item capire appieno il problema tramite l'analisi dei bisogni e delle fonti;
					\item classificare i requisiti;
					\item redigere i diagrammi dei casi d'uso;
					\item assegnare i requisiti a parti distinte del sistema;
					\item assicurarsi che i requisiti trovati siano conformi alle richieste del committente.
				\end{itemize}
				L'\insrole{Analista} non porta la soluzione, ma definisce il problema redigendo lo \insdoc{Studio di Fattibilità} e l'\insdoc{Analisi dei Requisiti}. Partecipa alla redazione del \insdoc{Piano di Qualifica} in quanto conosce l’ambito del progetto ed ha chiari i livelli di qualità richiesta e le procedure da applicare per ottenerla.
			\level{5}{Progettista}
				Il \insrole{Progettista} si occupa della progettazione. Ha responsabilità di scelta e decisione nei seguenti ambiti:
				\begin{itemize}
					\item produzione della descrizione di una soluzione soddisfacente per tutti gli \insglo{stakeholder};
					\item definizione dell'architettura del \insglo{prodotto} in modo tale che sia realizzabile con risorse date e costi prefissati;
					\item applicazione al \insglo{prodotto} di soluzioni note ed ottimizzate;
					\item applicazione al \insglo{prodotto} di soluzioni che lo rendano facilmente manutenibile;
					\item soddisfacimento dei requisiti con un sistema di qualità.
				\end{itemize}
				Il \insrole{Progettista} si occupa della redazione della \insdoc{Specifica Tecnica}, della \insdoc{Definizione di Prodotto} e delle sezioni del \insdoc{Piano di Qualifica } inerenti le metriche di verifica della programmazione.
			\level{5}{Programmatore}
				Il \insrole{Programmatore} si occupa delle attività di codifica. In particolare ha i seguenti compiti:
				\begin{itemize}
					\item deve seguire in modo rigoroso le soluzioni descritte dal progettista;
					\item deve provvedere a documentare il codice scritto;
					\item deve realizzare i test per la verifica e la validazione del codice stesso.
				\end{itemize}
				Il \insrole{Programmatore}, infine, si occupa della redazione del \insdoc{Manuale Utente}.
			\level{5}{Verificatore}
				Il \insrole{Verificatore} si occupa delle attività di verifica. Nello specifico le sue funzioni sono:
				\begin{itemize}
					\item assicurare che ogni attività venga svolta secondo le norme stabilite;
					\item verificare che ogni stadio del ciclo di vita del \insglo{prodotto} sia conforme.
				\end{itemize}
				Il verificatore redige la sezione del \insdoc{Piano di Qualifica} che illustra l’esito e la completezza delle verifiche effettuate.
		\level{4}{Rotazione ruoli}
			Ogni componente del gruppo dovrà svolgere ogni ruolo durante lo svolgimento del progetto. Nell'assegnare il ruolo ad ogni risorsa
			il \insrole{Project Manager} dovrà rispettare le seguenti regole:
			\begin{itemize}
				\item Evitare che ci siano conflitti d'interesse. Ovvero non potrà verificarsi che un membro svolga il ruolo di \insrole{verificatore}
				su qualcosa da lui redatto;
				\item La divisione dei ruoli in ore dovrà esser equa. Ciò significa che ognuno svolgerà un determinato ruolo per un tempo il più
				uguale possibile agli altri;
				\item Bisognerà evitare cambi di contesto frequenti. Ogni ruolo dovrà esser assegnato per il maggior tempo possibile ad ogni risorsa.
			\end{itemize}
		\level{4}{Aggiornamento dello stato dei ticket} \label{subsec:aggiornamentoStatoTicket}
			Per facilitare l'attività di organizzazione e supervisione del \insrole{Responsabile di Progetto} è necessario che ogni membro del \insglo{team} di sviluppo consulti quotidianamente la piattaforma TeamWork, accedendo al sito dell'applicazione o attivando l'invio delle notifiche, per sapere quali attività dovrà svolgere. Questo obbligo viene imposto anche per mettere a conoscenza tutto il gruppo delle attività in corso.\\
			Oltre a informarsi su quali compiti sono stati loro assegnati, però, è importante che il \insglo{team} segnali lo stato di avanzamento delle attività in corso. Ciò è possibile utilizzando il semplice meccanismo messo a disposizione da TeamWork, indicato nella guida per TeamWork presente in \hyperref[sec:teamwork]{Appendice~\ref*{sec:teamwork}}
	\level{3}{Strumenti}
		\level{4}{Ticketing di pianificazione}
			Per la gestione dei \insglo{ticket} di pianificazione si è scelto di utilizzare TeamWork perché permette:
			\begin{itemize}
				\item un facile ed intuitivo utilizzo;
				\item impostare \insglo{milestone} sul calendario;
				\item possiede un'applicazione mobile che permette una completa gestione e notifiche push per scadenze;
				\item permette la notifica automatica via email quando viene assegnato un task ad una risorsa;
				\item permette di assegnare una priorità ad ogni task;
				\item include un servizio di segnalazione dei rischi;
				\item la creazione automatica del \insglo{diagramma di Gantt} con possibilità di esportazione per il programma \insglo{GanttProject};
				\item permette un report automatico giornaliero delle attività svolte ed in ritardo inviato tramite email.
			\end{itemize}
			Sono stati valutati altri \insglo{software} come ad esempio Redmine, il quale fu ritenuto quasi altrettanto completo ed intuitivo. Tuttavia si è optato per TeamWork data la sua estrema semplicità di utilizzo.\\
		\level{4}{Creazione dei diagrammi di Gantt}
	Per la creazione dei diagrammi di Gantt è stato adottato il \insglo{software} \insglo{GanttProject}, \insglo{software} \insglo{open source} basato su piattaforma \insglo{Java}. Qui di seguito vengono elencate le principali caratteristiche che hanno portato alla scelta di questo strumento:
	\begin{itemize}
		\item portabilità, essendo un \insglo{software} basato su \insglo{Java};
		\item open-source;
		\item compatibile con \insglo{Microsoft Project};
		\item può generare grafici Work Breakdown Structure (WBS);
		\item fornisce la possibilità di creare diagrammi di Gantt;
		\item può generare grafici Program Evalutation and Review Tecnique (\insglo{PERT});
		\item permette di esportare tutto o solo una parte di grafico in formato \insglo{PDF}, \insglo{PNG}, HTML, CSV, e molti altri.
		\item in grado di gestire e generare grafici delle risorse assegnate.
	\end{itemize}

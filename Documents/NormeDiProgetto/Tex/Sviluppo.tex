\section{Sviluppo}
	\subsection{Analisi dei Requisiti}
		\subsubsection{Attività}
			\paragraph{Studio di Fattibilità}
				Lo \insdoc{Studio di Fattibilità} deve essere rapido e accurato.\\
				Il documento di \insdoc{Studio di Fattibilità} viene redatto dagli analisti, sulla base di ciò che emerge dalle prime due riunioni. Qui i membri del gruppo si confrontano su tematiche riguardanti i capitolati quali:
				\begin{itemize}
					\item il dominio applicativo e tecnologico;
					\item il rapporto costi/benefici:
					\begin{itemize}
						\item confronto tra mercato attuale e futuro;
						\item costo di produzione e redditività dell'investimento;
					\end{itemize}
					\item i rischi ai quali essi sono soggetti.
				\end{itemize}
			\paragraph{Analisi dei Requisiti}
				Durante l'\insdoc{Analisi dei Requisiti} vengono catalogati e descritti tutti i requisiti che il prodotto dovrà possedere. Tali requisiti sono ottenuti da una delle seguenti fonti (presentate in ordine decrescente di importanza):
				\begin{enumerate}
					\item capitolato d’appalto;
					\item incontri con il proponente;
					\item incontri con il committente;
					\item valutazioni interne al gruppo di lavoro.
				\end{enumerate}
				Tali requisiti sono raccolti all'interno del documento \insdoc{Analisi dei Requisiti}. All'interno di tale documento è anche indicato un modo per verificare i requisiti in esso raccolti.
		\subsubsection{Norme}
			\paragraph{Classificazione dei casi d'uso}
				I casi d'uso vengono identificati univocamente da una sigla nella forma UCYX.
				\begin{itemize}
					\item Y può assumere tre valori: N, A o D. N rappresenta tutti i casi d'uso inerenti a Norris, A rappresenta quelli inerenti l'applicazione Android e D rappresenta quelli che riguardano la Dashboard.
					\item X è un codice gerarchico numerico.
				\end{itemize}
				Sarà compito degli \insrole{analisti} identificare i casi d'uso e essi dovranno preoccuparsi di fornire un diagramma conforme allo standard UML per ogni caso d'uso non foglia. Quì di seguito sono riportate le informazioni da inserire nella parte testuale:
				\begin{itemize}
					\item titolo;
					\item attori (principali e secondari);
					\item precondizione;
					\item postcondizione;
					\item flusso principale degli eventi, dove si descrive il flusso dei casi d'uso figli; per ogni evento va specificato quanto segue:
					\begin{itemize}
						\item descrizione testuale;
						\item attori coinvolti;
						\item se l’evento è descritto nel dettaglio da un altro caso d’uso;
					\end{itemize}
					\item scenari alternativi; per ognuno di essi va specificato quanto segue:
					\begin{itemize}
						\item descrizione testuale;
						\item attori coinvolti;
						\item se lo scenario è descritto nel dettaglio da un altro caso d’uso.
					\end{itemize}
				\end{itemize}
			\paragraph{Classificazione dei requisiti}
				I requisiti dovranno essere classificati per tipo e importanza, utilizzando la seguente codifica: R[X][Y][Z].\\
				La codifica appena presentata deve essere interpretata nel modo seguente:
				\begin{enumerate}
					\item X indica l'importanza strategica del requisito e può assumere uno dei seguenti valori:
					\begin{itemize}
						\item 1: requisiti obbligatori;
						\item 2: requisiti desiderabili;
						\item 3: requisiti opzionali.
					\end{itemize}
					\item Y indica la tipologia del requisito e può assumere i seguenti valori:
					\begin{itemize}
						\item F: indica che si tratta di un requisito funzionale;
						\item P: indica che si tratta di un requisito prestazionale;
						\item Q: indica che si tratta di un requisito di qualità;
						\item V: indica che si tratta di un requisito di vincolo.
					\end{itemize}
					\item Z indica il codice gerarchico che identifica un requisito e dev'essere univoco.
				\end{enumerate}
		\subsubsection{Strumenti}
			\paragraph{Software per creazione di diagrammi UML}
			Per la modellazione di diagrammi UML (ad esempio diagrammi per use case) sono stati presi in considerazione soprattutto tre editor: Dia, Microsoft Visio, Astah community. Infine i membri del gruppo \groupname{} ha optato per adottare Astah community in quanto si tratta di un software open source, con supporto di UML 2.x e secondo una veloce analisi del team si è notato che esso ha un interfaccia più responsiva ed intuitiva degli altri software.
			\paragraph{Strumento per il tracciamento dei requisiti}
			Per mantenere una corretta catalogazione e implementare un corretto tracciamento dei requisiti, gli \insrole{Amministratori} hanno sviluppato un'apposita applicazione web. È possibile trovare informazioni più dettagliate in \autoref{sec:tracker}.
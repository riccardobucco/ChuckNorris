% !TEX encoding = UTF-8 Unicode
\section{Introduzione}

	\subsection{Scopo del documento}
		Tale documento ha lo scopo di pianificare il modo e i tempi in cui verranno svolte le attività dai membri del gruppo \groupname{}.\\
		In particolare, gli scopi di tale documento sono:
		\begin{itemize}
			\item descrivere il ciclo di vita scelto per il prodotto;
			\item fissare le varie fasi di sviluppo esplicandone scopo e dettagli;
			\item analizzare e gestire gli eventuali rischi;
			\item descrivere cosa dovrà esser fatto nelle varie fasi di sviluppo;
			\item preventivare l'impegno delle risorse;
			\item calcolare il consuntivo di utilizzo delle risorse durante lo svolgimento del progetto.
		\end{itemize}
	
	\level{2}{Glossario}
	Allo scopo di rendere più semplice la comprensione dei documenti ed evitare eventuali ambiguità, viene allegato il \insdoc{Glossario v6.00}, che contiene la spiegazione della terminologia tecnica e degli acronimi utilizzati. Per facilitare la lettura, i termini presenti all'interno di tale documento saranno marcati da una “G” maiuscola a pedice.
	

	\subsection{Riferimenti utili}
		\subsubsection{Glossario}
		Al fine di evitare ogni genere di ambiguità relativa al linguaggio o ai termini utilizzati nei documenti, si allega il documento \insdoc{Glossario v1.0}.
		\subsubsection{Riferimenti normativi}
			\begin{itemize}
				\item\textbf{Capitolato d'appalto C3:} \projectname{}: Real-time Business Intelligence. Reperibile all'indirizzo: \insuri{http://www.math.unipd.it/~tullio/IS-1/2014/Progetto/C3.pdf}
				\item\textbf{Norme di Progetto:} \insdoc{Norme Di Progetto v1.0}.
			\end{itemize}
		\subsubsection{Riferimenti informativi}
			\begin{itemize}
				\item \textbf{Analisi dei Requisiti:} \insdoc{Analisi Dei Requisiti v1.00};
				\item \textbf{Piano di Qualifica:} \insdoc{Piano Di Qualifica v1.00};
				\item \textbf{Studio di Fattibilità:} \insdoc{Studio Di Fattibilità v1.00};
				\item \textbf{Software Engineering - Ian Sommerville - 9th Edition (2010):} Part 4 - Software Management;
				\item \textbf{Slide dell’insegnamento Ingegneria del Software modulo A:} Il ciclo di vita del software, gestione di progetto.
			\end{itemize}
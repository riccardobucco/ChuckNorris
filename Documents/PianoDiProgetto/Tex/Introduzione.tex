% !TEX encoding = UTF-8 Unicode
\section{Introduzione}
	\subsection{Scopo del documento}
		Tale documento ha lo scopo di pianificare il modo e i tempi in cui verranno svolte le attività dai membri del gruppo \groupname{}. \\In particolare, gli scopi di tale documento sono:
		\begin{itemize}
			\item descrivere il ciclo di vita scelto per il prodotto;
			\item fissare le varie fasi di sviluppo esplicandone scopo e dettagli;
			\item analizzare a gestire gli eventuali rischi;
			\item descrivere cosa dovrà esser fatto nelle varie fasi di sviluppo;
			\item preventivare l'impegno delle risorse;
			\item calcolare il consuntivo di utilizzo delle risorse durante lo svolgimento del progetto.
		\end{itemize}
	\subsection{Riferimenti utili}
		\subsubsection{Glossario}
		Al fine di evitare ogni genere di ambiguità relativa al linguaggio o ai termini utilizzati nei documenti, si allega il documento \insfile{Glossario v1.0}.
		\subsubsection{Normativi}
			\begin{itemize}
				\item\textbf{Capitolato D'Appalto C3}: \projectname{}: Real-time Business Intelligence. Reperibile all'indirizzo: \insuri{http://www.math.unipd.it/~tullio/IS-1/2014/Progetto/C3.pdf}
				\item\textbf{Norme Di Progetto}: "Norme Di Progetto v\lastversion{}".
			\end{itemize}
		\subsubsection{Informativi}
			\begin{itemize}
				\item \textbf{Analisi dei Requisiti}: “Analisi Dei Requisiti v\lastversion{}”;
				\item \textbf{Piano di Qualifica}: “Piano Di Qualifica v\lastversion{}”;
				\item \textbf{Studio di Fattibilità}: “Studio Di Fattibilità v\lastversion{}”;
				\item \textbf{Software Engineering - Ian Sommerville - 9th Edition (2010)}: Part 4 - Software Management;
				\item \textbf{Slide dell’insegnamento Ingegneria del Software modulo A}: Il ciclo di vita del software, gestione di progetto.
			\end{itemize}
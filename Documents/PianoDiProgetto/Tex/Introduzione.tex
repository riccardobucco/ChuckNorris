% !TEX encoding = UTF-8 Unicode
\section{Introduzione}

	\subsection{Scopo del documento}
		Tale documento ha lo scopo di pianificare il modo e i tempi in cui verranno svolte le attività dai membri del gruppo \groupname{}.\\
		In particolare, gli scopi di tale documento sono:
		\begin{itemize}
			\item descrivere il ciclo di vita scelto per il prodotto;
			\item fissare le varie fasi di sviluppo esplicandone scopo e dettagli;
			\item analizzare e gestire gli eventuali rischi;
			\item descrivere cosa dovrà esser fatto nelle varie fasi di sviluppo;
			\item preventivare l'impegno delle risorse;
			\item calcolare il consuntivo di utilizzo delle risorse durante lo svolgimento del progetto.
		\end{itemize}
	
	\subsection{Glossario}
	Allo scopo di rendere più semplice la comprensione dei documenti ed evitare eventuali ambiguità, viene allegato il \insdoc{Glossario v1.00}, che contiene la spiegazione della terminologia tecnica e degli acronimi utilizzati. Per facilitare la lettura, i termini presenti all'interno di tale documento saranno marcati da una “G” maiuscola a pedice.	

	\subsection{Riferimenti utili}
		\subsubsection{Riferimenti normativi}
			\begin{itemize}
				\item \textbf{Capitolato d'appalto C3:} \projectname{}: Real-time Business Intelligence \\
					\insuri{http://www.math.unipd.it/~tullio/IS-1/2014/Progetto/C3.pdf};
				\item \textbf{Norme di Progetto:} \insdoc{Norme di Progetto v1.00}.
			\end{itemize}
		\subsubsection{Riferimenti informativi}
			\begin{itemize}
				\item \textbf{Materiale del corso di Ingegneria del Software:} \\
					\insuri{http://www.math.unipd.it/~tullio/IS-1/2014/};
				\item \textbf{Software Engineering 9th Edition - Ian Sommerville - Chapter 4. Software Management};
				\item \textbf{La gestione dei progetti software - Ercole F. Colonese:} \\
					\insuri{http://www.colonese.it/00-Manuali_Pubblicatii/03-ProjectManagement_v1.0.pdf};
				\item \textbf{Analisi dei Requisiti:} \insdoc{Analisi dei Requisiti v1.00};
				\item \textbf{Piano di Qualifica:} \insdoc{Piano di Qualifica v1.00};
				\item \textbf{Studio di Fattibilità:} \insdoc{Studio di Fattibilità v1.00}.
			\end{itemize}

% !TEX encoding = UTF-8 Unicode
\section{Introduzione}
	\subsection{Scopo del documento}
		Tale documento ha lo scopo di pianificare il modo e i tempi in cui verranno svolte le attività dai membri del gruppo \groupname. \\In particolare, gli scopi di tale documento sono:
		\begin{itemize}
			\item Descrivere il ciclo di vita scelto per il prodotto;
			\item Fissare le varie fasi di sviluppo esplicandone scopo e dettagli;
			\item Analizzare a gestire gli eventuali rischi;
			\item Descrivere cosa dovrà esser fatto nelle varie fasi di sviluppo;
			\item Preventivare l'impegno delle risorse;
			\item Calcolare il consuntivo di utilizzo delle risorse durante lo svolgimento del progetto.
		\end{itemize}
	\subsection{Scopo del prodotto}
		Lo scopo del prodotto è produrre un framework per lo stack tecnologico formato da Node.js, Express.js e Socket.io in grado di generare grafici i cui dati sono letti da sorgenti arbitrarie e che metta a disposizione funzioni di aggiornamento dei grafici lato server tramite tecnologia WebSocket.
	\subsection{Glossario}
		Al fine di evitare ogni genere di ambiguità relativa al linguaggio o ai termini utilizzati nei documenti, si allega il 
	\subsection{Riferimenti}
		\subsubsection{Normativi}
			\begin{itemize}
				\item\textbf{Capitolato D'Appalto C3}: \projectname: Real-time Business Intelligence. Reperibile all'indirizzo: \insuri{http://www.math.unipd.it/~tullio/IS-1/2014/Progetto/C3.pdf}
				\item\textbf{Norme Di Progetto}: "Norme di Progetto \lastversion".
			\end{itemize}
		\subsubsection{Informativi}
			\begin{itemize}
				\item \textbf{Analisi dei Requisiti}: “Analisi dei Requisiti \lastversion”;
				\item \textbf{Piano di Qualifica}: “Piano di Qualifica \lastversion”;
				\item \textbf{Studio di Fattibilità}: “Studio di Fattibilità \lastversion”;
				\item \textbf{Software Engineering - Ian Sommerville - 9th Edition (2010)}: Part 4 - Software Management;
				\item \textbf{Slide dell’insegnamento Ingegneria del Software modulo A}: Il ciclo di vita del software, gestione di progetto.
			\end{itemize}
	\subsection{Ciclo di vita}
		Il modello di ciclo di vita scelto per il prodotto è il \underline[modello incrementale].\\Si è deciso dunque di dividere lo sviluppo del progetto in varie fasi. Il termine di ognuna di esse è segnato da una milestone. Una milestone può essere o una scadenza di consegna di revisione o un incontro con il proponente. Si è preferito fissare milestone non a lunga distanza tra loro per ridurre i rischi e permettere di aver un resoconto da parte del proponente ad ogni fase. Per ogni fase di sviluppo vengono pianificati vari periodi di verifica per ogni attività, solitamente uno ogni 7 giorni con durata media di 2 giorni. Ogni verifica è importante perchè quella diventerà una baseline.
		\begin{description}
			\item[Fase A]: Questa fase è caratterizzata da 3 sottofasi:
				\begin{itemize}
					\item Individuazione/creazione degli strumenti per documentazione e di supporto;
					\item Creazione \insdoc{Norme di Progetto};
					\item Creazione documentazione (\insdoc{Studio di Fattibilità}, seguito da \insdoc{Analisi dei Requisiti}, \insdoc{Piano di Progetto}, \insdoc{Piano di Qualifica} e \insdoc{Glossario}).
				\end{itemize}
				In questa fase l'attività più onerosa sarà quella di analisi.\\Tra la seconda e la terza sottofase viene effettuato un'incontro con il proponente per chiarire le idee sulla comprensione del capitolato.\\Tale fase si conclude con la \insrev{Revisione dei Requisiti}. In tale modo si avrà un riscontro immediato sulle intenzioni del proponente.
			\item[Fase AD]: Caratterizzata dall’analisi di dettaglio. Saranno inseriti i nuovi requisiti individuati dagli analisti. I requisiti che non rispecchiavano le richieste del proponente saranno corretti, e verrà portato un incremento agli altri documenti se necessario.\\Tale fase si conclude con un incontro con il proponente che visionerà e confermerà le modifiche fatte.
			\item[Fase PA]: Caratterizzata dalla progettazione architetturale. Questa fase ha inizio quando il documento \insdoc{Analisi dei Requisiti} è nello stato tra acceptable e addressed (fine Fase AD) e termina con la visione del proponente. Verranno apportati incrementi ad alcuni prodotti dalle precedenti fasi, come \insdoc{Norme di Progetto}, \insdoc{Piano di progetto}, \insdoc{Glossario} e \insdoc{Piano di Qualifica}. Al proponente si prevede di mostrare il documento \insdoc{Specifica Tecnica}.
			\item[Fase PROB]: In questa fase si procede con la progettazione di dettaglio e la codifica dei requisiti obbligatori. Tale fase si conclude con la visione del proponente di un primo prototipo che dovrà soddisfare i requisiti obbligatori. Verranno apportati incrementi ai documenti prodotti dalle precedenti fasi.
			\item[Fase PRD]: Avrà inizio subito dopo la \insphase{Fase PROB}. In questa fase si procede con la progettazione di dettaglio e la codifica dei requisiti desiderabili. Tale fase si conclude con la scadenza della consegna di revisione. Alla \insrev{Revisione di Progettazione} si prevede di consegnare un primo prototipo che dovrà soddisfare i requisiti obbligatori e desiderabili ed il documento \insdoc{Specifica Tecnica} steso finora. Verranno apportati incrementi ai documenti prodotti dalle precedenti fasi.
			\item[Fase PROP]: Avrà inizio subito dopo la consegna per la \insrev{RP}. In questa fase si procede con la progettazione di dettaglio e la codifica dei requisiti opzionali. Tale fase si conclude con la consegna per la \insrev{Revisione di Qualifica}, nella quale sarà presentato il software che implementa tutti i requisiti (obbligatori, desiderabili e opzionali). Verranno apportati incrementi ai documenti prodotti dalle precedenti fasi.
			\item[Fase V]: Tale fase comincia non appena termina la \insphase{Fase PROP}. Caratterizzata dalla validazione. In questa fase il progetto avrà termine. Verrà quindi effettuata la validazione del software creato e successivamente verrà collaudato. Tale fase si conclude con la \insrev{Revisione di Accettazione}.
		\end{description}
		La scelta effettuata ci permette di spezzare facilmente ognuna di queste 7 macro-fasi in attività più piccole. Questo permette di avere maggior controllo sull'avanzamento del progetto, e soprattutto da la possibilità di applicare il PDCA molto frequentemente.\\Ad ognuna delle varie attività sono state associate una o più risorse. Delle sotto-attività è stato riportato unicamente il Gantt.
	\subsection{Scadenze}
		Le scadenze che il gruppo \groupname ha deciso di rispettare sono le seguenti:
		\begin{itemize}
			\item \insrev{Revisione dei requisiti}:\insdate{16}{02}{2015};
			\item \insrev{Revisione di Progettazione}:\insdate{24}{04}{2015};
			\item \insrev{Revisione di Qualifica}:\insdate{29}{05}{2015};
			\item \insrev{Revisione di Accettazione}:\insdate{18}{06}{2015};
		\end{itemize}

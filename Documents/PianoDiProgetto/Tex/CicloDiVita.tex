% !TEX encoding = UTF-8 Unicode
	\section{Ciclo di vita}
		Il modello di ciclo di vita scelto per il prodotto è il \underline{modello incrementale}.\\Si è deciso, dunque, di dividere lo sviluppo del progetto in varie fasi. Il termine di ognuna di esse è segnato da una milestone. Una milestone può essere o una scadenza di consegna di revisione o un incontro con il proponente. Si è preferito fissare milestone non a lunga distanza tra loro per ridurre i rischi e permettere di aver un resoconto da parte del proponente ad ogni fase. Per ogni fase di sviluppo vengono pianificati vari periodi di verifica per ogni attività, solitamente uno ogni 7 giorni con durata media di 2 giorni. Ogni verifica è importante perché quella diventerà una baseline.
		\begin{description}
			\item[Fase A:] Questa fase è caratterizzata da 3 sottofasi:
				\begin{itemize}
					\item individuazione/creazione degli strumenti per documentazione e di supporto;
					\item creazione \insdoc{Norme di Progetto};
					\item creazione documentazione (\insdoc{Studio di Fattibilità}, seguito da \insdoc{Analisi dei Requisiti}, \insdoc{Piano di Progetto}, \insdoc{Piano di Qualifica} e \insdoc{Glossario}).
				\end{itemize}
				In questa fase l'attività più onerosa sarà quella di analisi.\\Tra la seconda e la terza sottofase viene effettuato un incontro con il proponente per chiarire le idee sulla comprensione del capitolato.\\Tale fase si conclude con la \insrev{Revisione dei Requisiti}. In tale modo si avrà un riscontro immediato sulle intenzioni del proponente.
			\item[Fase AD:] Caratterizzata dall’analisi di dettaglio. In questa fase verranno consolidati i requisiti  e verrà portato un incremento o una correzione a tutti i documenti scritti fino ad ora, se necessario.\\Tale fase si conclude con un incontro con il proponente che visionerà e confermerà le modifiche fatte al documento \insdoc{Analisi dei Requisiti}.
			\item[Fase PA:] Caratterizzata dalla progettazione architetturale. Questa fase ha inizio quando il documento \insdoc{Analisi dei Requisiti} è nello stato tra \textit{acceptable} e \textit{addressed} (fine \insphase{fase AD}) e termina con la visione del proponente. Verranno apportati incrementi ad alcuni documenti prodotti nelle precedenti fasi, come \insdoc{Norme di Progetto}, \insdoc{Piano di Progetto}, \insdoc{Glossario} e \insdoc{Piano di Qualifica}. Al proponente si prevede di mostrare il documento \insdoc{Specifica Tecnica}.
			\item[Fase PROB:] In questa fase si procede con la progettazione di dettaglio e la codifica dei requisiti obbligatori. Tale fase si conclude con la visione del proponente di un primo prototipo che dovrà soddisfare i requisiti obbligatori. Verranno apportati incrementi ai documenti prodotti nelle precedenti fasi.
			\item[Fase PRD:] Avrà inizio subito dopo la \insphase{Fase PROB}. In questa fase si procede con la progettazione di dettaglio e la codifica dei requisiti desiderabili. Tale fase si conclude con la scadenza della consegna di revisione. Alla \insrev{Revisione di Progettazione} si prevede di consegnare un primo prototipo che dovrà soddisfare i requisiti obbligatori e desiderabili ed il documento \insdoc{Specifica Tecnica} steso fino a quel momento. Verranno apportati incrementi ai documenti prodotti nelle precedenti fasi.
			\item[Fase PROP:] Avrà inizio subito dopo la consegna per la \insrev{RP}. In questa fase si procede con la progettazione di dettaglio e la codifica dei requisiti opzionali. Tale fase si conclude con l'incontro con il proponente per mostrare il prototipo con tutti i requisiti implementati. Verranno apportati incrementi ai documenti prodotti nelle precedenti fasi.
			\item[Fase V:] Tale fase comincia non appena termina la \insphase{Fase PROP}. Essa è caratterizzata dalla validazione e quindi il lavoro più oneroso sarà quello dei \insrole{verificatori}. In questa fase il progetto avrà termine. Verrà quindi effettuata la validazione del software creato e successivamente verrà collaudato. Tale fase si conclude con la \insrev{RA}.
		\end{description}
		La scelta effettuata ci permette di spezzare facilmente ognuna di queste 7 macro-fasi in attività più piccole. Questo permette di avere maggior controllo sull'avanzamento del progetto, e soprattutto dà la possibilità di applicare il PDCA molto frequentemente.\\Ad ognuna delle varie attività sono state associate una o più risorse. Delle sotto-attività è stato riportato unicamente il Gantt.

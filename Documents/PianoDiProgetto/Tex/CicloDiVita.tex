% !TEX encoding = UTF-8 Unicode
\level{2}{Ciclo di vita}
		Nella scelta del ciclo di vita hanno rivestito particolare importanza le seguenti necessità:
		\begin{itemize}
			\item produrre "valore" ad ogni incremento, per avere in un tempo minore dei risultati che confermano il lavoro svolto oppure richiedono una revisione senza che questa causi eccessivi ritardi e dispendio di tempo e risorse;
			\item verificare in modo continuo le funzionalità implementate per rendere  sempre più stabile e di qualità;
			\item distribuire le risorse in un numero limitato di attività, per semplificare la valutazione e pianificazione di risorse e tempi;
			\item incrementare gradualmente le funzionalità messe a disposizione dal \insglo{prodotto}, con la possibilità di soddisfare in primis i requisiti con maggiore criticità.\\
		\end{itemize}
		
		La scelta è ricaduta sul \underline{\insglo{modello incrementale}}, ritenuto il più adatto alle esigenze del gruppo in questo progetto:
		\begin{itemize}
			\item È richiesto che il sistema sia scomposto in sottosistemi. Ciò porta a due caratteristiche positive:
			\begin{itemize}
				\item la gestione di risorse e tempi è più controllabile;
				\item i test possono esser fatti con un dettaglio maggiore e quindi saranno più esaustivi.
			\end{itemize}
			\item Prevede rilasci multipli e successivi portando quindi i seguenti vantaggi:
			\begin{itemize}
				\item possibilità di rilasciare prototipi;
				\item incremento di volta in volta delle funzionalità messe a disposizione dal \insglo{prodotto};
				\item soddisfacimento dei requisiti di maggiore importanza e attraversamento di più cicli di verifica da parte di quest'ultimi con il risultato di un maggiore miglioramento.
			\end{itemize}
			\item Ogni incremento fissa una baseline consolidando di conseguenza la sezione coinvolta. Tale cosa riduce il rischio di fallimento.
		\end{itemize}

% !TEX encoding = UTF-8 Unicode
\level{1}{Ciclo di vita}
		Il modello di ciclo di vita scelto per il \insglo{prodotto} è il \underline{\insglo{modello incrementale}}.\\
		I motivi di tale scelta sono legati alle sue proprietà:
		\begin{itemize}
			\item È richiesto che il sistema sia scomposto in sottosistemi. Ciò porta a due caratteristiche positive:
			\begin{itemize}
				\item in un breve lasso di tempo le risorse sono utilizzate in un numero limitato di attività rendendo più controllabile la gestione di esse e dei tempi;
				\item i test possono esser fatti con un dettaglio maggiore e quindi saranno più esaustivi.
			\end{itemize}
			\item Prevede rilasci multipli e successivi portando quindi i seguenti vantaggi:
			\begin{itemize}
				\item possibilità di rilasciare prototipi;
				\item incrementare di volta in volta le funzionalità messe a disposizione dal \insglo{prodotto};
				\item i primi rilasci riguarderanno il soddisfacimento dei requisiti di maggiore importanza facendo in modo che essi attraversino più cicli di verifica rendendoli sempre più raffinati e migliorati.
			\end{itemize}
			\item Ogni incremento fissa una baseline consolidando di conseguenza la seziomne coinvolta. Tale cosa riduce il rischio di fallimento;
			\item Grazie alla pianificazione di tali cicli incrementali si riesce ad avere un maggior controllo dei tempi e dei costi;
			\item Essendo i requisiti utente catalogati in base alla loro importanza, vengono soddisfatti in primis quelli con maggiore criticità.
		\end{itemize}
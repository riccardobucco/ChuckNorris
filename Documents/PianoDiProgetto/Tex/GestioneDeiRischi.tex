% !TEX encoding = UTF-8 Unicode
\level{2}{Gestione dei rischi}
	Al fine di realizzare un buon progetto \insglo{software} è opportuno, dopo aver individuato e compreso i rischi, prendere le misure preventive per gestirli. A tale scopo si è deciso di procedere per due fasi:
	\begin{itemize}
		\item \textbf{Pianificazione}: predisporre un piano che permetta di reagire, in caso un rischio si manifesti, in modo controllato ed efficace. L'obiettivo primario è quello di evitare i rischi, ma poiché non tutti sono evitabili, si cerca di mitigarne gli effetti.
		\item \textbf{Controllo}: tenere sotto controllo i rischi in modo tale da poterli prevenire o poter attuare contromisure pianificate.
	\end{itemize}
	Nelle sottosezioni a seguire viene fatta una descrizione esaustiva delle misure preventive prese per gestire i rischi.	

	\level{3}{Livello tecnologico}
		\level{4}{Tecnologie adottate sconosciute}
			\begin{itemize}
				\item \textbf{Pianificazione}: Sarà compito dell'\insrole{Amministratore} mettere a disposizione il materiale necessario ad ogni membro del gruppo, che si impegnerà a documentarsi, in maniera autonoma e il più completa possibile, sulle tecnologie da utilizzare.
				\item \textbf{Controllo}: Il \insrole{Responsabile di Progetto} ha il compito di monitorare costantemente il grado di conoscenza di ogni componente del \insglo{team} riguardo le tecnologie adottate.
			\end{itemize}
		\level{4}{Guasti hardware}
			\begin{itemize}
				\item \textbf{Pianificazione}: Allo scopo di evitare tale rischio si fa uso di \textit{\insglo{Dropbox}} e \textit{\insglo{repository}} per il backup dei dati. In questo modo ogni componente del \insglo{team} possiede una copia condivisa del lavoro svolto. In caso di guasto \insglo{hardware} verranno utilizzate le macchine messe a disposizione dai laboratori informatici o altri strumenti personali.
				\item \textbf{Controllo}: Ogni componente ha il compito di curare la propria strumentazione \insglo{hardware} e di effettuare il backup dei dati ad ogni aggiornamento importante. Inoltre, è utile controllare frequentemente che i dati di backup siano presenti ed aggiornati all'ultima versione.
			\end{itemize}
	\level{3}{Livello personale}
		\level{4}{Problemi personali dei componenti del team}
			\begin{itemize}
				\item \textbf{Pianificazione}: Ciascun membro del \insglo{team} si impegnerà ad avvertire, in caso di impegni personali, il \insrole{Responsabile di Progetto}, il quale ridurrà il carico di lavoro a quel componente e lo ripartirà tra le altre risorse disponibili, eseguendo una nuova pianificazione delle attività del periodo in considerazione.
				\item \textbf{Controllo}: Il \insrole{Responsabile di Progetto} verificherà costantemente se vi sono avvisi, riguardanti impegni o imprevisti, da parte dei componenti del gruppo. Tale rischio sarà, anche, controllato tramite l'uso di calendari di gruppo.
			\end{itemize}
		\level{4}{Problemi tra componenti del team}
			\begin{itemize}
				\item \textbf{Pianificazione}: Il \insrole{Project Manager} si dovrà occupare di risolvere eventuali contrasti o dispute, sottoponendo i componenti coinvolti ad un confronto civile per raggiungere un accordo; in caso il problema non venisse risolto dovrà pianificare le successive attività in modo da minimizzare il contatto tra le parti coinvolte.
				\item \textbf{Controllo}: Sarà compito del \insrole{Responsabile di Progetto} tenere sotto controllo i rapporti tra i vari membri del \insglo{team} e compito di tutti i componenti di riferire al \insrole{Project Manager} eventuali problemi di cui non è a conoscenza.
			\end{itemize}
	\level{3}{Livello organizzativo}
		\level{4}{Valutazione dei costi}
			\begin{itemize}
				\item \textbf{Pianificazione}: Il \insrole{Project Manager} verificherà ogni giorno l'avanzamento delle attività, tramite la \textit{\insglo{dashboard}}. Si presterà maggiore attenzione alle attività critiche, per le quali si predisporrà un periodo di slack in modo da evitare che un eventuale ritardo non posticipi le altre attività. Nel caso tale tempo di slack non bastasse, verrà rieseguita la pianificazione tenendo conto del ritardo accumulato che si dovrà recuperare.
				\item \textbf{Controllo}: Incarico del \insrole{Responsabile di Progetto} sarà accertarsi che le attività procedano come pianificato. Per facilitare tale compito si utilizzerà una \textit{\insglo{dashboard}}, tramite la quale ogni componente saprà quali attività deve svolgere ed entro quali date.
			\end{itemize}
	\level{3}{Livello degli strumenti}
		\level{4}{Inesperienza del team nell'utilizzo degli strumenti}
			\begin{itemize}
				\item \textbf{Pianificazione}: Il \insrole{Responsabile di Progetto} valuterà il livello di conoscenza di ogni componente del gruppo riguardo gli strumenti utilizzati. Ogni volta che si incontrerà la necessità di fare uso di un nuovo strumento, verrà comunicato a tutti i componenti del gruppo tramite \insglo{mailing list}. Ogni membro dovrà documentarsi autonomamente ed imparare ad utilizzare gli strumenti richiesti.
				\item \textbf{Controllo}: Sarà compito dell'\insrole{Amministratore} verificare se qualche membro del gruppo richiede aiuto riguardo la strumentazione. Ogni componente dovrà verificare se sono stati utilizzati nuovi strumenti.
			\end{itemize}
	\level{3}{Livello dei requisiti}
		\level{4}{Comprensione dei requisiti}
			\begin{itemize}
				\item \textbf{Pianificazione}: Durante l'attività di analisi dei requisiti il \insglo{team} effettuerà più incontri con il Proponente in modo da assicurarsi di aver compreso in modo chiaro ogni requisito necessario al corretto sviluppo del progetto.
				\item \textbf{Controllo}: Ad ogni revisione verranno consegnati tutti i documenti che il Proponente valuterà e saranno corretti gli eventuali errori o imprecisioni segnalati.
			\end{itemize}

% !TEX encoding = UTF-8 Unicode
\level{1}{Analisi dei rischi}
	Al fine di realizzare un buon progetto \insglo{software} è opportuno individuare i rischi potenziali che si possono presentare durante lo sviluppo, stimare la probabilità che un rischio si manifesti e valutarne le possibili conseguenze sul progetto: solo dopo questi passi si riuscirà a prendere misure preventive per poterli gestire. 
	Nella tabella sottostante si rappresentano i rischi individuati con la loro probabilità di occorrenza e i relativi livelli di rischio.
	Una descrizione dettagliata con la valutazione delle possibili conseguenze viene fatta nelle sottosezioni a seguire.
	\begin{table}[H]\centering \tabulinesep=3pt
			\begin{tabu}to \textwidth {|X[4]|X[3c]|X[4]|X[4]|}
				\hline
				Livello 			& Tipologia					& Probabilità di occorrenza 		& Livello di rischio  \\ \hline

				Tecnologico			& Tecnologie adottate sconosciute				& media 					& alto 		 \\ \cline{2-4}
				  					& Guasti \insglo{hardware}								& bassa 					& basso 	 \\ \hline
				Personale		 	& Problemi personali dei componenti del \insglo{team}	& media 					& medio 	 \\ \cline{2-4}
									& Problemi tra i componenti del \insglo{team}			& bassa 					& alto 		 \\ \hline
				Organizzativo	 	& Valutazione dei costi							& media 					& alto 		 \\ \hline
				Strumenti			& Inesperienza nell'utilizzo					& alta 						& alto 		 \\ \hline
				Requisiti		 	& Comprensione 									& media						& alto 		 \\ \hline
			\end{tabu}
		\caption{Riassunto analisi dei rischi}
	\end{table}

	\level{2}{Livello tecnologico}
		\level{3}{Tecnologie adottate sconosciute}
			\begin{itemize}
				\item \textbf{Descrizione}: le tecnologie adottate per lo sviluppo del progetto sono, in buona parte, sconosciute al gruppo. Non si esclude, quindi, la possibilità di trovare ostacoli nell'utilizzo di tali tecnologie.
				\item \textbf{Analisi}:
					\begin{itemize}
						\item \textit{Livello di rischio}: alto;
						\item \textit{Probabilità di occorrenza}: media;
						\item \textit{Possibili conseguenze}: possibili ritardi sulle date di consegna.
					\end{itemize}
			\end{itemize}
		\level{3}{Guasti hardware}
			\begin{itemize}
				\item \textbf{Descrizione}: È possibile che si incontrino problemi con le attrezzature \insglo{hardware} durante lo sviluppo del progetto che può portare alla perdita di dati.
				\item \textbf{Analisi}:
				\begin{itemize}
					\item \textit{Livello di rischio}: basso;
					\item \textit{Probabilità di occorrenza}: bassa;
					\item \textit{Possibili conseguenze}: difficoltà nel proseguimento del progetto e sostanziale rallentamento delle attività.
				\end{itemize}
			\end{itemize}
	\level{2}{Livello personale}
		\level{3}{Problemi personali dei componenti del team}
			\begin{itemize}
				\item \textbf{Descrizione}: Ciascun componente del \insglo{team} ha impegni e necessità personali. Nel gruppo c'è, inoltre, uno studente lavoratore che può non essere sempre disponibile a causa di impegni relativi al lavoro. È quindi inevitabile prevedere che alcuni membri del gruppo non siano disponibili in certi momenti.
				\item \textbf{Analisi}:
				\begin{itemize}
					\item \textit{Livello di rischio}: medio;
					\item \textit{Probabilità di occorrenza}: media;
					\item \textit{Possibili conseguenze}: possibili ritardi nello svolgimento delle attività, in caso di impegni imprevisti da parte di qualche componente del gruppo.
				\end{itemize}
			\end{itemize}
		\level{3}{Problemi tra componenti del team}
			\begin{itemize}
				\item \textbf{Descrizione}: Per la maggior parte dei componenti del \insglo{team} questa è la prima esperienza di lavoro in un gruppo numeroso; ciò può  innescare problemi di collaborazione e contrasti.
				\item \textbf{Analisi}:
				\begin{itemize}
					\item \textit{Livello di rischio}: alto;
					\item \textit{Probabilità di occorrenza}: bassa;
					\item \textit{Possibili conseguenze}: rallentamento delle attività, carico di lavoro più consistente e clima lavorativo poco proficuo.
				\end{itemize}
			\end{itemize}
	\level{2}{Livello organizzativo}
		\level{3}{Valutazione dei costi}
			\begin{itemize}
				\item \textbf{Descrizione}: Durante la \insglo{fase} di pianificazione è necessario definire le attività, gestire l'allocazione delle risorse e stimare le scadenze e i costi. A causa della poca esperienza da parte del \insglo{team} è possibile che la stima risulti sbagliata.
				\item \textbf{Analisi}:
				\begin{itemize}
					\item \textit{Livello di rischio}: alto;
					\item \textit{Probabilità di occorrenza}: media;
					\item \textit{Possibili conseguenze}: possibili ritardi dovuti alla sottostima dei costi delle attività o possibili sprechi di tempo in caso di sovrastima. 
				\end{itemize}
			\end{itemize}
	\level{2}{Livello degli strumenti}
		\level{3}{Inesperienza del team nell'utilizzo degli strumenti}
			\begin{itemize}
				\item \textbf{Descrizione}: È richiesto l'uso di molti strumenti \insglo{software} che il \insglo{team} non ha mai utilizzato e che non conosce. Alcuni di essi necessitano di tempo per essere appresi.
				\item \textbf{Analisi}:
				\begin{itemize}
					\item \textit{Livello di rischio}: alto;
					\item \textit{Probabilità di occorrenza}: alta;
					\item \textit{Possibili conseguenze}: lento avanzamento delle attività che richiedono l'utilizzo di tali strumenti.
				\end{itemize}
			\end{itemize}
	\level{2}{Livello dei requisiti}
		\level{3}{Comprensione dei requisiti}
			\begin{itemize}
				\item \textbf{Descrizione}: Nella \insglo{fase} di analisi è possibile che i requisiti del problema vengano fraintesi o non identificati correttamente. 
				\item \textbf{Analisi}:
				\begin{itemize}
					\item \textit{Livello di rischio}: alto;
					\item \textit{Probabilità di occorrenza}: media;
					\item \textit{Possibili conseguenze}: possibili divergenze tra la visione del problema da parte del gruppo e ciò che il Proponente ha richiesto.
				\end{itemize}
			\end{itemize}

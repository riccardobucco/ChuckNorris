\subsection{Fase PROP}
	\textbf{Periodo}: dal \insdate{19}{04}{2015} al \insdate{17}{05}{2015} \\Questa fase comincia subito dopo la scadenza della consegna per la \insrev{Revisione di Progetto} e termina con l'incontro con il proponente al fine di mostrare il prototipo con tutti i requisiti (obbligatori, desiderabili e opzionali) sviluppati. 
	\\Le attività di questa fase saranno le seguenti:
	\begin{itemize}
		\item\textbf{Definizione di Prodotto}: Viene steso il documento \insfile{Definizione di Prodotto 3.0}. Esso definisce la struttura interna del sistema e le relazioni dei componenti del prodotto relativi ai requisiti opzionali.
		\item\textbf{Manuale Utente e Manuale Amministratore}: Comincia la stesura dei manuali che forniranno indicazioni agli utilizzatori del sistema.
		\item\textbf{Incremento e Verifica Documenti}: Vengono eseguite modifiche ai documenti già scritti, se necessario.
		\item\textbf{Glossario}: Vengono aggiunti al file \insfile{Glossario.xml} i vocaboli dei quali si ritiene necessaria una definizione formale. Alla fine di questa fase vieni quindi generato il documento \insdoc{Glossario 6.0}.
	\end{itemize}
	\subsubsection{Diagramma di Gantt delle attività}
		%\includegraphics{PianoDiProgetto/Pics/FasePROP.png}
% !TEX encoding = UTF-8 Unicode
\subsection{Fase PRD: Progettazione Requisiti Desiderabili}
	\textbf{Periodo}: dal \insdate{08}{04}{2015} al \insdate{19}{04}{2015} \\Questa fase comincia con la fine della \insphase{Fase PROB} e termina con la scadenza della consegna della \insrev{Revisione di Progetto}.\\Le attività di questa fase saranno le seguenti:
	\begin{itemize}
		\item\textbf{Definizione di Prodotto}: Viene steso il documento \insdoc{Definizione di Prodotto v2.0}. Esso definisce la struttura interna del sistema e le relazioni dei componenti del prodotto relativi ai requisiti desiderabili.
		\item \textbf{Codifica}: con quest'attività inizia lo sviluppo da parte dei programmatori dei requisiti desiderabili. Sarà dunque seguito quanto riportato nel documento \insdoc{Definizione di Prodotto v2.0};
		\item \textbf{Esecuzione test}: verranno eseguiti automaticamente tutti i test di unità e integrazione previsti dal documento \insdoc{Piano di Qualifica v 5.0};
		\item\textbf{Manuale Utente e Manuale Amministratore}: Comincia la stesura dei manuali che forniranno indicazioni agli utilizzatori del sistema.
		\item\textbf{Incremento e Verifica Documenti}: Vengono eseguite modifiche ai documenti già scritti, se necessario.
		\item\textbf{Glossario}: Vengono aggiunti al file \insfile{Glossario.xml} i vocaboli dei quali si ritiene necessaria una definizione formale. Alla fine di questa fase vieni quindi generato il documento \insdoc{Glossario v5.0}.
	\end{itemize}
	\subsubsection{Diagramma di Gantt delle attività}
	\begin{figure}[H]\centering
		\includegraphics[width=\textwidth]{PianoDiProgetto/Pics/FasePRD.png}
	\caption{Gantt Fase PRD}
\end{figure}
% !TEX encoding = UTF-8 Unicode
\level{1}{Riscontro dei rischi}
	In questa appendice viene data una descrizione dettagliata per ogni \insglo{fase} del riscontro dei rischi individuati.


	\level{2}{Fase DB}
		\level{3}{Tecnologie adottate sconosciute}
		Il rischio non si è verificato in quanto le tecnologie utilizzate sono di facile comprensione ed in parte già conosciute.
		\level{3}{Guasti hardware}
		Ogni componente si è preso cura del proprio materiale e ciò ha favorito il fatto che non si siano verificati guasti \insglo{hardware}. Non è stato riscontrato alcun rallentamento o impedimento causato da un qualche guasto \insglo{hardware}.
		\level{3}{Problemi personali dei componenti del team}
		In questa prima \insglo{fase}, ogni componente ha avvertito il Responsabile di Progetto con largo anticipo in caso di impegni personali. Ognuno di questi rischi è stato ben gestito ed eliminato.
		\level{3}{Problemi tra componenti del team}
		In questa \insglo{fase} non si sono verificati problemi particolari.
		\level{3}{Valutazione dei costi}
		Si è verificato un ritardo sull’attività di analisi. Tuttavia, grazie ad una attenta pianificazione, non si sono verificati ritardi nella consegna.
		\level{3}{Inesperienza del team nell'utilizzo degli strumenti}
		Non si è verificato questo problema in quanto gli strumenti utilizzati nella \insglo{fase} sono stati di facile comprensione ed in parte già conosciuti.
		\level{3}{Comprensione dei requisiti}
		Durante questa \insglo{fase} si sono riscontrate difficoltà nella comprensione dei requisiti ma grazie agli incontri effettuati con il Proponente si è riusciti a capirli in modo chiaro.
		
	
	\level{2}{Fase DI}
		\level{3}{Tecnologie adottate sconosciute}
		Durante questa \insglo{fase} non si è verificato questo problema in quanto le tecnologie utilizzate sono di facile comprensione ed in parte già conosciute. 
		\level{3}{Guasti hardware}
		Ogni componente si è preso cura del proprio materiale e ciò ha favorito il fatto che non si siano verificati guasti \insglo{hardware}. Non è stato riscontrato alcun rallentamento o impedimento causato da un qualche guasto \insglo{hardware}.
		\level{3}{Problemi personali dei componenti del team}
		In questa \insglo{fase} non si sono riscontrati impegni personali da parte dei componenti del \insglo{team} che abbiano richiesto l'attivazione del Responsabile di Progetto per gestire ed eliminare il rischio.
		\level{3}{Problemi tra componenti del team}
		Durante la \insglo{fase} non si sono verificati problemi di questo tipo.
		\level{3}{Valutazione dei costi}
		Un rallentamento non previsto ha dilungato la durata della \insglo{fase} e si è dovuto ricorrere ad una rapida ripianificazione.
		\level{3}{Inesperienza del team nell'utilizzo degli strumenti}
		Non si è verificato questo problema in quanto gli strumenti utilizzati nella \insglo{fase} sono stati di facile comprensione ed in parte già conosciuti.
		\level{3}{Comprensione dei requisiti}
		Da questa \insglo{fase} i requisiti erano ben solidi e ciò ha portato ad un abbassamento del rischio passando da alto a medio. 



	\level{2}{Fase SD}
		\level{3}{Tecnologie adottate sconosciute}
		Nella terza \insglo{fase}, dove si stende la Specifica Tecnica, i componenti hanno dedicato più tempo alla formazione personale per recuperare le lacune nella conoscenza delle tecnologie utilizzate, riscontrando qualche rallentamento nella comprensione, in particolare del ruolo di \insglo{socket.io}.
		\level{3}{Guasti hardware}
		Ogni componente si è preso cura del proprio materiale e ciò ha favorito il fatto che non si siano verificati guasti \insglo{hardware}. Non è stato riscontrato alcun rallentamento o impedimento causato da un qualche guasto \insglo{hardware}.
		\level{3}{Problemi personali dei componenti del team}
		Nella terza \insglo{fase} ci sono stati dei ritardi imprevisti, dovuti agli impegni universitari dei componenti.
		\level{3}{Problemi tra componenti del team}
		Nella terza \insglo{fase} ci sono stati dei piccoli episodi di incomprensione e divergenza riguardante l’organizzazione in \insglo{fase} di progettazione, risolti prontamente tramite confronto e chiarimento tra le parti coinvolte.
		\level{3}{Valutazione dei costi}
		Si è stato riscontrato un rallentamento non previsto che ha dilungato la durata della \insglo{fase} ed è stato necessario ricorrere ad una rapida ripianificazione.
		\level{3}{Inesperienza del team nell'utilizzo degli strumenti}
		Non si è verificato questo problema in quanto gli strumenti utilizzati nella \insglo{fase} sono stati di facile comprensione ed in parte già conosciuti.
		\level{3}{Comprensione dei requisiti}
		In continuità con la seconda \insglo{fase}, i requisiti erano ben solidi e non c'è stato alcun problema riguardante questo rischio.

	\level{2}{Fase P}
		\level{3}{Tecnologie adottate sconosciute}
		In questa \insglo{fase}, durante la quale il gruppo ha cominciato l'attività di codifica vera e propria, si è effettivamente riscontrato quanto temuto: la conoscenza del dominio tecnologico non era adeguata. Ci si è infatti resi conto che parte di quanto era stato progettato in precedenza non combaciava con quanto alcune delle tecnologie richiedevano. A causa dunque della non completa conoscenza delle tecnologie adottate si è dovuto procedere alla ristesura di parte della progettazione. In ogni caso, il gruppo ha imparato la lezione, e le tecnologie che non erano state ben comprese sono state studiate in modo approfondito.
		\level{3}{Guasti hardware}
		Ogni componente si è preso cura del proprio materiale e ciò ha favorito il fatto che non si siano verificati guasti \insglo{hardware}. Non è stato riscontrato alcun rallentamento o impedimento causato da un qualche guasto \insglo{hardware}.
		\level{3}{Problemi personali dei componenti del team}
		Nella quarta \insglo{fase} il Responsabile di Progetto, preso in considerazione l’aumento del verificarsi di ritardi dovuti agli impegni personali, ha cercato di aumentare la frequenza di verifica e il cambiamento della pianificazione delle attività per arrivare a rispettare gli obiettivi previsti.
		\level{3}{Problemi tra componenti del team}
		Nella \insglo{fase} ci sono stati dei piccoli episodi di incomprensione e divergenza riguardante l’organizzazione in \insglo{fase} di progettazione, risolti prontamente tramite confronto e chiarimento tra le parti coinvolte.
		\level{3}{Valutazione dei costi}
		Nella quarta \insglo{fase} non si sono verificati particolari problemi di questo genere. Si è comunque riusciti a concludere la \insphase{Fase P} entro la scadenza della consegna per la \insrev{RP} ma si è deciso di consegnare tutta progettazione al dettaglio per intero alla \insrev{RQ} e non parzialmente come si era pensato in precedenza.
		\level{3}{Inesperienza del team nell'utilizzo degli strumenti}
		Non si è verificato questo problema in quanto gli strumenti utilizzati nella \insglo{fase} sono stati di facile comprensione ed in parte già conosciuti.
		\level{3}{Comprensione dei requisiti}
		Nessun problema ha riguardato la comprensione dei requisiti.

	\level{2}{Fase IP}
		\level{3}{Tecnologie adottate sconosciute}
		Non ci sono stati rallentamenti dovuti a ciò, poiché nella precedente \insglo{fase} sono state studiate in modo approfondito le tecnologie adottate.
		\level{3}{Guasti hardware}
		Ogni componente si è preso cura del proprio materiale e ciò ha favorito il fatto che non si siano verificati guasti \insglo{hardware}. Non è stato riscontrato alcun rallentamento o impedimento causato da un qualche guasto \insglo{hardware}.
		\level{3}{Problemi personali dei componenti del team}
		Durante la quinta \insglo{fase} vi sono stati alcuni problemi causati da malattie e impegni universitari non previsti. Questo talvolta ha creato piccoli problemi al \insrole{Responsabile di Progetto}, che però ha iniziato a gestire la situazione intervenendo sulla pianificazione e assegnando i vari compiti ai componenti del gruppo che al momento avevano disponibilità. Egli, inoltre, ha cercato di ottenere un piano dettagliato da parte di ciascun componente che descrivesse tutti gli impegni da lui previsti (e che nei mesi precedenti non era possibile sapere), per cercare di organizzare al meglio la pianificazione sul breve-medio termine.
		\level{3}{Problemi tra componenti del team}
		Durante la quinta \insglo{fase} i problemi tra componenti del \insglo{team} sono stati minimi, e riguardavano perlopiù divergenze d'opinione sui modi in cui implementare alcuni aspetti del \insglo{prodotto}.
		\level{3}{Valutazione dei costi}
		Durante la quinta \insglo{fase} vi sono stati dei rallentamenti causati da un fatto che non era stato previsto inizialmente: le correzioni effettuate dal committente e i relativi suggerimenti ci sono stati dati alcuni giorni dopo rispetto a quanto avevamo pianificato. Questo ci ha inizialmente rallentato, in quanto alcuni componenti risultavano senza nulla da fare per alcune ore. Quando però è stato evidente che il ritardo si sarebbe prolungato troppo, il \insrole{Responsabile di Progetto} è stato pronto a intervenire e a cambiare la pianificazione, cercando dunque di fare in modo che nessuno rimanesse senza fare nulla. Inoltre, grazie a uno slack di tempo che era stato fissato inizialmente in previsione di problemi di questo tipo, gli effetti sono stati minimi.
		\level{3}{Inesperienza del team nell'utilizzo degli strumenti}
		Non si è verificato questo problema in quanto gli strumenti utilizzati nella \insglo{fase} sono stati di facile comprensione ed in parte già conosciuti.
		\level{3}{Comprensione dei requisiti}
		Nessun problema ha riguardato la comprensione dei requisiti.

	\level{2}{Fase CP}
		\level{3}{Tecnologie adottate sconosciute}
		Non ci sono stati rallentamenti dovuti a ciò, poiché nelle fasi precedenti sono state studiate in modo approfondito le tecnologie adottate.
		\level{3}{Guasti hardware}
		Ogni componente si è preso cura del proprio materiale e ciò ha favorito il fatto che non si siano verificati guasti \insglo{hardware}. Non è stato riscontrato alcun rallentamento o impedimento causato da un qualche guasto \insglo{hardware}.
		\level{3}{Problemi personali dei componenti del team}
		Grazie ai cambiamenti di gestione di questo rischio da parte del \insrole{Responsabile di Progetto}, non ci sono stati particolari problemi in questa \insglo{fase}.
		\level{3}{Problemi tra componenti del team}
		Nella sesta \insglo{fase} non si sono verificati problemi: è evidente come la capacità di collaborare sia stata affinata non poco nel corso dello svolgimento del progetto.
		\level{3}{Valutazione dei costi}
		Durante la sesta \insglo{fase} abbiamo riscontrato problemi di valutazione dei tempi dovuti da un rischio non previsto. Infatti, dopo la consegna della valutazione da parte del committente, siamo dovuti ricorrere ad una riprogettazione dell'applicazione \insglo{Android} che ci ha costretti a consegnare un prototipo non completo come previsto.
		\level{3}{Inesperienza del team nell'utilizzo degli strumenti}
		Non si è verificato questo problema in quanto gli strumenti utilizzati nella \insglo{fase} sono stati di facile comprensione ed in parte già conosciuti.
		\level{3}{Comprensione dei requisiti}
		Nessun problema ha riguardato la comprensione dei requisiti.

	\level{2}{Fase PD}
		\level{3}{Tecnologie adottate sconosciute}
		Non ci sono stati rallentamenti dovuti a ciò, poiché nelle fasi precedenti sono state studiate in modo approfondito le tecnologie adottate.
		\level{3}{Guasti hardware}
		Durante questa \insglo{fase} uno dei componenti del \insglo{team} ha avuto un guasto \insglo{hardware} improvviso nel dispositivo con cui stava lavorando. Grazie alla propensione al salvataggio frequente delle modifiche e la disponibilità delle macchina virtuale con all'interno già preinstallati gli strumenti di lavoro, la perdita di tempo e risorse è stata minima.
		\level{3}{Problemi personali dei componenti del team}
		Nessun problema grave da evidenziare, in quanto tutti i componenti del \insglo{team} continuano a comunicare le disponibilità e gli impegni con largo anticipo al Responsabile di Progetto.
		\level{3}{Problemi tra componenti del team}
		Non ci sono stati particolari problemi tra i componenti del \insglo{team} da segnalare.
		\level{3}{Valutazione dei costi}
		Durante la \insglo{fase} il Responsabile di Progetto ha ritenuto necessario dedicare più ore all'attività di verifica per ridurre al minimo gli errori presenti nei documenti e nel codice; non ci sono stati però cambiamenti dovuti a errata pianificazione delle attività o ritardi imprevisti.
		\level{3}{Inesperienza del team nell'utilizzo degli strumenti}
		Non si è verificato questo problema in quanto gli strumenti utilizzati nella \insglo{fase} sono stati di facile comprensione ed in parte già conosciuti.
		\level{3}{Comprensione dei requisiti}
		Nessun problema ha riguardato la comprensione dei requisiti.

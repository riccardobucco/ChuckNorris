% !TEX encoding = UTF-8 Unicode
\level{1}{Consuntivo finale}
	Viene di seguito riportato, in formato tabellare, il consuntivo finale del progetto, indicante le spese effettivamente sopportate, sia per ruolo che per persona.
	
	\begin{table}[H]
				\begin{center}
					\begin{tabular}{| l | c | c |}
						\hline
						Ruolo 				& Ore 	& Costi  \\ \hline
						
						Project Manager		& 22		& \euro{} 660,00 \\
						Amministratore 		& 22	& \euro{} 440,00 	\\
						Analista	 		& 94		& \euro{} 2~350,00 \\
						Progettista 		& 198		& \euro{} 4~356,00  \\
						Programmatore		& 108				& \euro{} 1~620	\\
						Verificatore		& 291	& \euro{} 4~365,00	\\ \hline \hline
							
						Totale Preventivo	& 735 			& \euro{} 14~012,00 	\\ \hline
						Totale Consuntivo	& 735 			& \euro{} 13~791,00 	\\ \hline
						Differenza			& 0 			& \euro{} -221,00 	\\ \hline	
					\end{tabular}
				\end{center}
				\caption{Consuntivo finale per ruolo}
			\end{table}
	
	\begin{table}[H]
				\begin{center}
					\begin{tabular}{| l | c | c | c | c | c | c | c |}
						\hline
						Componente 					& PM	& Am	 & An 		& Pt 		& Pm 	& Ve 	& Ore Totali componente \\ \hline
						
						Bigarella Chiara 			& 3		& 0		& 10 		& 44 		& 20		& 28 		& 105 \\
						Bucco Riccardo 				& 4& 0		& 0			& 31 	& 20		& 50 		& 105 \\
						Carlon Chiara	 			& 2		& 8 & 21 		& 24			& 13		& 37	& 105 \\
						Dal Bianco Davide 			& 7		& 2 & 0			& 32 		& 17		& 47	& 105 \\
						Moretto Alessandro 			& 0		& 6 	& 20 		& 27 		& 12		& 17 		& 105 \\
						Pavanello Fabio Matteo	 	& 2		& 6 & 27	& 12	& 17		& 41	& 105 \\
						Rubin Marco					& 4 	& 0		& 16 		& 28			& 6		& 51	& 105 \\ \hline \hline
						
						Ore Totali Ruolo Preventivo	& 32 	& 29 	& 113 		& 176 		& 111		& 274		& 735\\ 
						Ore Totali Ruolo Consuntivo	& 22 	& 22 	& 94 		& 198 		& 108		& 291		& 735\\ \hline
					\end{tabular}
				\end{center}
				\caption{Consuntivo suddivisione ore di lavoro Fase SD}
			\end{table}
	
	\level{2}{Conclusioni}
		Come si può vedere, durante il progetto è stata sottostimata l'importanza della figura del \insrole{Verificatore}, attribuendo maggiore considerazione ad altri ruoli. Tale discrepanza è stata progressivamente appianata durante le singole fasi, portando ad un risparmio finale di \euro{} 221,00, che verrà detratto dal costo finale del progetto imputato al prooponente.
		
		
	
	% Verranno di seguito riportate le spese calcolate a posteriori sia per ruolo che per persona.
	% Sarà quindi riportato un bilancio per ogni \insglo{fase}. Esso potrà essere:
	% \begin{itemize}
		% \item \textbf{positivo}: se si è riusciti ad abbassare i costi e quindi il consuntivo è minore del preventivo;
		% \item \textbf{in pari}: se le spese totali previste sono rimaste inalterate anche se ci sono stati prolungamenti di tempo in alcuni ruoli e diminuzioni in altri. Il consuntivo e il preventivo sono identici;
		% \item \textbf{negativo}: se le modifiche, apportate durante la \insglo{fase} appena conclusa, dei tempi per ruolo hanno causato un aumento dei costi, ovvero quando il consuntivo supera il preventivo.
	% \end{itemize}

	% \level{2}{Fase SD}
		% \level{3}{Consuntivo parziale}
			% Verranno di seguito riportate le ore ed i costi effettivamente sostenute durante la \insphase{Fase SD}. Si ricordi che durante questa \insglo{fase} le spese sono rendicontate.
			% \begin{table}[H]
				% \begin{center}
					% \begin{tabular}{| l | c | c |}
						% \hline
						% Ruolo 				& Ore 	& Costi  \\ \hline
						
						% Project Manager		& 11 (-3) 		& \euro{} 330,00 (\euro{} -90,00)	\\
						% Amministratore 		& 13 (-5)		& \euro{} 260,00 (\euro{} -100,00)	\\
						% Analista	 		& 57 (-7)		& \euro{} 1~425,00 (\euro{} -175,00)	\\
						% Progettista 		& 63 (+8)		& \euro{} 1~386,00  (\euro{} +176,00)	\\
						% Programmatore		& 0				& \euro{} 0,00	\\
						% Verificatore		& 60 (+7)		& \euro{} 900,00 (\euro{} +105,00)	\\ \hline \hline
							
						% Totale Preventivo	& 204 			& \euro{} 4~301,00 	\\ \hline
						% Totale Consuntivo	& 204 			& \euro{} 4~217,00 	\\ \hline
						% Differenza			& - 			& \euro{} -84,00 	\\ \hline	
					% \end{tabular}
				% \end{center}
				% \caption{Consuntivo Fase SD}
			% \end{table}

			% Nella tabella seguente sono riportate le differenze tra le ore di lavoro previste per ogni componente con quelle realmente impiegate.
			% \begin{table}[H]
				% \begin{center}
					% \begin{tabular}{| l | c | c | c | c | c | c | c |}
						% \hline
						% Componente 					& PM	& Am	 & An 		& Pt 		& Pm 	& Ve 	& Ore Totali componente \\ \hline
						
						% Bigarella Chiara 			& 0		& 0		& 10 		& 17 		& 0		& 6 		& 33 \\
						% Bucco Riccardo 				& 7 (-3)& 0		& 0			& 17 (+3) 	& 0		& 11 		& 35 \\
						% Carlon Chiara	 			& 0		& 7 (-3)& 10 		& 0			& 0		& 11 (+3)	& 28 \\
						% Dal Bianco Davide 			& 0		& 3 (-1)& 0			& 17 		& 0		& 5	(+1)	& 25 \\
						% Moretto Alessandro 			& 0		& 0 	& 10 		& 12 		& 0		& 8 		& 30 \\
						% Pavanello Fabio Matteo	 	& 0		& 3 (-1)& 17 (-7)	& 0	(+5)	& 0		& 5	(+3)	& 25 \\
						% Rubin Marco					& 4 	& 0		& 10 		& 0			& 0		& 17 (+3)	& 28 \\ \hline \hline
						
						% Ore Totali Ruolo Preventivo	& 11 	& 13 	& 57 		& 63 		& 0		& 60		& 204\\ 
						% Ore Totali Ruolo Consuntivo	& 8 	& 8 	& 50 		& 71 		& 0		& 67		& 204\\ \hline
					% \end{tabular}
				% \end{center}
				% \caption{Consuntivo suddivisione ore di lavoro Fase SD}
			% \end{table}
		% \level{3}{Conclusioni}
			% Il gruppo \groupname{} in questa \insglo{fase} ha rilevato rallentamenti nel ruolo di \insrole{Progettista} ma nel contempo ha notato una riduzione dei tempi per l'attività di \insrole{Amministratore}, \insrole{Analista} e \insrole{Project Manager}. Nonostante questi cambiamenti di pianificazione, il bilancio è andato in positivo di \textbf{\euro{} 84,00}.
		% \level{3}{Preventivo a finire}
			% Da quanto riportato precedentemente si nota la possibilità di impiegare la parte di budget risparmiato nelle fasi successive. Il risparmio attuale di \textbf{\euro{} 84,00} (risparmiati in questa \insglo{fase}) consente di far fronte ad eventuali rallentamenti dovuti a rischi non preventivati o mal gestiti nelle fasi future.

	% \level{2}{Fase P} 
		% \level{3}{Consuntivo parziale}
			% Verranno di seguito riportate le ore ed i costi effettivamente sostenute durante la \insphase{Fase P}. Si ricordi che durante questa \insglo{fase} le spese sono rendicontate.
			% \begin{table}[H]
				% \begin{center}
					% \begin{tabular}{| l | c | c |}
						% \hline
						% Ruolo 				& Ore 	& Costi  \\ \hline
						
						% Project Manager		& 7 		& \euro{} 210,00 	\\
						% Amministratore 		& 9 		& \euro{} 180,00 	\\
						% Analista	 		& 25 (-2) 	& \euro{} 625,00 (\euro{} -50,00)	\\
						% Progettista 		& 61 (+2)	& \euro{} 1~342,00 (\euro{} +44,00) 	\\
						% Programmatore		& 45 		& \euro{} 675,00 	\\
						% Verificatore		& 77 		& \euro{} 1155,00 	\\ \hline \hline
						
						% Totale Preventivo	& 224 		& \euro{} 4~187,00 	\\ \hline
						% Totale Consuntivo	& 224 		& \euro{} 4~181,00 	\\ \hline
						% Differenza			& - 		& \euro{} -6,00 	\\ \hline
					% \end{tabular}
				% \end{center}
				% \caption{Consuntivo Fase P}
			% \end{table}


			% Nella tabella seguente sono riportate le differenze tra le ore di lavoro previste per ogni componente con quelle realmente impiegate.

			% \begin{table}[H]
				% \begin{center}
					% \begin{tabular}{| l | c | c | c | c | c | c | c |}
						% \hline
						% Componente 					& PM	& Am 	& An 	& Pt 		& Pm 		& Ve 	& Ore Totali componente \\ \hline
						
						% Bigarella Chiara 			& 0		& 0		& 0		& 8 		& 14 		& 9 		& 31 \\
						% Bucco Riccardo 				& 0		& 0		& 0		& 7 		& 12		& 15 		& 34 \\
						% Carlon Chiara	 			& 0		& 0		& 13 (-2) & 0		& 11 		& 6 		& 30 \\
						% Dal Bianco Davide 			& 7 	& 0		& 0		& 15 		& 0			& 11 		& 33 \\
						% Moretto Alessandro 			& 0		& 5 	& 0		& 15 		& 4 		& 10 		& 34 \\
						% Pavanello Fabio Matteo	 	& 0		& 4		& 12 	& 0			& 4 		& 13 		& 33 \\
						% Rubin Marco					& 0		& 0 	& 0		& 16 (+2)	& 0			& 13 		& 29 \\ \hline \hline
						
						% Ore Totali Ruolo Preventivo & 7 	& 9 	& 25 	& 61 		& 45 		& 77 		& 224	\\ 
						% Ore Totali Ruolo Consuntivo & 7 	& 9 	& 23    & 63		& 45 		& 77 		& 224	\\ \hline
					% \end{tabular}
				% \end{center}
				% \caption{Consuntivo suddivisione ore di lavoro Fase P}
			% \end{table}

		% \level{3}{Conclusioni} 
			% In questa \insglo{fase} non si sono riscontrati particolari problemi di pianificazione se non una piccola differenza tra il ruolo di \insrole{Analista} e \insrole{Progettista}. Tale differenza ha comunque portato un bilancio positivo di \textbf{\euro{} 6,00}.
		% \level{3}{Preventivo a finire}
			% Il totale risparmiato fino ad ora ammonta a \textbf{\euro{} 90,00} (\euro{} 84,00 nella \insphase{Fase SD} e \euro{} 6,00 in questa \insglo{fase}). Ciò ci consente di far fronte ad eventuali rallentamenti dovuti a rischi non preventivati o mal gestiti nelle fasi future.
	

	% \level{2}{Fase IP}
		% \level{3}{Consuntivo parziale}
			% Verranno di seguito riportate le ore ed i costi effettivamente sostenute durante la \insphase{Fase IP}. Si ricordi che durante questa \insglo{fase} le spese sono rendicontate.
			% \begin{table}[H]
				% \begin{center}
					% \begin{tabular}{| l | c | c |}
								% \hline
								% Ruolo 				& Ore 		& Costi  \\ \hline
							
								% Project Manager		& 5 (-2)		& \euro{} 150,00 (\euro{}-60,00) 	\\
								% Amministratore 		& 2 			& \euro{} 40,00 	\\
								% Analista	 		& 13 (-2)		& \euro{} 325,00 (\euro{}-75,00)	\\
								% Progettista 		& 21 (+9)		& \euro{} 462,00 (\euro{}+198,00) 	\\
								% Programmatore		& 19 (-2)		& \euro{} 285,00 (\euro{}-30,00)	\\
								% Verificatore		& 41 (-2)		& \euro{} 615,00 (\euro{}-30,00)	\\ \hline \hline
									
								% Totale Preventivo 	& 101 			& \euro{} 1~877,00 	\\ \hline
								% Totale Consuntivo	& 101			& \euro{} 1~880,00 	\\ \hline
								% Differenza			& 0				& \euro{} +3,00 	\\ \hline
							% \end{tabular}
				% \end{center}
				% \caption{Consuntivo Fase IP}
			% \end{table}


			% Nella tabella seguente sono riportate le differenze tra le ore di lavoro previste per ogni componente con quelle realmente impiegate.

			% \begin{table}[H]
				% \begin{center}
					% \begin{tabular}{| l | c | c | c | c | c | c | c |}
						% \hline
						% Componente 					& PM	& Am	& An 	& Pt 		& Pm 		& Ve 	& Ore Totali componente \\ \hline
						
						% Bigarella Chiara 			& 5 (-2) & 0	& 0		& 4	(+4)	& 4 (-2)	& 0		& 13 \\
						% Bucco Riccardo 				& 0		& 0		& 0		& 0			& 2			& 12 	& 14 \\
						% Carlon Chiara	 			& 0		& 2 	& 0		& 7 (+5)	& 2 		& 5 (-5) 	& 16 \\
						% Dal Bianco Davide 			& 0		& 0		& 0		& 0			& 5 		& 8 	& 13 \\
						% Moretto Alessandro 			& 0		& 0		& 13 (-3) & 0		& 0			& 0	(+3)	& 13 \\
						% Pavanello Fabio Matteo	 	& 0		& 0		& 0		& 0			& 3 		&11 	& 14 \\
						% Rubin Marco					& 0		& 0 	& 0		& 10 		& 3 		& 5		& 18 \\ \hline \hline
						
						% Ore Totali Ruolo Preventivo & 5 	& 2 	& 13 	& 21 		& 19 		& 41 	& 101\\ 
						% Ore Totali Ruolo Consuntivo & 3 	& 2 	& 10 	& 30 		& 17 		& 39 	& 101\\ \hline
					% \end{tabular}
				% \end{center}
				% \caption{Consuntivo suddivisione ore di lavoro Fase IP}
			% \end{table}

		% \level{3}{Conclusioni} 
			% In questa \insglo{fase} sono stati riscontrati dei problemi di pianificazione nel calcolo delle ore di \insrole{Progettista}. Tale differenza ha comunque portato un bilancio negativo di \textbf{\euro{} 3,00}.
		% \level{3}{Preventivo a finire}
			% Il totale risparmiato fino ad ora ammonta a \textbf{\euro{} 87,00} (\euro{} +84,00 nella \insphase{Fase SD}, \euro{} +6,00 nella \insphase{Fase P} e \euro{} -3,00 in questa \insglo{fase}). Ciò ci consente di far fronte ad eventuali rallentamenti dovuti a rischi non preventivati o mal gestiti nelle fasi future.
	
	% \level{2}{Fase CP}
		% \level{3}{Consuntivo parziale}
			% Verranno di seguito riportate le ore ed i costi effettivamente sostenute durante la \insphase{Fase CP}. Si ricordi che durante questa \insglo{fase} le spese sono rendicontate.
			% \begin{table}[H]
				% \begin{center}
					% \begin{tabular}{| l | c | c |}
						% \hline
						% Ruolo 				& Ore 		& Costi  \\ \hline
						
						% Project Manager		& 5 (-3)	& \euro{} 150,00 (\euro{}-90,00)	\\
						% Amministratore 		& 2 		& \euro{} 40,00 	\\
						% Analista	 		& 18 (-7)	& \euro{} 450,00 (\euro{}-175,00)	\\
						% Progettista 		& 22 (+5)	& \euro{} 484,00 (\euro{}+110,00) 	\\
						% Programmatore		& 30 (+3)	& \euro{} 450,00 (\euro{}+45,00)	\\
						% Verificatore		& 42 (+2)	& \euro{} 630,00 (\euro{}+30,00)	\\ \hline \hline
						
						% Totale Preventivo	& 119 		& \euro{} 2~204,00 	\\ \hline
						% Totale Consuntivo	& 119 		& \euro{} 2~124,00  \\ \hline
						% Differenza			& 0 		& \euro{} 80,00 	\\ \hline
					% \end{tabular}
				% \end{center}
				% \caption{Consuntivo Fase CP}
			% \end{table}


			% Nella tabella seguente sono riportate le differenze tra le ore di lavoro previste per ogni componente con quelle realmente impiegate.

			% \begin{table}[H]
				% \begin{center}
					% \begin{tabular}{| l | c | c | c | c | c | c | c |}
						% \hline
						% Componente 					& PM	& Am 	& An 	& Pt 		& Pm 	& Ve 		& Ore Totali componente \\ \hline
						
						% Bigarella Chiara 			& 0		& 0		& 0		& 6 (+5)		& 0		& 8 (-5)		& 14 \\
						% Bucco Riccardo 				& 0		& 0		& 0		& 4 		& 0		& 7 		& 11 \\
						% Carlon Chiara	 			& 0		& 2 	& 0		& 12 		& 0		& 4 		& 18 \\
						% Dal Bianco Davide 			& 0		& 0		& 0		& 0			& 12 	& 9 		& 21 \\
						% Moretto Alessandro 			& 0		& 0		& 0		& 0			& 11 	& 5			& 16 \\
						% Pavanello Fabio Matteo	 	& 5 (-3) & 0	& 5		& 0			& 7 (+3) & 5 		& 22 \\
						% Rubin Marco					& 0		& 0		& 13 (-7) & 0		& 0		& 4 (+7)	& 17 \\ \hline \hline
						
						% Ore Totali Ruolo Preventivo & 5 	& 2 	& 18 	& 22 		& 30 	& 42 		& 119\\ 
						% Ore Totali Ruolo Consuntivo & 2 	& 2 	& 11 	& 27 		& 33 	& 44 		& 119\\ \hline
					% \end{tabular}
				% \end{center}
				% \caption{Consuntivo suddivisione ore di lavoro Fase CP}
			% \end{table}

		% \level{3}{Conclusioni} 
			% In questa \insglo{fase} sono state riscontrate delle differenze di pianificazione principalmente nel calcolo delle ore di \insrole{Progettista} e \insrole{Analista}. Tale differenza ha comunque portato un bilancio positivo di \textbf{\euro{} 80,00}.
		% \level{3}{Preventivo a finire}
			% Il totale risparmiato fino ad ora ammonta a \textbf{\euro{} 167,00} (\euro{} +84,00 nella \insphase{Fase SD}, \euro{} +6,00 nella \insphase{Fase P}, \euro{} -3,00 nella \insphase{Fase IP} e \euro{} +80,00 in questa \insglo{fase}). Ciò ci consente di far fronte ad eventuali rallentamenti dovuti a rischi non preventivati o mal gestiti nelle fasi future.
	
	% \level{2}{Fase PD}
		% \level{3}{Consuntivo finale}
			% Verranno di seguito riportate le ore ed i costi effettivamente sostenute durante la \insphase{Fase PD}. Si ricordi che durante questa \insglo{fase} le spese sono rendicontate.
			% \begin{table}[H]
				% \begin{center}
					% \begin{tabular}{| l | c | c |}
						% \hline
						% Ruolo 				& Ore 		& Costi  \\ \hline
						
						% Project Manager		& 4(-2) 		& \euro{} 60,00 	\\
						% Amministratore 		& 3(-2) 		& \euro{} 20,00 	\\
						% Analista	 		& 0			& \euro{} 0,00	\\
						% Progettista 		& 9 (-2)		& \euro{} 198,00 (\euro{}-44,00)  	\\
						% Programmatore		& 17 (-4) 		& \euro{} 255,00 (\euro{}-60,00)	\\
						% Verificatore		& 54 (+10)		& \euro{} 810,00 (\euro{}+150,00) 	\\ \hline \hline
						
						% Totale Preventivo	& 87 		& \euro{} 1~443,00 	\\ \hline
						% Totale Consuntivo	&  87 			& \euro{}  1~422,00 	\\ \hline
						% Differenza			&  			& \euro{}  	21,00\\ \hline
					% \end{tabular}
				% \end{center}
				% \caption{Consuntivo Fase PD}
			% \end{table}


			% Nella tabella seguente sono riportate le differenze tra le ore di lavoro previste per ogni componente con quelle realmente impiegate.

			% \begin{table}[H]
				% \begin{center}
					% \begin{tabular}{| l | c | c | c | c | c | c | c |}
						% \hline
						% Componente 					& PM	& Am 	& An 	& Pt 		& Pm 	& Ve 		& Ore Totali componente \\ \hline
						
						% Bigarella Chiara 			& 0		& 0		& 0		& 0		& 5 (-1)		& 10 	& 14 \\
						% Bucco Riccardo 				& 0		& 0		& 0		& 0		& 7 (-1)			& 5 		& 11 \\
						% Carlon Chiara	 			& 2 	& 0		& 0		& 0		& 0			& 4 		& 13\\
						% Dal Bianco Davide 			& 0		& 0		& 0		& 0		& 0			& 13 		& 13  \\
						% Moretto Alessandro 			& 0		& 1 	& 0		& 0		& 0			& 11		& 12 \\
						% Pavanello Fabio Matteo	 	& 0		& 0		& 0		& 9 (-2) 	& 0			& 4		& 11  \\
						% Rubin Marco					& 0		& 0		& 0		& 0		& 5 (-2) 		& 10		& 13 \\ \hline \hline
						
						% Ore Totali Ruolo Preventivo 			& 4 	& 3 	& 0		& 7 	& 17 		& 54 		& 87\\
						% Ore Totali Ruolo Consuntivo 			& 2 	& 1 	& 0		& 7 	& 13 		& 54 		& 87\\ \hline
					% \end{tabular}
				% \end{center}
				% \caption{Consuntivo suddivisione ore di lavoro Fase PD}
			% \end{table}

		% \level{3}{Conclusioni}
		% In questa \insglo{fase} sono state riscontrate delle differenze di pianificazione principalmente nel calcolo delle ore di \insrole{Verificatore}. Tale differenza ha comunque portato un bilancio positivo di \textbf{\euro{} 21,00}.
		% \level{3}{Preventivo a finire}
		% Il totale risparmiato complessivamente nello svolgimento del progetto ammonta a \textbf{\euro{} 188,00} (\euro{} +84,00 nella \insphase{Fase SD}, \euro{} +6,00 nella \insphase{Fase P}, \euro{} -3,00 nella \insphase{Fase IP}, \euro{} +80,00 nella \insphase{Fase CP} e \euro{} 21,00 in questa \insglo{fase}). Questa cifra verrà oramai sottratta dal costo finale del progetto imputato al proponente.
	
	
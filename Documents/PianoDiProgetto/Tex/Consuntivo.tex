% !TEX encoding = UTF-8 Unicode
\level{1}{Consuntivo}
	Verranno di seguito riportate le spese calcolate a posteriori sia per ruolo che per persona.
	Sarà quindi riportato un bilancio per ogni fase. Esso potrà essere:
	\begin{itemize}
		\item \textbf{positivo}: se si è riusciti ad abbassare i costi e quindi il consuntivo è minore del preventivo;
		\item \textbf{in pari}: se le spese totali previste sono rimaste inalterate anche se ci sono stati prolungamenti di tempo in alcuni ruoli e diminuzioni in altri. Il consuntivo e il preventivo sono identici;
		\item \textbf{negativo}: se le modifiche, apportate durante la fase appena conclusa, dei tempi per ruolo hanno causato un aumento dei costi, ovvero quando il consuntivo supera il preventivo.
	\end{itemize}
	
	%\level{2}{Fase DB}
	%	\level{3}{Consuntivo parziale}
	%		Verranno di seguito riportate le ore ed i costi effettivamente sostenute durante la \insphase{Fase DB}. Si ricordi che durante questa fase le spese non sono rendicontato ma sono considerate di investimento.
	%		\begin{table}[H]
	%			\begin{center}
	%				\begin{tabular}{| l | c | c |}
	%					\hline
	%					Ruolo 				& Ore 			& Costi  \\ \hline
						
	%					Project Manager		& 32 			& \euro{} 960 	\\
	%					Amministratore 		& 54 (-4)		& \euro{} 1080 (-\euro{} 80)	\\
	%					Analista	 		& 109 (+4)		& \euro{} 2725 (+\euro{} 100)	\\
	%					Progettista 		& 0				& \euro{} 0 	\\
	%					Programmatore		& 0				& \euro{} 0		\\
	%					Verificatore		& 77			& \euro{} 1155 	\\ \hline \hline
						
	%					Totale Preventivo	& 272 			& \euro{} 5920 	\\ \hline
	%					Totale Consuntivo	& 272 			& \euro{} 5940 	\\ \hline
	%					Differenza			& - 			& \euro{} -20 	\\ \hline
	%				\end{tabular}
	%			\end{center}
	%			\caption{Consuntivo Fase DB}
	%		\end{table}
	%	\level{3}{Conclusioni}
	%		Il gruppo \groupname{} in questa fase ha rilevato rallentamenti nel ruolo di \insrole{Analista} ma nel contempo hanno notato una riduzione dei tempi per l'attività di \insrole{Amministratore}. Tali cambiamenti hanno generato un bilancio negativo di \euro{} 20. Ciò nonostante il team ha ritenuto questa una spesa (non a carico del committente) aggiuntiva ininfluente a fronte dell'esperienza acquisita. 

	\level{2}{Fase SD}
		\level{3}{Consuntivo parziale}
	Verranno di seguito riportate le ore ed i costi effettivamente sostenute durante la \insphase{Fase SD}. Si ricordi che durante questa fase le spese sono rendicontate.
	\begin{table}[H]
		\begin{center}
			\begin{tabular}{| l | c | c |}
				\hline
				Ruolo 				& Ore 	& Costi  \\ \hline
				
				Project Manager		& 11 (-3) 		& \euro{} 330,00 (\euro{} -90,00)	\\
				Amministratore 		& 13 (-5)		& \euro{} 260,00 (\euro{} -100,00)	\\
				Analista	 		& 57 (-7)		& \euro{} 1~425,00 (\euro{} -175,00)	\\
				Progettista 		& 63 (+8)		& \euro{} 1~386,00  (\euro{} +176,00)	\\
				Programmatore		& 0				& \euro{} 0,00	\\
				Verificatore		& 60 (+7)		& \euro{} 900,00 (\euro{} +105,00)	\\ \hline \hline
					
				Totale Preventivo	& 204 			& \euro{} 4~301,00 	\\ \hline
				Totale Consuntivo	& 204 			& \euro{} 4~217,00 	\\ \hline
				Differenza			& - 			& \euro{} -84,00 	\\ \hline	
			\end{tabular}
		\end{center}
		\caption{Consuntivo Fase SD}
	\end{table}
		\level{3}{Conclusioni}
			Il gruppo \groupname{} in questa fase ha rilevato rallentamenti nel ruolo di \insrole{Progettista} ma nel contempo ha notato una riduzione dei tempi per l'attività di \insrole{Amministratore}, \insrole{Analista} e \insrole{Project Manager}. Nonostante questi cambiamenti di pianificazione, il bilancio è andato in positivo di \textbf{\euro{} 84,00}.
		\level{3}{Preventivo a finire}
			Da quanto riportato precedentemente si nota la possibilità di impiegare la parte di budget risparmiato nelle fasi successive. Il risparmio attuale di \textbf{\euro{} 84,00} (risparmiati in questa fase) consente di far fronte ad eventuali rallentamenti dovuti a rischi non preventivati o mal gestiti nelle fasi future.

	\level{2}{Fase P} 
		\level{3}{Consuntivo parziale}
	Verranno di seguito riportate le ore ed i costi effettivamente sostenute durante la \insphase{Fase P}. Si ricordi che durante questa fase le spese sono rendicontate.
	\begin{table}[H]
		\begin{center}
			\begin{tabular}{| l | c | c |}
				\hline
				Ruolo 				& Ore 	& Costi  \\ \hline
				
				Project Manager		& 7 		& \euro{} 210,00 	\\
				Amministratore 		& 9 		& \euro{} 180,00 	\\
				Analista	 		& 25 (-2) 		& \euro{} 625,00 (\euro{} -50,00)	\\
				Progettista 		& 61 (+2)		& \euro{} 1~342,00 (\euro{} +44,00) 	\\
				Programmatore		& 45 		& \euro{} 675,00 	\\
				Verificatore		& 77 		& \euro{} 1155,00 	\\ \hline \hline
				
				Totale Preventivo	& 224 			& \euro{} 4~187,00 	\\ \hline
				Totale Consuntivo	& 224 			& \euro{} 4~181,00 	\\ \hline
				Differenza			& - 			& \euro{} -6,00 	\\ \hline
			\end{tabular}
		\end{center}
		\caption{Consuntivo Fase P}
	\end{table}
		\level{3}{Conclusioni} 
			In questa fase non si sono riscontrati particolari problemi di pianificazione se non una piccola differenza tra il ruolo di \insrole{Analista} e \insrole{Progettista}. Tale differenza ha comunque portato un bilancio positivo di \textbf{\euro{} 6,00}.
		\level{3}{Preventivo a finire}
			Il totale risparmiato fino ad ora ammonta a \textbf{\euro{} 90,00} (\euro{} 84,00 nella \insphase{Fase SD} e \euro{} 6,00 in questa fase). Ciò ci consente di far fronte ad eventuali rallentamenti dovuti a rischi non preventivati o mal gestiti nelle fasi future.
	

	%\level{2}{Fase IP}
	%	\level{3}{Consuntivo parziale}
	%Verranno di seguito riportate le ore ed i costi effettivamente sostenute durante la \insphase{Fase IP}. Si ricordi che durante questa fase le spese sono rendicontate.
	%\begin{table}[H]
	%	\begin{center}
			%\begin{tabular}{| l | c | c |}
			%			\hline
			%			Ruolo 				& Ore 			& Costi  	\\ \hline
						
			%			Project Manager		&  				& \euro{}  	\\
			%			Amministratore 		& 				& \euro{} 	\\
			%			Analista	 		& 				& \euro{} 	\\
			%			Progettista 		& 				& \euro{} 	\\
			%			Programmatore		& 				& \euro{}	\\
			%			Verificatore		& 				& \euro{} 	\\ \hline \hline
				
			%			Totale Preventivo	&  				& \euro{}  	\\ \hline
			%			Totale Consuntivo	&  				& \euro{}  	\\ \hline
			%			Differenza			&  				& \euro{}  	\\ \hline
			%		\end{tabular}
	%	\end{center}
	%	\caption{Consuntivo Fase IP}
	%\end{table}
	%	\level{3}{Conclusioni}
	%	\level{3}{Preventivo a finire}
	
	%\level{2}{Fase CP}
	%	\level{3}{Consuntivo parziale}
	%Verranno di seguito riportate le ore ed i costi effettivamente sostenute durante la \insphase{Fase CP}. Si ricordi che durante questa fase le spese sono rendicontate.
	%\begin{table}[H]
	%	\begin{center}
			%TODO tabular
	%	\end{center}
	%	\caption{Consuntivo Fase CP}
	%\end{table}
	%	\level{3}{Conclusioni}
	%	\level{3}{Preventivo a finire}
	
	%\level{2}{Fase PD}
	%	\level{3}{Consuntivo parziale}
	%Verranno di seguito riportate le ore ed i costi effettivamente sostenute durante la \insphase{Fase PD}. Si ricordi che durante questa fase le spese sono rendicontate.
	%\begin{table}[H]
	%	\begin{center}
			%TODO tabular
	%	\end{center}
	%	\caption{Consuntivo Fase PD}
	%\end{table}
	%	\level{3}{Conclusioni}
	%	\level{3}{Preventivo a finire}
	
	%\level{2}{Preventivo a finire}
	

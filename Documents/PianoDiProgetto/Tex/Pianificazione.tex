\level{2}{Pianificazione}
	Si è deciso di dividere lo sviluppo del progetto in varie fasi. Il termine di ognuna di esse è segnato da una \insglo{milestone}. Una \insglo{milestone} può essere o una scadenza di consegna di revisione o un incontro con il proponente. Si è preferito fissare \insglo{milestone} non a lunga distanza tra loro per ridurre i rischi e permettere di aver un resoconto da parte del proponente ad ogni \insglo{fase}. Per ogni \insglo{fase} di sviluppo vengono pianificati vari periodi di verifica per ogni attività, solitamente uno ogni 7 giorni con durata media di 2 giorni. Ogni verifica è importante perché quella diventerà una baseline.\\
	Il periodo di svolgimento del progetto è quindi stato diviso nelle seguenti fasi:
	\begin{description}
		\item[\insglo{Fase} DB (Documentation Beginning):] Questa \insglo{fase} è caratterizzata da 3 sottofasi:
			\begin{itemize}
				\item individuazione/creazione degli strumenti per documentazione e di supporto;
				\item creazione \insdoc{Norme di Progetto};
				\item creazione documentazione (\insdoc{Studio di Fattibilità}, seguito da \insdoc{Analisi dei Requisiti}, \insdoc{Piano di Progetto}, \insdoc{Piano di Qualifica} e \insdoc{Glossario}).
			\end{itemize}
			In questa \insglo{fase} l'attività più onerosa sarà quella di analisi.\\Tra la seconda e la terza sottofase viene effettuato un incontro con il proponente per chiarire le idee sulla comprensione del \insglo{capitolato}.\\Tale \insglo{fase} si conclude con la \insrev{Revisione dei Requisiti}. In tale modo si avrà un riscontro immediato sulle intenzioni del proponente.
		\item[\insglo{Fase} DI (Documentation Improvement):] Caratterizzata dall’analisi di dettaglio. In questa \insglo{fase} verranno consolidati i requisiti  e verrà portato un incremento o una correzione a tutti i documenti scritti fino ad ora, se necessario.\\Tale \insglo{fase} si conclude con un incontro con il proponente che visionerà e confermerà le modifiche fatte al documento \insdoc{Analisi dei Requisiti}.
		\item[\insglo{Fase} SD (\insglo{Software} Design):] Caratterizzata dalla progettazione architetturale. Questa \insglo{fase} ha inizio quando il documento \insdoc{Analisi dei Requisiti} è nello stato tra \textit{acceptable} e \textit{addressed} (fine \insphase{Fase DI}) e termina con la visione del proponente. Verranno apportati incrementi ad alcuni documenti prodotti nelle precedenti fasi, come \insdoc{Norme di Progetto}, \insdoc{Piano di Progetto}, \insdoc{Glossario} e \insdoc{Piano di Qualifica}. Al proponente si prevede di mostrare il documento \insdoc{Specifica Tecnica}.
		\item[\insglo{Fase} P (Prototyping):] In questa \insglo{fase} si procede con la progettazione di dettaglio e la codifica dei requisiti obbligatori. Tale \insglo{fase} si conclude con la visione del proponente di un primo prototipo che dovrà soddisfare i requisiti obbligatori e la scadenza della consegna per la \insrev{Revisione di Progettazione}. Verranno apportati incrementi ai documenti prodotti nelle precedenti fasi. Alla revisione si prevede di consegnare il documento \insdoc{Specifica Tecnica} steso fino a quel momento ma non la progettazione al dettaglio dei requisiti obbligatori che verrà consegnata per la \insrev{RQ} insieme al resto.
		\item[\insglo{Fase} IP (Increase of the Prototype):] Avrà inizio subito dopo la scadenza della consegna per la \insrev{Revisione di Progettazione}. In questa \insglo{fase} si procede con la progettazione di dettaglio e la codifica dei requisiti desiderabili. Tale \insglo{fase} si conclude con la visione del proponente di un secondo prototipo che dovrà soddisfare i requisiti obbligatori e desiderabili. Verranno apportati incrementi ai documenti prodotti nelle precedenti fasi.
		\item[\insglo{Fase} CP (Completion of the Product):] Avrà inizio subito dopo la consegna per la \insphase{Fase IP}. In questa \insglo{fase} si procede con la progettazione di dettaglio e la codifica dei requisiti opzionali. Tale \insglo{fase} si conclude con l'incontro con il proponente per mostrare il prototipo con tutti i requisiti implementati e con la scadenza della consegna per la \insrev{Revisione di Qualifica}. Verranno apportati incrementi ai documenti prodotti nelle precedenti fasi. Si prevede quindi di consegnare le ultime versioni dei documenti ed il documento \insdoc{Definizione di Prodotto} completo di tutta la progettazione al dettaglio ed il codice.
		\item[\insglo{Fase} PD (Product Delivery):] Tale \insglo{fase} comincia non appena termina la \insphase{Fase CP}. Essa è caratterizzata dalla validazione e quindi il lavoro più oneroso sarà quello dei \insrole{verificatori}. In questa \insglo{fase} il progetto avrà termine. Verrà quindi effettuata la validazione del \insglo{software} creato e successivamente verrà collaudato. Tale \insglo{fase} si conclude con la \insrev{RA}.
	\end{description}
	La scelta effettuata ci permette di spezzare facilmente ognuna di queste 7 macro-fasi in attività più piccole. Questo permette di avere maggior controllo sull'avanzamento del progetto, e soprattutto dà la possibilità di applicare il \insglo{PDCA} molto frequentemente.\\Ad ognuna delle varie attività sono state associate una o più risorse. Delle sotto-attività è stato riportato unicamente il Gantt.\\ 
	Di seguito saranno elencate le durate e le caratteristiche di ogni \insglo{fase}. I tempi sono stati pensati per permettere uno slack sufficiente per abbassare i rischi relativi alle tempistiche.
	\input{Tex/FaseDB.tex}
	% !TEX encoding = UTF-8 Unicode
\level{2}{Fase DI: Documentation Improvement}
\textbf{Periodo}: dal \insdate{16}{02}{2015} al \insdate{05}{03}{2015} \\
Questa \insglo{fase} comincia al termine della \insphase{Fase DB}. È caratterizzata da una nuova analisi di tutti i documenti redatti nella \insglo{fase} precedente e dalla correzione di questi in base alle richieste e segnalazioni del committente. Gli analisti provvedono, inoltre, all'individuazione di nuovi requisiti e alla correzione di quelli segnalati. Pertanto, i documenti vengono ampliati ed aggiornati alla versione 2.00.
\level{3}{Diagramma di Gantt delle attività}
\begin{center}
	\begin{figure}[H]\centering
		\includegraphics[width=\textwidth]{PianoDiProgetto/Pics/FaseDI.png}
		\caption{Gantt Fase DI}
	\end{figure}
\end{center}

	\input{Tex/FaseSD.tex}
	\input{Tex/FaseP.tex}
	% !TEX encoding = UTF-8 Unicode
\level{3}{Fase IP: Increase of the Prototype}
	\textbf{Periodo}: dal \insdate{20}{04}{2015} al \insdate{03}{05}{2015} \\Questa \insglo{fase} comincia con la fine della \insphase{Fase P} e termina con la visione del proponente di un secondo prototipo che dovrà soddisfare i requisiti obbligatori e desiderabili. .\\Le attività di questa \insglo{fase} saranno le seguenti:
	\begin{itemize}
		\item\textbf{Definizione di \insglo{Prodotto}}: Viene steso il documento \insdoc{Definizione di Prodotto v2.00}. Esso definisce la struttura interna del sistema e le relazioni dei componenti del \insglo{prodotto} relativi ai requisiti obbligatori e desiderabili.
		\item \textbf{Codifica}: con quest'attività inizia lo sviluppo da parte dei programmatori dei requisiti desiderabili. Sarà dunque seguito quanto riportato nel documento \insdoc{Definizione di Prodotto v2.00};
		\item \textbf{Esecuzione test}: verranno eseguiti automaticamente tutti i test di unità e integrazione previsti dal documento \insdoc{Piano di Qualifica v5.00};
		\item\textbf{Manuale Utente e Manuale Amministratore}: vengono ampliati ed aggiornati i manuali che forniranno indicazioni agli utilizzatori del sistema.
		\item\textbf{Incremento e Verifica Documenti}: Vengono eseguite modifiche ai documenti già scritti, se necessario, che passeranno alla versione 5.00.
		\item\textbf{Glossario}: Vengono aggiunti al file \insfile{Glossario.xml} i vocaboli dei quali si ritiene necessaria una definizione formale. Alla fine di questa \insglo{fase} viene quindi generato il documento \insdoc{Glossario v5.00}.
	\end{itemize}
	\level{4}{Diagramma di Gantt delle attività}
	\begin{figure}[H]\centering
		\includegraphics[width=\textwidth]{PianoDiProgetto/Pics/FaseIP.png}
	\caption{Gantt Fase IP}
\end{figure}

	% !TEX encoding = UTF-8 Unicode
\level{2}{Fase CP: Completion of the Product}
	\textbf{Periodo}: dal \insdate{19}{04}{2015} al \insdate{07}{05}{2015} \\Questa fase comincia subito dopo la scadenza della consegna per la \insrev{Revisione di Progetto} e termina con l'incontro con il proponente al fine di mostrare il prototipo con tutti i requisiti (obbligatori, desiderabili e opzionali) sviluppati. 
	\\Le attività di questa fase saranno le seguenti:
	\begin{itemize}
		\item\textbf{Definizione di Prodotto}: viene steso il documento \insdoc{Definizione di Prodotto v3.0}. Esso definisce la struttura interna del sistema e le relazioni dei componenti del prodotto relativi ai requisiti opzionali.
		\item \textbf{Codifica}: con quest'attività inizia lo sviluppo da parte dei programmatori dei requisiti opzionali. Sarà dunque seguito quanto riportato nel documento \insdoc{Definizione di Prodotto v3.00};
		\item \textbf{Esecuzione test}: verranno eseguiti automaticamente tutti i test di unità e integrazione previsti dal documento \insdoc{Piano di Qualifica v 6.00};
		\item\textbf{Manuale Utente e Manuale Amministratore}: comincia la stesura dei manuali che forniranno indicazioni agli utilizzatori del sistema.
		\item\textbf{Incremento e Verifica Documenti}: vengono eseguite modifiche ai documenti già scritti, se necessario.
		\item\textbf{Glossario}: vengono aggiunti al file \insfile{Glossario.xml} i vocaboli dei quali si ritiene necessaria una definizione formale. Alla fine di questa fase viene quindi generato il documento \insdoc{Glossario v6.00}.
	\end{itemize}
	\level{3}{Diagramma di Gantt delle attività}
	\begin{figure}[H]\centering
		\includegraphics[width=\textwidth]{PianoDiProgetto/Pics/FaseCP.png}
	\caption{Gantt Fase CP}
\end{figure}

	% !TEX encoding = UTF-8 Unicode
\level{2}{Fase PD: Product Delivery}
	\textbf{Periodo}: dal \insdate{24}{05}{2015} al \insdate{17}{06}{2015} \\Questa fase comincia con la fine della \insphase{Fase CP} e termina con la scadenza della consegna per la \insrev{RA}.
	\begin{itemize}
		\item \textbf{Incremento e Verifica}: se necessario verranno effettuati aggiornamenti ai vari documenti scritti;
		\item \textbf{Validazione}: viene verificato, attraverso tracciamento, di aver soddisfatto i requisiti presenti nel documento \insdoc{Analisi dei Requisiti v1.00};
		\item \textbf{Esecuzione test}: verranno eseguiti i test di sistema previsti dal documento \insdoc{Piano di Qualifica v 7.00};
		\item \textbf{Correzione bug}: i bug rilevati verranno risolti;
		\item \textbf{Collaudo}: viene eseguito e completamente collaudato il sistema creato.
	\end{itemize}
	\level{3}{Diagramma di Gantt delle attività}
		\begin{figure}[H]\centering
			\includegraphics[width=\textwidth]{PianoDiProgetto/Pics/FasePD.png}
		\caption{Gantt Fase PD}
\end{figure}

	

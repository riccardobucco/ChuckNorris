% !TEX encoding = UTF-8 Unicode
\level{3}{Fase IP: Increase of the Prototype}
	\textbf{Periodo}: dal \insdate{20}{04}{2015} al \insdate{03}{05}{2015} \\Questa \insglo{fase} comincia con la fine della \insphase{Fase P} e termina con la visione del proponente di un secondo prototipo che dovrà soddisfare i requisiti obbligatori e desiderabili. .\\Le attività di questa \insglo{fase} saranno le seguenti:
	\begin{itemize}
		\item\textbf{Definizione di \insglo{Prodotto}}: Viene steso il documento \insdoc{Definizione di Prodotto v2.00}. Esso definisce la struttura interna del sistema e le relazioni dei componenti del \insglo{prodotto} relativi ai requisiti obbligatori e desiderabili.
		\item \textbf{Codifica}: con quest'attività inizia lo sviluppo da parte dei programmatori dei requisiti desiderabili. Sarà dunque seguito quanto riportato nel documento \insdoc{Definizione di Prodotto v2.00};
		\item \textbf{Esecuzione test}: verranno eseguiti automaticamente tutti i test di unità e integrazione previsti dal documento \insdoc{Piano di Qualifica v5.00};
		\item\textbf{Manuale Utente e Manuale Amministratore}: vengono ampliati ed aggiornati i manuali che forniranno indicazioni agli utilizzatori del sistema.
		\item\textbf{Incremento e Verifica Documenti}: Vengono eseguite modifiche ai documenti già scritti, se necessario, che passeranno alla versione 5.00.
		\item\textbf{Glossario}: Vengono aggiunti al file \insfile{Glossario.xml} i vocaboli dei quali si ritiene necessaria una definizione formale. Alla fine di questa \insglo{fase} viene quindi generato il documento \insdoc{Glossario v5.00}.
	\end{itemize}
	\level{4}{Diagramma di Gantt delle attività}
	\begin{figure}[H]\centering
		\includegraphics[width=\textwidth]{PianoDiProgetto/Pics/FaseIP.png}
	\caption{Gantt Fase IP}
\end{figure}

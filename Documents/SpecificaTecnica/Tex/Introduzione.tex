% !TEX encoding = UTF-8 Unicode
\level{1}{Introduzione}

	\level{2}{Scopo del documento}
		Il presente documento ha lo scopo di definire la progettazione ad alto livello del prodotto \projectname{}.\\
		Vengono a tale scopo descritte le componenti, le classi e i design pattern utilizzari per la realizzazione del prodotto. Infine viene fatto il tracciamento tra le componenti e i requisiti individuati nel documento \insdoc{Analisi dei Requisiti v3.00}.

	\level{2}{Glossario}
	Allo scopo di rendere più semplice la comprensione dei documenti ed evitare eventuali ambiguità, viene allegato il \insdoc{Glossario v6.00}, che contiene la spiegazione della terminologia tecnica e degli acronimi utilizzati. Per facilitare la lettura, i termini presenti all'interno di tale documento saranno marcati da una “G” maiuscola a pedice.


	\level{2}{Riferimenti utili}
		\level{3}{Riferimenti normativi}
		\begin{itemize}
			\item\textbf{Capitolato d'appalto C3:} \projectname{}: Real-time Business Intelligence \\
				\insuri{http://www.math.unipd.it/~tullio/IS-1/2014/Progetto/C3.pdf};
			\item \textbf{Norme di Progetto:} \insdoc{Norme di Progetto v3.00}.
		\end{itemize}
		\level{3}{Riferimenti informativi}
		\begin{itemize}
			\item \textbf{Learning UML 2.0 - Kim Hamilton, Russell Miles - Chapter 2. Modelling Requirements: UseCases};
			\item \textbf{Diagramma delle classi:} \\ \insuri{http://www.math.unipd.it/~tullio/IS-1/2014/Dispense/E2a.pdf};
			\item \textbf{Diagramm dei package:} \\ \insuri{http://www.math.unipd.it/~tullio/IS-1/2014/Dispense/E2b.pdf};
			\item \textbf{Pattern MVC:} \\ \insuri{http://www.math.unipd.it/~rcardin/pdf/Design\%20Pattern\%20Architetturali\%20-\%20Model\%20View\%20Controller_4x4.pdf};
			\item \textbf{Pattern Observer:} \\ \insuri{http://www.math.unipd.it/~tullio/IS-1/2014/Dispense/E8.pdf}
			
		\end{itemize}

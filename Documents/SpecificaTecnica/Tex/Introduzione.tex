% !TEX encoding = UTF-8 Unicode
\level{1}{Introduzione}

	\level{2}{Scopo del documento}
		Il presente documento ha lo scopo di definire la progettazione ad alto livello del prodotto \projectname{}.\\
		Vengono a tale scopo descritte le componenti, le classi e i design pattern utilizzati per la realizzazione del prodotto. Infine viene fatto il tracciamento tra le componenti e i requisiti individuati nel documento \insdoc{Analisi dei Requisiti v3.00}.

	\level{2}{Glossario}
	Allo scopo di rendere più semplice la comprensione dei documenti ed evitare eventuali ambiguità, viene allegato il \insdoc{Glossario v6.00}, che contiene la spiegazione della terminologia tecnica e degli acronimi utilizzati. Per facilitare la lettura, i termini presenti all'interno di tale documento saranno marcati da una “G” maiuscola a pedice.


	\level{2}{Riferimenti utili}
		\level{3}{Riferimenti normativi}
		\begin{itemize}
			\item\textbf{Capitolato d'appalto C3:} \projectname{}: Real-time Business Intelligence \\
				\insuri{http://www.math.unipd.it/~tullio/IS-1/2014/Progetto/C3.pdf};
			\item \textbf{Norme di Progetto:} \insdoc{Norme di Progetto v3.00}.
		\end{itemize}
		\level{3}{Riferimenti informativi}
		\begin{itemize}
			\item \textbf{Learning UML 2.0 - Kim Hamilton, Russell Miles - Chapter 4. Modeling a System's Logical Structure: Introducing Classes and Class Diagrams};
			\item \textbf{Learning UML 2.0 - Kim Hamilton, Russell Miles - Chapter 5. Modeling a System's Logical Structure: Advanced Class Diagrams and Class Diagrams};
			\item \textbf{Diagramma delle classi:} \\ \insuri{http://www.math.unipd.it/~tullio/IS-1/2014/Dispense/E2a.pdf};
			\item \textbf{Diagramma dei package:} \\ \insuri{http://www.math.unipd.it/~tullio/IS-1/2014/Dispense/E2b.pdf};
			\item \textbf{Pattern MVC:} \\ \insuri{http://www.math.unipd.it/~rcardin/pdf/Design\%20Pattern\%20Architetturali\%20-\%20Model\%20View\%20Controller_4x4.pdf};
			\item \textbf{Pattern Observer:} \\ \insuri{http://www.math.unipd.it/~tullio/IS-1/2014/Dispense/E8.pdf};
			\item \textbf{Pattern Adapter:} \\ \insuri{http://www.math.unipd.it/~tullio/IS-1/2014/Dispense/E6.pdf};
			\item \textbf{Dependency Injection:} \\ \insuri{http://www.math.unipd.it/~tullio/IS-1/2014/Dispense/E9.pdf};
			\item \textbf{Pattern Bridge:} \\ \insuri{http://en.wikipedia.org/wiki/Bridge\_pattern}.
			\item \textbf{Node.js:} \\ \insuri{https://nodejs.org/api};
			\item \textbf{Socket.io:} \\ \insuri{http://socket.io/docs};
			\item \textbf{Express.js:} \\ \insuri{http://expressjs.org/4x/api.html};
			\item \textbf{HTML5:} \\ \insuri{http://www.w3schools.com/html/html5_intro.asp};
			\item \textbf{CSS3:} \\ \insuri{http://www.w3schools.com/css/default.asp};
			\item \textbf{Angular.js:} \\ \insuri{https://docs.angularjs.org/api};
			\item \textbf{Chart.js:} \\ \insuri{http://www.chartjs.org/docs/};
			\item \textbf{Dynatable.js:} \\ \insuri{https://www.dynatable.com/};
			\item \textbf{OpenStreetMap:} \\ \insuri{http://openstreetmap.it/impara/};
			\item \textbf{MPAndroidChart:}  
			  \begin{itemize}
				\item \insuri{https://github.com/PhilJay/MPAndroidChart};
				\item \insuri{http://code.tutsplus.com/tutorials/add-charts-to-your-android-app-using-mpandroidchart--cms-23335}.
			  \end{itemize}
			\item \textbf{Google Maps Android API v2:} \\ \insuri{https://developers.google.com/maps/documentation/android/start};
			
		\end{itemize}

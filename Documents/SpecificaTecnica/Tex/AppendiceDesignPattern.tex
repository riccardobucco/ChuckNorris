\level{1}{Design Patterns} \label{app:designpattern}
	\level{2}{Model View Controller} \label{app:MVC}
		Il Model-View-Controller (MVC) è un pattern architetturale utilizzato per dividere il codice in blocchi di funzionalità ben distinte, e viene utilizzato molto frequentemente nelle applicazioni in cui un insieme di informazioni deve essere rappresentato attraverso una interfaccia grafica.
		\level{3}{Componenti}
			MVC è quindi basato sul principio del disaccoppiamento dei tre oggetti di cui è composto, ovvero sulla riduzione del loro grado di dipendenza reciproca, allo scopo di fornire una maggiore robustezza, modularità e manutenibilità al software.\\
			Di seguito si riporta una breve descrizione dei componenti e delle loro caratteristiche. 
			\level{4}{Model}
				Il Model è il nucleo dell'applicazione. Definisce il modello dei dati realizzando la business logic, ovvero definisce gli oggetti secondo la logica di utilizzo dell'applicazione indicande anche le possibili operazioni effettuabili su di essi.\\
				Questo componente viene generalmente progettato attaverso tecniche object oriented.\\
				Nella struttura del pattern MVC, il Model è un componente passivo, che non ha relazioni uscenti forti, ma si occupa di notificare alla View eventuali aggiornamenti avvenuti in esso, solitamente attraverso l'implementazione del pattern Observer.
			\level{4}{View}
				La View si occupa di visualizzare i dati contenuti nel Model e fornisce l'interfaccia di interazione con utenti e agenti.\\
				Nella struttura del pattern MVC, la View cattura gli input dell'utente e li passa al Controller affinchè esegua le corrette operazioni sul Model. \\
				La View può essere soggetta a due tipi di aggiornamento:
				\begin{itemize}
					\item \textbf{Push model:} viene implementato attraverso il pattern Observer e soltanto quando MVC è usato in un solo ambiente di esecuzione; consiste nel fatto che sia il Model a emettere, senza essere sollecitato, la notifica in seguito alla quale la View verrà aggiornata.
					\item \textbf{Pull model:} viene implementato quando MVC è usato su diversi ambienti di esecuzione; in questo caso la View richiede al Model se sia necessario un aggiornamento solo al verificarsi di particolari eventi.
				\end{itemize}
			\level{4}{Controller}
				Il Controller è il collante fra la View e il Model. Si occupa infatti di implementare l'application logic, cioè l'insieme di operazioni eseguibili sul modello dei dati attraverso una particolare vista.\\
				Poichè potenzialmente possono esistere più View (si pensi per esempio ad una applicazione che dispone sia di una interfaccia web che di una applicazione per smartphone) è necessario che ad ogni View corrisponda un Controller.
 		\level{3}{Vantaggi nell'uso di MVC}
			Come detto, l'utilizzo del pattern MVC porta la riduzione delle dipendenze, offrendo diversi vantaggi. La modularità ottenuta permette il riutilizzo del Model in applicazioni con diverse View, oltre a rendere più semplice e rapida l'esecuzione di test. 
 		\level{3}{Svantaggi nell'uso di MVC}
			L'utilizzo del pattern MVC nello sviluppo di un software non ha solo benefici, ma ha anche un piccolo costo, derivante da una maggiore complessità e dalla necessità di un maggior numero di aggiornamenti.\\
  			La maggiore complessità è data dai livelli di indirezione introdotti dal pattern e dall'implementazione ad eventi necessaria per far comunicare i tre componenti. La maggiore frequenza degli aggiornamenti, invece, è un effetto collaterale della modularizzazione: frequenti cambiamenti al modello comportano spesso cambiamenti nelle View e quindi nei Controller.
	\level{2}{Singleton} \label{app:singleton}
		Lo scopo del pattern architetturale denominato Singleton è 
		assicurare l'esistenza di un'unica istanza di una classe e fornire
		un punto di accesso globale ad essa.\\
		Questo pattern è nato per rispondere alla necessità di non avere più
		istanze della stessa classe, anche nei linguaggi in cui non è possibile 
		usare una variabile globale, pur dando la possibilità alla classe di
		tener traccia di quella sua istanza.\\
		Il pattern Singleton è quindi applicabile ogniqualvolta debba esistere una 
		sola istanza di una certa classe in tutta l'applicazione, prestando però attenzione 
		al fatto che l'istanza sia estendibile tramite ereditarietà.\\
		Viene generalmente utilizzato per implementare altri pattern come Factory, 
		Builder e Façade.
		\level{3}{Vantaggi nell'uso del Singleton}
		L'utilizzo del Singleton nell'implementazione di un software apporta i seguenti 
		vantaggi:
		\begin{itemize}
		\item controllo completo di come e quando i client acedono all'interfaccia;
		\item evita l'utilizzo ingiustificato di variabili globali;
		\item consente di ridefinire le operazioni definite nel Singleton;
		\item permette di porre un limite massimo al numero di istanze di una certa classe.
		\end{itemize}
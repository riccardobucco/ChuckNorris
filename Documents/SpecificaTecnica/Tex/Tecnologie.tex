\level{1}{Tecnologie utilizzate}
In questa sezione sono elencate le principali tecnologie utilizzate per lo sviluppo del progetto. Per ognuna di esse viene descritto lo scopo per il quale vengono usate ed i vantaggi che se ne ricavano.
\level{2}{Node.js}
\level{3}{Descrizione}
L'utilizzo di \insglo{Node.js} è stato richiesto dal proponente. Si tratta di un sistema run-time cross-platform che utilizza l'engine Google V8 \insglo{JavaScript} per eseguire il codice; viene utilizzato nello sviluppo di applicazioni real-time, lato \insglo{server} e di rete. Nel progetto viene usato per la realizzazione della libreria come piattaforma di sviluppo.
\level{3}{Vantaggi}
I vantaggi nell'utilizzo di \insglo{Node.js} sono i seguenti:
\begin{itemize}
\item fornisce una semplice soluzione per lo sviluppo di programmi scalabili;
\item essendo leggero ed efficiente è particolarmente indicato per applicazioni real-time con un intensivo uso di dati;
\item usa un sistema I/O asincrono, non bloccante, basato su eventi; questo permette di sviluppare più facilmente sistemi responsivi.
\end{itemize}
\level{3}{Svantaggi}
Seguono gli svantaggi rilevati nell'utilizzo di \insglo{Node.js}:
\begin{itemize}
	\item essendo di recente creazione, alcuni moduli risultano instabili;
	\item essendo single-threaded, non è performante in caso di applicazioni con uso intensivo di CPU;
	\item utilizzando un sistema I/O asincrono, è possibile che l'uso di \insglo{callback} risulti eccessivo.
\end{itemize}

\level{2}{Socket.io}
\level{3}{Descrizione}
L'uso di \insglo{Socket.io} è stato richiesto dal proponente. È una libreria \insglo{JavaScript} che permette comunicazioni real-time bidirezionali e basate su eventi. Viene utilizzato per implementare la trasmissione degli aggiornamenti dei grafici.
\level{3}{Vantaggi}
L'utilizzo di \insglo{Socket.io} fornisce i seguenti vantaggi:
\begin{itemize}
\item permette di creare comunicazioni bi-direzionali tra \insglo{client} e \insglo{server};
\item gestisce la connessione in modo trasparente;
\item ogniqualvolta il \insglo{browser} lo supporti usa il protocollo \insglo{WebSocket}, il quale fornisce canali di comunicazione bidirezionali simultanei attraverso una singola connessione \insglo{TCP}.
\end{itemize}
\level{3}{Svantaggi}
L'utilizzo di \insglo{Socket.io} porta, anche, ad alcuni svantaggi:
\begin{itemize}
	\item richiede che, sia \insglo{client} che \insglo{server}, utilizzino le librerie di \insglo{Socket.io};
	\item essendo una tecnologia di recente creazione, soffre ancora di alcuni bug.
\end{itemize}

\level{2}{Express.js}
\level{3}{Descrizione}
\insglo{Express.js} è un \insglo{framework} di \insglo{Node.js} che mette a disposizione un \insglo{middleware} utilizzabile per la realizzazione dell'infrastruttura web. Il suo utilizzo è stato richiesto dal proponente.
\level{3}{Vantaggi}
\insglo{Express.js} permette di estendere il modulo \texttt{http} di \insglo{Node.js} rendendo più facile la gestione dell'indirizzamento del \insglo{server}, delle risposte, dei cookie e delle richieste di stato \insglo{HTTP}.
\level{3}{Svantaggi}
Non sono stati rilevati grossi svantaggi nell'uso di \insglo{Express.js}, il quale risulta essere un buon \insglo{framework} che fornisce molti metodi per estendere il modulo \insglo{http} di \insglo{Node.js}.

\level{2}{HTML5}
\level{3}{Descrizione}
Si è scelto di utilizzare HTML5 come \insglo{linguaggio di markup} per la creazione delle pagine web.
\level{3}{Vantaggi}
I vantaggi dell'uso di HTML5 sono i seguenti:
\begin{itemize}
\item è il linguaggio utilizzato dalla libreria \insglo{Chart.js} per produrre i grafici;
\item HTML5 è supportato da tutti i più recenti \insglo{browser};
\item assieme a CSS permette di creare pagine web più interattive;
\item è stato decretato standard dal W3C.
\end{itemize}
\level{3}{Svantaggi}
Nell'uso di HTML5, sono stati rilevati i seguenti svantaggi:
\begin{itemize}
\item è supportato solo dalle versioni più recenti dei \insglo{browser};
\item è un linguaggio ancora in via di sviluppo, il che comporta un possibile cambiamento dei costrutti; tuttavia la maggior parte di questo linguaggio è stata considerata stabile.
\end{itemize}

\level{2}{CSS3}
\level{3}{Descrizione}
È stato scelto di usare CSS3 come linguaggio di formattazione dei documenti HTML. Grazie ad esso è possibile fornire un \insglo{layout} alle pagine web per renderle accessibili ed usabili.
\level{3}{Vantaggi}
I vantaggi dell'uso di CSS3 sono i seguenti:
\begin{itemize}
\item permette il riuso del codice, il che comporta una maggiore velocità di caricamento delle pagine web;
\item consente una manutenzione più semplice e veloce;
\item permette la separazione tra la struttura e la presentazione.
\end{itemize}
\level{3}{Svantaggi}
L'unico svantaggio rilevato dall'uso di CSS3 è dato dal fatto che, essendo un linguaggio ancora in via di sviluppo, è supportato (non appieno) solo dai web \insglo{browser} più recenti.

\level{2}{Bower}
\level{3}{Descrizione}
Bower è un gestore di pacchetti per librerie Javascript che permette di recuperare e installare i pacchetti necessari per l'esecuzione di un sistema.
\level{3}{Vantaggi}
L'utilizzo di Bower fornisce i seguenti vantaggi:
\begin{itemize}
	\item evita che l'utente debba andare alla ricerca dei pacchetti, scaricarli ed installarli;
	\item è ottimizzato per il front-end;
	\item per dire cosa scaricare ed installare è sufficiente inserire le dipendenze in un file bower.json all'interno della cartella del sistema;
	\item facile da installare.
\end{itemize}
\level{3}{Svantaggi}
L'unico svantaggio rilevato consiste nel fatto che se due applicazioni hanno una dipendenza verso due versioni diverse dello stesso pacchetto, la dipendenza viene risolta verso una sola versione delle due creando potenziali problemi.


\level{2}{AngularJS}
\level{3}{Descrizione}
AngularJS è un \insglo{framework} \insglo{MVC} \insglo{JavaScript} che fornisce un metodo ben strutturato per la creazione di siti e applicazioni web. Questo \insglo{framework} è la libreria lato \insglo{client} ideale da usare insieme a \insglo{Node.js}.
\level{3}{Vantaggi}
L'utilizzo di AngularJS fornisce i seguenti vantaggi:
\begin{itemize}
\item fornisce un approccio pulito e strutturato per la creazione delle applicazioni lato \insglo{client};
\item utilizza gli oggetti \insglo{JavaScript};
\item facilita la corretta implementazione del \insglo{pattern} \insglo{MVC};
\item fornisce una semplice e flessibile interfaccia di filtraggio che permette di filtrare i dati nel passaggio da Model a View.
\end{itemize}
\level{3}{Svantaggi}
Uno degli svantaggi rilevato nell'uso di \insglo{Angular.js} è la documentazione povera di esempi d'utilizzo. Tuttavia, questo può essere risolto tramite un'approfondita ricerca nel web. Un altro svantaggio deriva dal fatto che il codice non è valido secondo lo standard del W3C. 

\level{2}{Google Chart}
\level{3}{Descrizione}
È stato scelto di usare la libreria \insglo{JavaScript} sviluppata da Google per la creazione di grafici in HTML5.
\level{3}{Vantaggi}
I vantaggi dell'uso di \insglo{Google chart} sono i seguenti:
\begin{itemize}
\item consente un facile inserimento dei grafici all'interno di pagine HTML attraverso l'utilizzo di tag <div>;
\item consente un'ampia personalizzazione delle impostazioni dei grafici;
\item fornisce un'ampia galleria di grafici con cui poter eventualmente estendere il \insglo{framework};
\item essendo modulare, permette di caricare solo i grafici di cui si ha bisogno.
\end{itemize}
\level{3}{Svantaggi}
Non sono stati riscontrati grossi svantaggi nell'uso di tale libreria, la quale è accompagnata da una documentazione organizzata e ben fornita, anche, di numerosi esempi.

\level{2}{DataTables}
\level{3}{Descrizione}
È stato scelto di usare la libreria \insglo{JavaScript} \insglo{DataTables} per la creazione di tabelle in HTML5. Tale libreria permette la creazione di tabelle dinamiche ed interattive. \insglo{DataTables} fornisce delle \insglo{API} per la manipolazione delle tabelle e per l'accesso ai dati contenuti in esse. 
\level{3}{Vantaggi}
L'utilizzo di \insglo{DataTables} offre i seguenti vantaggi:
\begin{itemize}
\item fornisce numerose opzioni aggiuntive ed estensioni, che rendono ampiamente personalizzabili le tabelle create;
\item fornisce metodi di ordinamento dei dati, che possono essere personalizzati a seconda delle necessità;
\item permette il filtraggio dei dati, basato sui dati o sulle righe;
\item permette la personalizzazione delle opzioni di impaginamento delle tabelle, in modo che siano facili da consultare per l'utente finale.
\end{itemize}
\level{3}{Svantaggi}
Nell'uso di \insglo{DataTables} non sono stati rilevati grossi svantaggi. \insglo{DataTables} offre un'ampia documentazione piena di esempi d'uso. Inoltre, grazie al forum, presente sul sito web ufficiale, si possono ottenere risposte ad eventuali problemi o dubbi che sorgono durante l'utilizzo della libreria.

\level{2}{OpenStreetMap}
\level{3}{Descrizione}
È stato scelto di utilizzare OpenStreetMap, per la visualizzazione di Map Charts, un progetto collaborativo finalizzato alla creazione di dati cartografici, liberi e gratuiti a chiunque. Il progetto è stato lanciato poiché molte delle mappe che normalmente si pensano libere, hanno in realtà restrizioni legali o tecniche. OpenStreetMap mette, inoltre, a disposizione i geodati su cui le mappe si basano, al contrario di moltre altre librerie, come Google Maps \insglo{API}.
\level{3}{Vantaggi}
L'utilizzo di OpenStreetMap offre i seguenti vantaggi:
\begin{itemize}
\item libertà: i dati geografici e le mappe di OpenStreetMap sono rilasciati con licenza creata specificamente per proteggere i \insglo{database}, la quale permette la copia, la modifica e la ridistribuzione delle mappe, inoltre sono completamente gratuite, quindi non occorre pagare per usufruire di funzionalità aggiuntive;
\item affidabilità: le mappe sono prive di errori inseriti volutamente per riconoscere le mappe copiate;
\item copertura mondiale: essendo un progetto collaborativo a livello mondiale, offre mappe molto precise di tutto il mondo;
\item flessibilità: i dati di OpenStreetMap possono essere usati in molte maniere diverse, principalmente per creare mappe ma anche per copiarle, trasferirle e modificarle per qualsiasi scopo.
\end{itemize}
\level{3}{Svantaggi}
\begin{itemize}
\item le mappe possono risultare imprecise, in quanto i dati non sono controllati sistematicamente;
\item il procedimento per poter cambiare il colore dei \textit{marker} è complesso.
\end{itemize}

\level{2}{MPAndroidChart}
\level{3}{Descrizione}
Si è scelto di utilizzare MPAndroidChart, una libreria per la creazione di chart in \insglo{Android}. Essa viene usata per la creazione di \insglo{bar chart} e \insglo{line chart}, ma ne supporta molti altri come pie chart, radar chart e scatter chart. Tale libreria fornisce anche delle funzioni di scaling (ridimensionamento), dragging (trascinamento), selecting (selezione) e di animazione. La scelta di una libreria per la creazione di chart in \insglo{Android} è ricaduta su MPAndroidChart principalmente per le poche librerie a disposizione, tuttavia sono comunque stati trovati alcuni aspetti positivi.
\level{3}{Vantaggi}
MPAndroidChart fornisce i seguenti benefici:
\begin{itemize}
\item ridimensionamento su entrambi gli assi di un chart, tramite touch-\insglo{gesture}, in modo separato;
\item salvataggio dei chart su SD-Card come immagini;
\item template di impostazioni predefinite;
\item legende create automaticamente ma personalizzabili;
\item grafici ampiamente personalizzabili.
\end{itemize}
\level{3}{Svantaggi}
Non sono stati trovati svantaggi nell'uso di tale libreria. Essa è accompagnata da un'ampia documentazione, che ci ha permesso uno studio più rapido e più semplice grazie ai numerosi esempi forniti.

\level{2}{GoogleMaps Android API v2}
\level{3}{Descrizione}
Si è scelto di utilizzare GoogleMaps \insglo{Android} \insglo{API} v2 per la visualizzazione di mappe. Tale libreria permette di aggiungere mappe basate sui dati di Google Maps, alla propria applicazione. Le \insglo{API} gestiscono automaticamente l'accesso ai \insglo{server} di Google Maps, il download dei dati, la visualizzazione della mappa e le risposte alle \insglo{gesture}. Inoltre è possibile usare tali \insglo{API} per aggiungere markers e per cambiare la view di una particolare area di una mappa.
\level{3}{Vantaggi}
L'utilizzo di GoogleMaps \insglo{Android} \insglo{API} v2 fornisce i seguenti vantaggi:
\begin{itemize}
\item è possibile inserire una mappa in un'\insglo{activity} come un fragment, con un semplice frammento di codice \insglo{XML};
\item fornisce nuove funzionalità come: mappe 3D, satellite, terreno, mappe ibride e transizioni animate
\item offre la possibilità di aggiungere la visualizzazione “StreetView”.
\end{itemize}
\level{3}{Svantaggi}
Non sono stati riscontrati grossi svantaggi nell'uso di tale libreria, la quale è accompagnata da una documentazione ben organizzata. Inoltre, si possono trovare facilmente online, molti esempi e tutorial.

\level{1}{Tecnologie utilizzate}
In questa sezione sono elencate le principali tecnologie utilizzate per lo sviluppo del progetto. Per ognuna di esse viene descritto lo scopo per il quale vengono usate ed i vantaggi che se ne ricavano.
\level{2}{Node.js}
\level{3}{Descrizione}
L'utilizzo di Node.js è stato richiesto dal proponente. Si tratta di un sistema run-time cross-platform che utilizza l'engine Google V8 JavaScript per eseguire il codice; viene utilizzato nello sviluppo di applicazioni real-time, lato server e di rete. Nel progetto viene usato per la realizzazione della libreria come piattaforma di sviluppo.
\level{3}{Vantaggi}
\begin{itemize}
\item fornisce una semplice soluzione per lo sviluppo di programmi scalabili;
\item essendo leggero ed efficiente è particolarmente indicato per applicazioni real-time con un intensivo uso di dati;
\item usa un sistema I/O asincrono, non bloccante, basato su eventi; questo permette di sviluppare più facilmente sistemi responsivi.
\end{itemize}

\level{2}{Socket.io}
\level{3}{Descrizione}
L'uso di Socket.io è stato richiesto dal proponente. È una libreria JavaScript che permette comunicazioni real-time bidirezionali e basate su eventi. Viene utilizzato per implementare la trasmissione degli aggiornamenti dei grafici.
\level{3}{Vantaggi}
\begin{itemize}
\item permette di creare comunicazioni bi-direzionali tra client e server;
\item gestisce la connessione in modo trasparente;
\item ogniqualvolta il browser lo supporti usa il protocollo WebSocket, il quale fornisce canali di comunicazione bidirezionali simultanei attraverso una singola connessione TCP.
\end{itemize}

\level{2}{Express.js}
\level{3}{Descrizione}
Express.js è un framework di Node.js che mette a disposizione un middleware utilizzabile per la realizzazione dell'infrastruttura web. Il suo utilizzo è stato richiesto dal proponente.

\level{3}{Vantaggi}
Express.js permette di estendere il modulo \texttt{http} di Node.js rendendo più facile la gestione dell'indirizzamento del server, delle risposte, dei cookie e delle richieste di stato HTTP.

\level{2}{HTML5}
\level{3}{Descrizione}
Si è scelto di utilizzare HTML5 come linguaggio di markup per la creazione delle pagine web.
\level{3}{Vantaggi}
\begin{itemize}
\item è il linguaggio utilizzato dalla libreria Chart.js per produrre i grafici;
\item HTML5 è supportato da tutti i più recenti browser;
\item assieme a CSS permette di creare pagine web più interattive;
\item è stato decretato standard dal W3C.
\end{itemize}

\level{2}{CSS3}
\level{3}{Descrizione}
È stato scelto di usare CSS3 come linguaggio di formattazione dei documenti HTML. Grazie ad esso è possibile fornire un layout alle pagine web per renderle accessibili ed usabili.
\level{3}{Vantaggi:}
\begin{itemize}
\item permette il riuso del codice, il che comporta una maggiore velocità di caricamento delle pagine web;
\item consente una manutenzione più semplice e veloce;
\item permette la separazione tra la struttura e la presentazione.
\end{itemize}

\level{2}{AngularJS}
\level{3}{Descrizione}
AngularJS è un framework MVC JavaScript che fornisce un metodo ben strutturato per la creazione di siti e applicazioni web. Questo framework è la libreria lato client ideale da usare insieme a Node.js.
\level{3}{Vantaggi}
\begin{itemize}
\item fornisce un approccio pulito e strutturato per la creazione delle applicazioni lato client;
\item utilizza gli oggetti JavaScript;
\item facilita la corretta implementazione del pattern MVC;
\item fornisce una semplice e flessibile interfaccia di filtraggio che permette di filtrare i dati nel passaggio da Model a View.
\end{itemize}

\level{2}{Chart.js}
\level{3}{Descrizione}
È stato scelto di usare la libreria JavaScript Chart.js per la creazione di grafici in HTML5.
\level{3}{Vantaggi}
\begin{itemize}
\item usa l'elemento canvas di HTML5, il quale è supportato da tutti i browser più recenti;
\item è libero da dipendenze ed è leggero;
\item fornisce un sistema responsive (ad esempio i grafici vengono ridimensionati se la finestra del browser cambia dimensioni);
\item essendo modulare, permette di caricare solo i grafici di cui si ha bisogno.
\end{itemize}

\level{2}{Dynatable}
\level{3}{Descrizione}
È stato scelto di usare la libreria JavaScript Dynatable per la creazione di tabelle in HTML5. Tale libreria permette la creazione di tabelle dinamiche ed interattive. Dynatable fornisce un framework per implementare i metodi di filtraggio, ricerca ed ordinamento dei dati contenuti nelle tabelle. Inoltre, mette a disposizione un metodo di impaginazione che permette una rappresentazione più compatta ma comprensibile per tabelle che contengono grandi quantità di dati.
\level{3}{Vantaggi}
\begin{itemize}
\item poichè i dati vengono normalizzati in un array di oggetti JSON, le interazioni, che agiscono su tali oggetti, risultano più rapide ed efficienti;
\item le operazioni DOM, come lettura e scrittura, sono raggruppate tra di loro, favorendo così la velocità delle interazioni con la tabella;
\item le operazioni, come filtraggio ed ordinamento, sono semplici funzioni JavaScript ed agiscono direttamente sugli oggetti JSON.
\end{itemize}

\level{2}{OpenStreetMap}
\level{3}{Descrizione}
È stato scelto di utilizzare OpenStreetMap, per la visualizzazione di Map Charts, un progetto collaborativo finalizzato alla creazione di dati cartografici, liberi e gratuiti a chiunque. Il progetto è stato lanciato poichè molte delle mappe che normalmente si pensano libere, hanno in realtà restrizioni legali o tecniche. OpenStreetMap mette, inoltre, a disposizione i geodati su cui le mappe si basano, al contrario di moltre altre librerie, come Google Maps API.
\level{3}{Vantaggi}
\begin{itemize}
\item libertà: i dati geografici e le mappe di OpenStreetMap sono rilasciati con licenza creata specificamente per proteggere i database, la quale permette la copia, la modifica e la ridistribuzione delle mappe, inoltre sono completamente gratuite, quindi non occorre pagare per usufruire di funzionalità aggiuntive;
\item affidabilità: le mappe sono prive di errori inseriti volutamente per riconiscere le mappe copiate;
\item copertura mondiale: essendo un progetto collaborativo a livello mondiale, offre mappe molto precise di tutto il mondo;
\item flessibilità: i dati di OpenStreetMap possono essere usati in molte maniere diverse, principalmente per creare mappe ma anche per copiarle, trasferirle e modificarle per qualsiasi scopo.
\end{itemize}

\level{2}{MPAndroidChart}
\level{3}{Descrizione}
Si è scelto di utilizzare MPAndroidChart, una libreria per la creazione di chart in Android. Essa viene usata per la creazione di bar chart e line chart, ma ne supporta molti altri come pie chart, radar chart e scatter chart. Tale libreria fornisce anche delle funzioni di scaling (ridimensionamento), dragging (trascinamento), selecting (selezione) e di animazione. La scelta di una libreria per la creazione di chart in Android è ricaduta su MPAndroidChart principalmente per le poche librerie a disposizione, tuttavia sono comunque stati trovati alcuni aspetti positivi.
level{3}{Vantaggi}
MPAndroidChart fornisce i seguenti benefici:
\begin{itemize}
\item ridimensionamento su entrambi gli assi di un chart, tramite touch-gesture, in modo separato;
\item salvataggio dei chart su SD-Card come immagini;
\item template di impostazioni predefinite;
\item legende create automaticamente ma personalizzabili;
\item grafici ampiamente personalizzabili.
\end{itemize}

\level{2}{GoogleMaps Android API v2}
\level{3}{Descrizione}
Si è scelto di utilizzare GoogleMaps Android API v2 per la visualizzazione di mappe. Tale libreria permette di aggiungere mappe basate sui dati di Google Maps, alla propria applicazione. Le API gestiscono automaticamente l'accesso ai server di Google Maps, il download dei dati, la visualizzazione della mappa e le risposte alle gesture. Inoltre è possibile usare tali API per aggiungere markers e per cambiare la view di una particolare area di una mappa.
\level{3}{Vantaggi}
\begin{itemize}
\item è possibile inserire una mappa in un'activity come un fragment, con un semplice frammento di codice XML;
\item fornisce nuove funzionalità come: mappe 3D, satellite, terreno, mappe ibride e transizioni animate
\item offre la possibilità di aggiungere la visualizzazione “StreetView“.
\end{itemize}
\level{1}{Stime di fattibilità e di bisogno di risorse}
Dopo aver definito l'architettura a un sufficiente livello di dettaglio, il team di sviluppo è in grado di fornire una stima della fattibilità.\\
Grazie alle tecnologie individuate si prevede di riuscire ad adempiere a tutte le richieste fatte dal proponente.\\
In questi mesi, infatti, il gruppo ha studiato le tecnologie necessarie per implementare il prodotto, prima sconosciute, capendo il loro utilizzo e i loro pregi e difetti, grazie anche all'interazione col proponente. Anche il fatto che uno dei membri del team stia seguendo un corso di Programmazione Embedded ha aiutato molto nella realizzazione dell'architettura dell'applicazione Android richiesta. \\
Altri fattori che ci avvantaggiano sono che l'architettura del prodotto richiesto sia simile a quella di altri prodotti sul mercato, come Agenda (\url{https://github.com/rschmukler/agenda}), e la presenza in rete di numerose librerie per la gestione della parte grafica ed esempi sull'implementazione degli aggiornamenti in tempo reale.\\
Nell'ottica della suddivisione del lavoro, il fatto che il prodotto abbia un elevato livello di modularità agevolerà il \insrole{Responsabile di Progetto} nell'assegnazione dei prossimi task di codifica.\\
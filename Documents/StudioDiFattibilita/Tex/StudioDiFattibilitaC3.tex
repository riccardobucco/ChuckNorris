% !TEX encoding = UTF-8 Unicode
\level{1}{Studio di fattibilità del capitolato C3}
	\level{2}{Descrizione del capitolato}
		Si è scelto il \insglo{capitolato}, denominato \projectname, presentato da CoffeeStrap, il quale pone come obiettivo la creazione di un \insglo{framework}, basato sulle tecnologie \insglo{Node.js}, \insglo{Express.js} e \insglo{Socket.io}, per la generazione di diversi tipi di grafici a partire da dati provenienti da sorgenti arbitrarie.\\
		Il \insglo{framework} creato deve fornire delle \insglo{API} per la configurazione programmatica della rappresentazione grafica e dei dati che si vogliono utilizzare, oltre a funzioni di aggiornamento dei grafici lato \insglo{server} tramite \insglo{WebSocket}.
	\level{2}{Studio del dominio}
		Per sviluppare il \insglo{capitolato} in esame occorrono capacità di tipo tecnologico e di comprensione dell'ambito in cui l'applicazione verrà utilizzata. Si descrivono di seguito il dominio tecnologico e quello applicativo.
		\level{3}{Dominio applicativo}
			\projectname{} si propone come versatile strumento di supporto delle attività di business all'interno delle imprese, in particolare nell'attività di analisi dei dati necessaria all'elaborazione di strategie di gestione interna o di mercato, utilizzabile oltre che da esperti di dominio anche da figure professionali non in possesso di particolare curriculum tecnico. Lo scopo è, quindi, minimizzare le risorse e massimizzare l'efficienza, semplificando l'utilizzo della notevole quantità di informazioni derivanti dalla rete attraverso grafici e tabelle.
		\level{3}{Dominio tecnologico}
			Il dominio tecnologico, da quanto è emerso dal \insglo{capitolato}, riguarda la generazione rapida di \insglo{client} web per la visualizzazione di grafici aggiornabili in tempo reale.\\
			Le conoscenze necessarie per l'implementazione del \insglo{prodotto} sono:
			\begin{itemize}
				\item lo stack tecnologico formato da \insglo{Node.js}, \insglo{Express.js} e \insglo{Socket.io};
				\item il linguaggio \insglo{JavaScript};
				\item la programmazione di componenti grafiche, attraverso l'uso di \insglo{Angular.js};
				\item la manipolazione di documenti basati su dati;
				\item il sistema operativo \insglo{Android} per la creazione di una app mobile.
			\end{itemize}
	\level{2}{Valutazione del capitolato} 
		Il \insglo{capitolato} presenta i seguenti aspetti che il \insglo{team} ha ritenuto positivi:
		\begin{itemize}
			\item l'acquisizione di conoscenze riguardanti tecnologie strategiche per l'inserimento nel mondo del lavoro;
			\item l'interesse verso il dominio applicativo, che si prevede possa riscontrare successo sul mercato;
			\item la disponibilità di una ricca documentazione riguardante gli strumenti che verranno utilizzati;
			\item la presentazione sufficientemente chiara dei requisiti opzionali ed obbligatori.
		\end{itemize}
		Sono stati riscontrati dal gruppo anche i seguenti aspetti ritenuti, invece, negativi:
		\begin{itemize}
			\item l'utilizzo di tecnologie delle quali il gruppo ha scarsa conoscenza;
		\end{itemize}
		\level{3}{Individuazione dei rischi}
			L'individuazione dei rischi legati alla realizzazione del \insglo{capitolato} scelto è presentata nel documento
			\insdoc{Piano di Progetto 1.00}.
		\level{3}{Analisi del mercato}
			Si cita una frase del \insglo{capitolato}:
			\begin{quotation}
				«la notevole mole di informazioni rese disponibili dalla velocità di rete e dalla potenza di calcolo dei nuovi sistemi oggi ha impegnato progressivamente gli sviluppatori a rivedere e rivoluzionare i tradizionali paradigmi di analisi dei dati a supporto delle attività di business. [...]
				Il processo di decisione da parte di queste funzioni aziendali dev’essere fortemente supportato da dati che descrivano il più precisamente possibile il funzionamento dell’azienda in tutti i suoi aspetti. Mancando tali figure di particolare curriculum tecnico, esse hanno bisogno di utilizzare strumenti facilmente fruibili e leggere i dati in output in maniera semplice ed immediata, tramite l’utilizzo di rappresentazioni grafiche quali grafici e tabelle.»
			\end{quotation}
			Da quanto si può apprendere dai paragrafi riportati, è possibile affermare che il \insglo{prodotto} che ci si prefigge di realizzare avrà buone probabilità di riscontrare l'interesse del mercato poiché permetterà di svolgere in modo più semplice e veloce le attività di business.
	\level{2}{Stima della fattibilità}
		Il gruppo \groupname{} ritiene di avere le capacità per poter portare a termine il progetto e per poter soddisfare tutti i requisiti obbligatori richiesti. Il gruppo non ha mai lavorato con alcune delle tecnologie sopra indicate ma intende approfondire le proprie conoscenze a riguardo fino ad acquisirne una sufficiente padronanza. Inoltre ritiene di possedere le conoscenze necessarie per riuscire a comprendere le principali problematiche e trova che la possibilità di arricchire le competenze di nuove tecnologie sia un ottimo incentivo per implementare il \insglo{prodotto} richiesto.

\section{Studio di fattibilità del capitolato C3}
	\subsection{Descrizione del capitolato}
		Si è scelto il capitolato, denominato \projectname, presentato da CoffeStrap, il quale pone come obiettivo la creazione di un framework, basato sulle tecnologie Node.js, Express.js e Socket.io, per la generazione di diversi tipi di grafici a partire da dati provenienti da sorgenti arbitrarie.\\
		Il framework creato deve fornire delle API per la configurazione programmatica della rappresentazione grafica e dei dati che si vogliono utilizzare, oltre a funzioni di aggiornamento dei grafici lato server tramite WebSocket.
	\subsection{Studio del dominio}
		Per sviluppare il capitolato in esame occorrono capacità di tipo tecnologico e di comprensione dell'ambito in cui l'applicazione verrà utilizzata. Si descrivono di seguito il dominio tecnologico e quello applicativo.
		\subsubsection{Dominio applicativo}
			\projectname si propone come versatile strumento di supporto delle attività di business all'interno delle imprese, in particolare nell'attività di analisi dei dati necessaria all'elaborazione di strategie di gestione interna o di mercato, utilizzabile oltre che da esperti di dominio anche da figure professionali non in possesso di particolare curriculum tecnico. Lo scopo è, quindi, minimizzare le risorse e massimizzare l'efficienza, semplificando l'utilizzo della notevole quantità di informazioni derivanti dalla rete attraverso grafici e tabelle.
		\subsubsection{Dominio tecnologico}
			Il dominio tecnologico, da quanto è emerso dal capitolato, riguarda la generazione rapida di client web per la visualizzazione di grafici aggiornabili in tempo reale.\\
			Le conoscenze necessarie per l'implementazione del prodotto sono:
			\begin{itemize}
				\item lo stack tecnologico formato da Node.js, Express.js e Socket.io;
				\item il linguaggio JavaScript;
				\item la programmazione di componenti grafiche, attraverso l'uso di Angular.js;
				\item la manipolazione di documenti basati su dati;
				\item il sistema operativo Android per la creazione di una app mobile;
			\end{itemize}
	\subsection{Valutazione del capitolato} 
		Il capitolato presenta i seguenti aspetti che il team ha ritenuto positivi:
		\begin{itemize}
			\item l'acquisizione di conoscenze riguardanti tecnologie strategiche per l'inserimento nel mondo del lavoro;
			\item l'interesse verso il dominio applicativo, che si prevede possa riscontrare successo sul mercato;
			\item la disponibilità di una ricca documentazione riguardante gli strumenti che verranno utilizzati;
			\item la presentazione sufficientemente chiara dei requisiti opzionali ed obbligatori.
		\end{itemize}
		Sono stati riscontrati dal gruppo anche i seguenti aspetti ritenuti, invece, negativi:
		\begin{itemize}
			\item l'utilizzo di tecnologie delle quali il gruppo ha scarsa conoscenza;
		\end{itemize}
		\subsubsection{Individuazione dei rischi}
			L'individuazione dei rischi legati alla realizzazione del capitolato scelto è presentata nel documento
			``Piano di Progetto \lastversion''.
			%\insdoc{Piano Di Progetto \lastversion}.
		\subsubsection{Analisi del mercato}
			Si cita una frase del capitolato:
			\begin{quotation}
				<<la notevole mole di informazioni rese disponibili dalla velocità di rete e dalla potenza di calcolo dei nuovi sistemi oggi ha impegnato progressivamente gli sviluppatori a rivedere e rivoluzionare i tradizionali paradigmi di analisi dei dati a supporto delle attività di business. [...]
				Il processo di decisione da parte di queste funzioni aziendali dev’essere fortemente supportato da dati che descrivano il più precisamente possibile il funzionamento dell’azienda in tutti i suoi aspetti. Mancando tali figure di particolare curriculum tecnico, esse hanno bisogno di utilizzare strumenti facilmente fruibili e leggere i dati in output in maniera semplice ed immediata, tramite l’utilizzo di rappresentazioni grafiche quali grafici e tabelle.>>
			\end{quotation}
			Da quanto si può apprendere dai paragrafi riportati, è possibile affermare che il prodotto che ci si prefigge di realizzare avrà buone probabilità di riscontrare l'interesse del mercato poichè permetterà di svolgere in modo più semplice e veloce le attività di business.
	\subsection{Stima della fattibilità}
		Il gruppo \groupname\ ritiene di avere le capacità per poter portare a termine il progetto e per poter soddisfare tutti i requisiti obbligatori richiesti. Il gruppo non ha mai lavorato con alcune delle tecnologie sopra indicate ma intende approfondire le proprie conoscenze a riguardo fino ad acquisirne una sufficiente padronanza. Inoltre ritiene di possedere le conoscenze necessarie per riuscire a comprendere le principali problematiche e trova che la possibilità di arricchire le competenze di nuove tecnologie sia un ottimo incentivo per implementare il prodotto richiesto.

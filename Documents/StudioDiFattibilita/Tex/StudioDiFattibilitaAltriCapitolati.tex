\section{Studio di fattibilità degli altri capitolati}
	\subsection{Capitolato C1 - BDSMApp: Big Data Social Monitoring App}
		\subsubsection{Descrizione del capitolato}
			Il capitolato d'appalto C1 pone come obbiettivo la creazione di un' infrastruttura per l'interrogazione di agglomerati di dati provenienti dai principali social network.\\
			L'applicazione richiesta prevede un'interfaccia web per utente e una seconda parte che fornisca servizi REST.
		\subsubsection{Valutazione generale}
			Il capitolato C1 è stato valutato con attenzione dal gruppo, ritenendolo il secondo di maggiore interesse, in quanto l'analisi di agglomerati di dati come quelli provenienti dai social network è un'attività ormai frequente e piuttosto richiesta sul mercato.
		\subsubsection{Individuazione dei rischi}
			Si è ritenuto che le molte decisioni che, a prima vista, vengono lasciate a cura del team, potrebbero non portare ad una valida analisi dei requisiti e quindi al mancato soddisfacimento degli obiettivi desiderati dal proponente.
	\subsection{Capitolato C2 - GUS: Glass (Uni) Scanner}
		\subsubsection{Descrizione del capitolato}
			Il capitolato d'appalto C2 pone come obiettivo la creazione di un'applicazione che, attraverso la scansione di un pannello di vetro, acquisisca un'immagine, la analizzi e sia in grado di catalogarla come conforme o meno rispetto a delle \textit{ricette} basate su norme in vigore.
		\subsubsection{Valutazione generale}
			Il progetto esposto non ha suscitato l'interesse da parte del team, in quanto è stato ritenuto troppo specifico e inadatto per l'acquisizione di nuove conoscenze tecnologiche.
		\subsubsection{Individuazione dei rischi}
			\begin{itemize}
				\item Possibili problemi nella gestione di tutte le casistiche richieste per il controllo della qualità.
				\item Totale mancanza di conoscenza del dominio applicativo.
			\end{itemize}
	\subsection{Capitolato C4 - Premi: Software di presentazione ``better than Prezi''}
		\subsubsection{Descrizione del capitolato}
			Il capitolato d'appalto C4 pone come obbiettivo la creazione di un software per un sistema di presentazione di slide più innovativo rispetto a prodotti quali \textit{PowerPoint}, \textit{Impress} e \textit{Keynote}; un software che permetta la creazione di presentazioni aperte a più possibilità, che dia modo al presentatore di seguire più fili conduttori, in modo simile al software \textit{Prezi}.
		\subsubsection{Valutazione generale}
			Il team ha ritenuto poco interessante l'oggetto del capitolato e poco adatto per l'acquisizione di nuove conoscenze tecnologiche. Inoltre il prodotto richiesto è stato giudicato di scarso interesse per il mercato, data la presenza di software simili già ampiamente diffusi.
		\subsubsection{Individuazione dei rischi}
			\begin{itemize}
				\item Possibili difficoltà nell'implementazione di un prodotto più innovativo rispetto a quelli già presenti
				sul mercato.
				\item Presenza, nel capitolato, di requisiti obbligatori molto generici che potrebbero portare a non soddisfare 
				le attese del proponente.
			\end{itemize}
	\subsection{Capitolato C5 - sHike: A smart cloud and mobile platform appliance for the safety and health in mountain hiking}
		\subsubsection{Descrizione del capitolato}
			Il capitolato d'appalto C5 pone come obbiettivo la creazione di una applicazione eseguibile su un dispositivo mobile indossabile. Tale applicazione deve funzionare su sistema operativo Android ed è destinata all'utilizzo da parte degli escursionisti; fra le sue diverse funzioni devono quindi comparire la gestione di mappe, funzioni di navigazione e funzioni di controllo dei parametri fisici dell'utilizzatore.
		\subsubsection{Valutazione generale}
		Sebbene a prima vista il prodotto esposto nel capitolato risulti accattivante, il team ha ritenuto che il progetto non abbia un grande valore \textit{sociale} visto che può essere facilmente sostituito con una o più applicazioni per smartphone. A discapito di questo capitolato, inoltre, il team ha ritenuto che le tecnologie da imparare siano \textit{di nicchia} (Extention WearIT API) o non particolarmente interessanti (Android 4.4.2).
		\subsubsection{Individuazione dei rischi}
			\begin{itemize}
				\item \'{E} stato ritenuto che la scarsa conoscenza, unita al poco interesse del gruppo nello sviluppo del
				prodotto proposto, avrebbe potuto portare al mancato soddisfacimento degli obiettivi richiesti.
			\end{itemize}
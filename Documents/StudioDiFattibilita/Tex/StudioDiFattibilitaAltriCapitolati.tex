% !TEX encoding = UTF-8 Unicode
\section{Studio di fattibilità degli altri capitolati}
	\subsection{Capitolato C1 - BDSMApp: Big Data Social Monitoring App}
		\subsubsection{Descrizione del capitolato}
			Il capitolato d'appalto C1 pone come obbiettivo la creazione di un'infrastruttura che permetta di interrogare big data provenienti dai principali social network (Facebook, Twitter e Instagram).\\
			L'applicazione richiesta prevede due parti:
			\begin{enumerate}
				\item un'interfaccia web per utente che permetta consultazione e interrogazione;
				\item servizi REST interrogabili.
			\end{enumerate}
		\subsubsection{Valutazione generale}
			Il capitolato C1 è ritenuto di grande interesse in quanto l'analisi di agglomerati di dati come quelli provenienti dai social network è un'attività ormai frequente e piuttosto richiesta sul mercato. Lo sviluppo di capacità in questo ambito è pertanto un buon investimento per il nostro futuro.
		\subsubsection{Individuazione dei rischi}
			All'interno del capitolato viene concessa molta libertà su quanto deve essere fatto. Tale cosa è a prima vista positiva. Tuttavia, si è ritiene che essa può comportare una maggiore difficoltà nell'individuare gli obiettivi desiderati dal proponente. L'Analisi dei Requisiti può dunque risultare molto difficile.
	\subsection{Capitolato C2 - GUS: Glass (Uni) Scanner}
		\subsubsection{Descrizione del capitolato}
			Il capitolato d'appalto C2 pone come obiettivo la creazione di un'applicazione che sia in grado di analizzare un'immagine ricavata dalla scansione di un pannello di vetro. Essa, in particolare, deve essere in grado di rilevare qualsiasi tipo di non conformità.
		\subsubsection{Valutazione generale}
			Tale progetto non suscita l'interesse da parte del team, in quanto lo si ritiene troppo settoriale. Esso, dal nostro punto di vista, è inadatto per l'acquisizione di nuove conoscenze e competenze tecnologiche che ci possono servire nel futuro.
		\subsubsection{Individuazione dei rischi}
			Il capitolato è rischioso per i seguenti motivi:
			\begin{itemize}
				\item gestire e riconoscere tutti i possibili problemi che un pannello di vetro può presentare sembra difficile dopo una breve analisi;
				\item i componenti del nostro team hanno totale mancanza di conoscenza del dominio applicativo.
			\end{itemize}
	\subsection{Capitolato C4 - Premi: Software di presentazione "better than Prezi"}
		\subsubsection{Descrizione del capitolato}
			Il capitolato d'appalto C4 pone come obbiettivo la creazione di un software per un sistema di presentazione di slide innovativo rispetto a prodotti quali PowerPoint, Impress e Keynote. Tale software deve permettere la creazione di presentazioni aperte a più possibilità, che diano modo al presentatore di seguire più fili conduttori, in modo simile al software Prezi.
		\subsubsection{Valutazione generale}
			Il team ha ritenuto poco interessante l'oggetto del capitolato e poco adatto per l'acquisizione di nuove conoscenze tecnologiche. Inoltre, il prodotto richiesto è stato giudicato di scarso interesse per il mercato, data la presenza di software simili già ampiamente diffusi.
		\subsubsection{Individuazione dei rischi}
			Anche questo capitolato, come il C1, è considerato da noi molto rischioso a causa della presenza di requisiti obbligatori molto generici e vaghi. Se l'iterazione con il proponente non dovesse essere fatta in modo appropriato, è probabile un mancato soddisfacimento delle attese.
	\subsection{Capitolato C5 - sHike: A smart cloud and mobile platform appliance for the safety and health in mountain hiking}
		\subsubsection{Descrizione del capitolato}
			Il capitolato d'appalto C5 pone come obbiettivo la creazione di una applicazione eseguibile su un dispositivo mobile indossabile. Tale applicazione deve funzionare su sistema operativo Android ed è destinata all'utilizzo da parte degli escursionisti: fra le funzioni che essa deve implementare devono quindi comparire la gestione delle mappe e della navigazione, oltre che funzioni di controllo dei parametri fisici dell'utilizzatore.
		\subsubsection{Valutazione generale}
		Sebbene a prima vista il prodotto esposto nel capitolato risulti accattivante, il team ha ritenuto che il progetto non abbia un grande valore sociale, visto che può essere facilmente sostituito con una o più applicazioni per smartphone. A discapito di questo capitolato, inoltre, il team ha ritenuto che le tecnologie da imparare siano di nicchia (Extention WearIT API) o non particolarmente interessanti (Android 4.4.2).
		\subsubsection{Individuazione dei rischi}
			Vengono di seguito riportati i rischi ai quali è soggetto tale capitolato.
			\begin{itemize}
				\item Uno dei principali rischi di questo capitolato è il fatto che sia molto accattivante: è probabile che molti gruppi scelgano di svolgere tale progetto, aumentando la concorrenza e dunque il rischio di venire esclusi alla Revisione dei Requisiti.
				\item La maggior parte dei membri del gruppo è poco interessata nel progetto, oltre ad essere in possesso di poca della conoscenza necessaria allo svolgimento del capitolato.
			\end{itemize}

% !TEX encoding = UTF-8 Unicode
\level{1}{Middleware in Express.js} \label{app:express}
	\level{2}{Cos'è un middleware?}
		In \insglo{Express.js} un \insglo{middleware}, molto semplicemente, è un insieme di funzioni che gestiscono le richieste al \insglo{server}: esse sono invocate dal routing layer di \insglo{Express.js} quando vengono fatte determinate richieste \insglo{HTTP} a certi endpoint. Nello specifico, quando arriva una richiesta (rappresentata da un oggetto ServerRequest) questa viene passata alla prima funzione dello stack dei \insglo{middleware} (assieme a un oggetto ServerResponse, che rappresenta la risposta del \insglo{server} alla richiesta): ogni \insglo{middleware} può decidere se gestire la richiesta in qualche modo (per esempio invocando dei metodi sull'oggetto ServerResponse) oppure no. Quando un \insglo{middleware} ha terminato la sua specifica gestione di una richiesta, esso può passare l'oggetto ServerRequest al \insglo{middleware} che lo segue nello stack (tramite la \insglo{callback} \texttt{next()}). Un \insglo{middleware}, dunque, può essere usato ovunque si voglia gestire in qualche modo logico tutte le richieste di un certo tipo.\\
		Riportiamo in seguito la descrizione che fornisce il sito \insuri{expressjs.com} di \insglo{middleware} (riportata in lingua originale):
		\begin{quote}
			\insglo{Middleware} is a function with access to the request object (\texttt{req}), the response object (\texttt{res}), and the next \insglo{middleware} in line in the request-response cycle of an Express application, commonly denoted by a variable named \texttt{next}. \insglo{Middleware} can:
			\begin{itemize}
				\item execute any code;
				\item make changes to the request and the response objects;
				\item end the request-response cycle;
				\item call the next \insglo{middleware} in the stack.
			\end{itemize}
			If the current \insglo{middleware} does not end the request-response cycle, it must call \texttt{next()} to pass control to the next \insglo{middleware}, otherwise the request will be left hanging.
		\end{quote}
	\level{2}{Ottenere un middleware usando le API interne di Norris}
		Come specificato nel \insglo{capitolato} d'appalto:
		\begin{quote}
			l'uso del \insglo{framework} (\insglo{Norris} ndr) dovrà essere compatibile con l’utilizzo standard dei \insglo{middleware} di Express, versione 4.x (\insuri{http://expressjs.com/4x/api.html\#middleware})
		\end{quote}
		Ne deriva che l'utilizzatore finale del \insglo{prodotto} deve poter utilizzare il comando
		\begin{lstlisting}
			app.use([path,] function [, function...])
		\end{lstlisting}
		all'interno del proprio \insglo{server} Express: con tale comando egli può fissare un certo percorso (\texttt{path}) come punto di mount di uno o più \insglo{middleware}. Tali \insglo{middleware} possono essere definiti dallo sviluppatore stesso oppure possono essere forniti da terzi (come nel caso di \insglo{Norris}).\\
		L'utilizzatore di \insglo{Norris} ha bisogno dunque di un modo per poter ottenere il \insglo{middleware} \insglo{Norris} e poterne fare il mount su di un path a sua scelta. Sulla base di tutte queste considerazioni possiamo concludere che l'ottenimento di un \insglo{middleware} è a tutti gli effetti una funzionalità che il \insglo{prodotto} \insglo{Norris} deve fornire allo sviluppatore.\\
		Segue un esempio immaginario e sperimentale dell'uso di tale funzionalità da parte dello sviluppatore utilizzatore di \insglo{Norris}:
		\begin{lstlisting}
			var express = require('express');
			var app = express();
			(*var \ignoreglo{norris} = require('\ignoreglo{norris}');*)
			(*// load the \ignoreglo{norris} \ignoreglo{middleware}*)
			(*app.use('/', \ignoreglo{norris}.\ignoreglo{middleware}());*)
		\end{lstlisting}

\section{Casi d'uso}
L’analisi del capitolato, l’incontro con il proponente e la discussione tra gli \insrole{Analisti} hanno portato all'individuazione dei casi d'uso riportati di seguito. 
I casi d'uso sono suddivisi in tre categorie, in quanto il progetto prevede la creazione di tre sistemi differenti:
\begin{itemize}
	\item Norris: sono i casi d'uso inerenti alle funzionalità offerte dal framework;
	\item Applicazione Android: sono i casi d'uso inerenti all'utilizzo dell'applicazione Android;
	\item Dashboard: sono i casi d'uso inerenti all'utilizzo della dashboard.
\end{itemize}
Ogni caso d'uso è identificato da un codice univoco gerarchico, composto nel seguente modo:
\begin{itemize}
	\item [] UC[codice categoria][codice identificativo del padre].[codice progressivo di livello]
\end{itemize}
Il codice progressivo può includere diversi livelli di gerarchia separati da un punto.

\subsection{Norris}
\input{UseCase/UCN.tex}

\subsection{Applicazione Android}
\input{UseCase/UCA.tex}

\subsection{Dashboard}
\input{UseCase/UCD.tex}

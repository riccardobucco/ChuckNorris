% !TEX encoding = UTF-8 Unicode
\section{Introduzione}
	\subsection{Scopo del documento}
		Il presente documento ha lo scopo di analizzare, specificare e classificare i requisiti del sistema Norris.\\
		Le funzionalità che il sistema deve possedere (o, da un diverso punto di vista, i bisogni che l'utente del prodotto deve essere in grado di soddisfare) sono stati ricavati da un'attenta analisi del capitolato d'appalto e da quanto è emerso da alcuni incontri con il proponente.\\
		Il presente documento è destinato tanto al Kaizen Team quanto al proponente. Esso, infatti, stabilisce le basi per un accordo tra fornitore e proponente su cosa il prodotto deve fare. Inoltre, grazie all'analisi svolta in seguito, i costi di sviluppo possono essere considerevolmente diminuiti: individuare fin da subito fraintendimenti, inconsistenze o mancanze è molto più economico che farlo in seguito.
	\subsection{Glossario}
		Il presente documento contiene:
		\begin{itemize}
			\item termini tecnici, di dominio o comunque non di uso comune;
			\item abbreviazioni; 
			\item acronimi.
		\end{itemize}
		Il significato di tali elementi può essere talvolta non chiaro o comunque soggetto ad ambiguità. Per ottenere le definizioni di questi vocaboli e ulteriori chiarimenti su di essi si rimanda al documento Glossario.pdf.\\
		Tutti i termini, le abbreviazioni o gli acronimi presenti all'interno di Glossario.pdf sono rappresentati tramite [DA FARE].
	\subsection{Riferimenti}
		\subsubsection{Elenco dei riferimenti normativi}
		\subsubsection{Elenco dei riferimenti informativi}

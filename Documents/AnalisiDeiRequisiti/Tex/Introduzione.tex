% !TEX encoding = UTF-8 Unicode
\section{Introduzione}

	\subsection{Scopo del documento}
		Il presente documento ha lo scopo di analizzare, specificare e classificare i requisiti del sistema Norris.\\
		Le funzionalità che il sistema deve possedere (o, da un diverso punto di vista, i bisogni che l'utente del prodotto deve essere in grado di soddisfare) sono stati ricavati da un'attenta analisi del capitolato d'appalto e da quanto è emerso da alcuni incontri con il proponente.\\
		Il presente documento è destinato tanto al Kaizen Team quanto al proponente. Esso, infatti, stabilisce le basi per un accordo tra fornitore e proponente su cosa il prodotto deve fare. Inoltre, grazie all'analisi svolta in seguito, i costi di sviluppo possono essere considerevolmente diminuiti: individuare fin da subito fraintendimenti, inconsistenze o mancanze è molto più economico che farlo in seguito.
	
	\level{2}{Glossario}
	Allo scopo di rendere più semplice la comprensione dei documenti ed evitare eventuali ambiguità, viene allegato il \insdoc{Glossario v6.00}, che contiene la spiegazione della terminologia tecnica e degli acronimi utilizzati. Per facilitare la lettura, i termini presenti all'interno di tale documento saranno marcati da una “G” maiuscola a pedice.


	\subsection{Riferimenti utili}
		\subsubsection{Riferimenti normativi}
		\begin{itemize}
			\item\textbf{Capitolato d'appalto C3:} \projectname{}: Real-time Business Intelligence \\
				\insuri{http://www.math.unipd.it/~tullio/IS-1/2014/Progetto/C3.pdf};
			\item \textbf{Norme di Progetto:} \insdoc{Norme di Progetto v1.00}.
		\end{itemize}
		\subsubsection{Riferimenti informativi}
		\begin{itemize}
			\item \textbf{Learning UML 2.0 - Kim Hamilton, Russell Miles - Chapter 2. Modelling Requirements: UseCases};
			\item \textbf{Software Engineering 9 th Edition - Ian Sommerville - Chapter 4. Requirements engineering};
			\item \textbf{Materiale del corso di Ingegneria del Software:} \\
				\insuri{http://www.math.unipd.it/~tullio/IS-1/2014/}.
		\end{itemize}

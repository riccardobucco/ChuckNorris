\section{Requisiti}

Di seguito sono riportati tutti i requisiti, i quali sono stati individuati dopo un'analisi sulla base del capitolato, dei casi d'uso, degli incontri con il proponente e delle riunioni interne.


I requisiti sono individuati in modo univoco secondo il seguente codice:
\begin{center}
	R[importanza][categoria][codice]
\end{center}
\begin{itemize}

	\item \textbf{Importanza} assume i seguenti valori:
		\begin{itemize}
			\item[1] Requisito obbligatorio
			\item[2] Requisito desiderabile
			\item[3] Requisito opzionale
		\end{itemize}

	\item \textbf{Categoria} assume i seguenti valori:
		\begin{itemize}
			\item[F] Requisito funzionale
			\item[P] Requisito prestazionale
			\item[Q] Requisito di qualità
			\item[V] Requisito di vincolo
		\end{itemize}
	\item \textbf{Codice} è un identificativo univoco espresso in modo gerarchico.
\end{itemize}

\subsection{Elenco delle fonti}

				\begin{longtabu} spread 1cm [c]{|X[-1,l]|X[-1,l]|}
					\hline
					\rowfont{\bf \centering}
					Codice &
					Dettaglio \\
					\hline
					\endhead
					
					Capitolato &
                Capitolato d'appalto C3\\\hline UCA0 &
                UCA0 Utilizzo dell'applicazione Android\\\hline UCA1 &
                UCA1 Accedere a un istanza di Norris\\\hline UCA1.1 &
                UCA1.1 Inserire indirizzo di un istanza di Norris\\\hline UCA1.2 &
                UCA1.2 Inserire username\\\hline UCA1.3 &
                UCA1.3 Inserire password\\\hline UCA2 &
                UCA2 Visualizzare elenco grafici\\\hline UCA2.1 &
                UCA2.1 Visualizzare ID grafici\\\hline UCA2.2 &
                UCA2.2 Visualizzare titolo grafici\\\hline UCA2.3 &
                UCA2.3 Visualizzare tipo grafici\\\hline UCA2.4 &
                UCA2.4 Visualizzare descrizione grafici\\\hline UCA3 &
                UCA3 Visualizzare singolo grafico\\\hline UCA3.1 &
                UCA3.1 Selezione di un grafico dall'elenco di grafici esistenti\\\hline UCA3.2 &
                UCA3.2 Visualizzare il grafico selezionato\\\hline UCA4 &
                UCA4 Visualizzare errore dati di accesso non validi\\\hline UCA4.1 &
                UCA4.1 Visualizzare errore indirizzo dell istanza Norris non valido\\\hline UCA4.2 &
                UCA4.2 Visualizzare errore credenziali di accesso non valide\\\hline UCD0 &
                UCD0 Utilizzo della dashboard\\\hline UCD1 &
                UCD1 Visualizzazione posizione autobus APS in tempo reale\\\hline UCD2 &
                UCD2 Visualizzazione numero di autobus attivi per ciascuna linea\\\hline UCD2.1 &
                UCD2.1 Selezione della linea della quale si vuole visualizzare il numero di bus\\\hline UCD2.2 &
                UCD2.2 Visualizzare il numero di autobus attivi per una linea\\\hline UCD3 &
                UCD3 Filtrare autobus per linea di appartenenza\\\hline UCD3.1 &
                UCD3.1 Selezionare le linee che si intende visualizzare\\\hline UCD3.2 &
                UCD3.2 Visualizzare i bus appartenenti alle linee scelte\\\hline UCN0 &
                UCN0 Utilizzo del framework Norris\\\hline UCN1 &
                UCN1 Utilizzo API interne di Norris\\\hline UCN1.1 &
                UCN1.1 Creare modello chart\\\hline UCN1.1.1 &
                UCN1.1.1 Inserire il titolo del chart\\\hline UCN1.1.2 &
                UCN1.1.2 Scegliere una descrizione per il chart\\\hline UCN1.10 &
                UCN1.10 Visualizzare errore tipologia di aggiornamento non valida\\\hline UCN1.11 &
                UCN1.11 Visualizzare errore raggiunto limite grafici inseribili in una pagina\\\hline UCN1.12 &
                UCN1.12 Fornire funzioni di autenticazione\\\hline UCN1.12.1 &
                UCN1.12.1 Fornire funzione di login\\\hline UCN1.12.2 &
                UCN1.12.2 Fornire funzione di logout\\\hline UCN1.12.3 &
                UCN1.12.3 Fornire funzione di verifica autenticazione\\\hline UCN1.2 &
                UCN1.2 Aggiornare un modello chart\\\hline UCN1.2.1 &
                UCN1.2.1 Aggiornare un modello bar chart\\\hline UCN1.2.1.1 &
                UCN1.2.1.1 Aggiornare un modello bar chart selezionato con metodo in place\\\hline UCN1.2.2 &
                UCN1.2.2 Aggiornare un modello line chart\\\hline UCN1.2.2.1 &
                UCN1.2.2.1 Aggiornare un modello line chart con metodo in place\\\hline UCN1.2.2.2 &
                UCN1.2.2.2 Aggiornare un modello line chart con metodo stream\\\hline UCN1.2.3 &
                UCN1.2.3 Aggiornare un modello map chart\\\hline UCN1.2.3.1 &
                UCN1.2.3.1 Aggiornare un modello map chart con metodo in place\\\hline UCN1.2.3.2 &
                UCN1.2.3.2 Aggiornare un modello map chart con metodo movie\\\hline UCN1.2.4 &
                UCN1.2.4 Aggiornare un modello table\\\hline UCN1.2.4.1 &
                UCN1.2.4.1 Aggiornare un modello table con metodo in place\\\hline UCN1.2.4.2 &
                UCN1.2.4.2 Aggiornare un modello table con metodo stream\\\hline UCN1.3 &
                UCN1.3 Creare pagina\\\hline UCN1.3.1 &
                UCN1.3.1 Inserire il titolo della pagina\\\hline UCN1.3.2 &
                UCN1.3.2 Scegliere le opzioni di visualizzazione della pagina\\\hline UCN1.3.2.1 &
                UCN1.3.2.1 Scegliere il massimo numero di grafici visualizzabili su una riga\\\hline UCN1.3.2.2 &
                UCN1.3.2.2 Scegliere il massimo numero di grafici visualizzabili su una colonna\\\hline UCN1.3.3 &
                UCN1.3.3 Aggiungere grafico alla pagina\\\hline UCN1.4 &
                UCN1.4 Creare modello bar chart\\\hline UCN1.4.1 &
                UCN1.4.1 Scegliere opzioni bar chart\\\hline UCN1.4.1.1 &
                UCN1.4.1.1 Scegliere il colore di ciascun set di barre\\\hline UCN1.4.1.2 &
                UCN1.4.1.2 Scegliere l'orientamento delle barre\\\hline UCN1.4.1.3 &
                UCN1.4.1.3 Scegliere il formato di stampa dei valori del bar chart\\\hline UCN1.4.1.3.1 &
                UCN1.4.1.3.1 Scegliere la dimensione dello spazio tra due serie del bar chart\\\hline UCN1.4.1.3.2 &
                UCN1.4.1.3.2 Scegliere la dimensione dello spazio tra due valori del bar chart\\\hline UCN1.4.1.4 &
                UCN1.4.1.4 Scegliere il nome di ciascun set di barre\\\hline UCN1.4.1.5 &
                UCN1.4.1.5 Scegliere le opzioni riguardanti la legenda di un bar chart\\\hline UCN1.4.1.5.1 &
                UCN1.4.1.5.1 Scegliere se la legenda è visualizzata\\\hline UCN1.4.1.5.2 &
                UCN1.4.1.5.2 Scegliere la posizione in cui è visualizzata la legenda\\\hline UCN1.4.1.6 &
                UCN1.4.1.6 Scegliere le opzioni riguardanti il piano cartesiano del bar chart\\\hline UCN1.4.1.6.1 &
                UCN1.4.1.6.1 Scegliere il nome degli assi\\\hline UCN1.4.1.6.2 &
                UCN1.4.1.6.2 Scegliere se le linee della griglia sono visualizzate\\\hline UCN1.4.1.7 &
                UCN1.4.1.7 Scegliere il massimo numero di barre da visualizzare per ogni serie\\\hline UCN1.4.2 &
                UCN1.4.2 Inserire dati bar chart\\\hline UCN1.4.2.1 &
                UCN1.4.2.1 Inserire i valori indipendenti nel bar chart\\\hline UCN1.4.2.2 &
                UCN1.4.2.2 Inserire i valori dipendenti per ciascun set di barre\\\hline UCN1.5 &
                UCN1.5 Creare modello line chart\\\hline UCN1.5.1 &
                UCN1.5.1 Scegliere opzioni line chart\\\hline UCN1.5.1.1 &
                UCN1.5.1.1 Scegliere il colore di ciascuna linea\\\hline UCN1.5.1.2 &
                UCN1.5.1.2 Scegliere il formato di stampa dei valori del line chart\\\hline UCN1.5.1.2.1 &
                UCN1.5.1.2.1 Scegliere la dimensione dei punti del line chart\\\hline UCN1.5.1.2.2 &
                UCN1.5.1.2.2 Scegliere se la linea del line chart è curva o segmentata\\\hline UCN1.5.1.3 &
                UCN1.5.1.3 Scegliere il nome di ciascuna linea\\\hline UCN1.5.1.4 &
                UCN1.5.1.4 Scegliere il massimo numero di punti visualizzati sull'asse dei valori indipendenti\\\hline UCN1.5.1.5 &
                UCN1.5.1.5 Scegliere le opzioni riguardanti la legenda di un line chart\\\hline UCN1.5.1.5.1 &
                UCN1.5.1.5.1 Scegliere se la legenda è visualizzata\\\hline UCN1.5.1.5.2 &
                UCN1.5.1.5.2 Scegliere la posizione in cui è visualizzata la legenda\\\hline UCN1.5.1.6 &
                UCN1.5.1.6 Scegliere le opzioni riguardanti il piano cartesiano del line chart\\\hline UCN1.5.1.6.1 &
                UCN1.5.1.6.1 Scegliere il nome degli assi\\\hline UCN1.5.1.6.2 &
                UCN1.5.1.6.2 Scegliere se le linee della griglia sono visualizzate\\\hline UCN1.5.2 &
                UCN1.5.2 Inserire dati line chart\\\hline UCN1.5.2.1 &
                UCN1.5.2.1 Inserire i valori indipendenti nel line chart\\\hline UCN1.5.2.2 &
                UCN1.5.2.2 Inserire i valori dipendenti per ciascuna linea\\\hline UCN1.6 &
                UCN1.6 Creare modello map chart\\\hline UCN1.6.1 &
                UCN1.6.1 Scegliere opzioni map chart\\\hline UCN1.6.1.1 &
                UCN1.6.1.1 Scegliere il formato di stampa dei valori del map chart\\\hline UCN1.6.1.1.1 &
                UCN1.6.1.1.1 Scegliere la forma dei marcatori del map chart\\\hline UCN1.6.1.2 &
                UCN1.6.1.2 Scegliere le dimensioni dell'area mostrata sulla mappa\\\hline UCN1.6.1.3 &
                UCN1.6.1.3 Scegliere il massimo numero di punti da visualizzare per ogni serie\\\hline UCN1.6.1.4 &
                UCN1.6.1.4 Scegliere le coordinate del punto centrale della mappa\\\hline UCN1.6.1.5 &
                UCN1.6.1.5 Scegliere il colore di ciascuna serie di punti\\\hline UCN1.6.1.6 &
                UCN1.6.1.6 Scegliere il nome di ciascuna serie di punti\\\hline UCN1.6.1.7 &
                UCN1.6.1.7 Scegliere le opzioni riguardanti la legenda di un map chart\\\hline UCN1.6.1.7.1 &
                UCN1.6.1.7.1 Scegliere se la legenda è visualizzata\\\hline UCN1.6.1.7.2 &
                UCN1.6.1.7.2 Scegliere la posizione in cui è visualizzata la legenda\\\hline UCN1.6.2 &
                UCN1.6.2 Inserire nuova serie di punti\\\hline UCN1.7 &
                UCN1.7 Creare modello table\\\hline UCN1.7.1 &
                UCN1.7.1 Scegliere opzioni table\\\hline UCN1.7.1.1 &
                UCN1.7.1.1 Scegliere il formato di una cella della table\\\hline UCN1.7.1.1.1 &
                UCN1.7.1.1.1 Scegliere il colore del testo contenuto in una cella\\\hline UCN1.7.1.1.2 &
                UCN1.7.1.1.2 Scegliere il colore dello sfondo di una cella\\\hline UCN1.7.1.2 &
                UCN1.7.1.2 Scegliere l'intestazione di una singola colonna\\\hline UCN1.7.1.3 &
                UCN1.7.1.3 Scegliere il massimo numero di righe da visualizzare\\\hline UCN1.7.1.4 &
                UCN1.7.1.4 Scegliere se è possibile ordinare le righe rispetto ai valori di una colonna\\\hline UCN1.7.1.5 &
                UCN1.7.1.5 Scegliere la posizione in cui vengono aggiunte nuove righe\\\hline UCN1.7.1.6 &
                UCN1.7.1.6 Scegliere se le linee della tabella sono visualizzate\\\hline UCN1.7.2 &
                UCN1.7.2 Inserire nuova riga\\\hline UCN1.8 &
                UCN1.8 Ottenere middleware\\\hline UCN1.9 &
                UCN1.9 Visualizzare errore dati non corretti\\\hline UCN2 &
                UCN2 Utilizzo API esterne di Norris\\\hline UCN2.1 &
                UCN2.1 Accedere a un istanza di Norris\\\hline UCN2.1.1 &
                UCN2.1.1 Inserire indirizzo di un istanza di Norris\\\hline UCN2.1.2 &
                UCN2.1.2 Inserire username\\\hline UCN2.1.3 &
                UCN2.1.3 Inserire password\\\hline UCN2.2 &
                UCN2.2 Ottenere lista dei grafici presenti in un'istanza di Norris\\\hline UCN2.2.1 &
                UCN2.2.1 Ottenere l'ID di ciascun grafico\\\hline UCN2.2.2 &
                UCN2.2.2 Ottenere il titolo di ciascun grafico\\\hline UCN2.2.3 &
                UCN2.2.3 Ottenere il tipo di ciascun grafico\\\hline UCN2.2.4 &
                UCN2.2.4 Ottenere la descrizione di ciascun grafico\\\hline UCN2.3 &
                UCN2.3 Ottenere un grafico presente in un'istanza di Norris\\\hline UCN2.4 &
                UCN2.4 Scollegarsi da un istanza di Norris\\\hline UCN3 &
                UCN3 Utilizzo API fornite da Chuck\\\hline UCN3.1 &
                UCN3.1 Selezionare il modello di grafico che si vuole rappresentare\\\hline UCN3.2 &
                UCN3.2 Scegliere il tag HTML nel quale si vuole inserire la rappresentazione del grafico\\\hline UCN3.3 &
                UCN3.3 Cambiare opzioni di visualizzazione di un grafico\\\hline UCN3.4 &
                UCN3.4 Cambiare opzioni di visualizzazione di un bar chart\\\hline UCN3.4.1 &
                UCN3.4.1 Cambiare il colore di un set di barre\\\hline UCN3.4.2 &
                UCN3.4.2 Cambiare le opzioni riguardanti la legenda di un bar chart\\\hline UCN3.4.2.1 &
                UCN3.4.2.1 Cambiare il fatto che la legenda sia visualizzata o meno\\\hline UCN3.4.2.2 &
                UCN3.4.2.2 Cambiare la posizione in cui è visualizzata la legenda\\\hline UCN3.4.3 &
                UCN3.4.3 Cambiare il fatto che la griglia del piano cartesiano sia visualizzata o meno\\\hline UCN3.5 &
                UCN3.5 Cambiare opzioni di visualizzazione di un line chart\\\hline UCN3.5.1 &
                UCN3.5.1 Cambiare il colore di una linea\\\hline UCN3.5.2 &
                UCN3.5.2 Cambiare le opzioni riguardanti la legenda di un line chart\\\hline UCN3.5.2.1 &
                UCN3.5.2.1 Cambiare il fatto che la legenda sia visualizzata o meno\\\hline UCN3.5.2.2 &
                UCN3.5.2.2 Cambiare la posizione in cui è visualizzata la legenda\\\hline UCN3.5.3 &
                UCN3.5.3 Cambiare il fatto che la griglia del piano cartesiano sia visualizzata o meno\\\hline UCN3.6 &
                UCN3.6 Cambiare opzioni di visualizzazione di un map chart\\\hline UCN3.6.1 &
                UCN3.6.1 Cambiare il colore di una serie di punti\\\hline UCN3.6.2 &
                UCN3.6.2 Cambiare le opzioni riguardanti la legenda di un map chart\\\hline UCN3.6.2.1 &
                UCN3.6.2.1 Cambiare il fatto che la legenda sia visualizzata o meno\\\hline UCN3.6.2.2 &
                UCN3.6.2.2 Cambiare la posizione in cui è visualizzata la legenda\\\hline UCN3.7 &
                UCN3.7 Cambiare opzioni di visualizzazione di una table\\\hline UCN3.7.1 &
                UCN3.7.1 Cambiare le opzioni riguardanti il formato di una cella\\\hline UCN3.7.1.1 &
                UCN3.7.1.1 Cambiare il colore del testo contenuto in una cella\\\hline UCN3.7.1.2 &
                UCN3.7.1.2 - Cambiare il colore di sfondo di una cella\\\hline UCN3.8 &
                UCN3.8 Accedere a un istanza di Norris\\\hline UCN3.8.1 &
                UCN3.8.1 Inserire indirizzo di un istanza di Norris\\\hline UCN3.8.2 &
                UCN3.8.2 Inserire Username\\\hline UCN3.8.3 &
                UCN3.8.3 Inserire password\\\hline UCN3.9 &
                UCN3.9 Scollegarsi da un istanza di Norris\\\hline                 \caption{Fonti}
				\end{longtabu}

\subsection{Elenco dei requisiti}

				
				\begin{longtabu}[c]{|X[0.7,l]|X[l]|X[2.5,l]|X[l]|}
				    \hline
					\rowfont{\bf \centering}
				    Codice &
					Tipologia &
					Descrizione &
					Fonti \\
					\hline
					\endhead
					
					
                RRF1 & 
                \parbox[t]{4cm}{ Obbligatorio \\ Funzionale} & Il framework deve permettere la creazione di grafici tramite le API interne & \parbox[t]{4cm}{Capitolato \\ UCN0 \\ UCN1 \\ UCN1.1 }  \\ 
                \hline
                
                RRF1.1 & 
                \parbox[t]{4cm}{ Obbligatorio \\ Funzionale} & Il framework deve permettere la creazione di bar chart & \parbox[t]{4cm}{Capitolato \\ UCN1.4 }  \\ 
                \hline
                
                RRF1.2 & 
                \parbox[t]{4cm}{ Obbligatorio \\ Funzionale} & Il framework deve permettere la creazione di line chart & \parbox[t]{4cm}{Capitolato \\ UCN1.5 }  \\ 
                \hline
                
                RRF1.3 & 
                \parbox[t]{4cm}{ Obbligatorio \\ Funzionale} & Il framework deve permettere la creazione di map chart & \parbox[t]{4cm}{Capitolato \\ UCN1.6 }  \\ 
                \hline
                
                RRF1.4 & 
                \parbox[t]{4cm}{ Obbligatorio \\ Funzionale} & Il framework deve permettere la creazione di table & \parbox[t]{4cm}{Capitolato \\ UCN1.7 }  \\ 
                \hline
                
                RRF1.5 & 
                \parbox[t]{4cm}{ Obbligatorio \\ Funzionale} & Ogni grafico deve permettere la scelta delle impostazioni & \parbox[t]{4cm}{Capitolato \\ UCN1.4.1 \\ UCN1.5.1 \\ UCN1.6.1 \\ UCN1.7.1 }  \\ 
                \hline
                
                RRF1.5.1 & 
                \parbox[t]{4cm}{ Obbligatorio \\ Funzionale} & Il grafico line chart deve permettere la scelta delle impostazioni riguardanti la legenda & \parbox[t]{4cm}{Capitolato \\ UCN1.5.1.5 }  \\ 
                \hline
                
                RRF1.5.1.1 & 
                \parbox[t]{4cm}{ Obbligatorio \\ Funzionale} & Deve essere possibile scegliere se la legenda è visualizzata o nascosta & \parbox[t]{4cm}{Capitolato \\ UCN1.5.1.5.1 }  \\ 
                \hline
                
                RRF1.5.1.2 & 
                \parbox[t]{4cm}{ Obbligatorio \\ Funzionale} & Deve essere possibile scegliere la posizione in cui è visualizzata la legenda & \parbox[t]{4cm}{Capitolato \\ UCN1.5.1.5.2 }  \\ 
                \hline
                
                RRF1.5.2 & 
                \parbox[t]{4cm}{ Obbligatorio \\ Funzionale} & Il grafico bar chart deve permettere la scelta delle impostazioni riguardanti la legenda & \parbox[t]{4cm}{Capitolato \\ UCN1.4.1.5 }  \\ 
                \hline
                
                RRF1.5.2.1 & 
                \parbox[t]{4cm}{ Obbligatorio \\ Funzionale} & Deve essere possibile scegliere se la legenda è visualizzata o nascosta & \parbox[t]{4cm}{Capitolato \\ UCN1.4.1.5.1 }  \\ 
                \hline
                
                RRF1.5.2.2 & 
                \parbox[t]{4cm}{ Obbligatorio \\ Funzionale} & Deve essere possibile scegliere la posizione in cui è visualizzata la legenda & \parbox[t]{4cm}{Capitolato \\ UCN1.4.1.5.2 }  \\ 
                \hline
                
                RRF1.5.3 & 
                \parbox[t]{4cm}{ Obbligatorio \\ Funzionale} & Il grafico map chart deve permettere la scelta delle impostazioni riguardanti la legenda & \parbox[t]{4cm}{Capitolato \\ UCN1.6.1.7 }  \\ 
                \hline
                
                RRF1.5.3.1 & 
                \parbox[t]{4cm}{ Obbligatorio \\ Funzionale} & Deve essere possibile scegliere se la legenda è visualizzata o nascosta & \parbox[t]{4cm}{Capitolato \\ UCN1.6.1.7.1 }  \\ 
                \hline
                
                RRF1.5.3.2 & 
                \parbox[t]{4cm}{ Obbligatorio \\ Funzionale} & Deve essere possibile scegliere la posizione in cui è visualizzata la legenda & \parbox[t]{4cm}{Capitolato \\ UCN1.6.1.7.2 }  \\ 
                \hline
                
                RDF1.5.4 & 
                \parbox[t]{4cm}{ Desiderabile \\ Funzionale} & Ogni grafico deve permettere l'inserimento di una descrizione testuale & \parbox[t]{4cm}{UCN1.1.2 }  \\ 
                \hline
                
                RRF1.5.5 & 
                \parbox[t]{4cm}{ Obbligatorio \\ Funzionale} & Un grafico line chart deve consentire la scelta delle impostazioni riguardanti il piano cartesiano & \parbox[t]{4cm}{Capitolato \\ UCN1.5.1.6 }  \\ 
                \hline
                
                RRF1.5.5.1 & 
                \parbox[t]{4cm}{ Obbligatorio \\ Funzionale} & Deve essere possibile inserire il nome dei due assi cartesiani in un line chart & \parbox[t]{4cm}{Capitolato \\ UCN1.5.1.6.1 }  \\ 
                \hline
                
                RRF1.5.5.2 & 
                \parbox[t]{4cm}{ Obbligatorio \\ Funzionale} & Deve essere possibile scegliere se le linee della griglia sono visualizzate o nascoste in un line chart & \parbox[t]{4cm}{Capitolato \\ UCN1.5.1.6.2 }  \\ 
                \hline
                
                RRF1.5.6 & 
                \parbox[t]{4cm}{ Obbligatorio \\ Funzionale} & Un grafico bar chart deve consentire la scelta delle impostazioni riguardanti il piano cartesiano & \parbox[t]{4cm}{Capitolato \\ UCN1.4.1.6 }  \\ 
                \hline
                
                RRF1.5.6.1 & 
                \parbox[t]{4cm}{ Obbligatorio \\ Funzionale} & Deve essere possibile inserire il nome dei due assi cartesiani in un bar chart & \parbox[t]{4cm}{Capitolato \\ UCN1.4.1.6.1 }  \\ 
                \hline
                
                RRF1.5.6.2 & 
                \parbox[t]{4cm}{ Obbligatorio \\ Funzionale} & Deve essere possibile scegliere se le linee della griglia sono visualizzate o nascoste in un bar chart & \parbox[t]{4cm}{Capitolato \\ UCN1.4.1.6.2 }  \\ 
                \hline
                
                RRF1.5.7 & 
                \parbox[t]{4cm}{ Obbligatorio \\ Funzionale} & Per ogni grafico deve essere possibile scegliere il formato di stampa dei dati & \parbox[t]{4cm}{Capitolato \\ UCN1.4.1.3 \\ UCN1.5.1.2 \\ UCN1.6.1.1 \\ UCN1.7.1.1 }  \\ 
                \hline
                
                RRF1.5.7.1 & 
                \parbox[t]{4cm}{ Obbligatorio \\ Funzionale} & La table deve permettere la scelta del colore del testo di ogni cella & \parbox[t]{4cm}{Capitolato \\ UCN1.7.1.1.1 }  \\ 
                \hline
                
                RRF1.5.7.2 & 
                \parbox[t]{4cm}{ Obbligatorio \\ Funzionale} & La table deve consentire la scelta del colore di sfondo di ogni cella & \parbox[t]{4cm}{Capitolato \\ UCN1.7.1.1.2 }  \\ 
                \hline
                
                RDF1.5.7.3 & 
                \parbox[t]{4cm}{ Desiderabile \\ Funzionale} & Il bar chart deve consentire di impostare la dimensione dello spazio tra due serie & \parbox[t]{4cm}{UCN1.4.1.3.1 }  \\ 
                \hline
                
                RRF1.5.7.4 & 
                \parbox[t]{4cm}{ Obbligatorio \\ Funzionale} & Il bar chart deve consentire di impostare la dimensione dello spazio tra due valori & \parbox[t]{4cm}{UCN1.4.1.3.2 }  \\ 
                \hline
                
                RDF1.5.7.5 & 
                \parbox[t]{4cm}{ Desiderabile \\ Funzionale} & Deve esser possibile scegliere la forma dei marcatori del map chart & \parbox[t]{4cm}{UCN1.6.1.1.1 }  \\ 
                \hline
                
                RDF1.5.7.6 & 
                \parbox[t]{4cm}{ Desiderabile \\ Funzionale} & Deve essere possibile impostare la dimensione dei punti nel line chart & \parbox[t]{4cm}{UCN1.5.1.2.1 }  \\ 
                \hline
                
                RRF1.5.7.7 & 
                \parbox[t]{4cm}{ Obbligatorio \\ Funzionale} & Deve esser possibile scegliere se la linea di un line chart è curva o segmentata & \parbox[t]{4cm}{UCN1.5.1.2.2 }  \\ 
                \hline
                
                RRF1.5.8 & 
                \parbox[t]{4cm}{ Obbligatorio \\ Funzionale} & Il bar chart deve permettere la scelta dell'orientamento delle barre tra verticale e orizzontale & \parbox[t]{4cm}{Capitolato \\ UCN1.4.1.2 }  \\ 
                \hline
                
                RRF1.5.9 & 
                \parbox[t]{4cm}{ Obbligatorio \\ Funzionale} & Il map chart deve permette la scelta delle dimensioni dell'area visualizzata & \parbox[t]{4cm}{Capitolato \\ UCN1.6.1.2 }  \\ 
                \hline
                
                RRF1.5.10 & 
                \parbox[t]{4cm}{ Obbligatorio \\ Funzionale} & Il map chart deve permettere la scelta del punto centrale della mappa & \parbox[t]{4cm}{Capitolato \\ UCN1.6.1.4 }  \\ 
                \hline
                
                RRF1.5.11 & 
                \parbox[t]{4cm}{ Obbligatorio \\ Funzionale} & Line chart, bar chart e map chart devono permettere la scelta del massimo numero di dati visualizzati per ogni serie & \parbox[t]{4cm}{Capitolato \\ UCN1.5.1.4 \\ UCN1.6.1.3 \\ UCN1.4.1.7 }  \\ 
                \hline
                
                RRF1.5.12 & 
                \parbox[t]{4cm}{ Obbligatorio \\ Funzionale} & La table deve permettere la scelta del massimo numero di righe visualizzate & \parbox[t]{4cm}{Capitolato \\ UCN1.7.1.3 }  \\ 
                \hline
                
                RRF1.5.13 & 
                \parbox[t]{4cm}{ Obbligatorio \\ Funzionale} & La table deve permettere la scelta di rendere disponibile l'ordinamento per colonna & \parbox[t]{4cm}{Capitolato \\ UCN1.7.1.4 }  \\ 
                \hline
                
                RRF1.5.14 & 
                \parbox[t]{4cm}{ Obbligatorio \\ Funzionale} & La table deve consentire la scelta della posizione in cui vengono aggiunte nuove righe & \parbox[t]{4cm}{Capitolato \\ UCN1.7.1.5 }  \\ 
                \hline
                
                RRF1.5.15 & 
                \parbox[t]{4cm}{ Obbligatorio \\ Funzionale} & La table deve consentire la scelta dell'intestazione di ciascuna colonna & \parbox[t]{4cm}{Capitolato \\ UCN1.7.1.2 }  \\ 
                \hline
                
                RRF1.5.16 & 
                \parbox[t]{4cm}{ Obbligatorio \\ Funzionale} & La table deve permettere la scelta se le linee della tabella sono visualizzate o nascoste & \parbox[t]{4cm}{Capitolato \\ UCN1.7.1.6 }  \\ 
                \hline
                
                RRF1.5.17 & 
                \parbox[t]{4cm}{ Obbligatorio \\ Funzionale} & Ogni grafico deve permettere l'impostazione del titolo & \parbox[t]{4cm}{Capitolato \\ UCN1.1.1 }  \\ 
                \hline
                
                RRF1.5.18 & 
                \parbox[t]{4cm}{ Obbligatorio \\ Funzionale} & I grafici line chart devono permettere la scelta del colore per ciascun set di dati & \parbox[t]{4cm}{Capitolato \\ UCN1.5.1.1 }  \\ 
                \hline
                
                RRF1.5.19 & 
                \parbox[t]{4cm}{ Obbligatorio \\ Funzionale} & I grafici bar chart devono permettere la scelta del colore per ciascun set di dati & \parbox[t]{4cm}{Capitolato \\ UCN1.4.1.1 }  \\ 
                \hline
                
                RRF1.5.20 & 
                \parbox[t]{4cm}{ Obbligatorio \\ Funzionale} & I grafici map chart devono permettere la scelta del colore per ciascun set di dati & \parbox[t]{4cm}{}  \\ 
                \hline
                
                RRF1.5.21 & 
                \parbox[t]{4cm}{ Obbligatorio \\ Funzionale} & I grafici line chart devono permettere la scelta del nome per ciascun set di dati & \parbox[t]{4cm}{Capitolato \\ UCN1.5.1.3 }  \\ 
                \hline
                
                RRF1.5.22 & 
                \parbox[t]{4cm}{ Obbligatorio \\ Funzionale} & I grafici bar chart devono permettere la scelta del nome per ciascun set di dati & \parbox[t]{4cm}{Capitolato \\ UCN1.4.1.4 }  \\ 
                \hline
                
                RRF1.5.23 & 
                \parbox[t]{4cm}{ Obbligatorio \\ Funzionale} & I grafici map chart devono permettere la scelta del nome per ciascun set di dati & \parbox[t]{4cm}{Capitolato \\ UCN1.6.1.6 }  \\ 
                \hline
                
                RRF1.6 & 
                \parbox[t]{4cm}{ Obbligatorio \\ Funzionale} & Per ogni grafico deve essere permesso l'inserimento dei dati & \parbox[t]{4cm}{Capitolato \\ UCN1.4.2 \\ UCN1.4.2.1 \\ UCN1.4.2.2 \\ UCN1.5.2 \\ UCN1.5.2.1 \\ UCN1.5.2.2 \\ UCN1.6.2 \\ UCN1.7.2 }  \\ 
                \hline
                
                RRF1.7 & 
                \parbox[t]{4cm}{ Obbligatorio \\ Funzionale} & Deve esser visualizzato un errore qualora i dati passati per l'aggiornamento di un grafico siano scorretti & \parbox[t]{4cm}{UCN1.9 }  \\ 
                \hline
                
                RRF1.8 & 
                \parbox[t]{4cm}{ Obbligatorio \\ Funzionale} & Deve esser visualizzato un errore qualora i dati passati per la creazione di un grafico siano scorretti & \parbox[t]{4cm}{UCN1.9 }  \\ 
                \hline
                
                RRF1.9 & 
                \parbox[t]{4cm}{ Obbligatorio \\ Funzionale} & Deve esser visualizzato un errore qualora si cerchi di inserire un grafico oltre il limite consentito dalla pagina & \parbox[t]{4cm}{UCN1.11 }  \\ 
                \hline
                
                RRF1.10 & 
                \parbox[t]{4cm}{ Obbligatorio \\ Funzionale} & Deve esser visualizzato un errore qualora si cerchi di aggiornare un grafico con un tipo di aggiornamento da lui non permesso & \parbox[t]{4cm}{UCN1.10 }  \\ 
                \hline
                
                RRF2 & 
                \parbox[t]{4cm}{ Obbligatorio \\ Funzionale} & Il framework deve permettere l'aggiornamento di un grafico tramite le API interne & \parbox[t]{4cm}{Capitolato \\ UCN0 \\ UCN1 \\ UCN1.2 \\ UCN1.2.1 \\ UCN1.2.2 \\ UCN1.2.3 \\ UCN1.2.4 }  \\ 
                \hline
                
                RRF2.1 & 
                \parbox[t]{4cm}{ Obbligatorio \\ Funzionale} & Un grafico line chart deve consentire l'aggiornamento stream & \parbox[t]{4cm}{Capitolato \\ UCN1.2.2.2 }  \\ 
                \hline
                
                RRF2.2 & 
                \parbox[t]{4cm}{ Obbligatorio \\ Funzionale} & Un grafico table deve consentire l'aggiornamento stream & \parbox[t]{4cm}{Capitolato \\ UCN1.2.4.2 }  \\ 
                \hline
                
                RRF2.3 & 
                \parbox[t]{4cm}{ Obbligatorio \\ Funzionale} & Un grafico map chart deve consentire l'aggiornamento movie & \parbox[t]{4cm}{Capitolato \\ UCN1.2.3.2 }  \\ 
                \hline
                
                RRF2.4 & 
                \parbox[t]{4cm}{ Obbligatorio \\ Funzionale} & Un grafico bar chart deve consentire l'aggiornamento in place & \parbox[t]{4cm}{Capitolato \\ UCN1.2.1.1 }  \\ 
                \hline
                
                RRF2.5 & 
                \parbox[t]{4cm}{ Obbligatorio \\ Funzionale} & Un grafico line chart deve consentire l'aggiornamento in place & \parbox[t]{4cm}{Capitolato \\ UCN1.2.2.1 }  \\ 
                \hline
                
                RRF2.6 & 
                \parbox[t]{4cm}{ Obbligatorio \\ Funzionale} & Un grafico map chart deve consentire l'aggiornamento in place & \parbox[t]{4cm}{Capitolato \\ UCN1.2.3.1 }  \\ 
                \hline
                
                RRF2.7 & 
                \parbox[t]{4cm}{ Obbligatorio \\ Funzionale} & Un grafico table deve consentire l'aggiornamento in place & \parbox[t]{4cm}{Capitolato \\ UCN1.2.4.1 }  \\ 
                \hline
                
                RRF3 & 
                \parbox[t]{4cm}{ Obbligatorio \\ Funzionale} & Il framework deve permettere la creazione di una pagina HTML contenente alcuni grafici tramite le API interne & \parbox[t]{4cm}{Capitolato \\ UCN0 \\ UCN1 \\ UCN1.3 }  \\ 
                \hline
                
                RRF3.1 & 
                \parbox[t]{4cm}{ Obbligatorio \\ Funzionale} & Ogni pagina deve permettere l'inserimento del titolo & \parbox[t]{4cm}{Capitolato \\ UCN1.3.1 }  \\ 
                \hline
                
                RRF3.2 & 
                \parbox[t]{4cm}{ Obbligatorio \\ Funzionale} & Ogni pagina deve consentire la scelta delle opzioni di visualizzazione & \parbox[t]{4cm}{Capitolato \\ UCN1.3.2 }  \\ 
                \hline
                
                RRF3.2.1 & 
                \parbox[t]{4cm}{ Obbligatorio \\ Funzionale} & Deve essere possibile impostare il massimo numero di grafici visualizzabile per riga & \parbox[t]{4cm}{Capitolato \\ UCN1.3.2.1 }  \\ 
                \hline
                
                RRF3.2.2 & 
                \parbox[t]{4cm}{ Obbligatorio \\ Funzionale} & Deve essere possibile impostare il massimo numero di grafici visualizzabile per colonna & \parbox[t]{4cm}{Capitolato \\ UCN1.3.2.2 }  \\ 
                \hline
                
                RRF3.3 & 
                \parbox[t]{4cm}{ Obbligatorio \\ Funzionale} & Ogni pagina deve consentire l'aggiunta di grafici & \parbox[t]{4cm}{Capitolato \\ UCN1.3.3 }  \\ 
                \hline
                
                RRF4 & 
                \parbox[t]{4cm}{ Obbligatorio \\ Funzionale} & Il framework deve fornire un middleware per Express.js tramite le API interne & \parbox[t]{4cm}{Capitolato \\ UCN0 \\ UCN1 \\ UCN1.8 }  \\ 
                \hline
                
                RRF5 & 
                \parbox[t]{4cm}{ Obbligatorio \\ Funzionale} & Il framework deve permettere di ottenere tramite le API esterne le informazioni sui grafici creati con le API interne & \parbox[t]{4cm}{Capitolato \\ UCN0 \\ UCN2 }  \\ 
                \hline
                
                RRF5.1 & 
                \parbox[t]{4cm}{ Obbligatorio \\ Funzionale} & Le API esterne devono fornire la lista dei grafici presenti & \parbox[t]{4cm}{Capitolato \\ UCN2.2 }  \\ 
                \hline
                
                RRF5.1.1 & 
                \parbox[t]{4cm}{ Obbligatorio \\ Funzionale} & La lista dei grafici deve fornire ID, titolo, tipo e descrizione di ciascun grafico & \parbox[t]{4cm}{UCN2.2.1 \\ UCN2.2.2 \\ UCN2.2.3 \\ UCN2.2.4 }  \\ 
                \hline
                
                RRF5.2 & 
                \parbox[t]{4cm}{ Obbligatorio \\ Funzionale} & Le API esterne devono fornire i grafici con relativi aggiornamenti & \parbox[t]{4cm}{Capitolato \\ UCN2.3 }  \\ 
                \hline
                
                RRF5.3 & 
                \parbox[t]{4cm}{ Obbligatorio \\ Funzionale} & Deve essere possibile accedere a un'istanza di Norris tramite username e password utilizzando le API esterne & \parbox[t]{4cm}{UCN2.1 \\ UCN2.1.1 \\ UCN2.1.2 \\ UCN2.1.3 }  \\ 
                \hline
                
                RRF5.4 & 
                \parbox[t]{4cm}{ Obbligatorio \\ Funzionale} & Deve essere possibile scollegarsi da un'istanza di Norris utilizzando le API esterne & \parbox[t]{4cm}{UCN2.4 }  \\ 
                \hline
                
                RRF6 & 
                \parbox[t]{4cm}{ Obbligatorio \\ Funzionale} & Il framework deve fornire una libreria per l'inserimento di grafici all'interno di pagine HTML & \parbox[t]{4cm}{UCN0 \\ UCN3 }  \\ 
                \hline
                
                RRF6.1 & 
                \parbox[t]{4cm}{ Obbligatorio \\ Funzionale} & La libreria deve consentire l'inserimento di un grafico all'interno di un determinato tag HTML & \parbox[t]{4cm}{UCN3.1 \\ UCN3.2 }  \\ 
                \hline
                
                RDF6.2 & 
                \parbox[t]{4cm}{ Desiderabile \\ Funzionale} & La libreria deve consentire la modifica delle principali impostazioni di un grafico & \parbox[t]{4cm}{UCN3.3 \\ UCN3.4 \\ UCN3.5 \\ UCN3.6 \\ UCN3.7 }  \\ 
                \hline
                
                RDF6.2.1 & 
                \parbox[t]{4cm}{ Desiderabile \\ Funzionale} & Per un grafico line chart la libreria deve consentire la modifica della scelta se visualizzare o meno la legenda & \parbox[t]{4cm}{UCN3.5.2 \\ UCN3.5.2.1 }  \\ 
                \hline
                
                RDF6.2.2 & 
                \parbox[t]{4cm}{ Desiderabile \\ Funzionale} & Per un grafico bar chart la libreria deve consentire la modifica della scelta se visualizzare o meno la legenda & \parbox[t]{4cm}{UCN3.4.2 \\ UCN3.4.2.1 }  \\ 
                \hline
                
                RDF6.2.3 & 
                \parbox[t]{4cm}{ Desiderabile \\ Funzionale} & Per un grafico map chart la libreria deve consentire la modifica della scelta se visualizzare o meno la legenda & \parbox[t]{4cm}{UCN3.6.2 \\ UCN3.6.2.1 }  \\ 
                \hline
                
                RDF6.2.4 & 
                \parbox[t]{4cm}{ Desiderabile \\ Funzionale} & Per un grafico line chart la libreria deve consentire la modifica della posizione in cui è visualizzata la legenda & \parbox[t]{4cm}{UCN3.5.2 \\ UCN3.5.2.2 }  \\ 
                \hline
                
                RDF6.2.5 & 
                \parbox[t]{4cm}{ Desiderabile \\ Funzionale} & Per un grafico bar chart a libreria deve consentire la modifica della posizione in cui è visualizzata la legenda & \parbox[t]{4cm}{UCN3.4.2 \\ UCN3.4.2.2 }  \\ 
                \hline
                
                RDF6.2.6 & 
                \parbox[t]{4cm}{ Desiderabile \\ Funzionale} & Per un grafico map chart la libreria deve consentire la modifica della posizione in cui è visualizzata la legenda & \parbox[t]{4cm}{UCN3.6.2 \\ UCN3.6.2.2 }  \\ 
                \hline
                
                RDF6.2.7 & 
                \parbox[t]{4cm}{ Desiderabile \\ Funzionale} & Per un grafico line chart la libreria deve consentire la modifica della scelta se la griglia del piano cartesiano è visualizzata o meno & \parbox[t]{4cm}{UCN3.5.3 }  \\ 
                \hline
                
                RDF6.2.8 & 
                \parbox[t]{4cm}{ Desiderabile \\ Funzionale} & Per un grafico bar chart la libreria deve consentire la modifica della scelta se la griglia del piano cartesiano è visualizzata o meno & \parbox[t]{4cm}{UCN3.4.3 }  \\ 
                \hline
                
                RDF6.2.9 & 
                \parbox[t]{4cm}{ Desiderabile \\ Funzionale} & Per un grafico line chart la libreria deve consentire la modifica del colore di ciascun set di dati & \parbox[t]{4cm}{UCN3.5.1 }  \\ 
                \hline
                
                RDF6.2.10 & 
                \parbox[t]{4cm}{ Desiderabile \\ Funzionale} & Per un grafico bar chart la libreria deve consentire la modifica del colore di ciascun set di dati & \parbox[t]{4cm}{UCN3.4.1 }  \\ 
                \hline
                
                RDF6.2.11 & 
                \parbox[t]{4cm}{ Desiderabile \\ Funzionale} & Per un grafico map chart la libreria deve consentire la modifica del colore di ciascun set di dati & \parbox[t]{4cm}{UCN3.6.1 }  \\ 
                \hline
                
                RDF6.2.12 & 
                \parbox[t]{4cm}{ Desiderabile \\ Funzionale} & La libreria deve consentire la modifica del colore di sfondo di ogni cella per la table & \parbox[t]{4cm}{UCN3.7.1 \\ UCN3.7.1.2 }  \\ 
                \hline
                
                RDF6.2.13 & 
                \parbox[t]{4cm}{ Desiderabile \\ Funzionale} & La libreria deve consentire la modifica del colore del testo contenuto in ogni cella per la table & \parbox[t]{4cm}{UCN3.7.1 \\ UCN3.7.1.1 }  \\ 
                \hline
                
                RRF6.3 & 
                \parbox[t]{4cm}{ Obbligatorio \\ Funzionale} & Deve essere possibile accedere a un'istanza di Norris tramite username e password utilizzando le API fornite da Chuck & \parbox[t]{4cm}{UCN3.8 \\ UCN3.8.1 \\ UCN3.8.2 \\ UCN3.8.3 }  \\ 
                \hline
                
                RRF6.4 & 
                \parbox[t]{4cm}{ Obbligatorio \\ Funzionale} & Deve essere possibile scollegarsi da un'istanza di Norris utilizzando le API fornite da Chuck & \parbox[t]{4cm}{UCN3.9 }  \\ 
                \hline
                
                RRF7 & 
                \parbox[t]{4cm}{ Obbligatorio \\ Funzionale} & Deve essere fornita una dashboard per visualizzare in tempo reali gli autobus dell'APS & \parbox[t]{4cm}{Capitolato \\ UCD0 \\ UCD1 }  \\ 
                \hline
                
                RRF7.1 & 
                \parbox[t]{4cm}{ Obbligatorio \\ Funzionale} & Deve essere possibile la visualizzazione del numero di autobus attivi per ciascuna linea & \parbox[t]{4cm}{Capitolato \\ UCD2 \\ UCD2.1 \\ UCD2.2 }  \\ 
                \hline
                
                RRF7.2 & 
                \parbox[t]{4cm}{ Obbligatorio \\ Funzionale} & Deve essere possibile filtrare gli autobus per linea di appartenenza e visualizzarli di conseguenza & \parbox[t]{4cm}{Capitolato \\ UCD3 \\ UCD3.1 \\ UCD3.2 }  \\ 
                \hline
                
                ROF8 & 
                \parbox[t]{4cm}{ Opzionale \\ Funzionale} & Deve essere fornita un'applicazione Android per la visualizzazione dei grafici & \parbox[t]{4cm}{Capitolato \\ UCA0 }  \\ 
                \hline
                
                RRF8.1 & 
                \parbox[t]{4cm}{ Obbligatorio \\ Funzionale} & Deve essere possibile accedere a un'istanza di Norris tramite username e password & \parbox[t]{4cm}{Capitolato \\ UCA1 \\ UCA1.1 \\ UCA1.2 \\ UCA1.3 }  \\ 
                \hline
                
                RRF8.2 & 
                \parbox[t]{4cm}{ Obbligatorio \\ Funzionale} & Deve essere possibile visualizzare l'elenco dei grafici esistenti nell'istanza Norris con relativo ID, titolo, tipo e descrizione & \parbox[t]{4cm}{Capitolato \\ UCA2 \\ UCA2.1 \\ UCA2.2 \\ UCA2.3 \\ UCA2.4 }  \\ 
                \hline
                
                RRF8.3 & 
                \parbox[t]{4cm}{ Obbligatorio \\ Funzionale} & Deve essere possibile selezionare e visualizzare un singolo grafico dell'istanza di Norris & \parbox[t]{4cm}{Capitolato \\ UCA3 \\ UCA3.1 \\ UCA3.2 }  \\ 
                \hline
                
                RRF8.4 & 
                \parbox[t]{4cm}{ Obbligatorio \\ Funzionale} & Deve essere visualizzato un errore quando viene immesso un indirizzo per un'istanza di Norris non valido & \parbox[t]{4cm}{UCA4 \\ UCA4.1 }  \\ 
                \hline
                
                RRF8.5 & 
                \parbox[t]{4cm}{ Obbligatorio \\ Funzionale} & Deve essere visualizzato un errore quando vengono immessi dati di accesso (username e password) per un'istanza di Norris non validi & \parbox[t]{4cm}{UCA4 \\ UCA4.2 }  \\ 
                \hline
                
                RRF9 & 
                \parbox[t]{4cm}{ Obbligatorio \\ Funzionale} & Il framework deve permettere al suo utilizzatore di fornire delle funzioni tramite le quali si può gestire l'autenticazione & \parbox[t]{4cm}{UCN1.12 \\ UCN1.12.1 \\ UCN1.12.2 \\ UCN1.12.3 }  \\ 
                \hline
                
                RRC10 & 
                \parbox[t]{4cm}{ Obbligatorio \\ Di vincolo} & Si deve utilizzare il framework Express.js per la realizzazione dell'infrastruttura web & \parbox[t]{4cm}{Capitolato }  \\ 
                \hline
                
                RRC11 & 
                \parbox[t]{4cm}{ Obbligatorio \\ Di vincolo} & Si deve utilizzare la libreria Socket.io per la componente WebSocket che realizza le notifiche push degli aggiornamenti & \parbox[t]{4cm}{Capitolato }  \\ 
                \hline
                
                RRC12 & 
                \parbox[t]{4cm}{ Obbligatorio \\ Di vincolo} & L'uso del framework deve essere compatibile con l'utilizzo standard dei middleware di Express.js & \parbox[t]{4cm}{Capitolato }  \\ 
                \hline
                
                RRQ13 & 
                \parbox[t]{4cm}{ Obbligatorio \\ Di qualità} & Il progetto deve essere pubblicato su GitHub & \parbox[t]{4cm}{Capitolato }  \\ 
                \hline
                
                RRQ14 & 
                \parbox[t]{4cm}{ Obbligatorio \\ Di qualità} & Devono essere utilizzate le issue di GitHub per la segnalazione dei bug & \parbox[t]{4cm}{Capitolato }  \\ 
                \hline
                
                RRC15 & 
                \parbox[t]{4cm}{ Obbligatorio \\ Di vincolo} & Il framework deve essere compatibile con la versione 38.0.x o superiori di Chrome & \parbox[t]{4cm}{Capitolato }  \\ 
                \hline
                
                RRC16 & 
                \parbox[t]{4cm}{ Obbligatorio \\ Di vincolo} & Il framework deve essere compatibile con la versione 32.x o superiori di Firefox & \parbox[t]{4cm}{Capitolato }  \\ 
                \hline
                
                RRC17 & 
                \parbox[t]{4cm}{ Obbligatorio \\ Di vincolo} & Ciascun grafico deve venire aggiornato in modo indipendente dagli altri grafici & \parbox[t]{4cm}{Capitolato }  \\ 
                \hline
                
                RDC18 & 
                \parbox[t]{4cm}{ Desiderabile \\ Di vincolo} & Deve essere utilizzato Angular.js per le componenti grafiche di front-end & \parbox[t]{4cm}{Capitolato }  \\ 
                \hline
                
                RRQ19 & 
                \parbox[t]{4cm}{ Obbligatorio \\ Di qualità} & Deve essere fornito il manuale d'uso del framework & \parbox[t]{4cm}{Capitolato }  \\ 
                \hline
                
                RDQ20 & 
                \parbox[t]{4cm}{ Desiderabile \\ Di qualità} & Deve essere fornito il manuale d'uso del framework in lingua inglese & \parbox[t]{4cm}{Capitolato }  \\ 
                \hline
                
                RDQ21 & 
                \parbox[t]{4cm}{ Desiderabile \\ Di qualità} & Ci si avvale del continuous deployment su Heroku & \parbox[t]{4cm}{Capitolato }  \\ 
                \hline
                
                RRC22 & 
                \parbox[t]{4cm}{ Obbligatorio \\ Di vincolo} & Norris deve essere un framework basato su Node.js & \parbox[t]{4cm}{Capitolato }  \\ 
                \hline
                                \caption{Requisiti}
				\end{longtabu}

\subsection{Tracciamento requisiti-fonti}

				\begin{longtabu} spread 1cm [c]{|X[l]|X[l]|}
					\hline
					\rowfont{\bf \centering}
					Requisito &
					Fonti \\
					\hline
					\endhead
					
					RRF1 & \parbox[t]{4cm}{ Capitolato \\ UCN0 \\ UCN1 \\ UCN1.1 } \\ 
                \hline
                RRF1.1 & \parbox[t]{4cm}{ Capitolato \\ UCN1.4 } \\ 
                \hline
                RRF1.5 & \parbox[t]{4cm}{ Capitolato \\ UCN1.6.1 \\ UCN1.7.1 \\ UCN1.5.1 \\ UCN1.4.1 } \\ 
                \hline
                RRF1.5.17 & \parbox[t]{4cm}{ Capitolato \\ UCN1.1.1 } \\ 
                \hline
                RRF1.5.1 & \parbox[t]{4cm}{ Capitolato \\ UCN1.5.1.5 } \\ 
                \hline
                RRF1.5.1.1 & \parbox[t]{4cm}{ Capitolato \\ UCN1.5.1.5.1 } \\ 
                \hline
                RRF1.5.1.2 & \parbox[t]{4cm}{ Capitolato \\ UCN1.5.1.5.2 } \\ 
                \hline
                RRF1.5.5 & \parbox[t]{4cm}{ Capitolato \\ UCN1.5.1.6 } \\ 
                \hline
                RRF1.5.5.1 & \parbox[t]{4cm}{ Capitolato \\ UCN1.5.1.6.1 } \\ 
                \hline
                RRF1.5.5.2 & \parbox[t]{4cm}{ Capitolato \\ UCN1.5.1.6.2 } \\ 
                \hline
                RDF1.5.4 & \parbox[t]{4cm}{ UCN1.1.2 } \\ 
                \hline
                RRF1.5.18 & \parbox[t]{4cm}{ Capitolato \\ UCN1.5.1.1 } \\ 
                \hline
                RRF1.5.7 & \parbox[t]{4cm}{ Capitolato \\ UCN1.4.1.3 \\ UCN1.5.1.2 \\ UCN1.6.1.1 \\ UCN1.7.1.1 } \\ 
                \hline
                RRF1.5.7.1 & \parbox[t]{4cm}{ Capitolato \\ UCN1.7.1.1.1 } \\ 
                \hline
                RRF1.5.7.2 & \parbox[t]{4cm}{ Capitolato \\ UCN1.7.1.1.2 } \\ 
                \hline
                RDF1.5.7.3 & \parbox[t]{4cm}{ UCN1.4.1.3.1 } \\ 
                \hline
                RRF1.5.7.4 & \parbox[t]{4cm}{ UCN1.4.1.3.2 } \\ 
                \hline
                RDF1.5.7.6 & \parbox[t]{4cm}{ UCN1.5.1.2.1 } \\ 
                \hline
                RRF1.5.7.7 & \parbox[t]{4cm}{ UCN1.5.1.2.2 } \\ 
                \hline
                RDF1.5.7.5 & \parbox[t]{4cm}{ UCN1.6.1.1.1 } \\ 
                \hline
                RRF1.5.8 & \parbox[t]{4cm}{ Capitolato \\ UCN1.4.1.2 } \\ 
                \hline
                RRF1.5.21 & \parbox[t]{4cm}{ Capitolato \\ UCN1.5.1.3 } \\ 
                \hline
                RRF1.5.9 & \parbox[t]{4cm}{ Capitolato \\ UCN1.6.1.2 } \\ 
                \hline
                RRF1.5.10 & \parbox[t]{4cm}{ Capitolato \\ UCN1.6.1.4 } \\ 
                \hline
                RRF1.5.11 & \parbox[t]{4cm}{ UCN1.5.1.4 \\ Capitolato \\ UCN1.6.1.3 \\ UCN1.4.1.7 } \\ 
                \hline
                RRF1.5.12 & \parbox[t]{4cm}{ Capitolato \\ UCN1.7.1.3 } \\ 
                \hline
                RRF1.5.13 & \parbox[t]{4cm}{ Capitolato \\ UCN1.7.1.4 } \\ 
                \hline
                RRF1.5.14 & \parbox[t]{4cm}{ Capitolato \\ UCN1.7.1.5 } \\ 
                \hline
                RRF1.5.15 & \parbox[t]{4cm}{ Capitolato \\ UCN1.7.1.2 } \\ 
                \hline
                RRF1.5.16 & \parbox[t]{4cm}{ Capitolato \\ UCN1.7.1.6 } \\ 
                \hline
                RRF1.5.2 & \parbox[t]{4cm}{ Capitolato \\ UCN1.4.1.5 } \\ 
                \hline
                RRF1.5.2.1 & \parbox[t]{4cm}{ Capitolato \\ UCN1.4.1.5.1 } \\ 
                \hline
                RRF1.5.2.2 & \parbox[t]{4cm}{ Capitolato \\ UCN1.4.1.5.2 } \\ 
                \hline
                RRF1.5.3 & \parbox[t]{4cm}{ Capitolato \\ UCN1.6.1.7 } \\ 
                \hline
                RRF1.5.3.1 & \parbox[t]{4cm}{ Capitolato \\ UCN1.6.1.7.1 } \\ 
                \hline
                RRF1.5.3.2 & \parbox[t]{4cm}{ Capitolato \\ UCN1.6.1.7.2 } \\ 
                \hline
                RRF1.5.6 & \parbox[t]{4cm}{ Capitolato \\ UCN1.4.1.6 } \\ 
                \hline
                RRF1.5.6.1 & \parbox[t]{4cm}{ Capitolato \\ UCN1.4.1.6.1 } \\ 
                \hline
                RRF1.5.6.2 & \parbox[t]{4cm}{ Capitolato \\ UCN1.4.1.6.2 } \\ 
                \hline
                RRF1.5.19 & \parbox[t]{4cm}{ Capitolato \\ UCN1.4.1.1 } \\ 
                \hline
                RRF1.5.20 & \parbox[t]{4cm}{ } \\ 
                \hline
                RRF1.5.22 & \parbox[t]{4cm}{ Capitolato \\ UCN1.4.1.4 } \\ 
                \hline
                RRF1.5.23 & \parbox[t]{4cm}{ Capitolato \\ UCN1.6.1.6 } \\ 
                \hline
                RRF1.6 & \parbox[t]{4cm}{ Capitolato \\ UCN1.4.2 \\ UCN1.5.2 \\ UCN1.6.2 \\ UCN1.7.2 \\ UCN1.5.2.1 \\ UCN1.5.2.2 \\ UCN1.4.2.1 \\ UCN1.4.2.2 } \\ 
                \hline
                RRF1.2 & \parbox[t]{4cm}{ Capitolato \\ UCN1.5 } \\ 
                \hline
                RRF1.3 & \parbox[t]{4cm}{ Capitolato \\ UCN1.6 } \\ 
                \hline
                RRF1.4 & \parbox[t]{4cm}{ Capitolato \\ UCN1.7 } \\ 
                \hline
                RRF1.7 & \parbox[t]{4cm}{ UCN1.9 } \\ 
                \hline
                RRF1.8 & \parbox[t]{4cm}{ UCN1.9 } \\ 
                \hline
                RRF1.9 & \parbox[t]{4cm}{ UCN1.11 } \\ 
                \hline
                RRF1.10 & \parbox[t]{4cm}{ UCN1.10 } \\ 
                \hline
                RRF2 & \parbox[t]{4cm}{ Capitolato \\ UCN0 \\ UCN1 \\ UCN1.2 \\ UCN1.2.1 \\ UCN1.2.3 \\ UCN1.2.4 \\ UCN1.2.2 } \\ 
                \hline
                RRF2.1 & \parbox[t]{4cm}{ Capitolato \\ UCN1.2.2.2 } \\ 
                \hline
                RRF2.2 & \parbox[t]{4cm}{ Capitolato \\ UCN1.2.4.2 } \\ 
                \hline
                RRF2.3 & \parbox[t]{4cm}{ Capitolato \\ UCN1.2.3.2 } \\ 
                \hline
                RRF2.4 & \parbox[t]{4cm}{ Capitolato \\ UCN1.2.1.1 } \\ 
                \hline
                RRF2.5 & \parbox[t]{4cm}{ Capitolato \\ UCN1.2.2.1 } \\ 
                \hline
                RRF2.6 & \parbox[t]{4cm}{ Capitolato \\ UCN1.2.3.1 } \\ 
                \hline
                RRF2.7 & \parbox[t]{4cm}{ Capitolato \\ UCN1.2.4.1 } \\ 
                \hline
                RRF3 & \parbox[t]{4cm}{ Capitolato \\ UCN0 \\ UCN1 \\ UCN1.3 } \\ 
                \hline
                RRF3.1 & \parbox[t]{4cm}{ UCN1.3.1 \\ Capitolato } \\ 
                \hline
                RRF3.2 & \parbox[t]{4cm}{ Capitolato \\ UCN1.3.2 } \\ 
                \hline
                RRF3.2.1 & \parbox[t]{4cm}{ Capitolato \\ UCN1.3.2.1 } \\ 
                \hline
                RRF3.2.2 & \parbox[t]{4cm}{ Capitolato \\ UCN1.3.2.2 } \\ 
                \hline
                RRF3.3 & \parbox[t]{4cm}{ Capitolato \\ UCN1.3.3 } \\ 
                \hline
                RRF4 & \parbox[t]{4cm}{ Capitolato \\ UCN0 \\ UCN1 \\ UCN1.8 } \\ 
                \hline
                RRF5 & \parbox[t]{4cm}{ Capitolato \\ UCN0 \\ UCN2 } \\ 
                \hline
                RRF5.1 & \parbox[t]{4cm}{ Capitolato \\ UCN2.2 } \\ 
                \hline
                RRF5.1.1 & \parbox[t]{4cm}{ UCN2.2.1 \\ UCN2.2.2 \\ UCN2.2.3 \\ UCN2.2.4 } \\ 
                \hline
                RRF5.2 & \parbox[t]{4cm}{ Capitolato \\ UCN2.3 } \\ 
                \hline
                RRF5.3 & \parbox[t]{4cm}{ UCN2.1 \\ UCN2.1.1 \\ UCN2.1.2 \\ UCN2.1.3 } \\ 
                \hline
                RRF5.4 & \parbox[t]{4cm}{ UCN2.4 } \\ 
                \hline
                RRF6 & \parbox[t]{4cm}{ UCN0 \\ UCN3 } \\ 
                \hline
                RRF6.1 & \parbox[t]{4cm}{ UCN3.1 \\ UCN3.2 } \\ 
                \hline
                RDF6.2 & \parbox[t]{4cm}{ UCN3.3 \\ UCN3.4 \\ UCN3.5 \\ UCN3.6 \\ UCN3.7 } \\ 
                \hline
                RDF6.2.1 & \parbox[t]{4cm}{ UCN3.5.2 \\ UCN3.5.2.1 } \\ 
                \hline
                RDF6.2.4 & \parbox[t]{4cm}{ UCN3.5.2 \\ UCN3.5.2.2 } \\ 
                \hline
                RDF6.2.7 & \parbox[t]{4cm}{ UCN3.5.3 } \\ 
                \hline
                RDF6.2.9 & \parbox[t]{4cm}{ UCN3.5.1 } \\ 
                \hline
                RDF6.2.12 & \parbox[t]{4cm}{ UCN3.7.1 \\ UCN3.7.1.2 } \\ 
                \hline
                RDF6.2.13 & \parbox[t]{4cm}{ UCN3.7.1 \\ UCN3.7.1.1 } \\ 
                \hline
                RDF6.2.10 & \parbox[t]{4cm}{ UCN3.4.1 } \\ 
                \hline
                RDF6.2.11 & \parbox[t]{4cm}{ UCN3.6.1 } \\ 
                \hline
                RDF6.2.8 & \parbox[t]{4cm}{ UCN3.4.3 } \\ 
                \hline
                RDF6.2.5 & \parbox[t]{4cm}{ UCN3.4.2 \\ UCN3.4.2.2 } \\ 
                \hline
                RDF6.2.6 & \parbox[t]{4cm}{ UCN3.6.2 \\ UCN3.6.2.2 } \\ 
                \hline
                RDF6.2.2 & \parbox[t]{4cm}{ UCN3.4.2 \\ UCN3.4.2.1 } \\ 
                \hline
                RDF6.2.3 & \parbox[t]{4cm}{ UCN3.6.2 \\ UCN3.6.2.1 } \\ 
                \hline
                RRF6.3 & \parbox[t]{4cm}{ UCN3.8 \\ UCN3.8.1 \\ UCN3.8.2 \\ UCN3.8.3 } \\ 
                \hline
                RRF6.4 & \parbox[t]{4cm}{ UCN3.9 } \\ 
                \hline
                RRF7 & \parbox[t]{4cm}{ Capitolato \\ UCD0 \\ UCD1 } \\ 
                \hline
                RRF7.1 & \parbox[t]{4cm}{ Capitolato \\ UCD2 \\ UCD2.2 \\ UCD2.1 } \\ 
                \hline
                RRF7.2 & \parbox[t]{4cm}{ Capitolato \\ UCD3 \\ UCD3.1 \\ UCD3.2 } \\ 
                \hline
                ROF8 & \parbox[t]{4cm}{ Capitolato \\ UCA0 } \\ 
                \hline
                RRF8.1 & \parbox[t]{4cm}{ Capitolato \\ UCA1 \\ UCA1.1 \\ UCA1.2 \\ UCA1.3 } \\ 
                \hline
                RRF8.2 & \parbox[t]{4cm}{ Capitolato \\ UCA2 \\ UCA2.1 \\ UCA2.2 \\ UCA2.3 \\ UCA2.4 } \\ 
                \hline
                RRF8.3 & \parbox[t]{4cm}{ Capitolato \\ UCA3 \\ UCA3.1 \\ UCA3.2 } \\ 
                \hline
                RRF8.4 & \parbox[t]{4cm}{ UCA4 \\ UCA4.1 } \\ 
                \hline
                RRF8.5 & \parbox[t]{4cm}{ UCA4 \\ UCA4.2 } \\ 
                \hline
                RRC22 & \parbox[t]{4cm}{ Capitolato } \\ 
                \hline
                RRC10 & \parbox[t]{4cm}{ Capitolato } \\ 
                \hline
                RRC11 & \parbox[t]{4cm}{ Capitolato } \\ 
                \hline
                RRC12 & \parbox[t]{4cm}{ Capitolato } \\ 
                \hline
                RRQ13 & \parbox[t]{4cm}{ Capitolato } \\ 
                \hline
                RRQ14 & \parbox[t]{4cm}{ Capitolato } \\ 
                \hline
                RRC15 & \parbox[t]{4cm}{ Capitolato } \\ 
                \hline
                RRC16 & \parbox[t]{4cm}{ Capitolato } \\ 
                \hline
                RRC17 & \parbox[t]{4cm}{ Capitolato } \\ 
                \hline
                RDC18 & \parbox[t]{4cm}{ Capitolato } \\ 
                \hline
                RRQ19 & \parbox[t]{4cm}{ Capitolato } \\ 
                \hline
                RDQ20 & \parbox[t]{4cm}{ Capitolato } \\ 
                \hline
                RDQ21 & \parbox[t]{4cm}{ Capitolato } \\ 
                \hline
                RRF9 & \parbox[t]{4cm}{ UCN1.12 \\ UCN1.12.1 \\ UCN1.12.2 \\ UCN1.12.3 } \\ 
                \hline
                                \caption{Tracciamento requisiti-fonti}
				\end{longtabu}

\subsection{Tracciamento fonti-requisiti}

				\begin{longtabu} spread 1cm [c]{|X[l]|X[l]|X[l]|}
					\hline
					\rowfont{\bf \centering}
					Codice &
					Dettaglio &
					Requisiti abbinati \\
					\hline
					\endhead
					
					Capitolato & Capitolato d'appalto C3 & \parbox[t]{4cm}{ RRF1 \\ RRF2 \\ RRF3 \\ RRF4 \\ RRF5 \\ RRF1.1 \\ RRF1.5.17 \\ RRF1.5 \\ RRF1.5.1 \\ RRF1.5.5 \\ RRF1.5.5.1 \\ RRF1.5.5.2 \\ RRF1.5.1.1 \\ RRF1.5.1.2 \\ RRF1.5.18 \\ RRF1.5.7 \\ RRF1.5.8 \\ RRF1.5.21 \\ RRF1.5.9 \\ RRF1.5.10 \\ RRF1.5.12 \\ RRF1.5.13 \\ RRF1.5.14 \\ RRF1.5.16 \\ RRF1.5.15 \\ RRF1.5.7.1 \\ RRF1.5.7.2 \\ RRF1.6 \\ RRF2.1 \\ RRF2.2 \\ RRF2.3 \\ RRF3.1 \\ RRF3.2 \\ RRF3.3 \\ RRF3.2.1 \\ RRF3.2.2 \\ RRF5.1 \\ RRF5.2 \\ RRF7 \\ RRF7.1 \\ RRF7.2 \\ ROF8 \\ RRF8.1 \\ RRF8.2 \\ RRF8.3 \\ RRC22 \\ RRC10 \\ RRC11 \\ RRC12 \\ RRQ13 } \\ \hline
                                        Capitolato & Capitolato d'appalto C3 & \parbox[t]{4cm}{RRQ14 \\ RRC15 \\ RRC16 \\ RRC17 \\ RDC18 \\ RRQ19 \\ RDQ20 \\ RDQ21 \\ RRF1.2 \\ RRF1.3 \\ RRF1.4 \\ RRF2.4 \\ RRF2.5 \\ RRF2.6 \\ RRF2.7 \\ RRF1.5.2 \\ RRF1.5.3 \\ RRF1.5.2.1 \\ RRF1.5.3.1 \\ RRF1.5.2.2 \\ RRF1.5.3.2 \\ RRF1.5.6 \\ RRF1.5.6.1 \\ RRF1.5.6.2 \\ RRF1.5.11 \\ RRF1.5.19 \\ RRF1.5.22 \\ RRF1.5.23 }\\
                \hline
                UCA0 & UCA0 Utilizzo dell'applicazione Android & \parbox[t]{4cm}{ ROF8 }\\
                \hline
                UCA1 & UCA1 Accedere a un istanza di Norris & \parbox[t]{4cm}{ RRF8.1 }\\
                \hline
                UCA1.1 & UCA1.1 Inserire indirizzo di un istanza di Norris & \parbox[t]{4cm}{ RRF8.1 }\\
                \hline
                UCA1.2 & UCA1.2 Inserire username & \parbox[t]{4cm}{ RRF8.1 }\\
                \hline
                UCA1.3 & UCA1.3 Inserire password & \parbox[t]{4cm}{ RRF8.1 }\\
                \hline
                UCA2 & UCA2 Visualizzare elenco grafici & \parbox[t]{4cm}{ RRF8.2 }\\
                \hline
                UCA2.1 & UCA2.1 Visualizzare ID grafici & \parbox[t]{4cm}{ RRF8.2 }\\
                \hline
                UCA2.2 & UCA2.2 Visualizzare titolo grafici & \parbox[t]{4cm}{ RRF8.2 }\\
                \hline
                UCA2.3 & UCA2.3 Visualizzare tipo grafici & \parbox[t]{4cm}{ RRF8.2 }\\
                \hline
                UCA2.4 & UCA2.4 Visualizzare descrizione grafici & \parbox[t]{4cm}{ RRF8.2 }\\
                \hline
                UCA3 & UCA3 Visualizzare singolo grafico & \parbox[t]{4cm}{ RRF8.3 }\\
                \hline
                UCA3.1 & UCA3.1 Selezione di un grafico dall'elenco di grafici esistenti & \parbox[t]{4cm}{ RRF8.3 }\\
                \hline
                UCA3.2 & UCA3.2 Visualizzare il grafico selezionato & \parbox[t]{4cm}{ RRF8.3 }\\
                \hline
                UCA4 & UCA4 Visualizzare errore dati di accesso non validi & \parbox[t]{4cm}{ RRF8.4 \\ RRF8.5 }\\
                \hline
                UCA4.1 & UCA4.1 Visualizzare errore indirizzo dell istanza Norris non valido & \parbox[t]{4cm}{ RRF8.4 }\\
                \hline
                UCA4.2 & UCA4.2 Visualizzare errore credenziali di accesso non valide & \parbox[t]{4cm}{ RRF8.5 }\\
                \hline
                UCD0 & UCD0 Utilizzo della dashboard & \parbox[t]{4cm}{ RRF7 }\\
                \hline
                UCD1 & UCD1 Visualizzazione posizione autobus APS in tempo reale & \parbox[t]{4cm}{ RRF7 }\\
                \hline
                UCD2 & UCD2 Visualizzazione numero di autobus attivi per ciascuna linea & \parbox[t]{4cm}{ RRF7.1 }\\
                \hline
                UCD2.1 & UCD2.1 Selezione della linea della quale si vuole visualizzare il numero di bus & \parbox[t]{4cm}{ RRF7.1 }\\
                \hline
                UCD2.2 & UCD2.2 Visualizzare il numero di autobus attivi per una linea & \parbox[t]{4cm}{ RRF7.1 }\\
                \hline
                UCD3 & UCD3 Filtrare autobus per linea di appartenenza & \parbox[t]{4cm}{ RRF7.2 }\\
                \hline
                UCD3.1 & UCD3.1 Selezionare le linee che si intende visualizzare & \parbox[t]{4cm}{ RRF7.2 }\\
                \hline
                UCD3.2 & UCD3.2 Visualizzare i bus appartenenti alle linee scelte & \parbox[t]{4cm}{ RRF7.2 }\\
                \hline
                UCN0 & UCN0 Utilizzo del framework Norris & \parbox[t]{4cm}{ RRF1 \\ RRF2 \\ RRF3 \\ RRF4 \\ RRF5 \\ RRF6 }\\
                \hline
                UCN1 & UCN1 Utilizzo API interne di Norris & \parbox[t]{4cm}{ RRF1 \\ RRF2 \\ RRF3 \\ RRF4 }\\
                \hline
                UCN1.1 & UCN1.1 Creare modello chart & \parbox[t]{4cm}{ RRF1 }\\
                \hline
                UCN1.1.1 & UCN1.1.1 Inserire il titolo del chart & \parbox[t]{4cm}{ RRF1.5.17 }\\
                \hline
                UCN1.1.2 & UCN1.1.2 Scegliere una descrizione per il chart & \parbox[t]{4cm}{ RDF1.5.4 }\\
                \hline
                UCN1.10 & UCN1.10 Visualizzare errore tipologia di aggiornamento non valida & \parbox[t]{4cm}{ RRF1.10 }\\
                \hline
                UCN1.11 & UCN1.11 Visualizzare errore raggiunto limite grafici inseribili in una pagina & \parbox[t]{4cm}{ RRF1.9 }\\
                \hline
                UCN1.12 & UCN1.12 Fornire funzioni di autenticazione & \parbox[t]{4cm}{ RRF9 }\\
                \hline
                UCN1.12.1 & UCN1.12.1 Fornire funzione di login & \parbox[t]{4cm}{ RRF9 }\\
                \hline
                UCN1.12.2 & UCN1.12.2 Fornire funzione di logout & \parbox[t]{4cm}{ RRF9 }\\
                \hline
                UCN1.12.3 & UCN1.12.3 Fornire funzione di verifica autenticazione & \parbox[t]{4cm}{ RRF9 }\\
                \hline
                UCN1.2 & UCN1.2 Aggiornare un modello chart & \parbox[t]{4cm}{ RRF2 }\\
                \hline
                UCN1.2.1 & UCN1.2.1 Aggiornare un modello bar chart & \parbox[t]{4cm}{ RRF2 }\\
                \hline
                UCN1.2.1.1 & UCN1.2.1.1 Aggiornare un modello bar chart selezionato con metodo in place & \parbox[t]{4cm}{ RRF2.4 }\\
                \hline
                UCN1.2.2 & UCN1.2.2 Aggiornare un modello line chart & \parbox[t]{4cm}{ RRF2 }\\
                \hline
                UCN1.2.2.1 & UCN1.2.2.1 Aggiornare un modello line chart con metodo in place & \parbox[t]{4cm}{ RRF2.5 }\\
                \hline
                UCN1.2.2.2 & UCN1.2.2.2 Aggiornare un modello line chart con metodo stream & \parbox[t]{4cm}{ RRF2.1 }\\
                \hline
                UCN1.2.3 & UCN1.2.3 Aggiornare un modello map chart & \parbox[t]{4cm}{ RRF2 }\\
                \hline
                UCN1.2.3.1 & UCN1.2.3.1 Aggiornare un modello map chart con metodo in place & \parbox[t]{4cm}{ RRF2.6 }\\
                \hline
                UCN1.2.3.2 & UCN1.2.3.2 Aggiornare un modello map chart con metodo movie & \parbox[t]{4cm}{ RRF2.3 }\\
                \hline
                UCN1.2.4 & UCN1.2.4 Aggiornare un modello table & \parbox[t]{4cm}{ RRF2 }\\
                \hline
                UCN1.2.4.1 & UCN1.2.4.1 Aggiornare un modello table con metodo in place & \parbox[t]{4cm}{ RRF2.7 }\\
                \hline
                UCN1.2.4.2 & UCN1.2.4.2 Aggiornare un modello table con metodo stream & \parbox[t]{4cm}{ RRF2.2 }\\
                \hline
                UCN1.3 & UCN1.3 Creare pagina & \parbox[t]{4cm}{ RRF3 }\\
                \hline
                UCN1.3.1 & UCN1.3.1 Inserire il titolo della pagina & \parbox[t]{4cm}{ RRF3.1 }\\
                \hline
                UCN1.3.2 & UCN1.3.2 Scegliere le opzioni di visualizzazione della pagina & \parbox[t]{4cm}{ RRF3.2 }\\
                \hline
                UCN1.3.2.1 & UCN1.3.2.1 Scegliere il massimo numero di grafici visualizzabili su una riga & \parbox[t]{4cm}{ RRF3.2.1 }\\
                \hline
                UCN1.3.2.2 & UCN1.3.2.2 Scegliere il massimo numero di grafici visualizzabili su una colonna & \parbox[t]{4cm}{ RRF3.2.2 }\\
                \hline
                UCN1.3.3 & UCN1.3.3 Aggiungere grafico alla pagina & \parbox[t]{4cm}{ RRF3.3 }\\
                \hline
                UCN1.4 & UCN1.4 Creare modello bar chart & \parbox[t]{4cm}{ RRF1.1 }\\
                \hline
                UCN1.4.1 & UCN1.4.1 Scegliere opzioni bar chart & \parbox[t]{4cm}{ RRF1.5 }\\
                \hline
                UCN1.4.1.1 & UCN1.4.1.1 Scegliere il colore di ciascun set di barre & \parbox[t]{4cm}{ RRF1.5.19 }\\
                \hline
                UCN1.4.1.2 & UCN1.4.1.2 Scegliere l'orientamento delle barre & \parbox[t]{4cm}{ RRF1.5.8 }\\
                \hline
                UCN1.4.1.3 & UCN1.4.1.3 Scegliere il formato di stampa dei valori del bar chart & \parbox[t]{4cm}{ RRF1.5.7 }\\
                \hline
                UCN1.4.1.3.1 & UCN1.4.1.3.1 Scegliere la dimensione dello spazio tra due serie del bar chart & \parbox[t]{4cm}{ RDF1.5.7.3 }\\
                \hline
                UCN1.4.1.3.2 & UCN1.4.1.3.2 Scegliere la dimensione dello spazio tra due valori del bar chart & \parbox[t]{4cm}{ RRF1.5.7.4 }\\
                \hline
                UCN1.4.1.4 & UCN1.4.1.4 Scegliere il nome di ciascun set di barre & \parbox[t]{4cm}{ RRF1.5.22 }\\
                \hline
                UCN1.4.1.5 & UCN1.4.1.5 Scegliere le opzioni riguardanti la legenda di un bar chart & \parbox[t]{4cm}{ RRF1.5.2 }\\
                \hline
                UCN1.4.1.5.1 & UCN1.4.1.5.1 Scegliere se la legenda è visualizzata & \parbox[t]{4cm}{ RRF1.5.2.1 }\\
                \hline
                UCN1.4.1.5.2 & UCN1.4.1.5.2 Scegliere la posizione in cui è visualizzata la legenda & \parbox[t]{4cm}{ RRF1.5.2.2 }\\
                \hline
                UCN1.4.1.6 & UCN1.4.1.6 Scegliere le opzioni riguardanti il piano cartesiano del bar chart & \parbox[t]{4cm}{ RRF1.5.6 }\\
                \hline
                UCN1.4.1.6.1 & UCN1.4.1.6.1 Scegliere il nome degli assi & \parbox[t]{4cm}{ RRF1.5.6.1 }\\
                \hline
                UCN1.4.1.6.2 & UCN1.4.1.6.2 Scegliere se le linee della griglia sono visualizzate & \parbox[t]{4cm}{ RRF1.5.6.2 }\\
                \hline
                UCN1.4.1.7 & UCN1.4.1.7 Scegliere il massimo numero di barre da visualizzare per ogni serie & \parbox[t]{4cm}{ RRF1.5.11 }\\
                \hline
                UCN1.4.2 & UCN1.4.2 Inserire dati bar chart & \parbox[t]{4cm}{ RRF1.6 }\\
                \hline
                UCN1.4.2.1 & UCN1.4.2.1 Inserire i valori indipendenti nel bar chart & \parbox[t]{4cm}{ RRF1.6 }\\
                \hline
                UCN1.4.2.2 & UCN1.4.2.2 Inserire i valori dipendenti per ciascun set di barre & \parbox[t]{4cm}{ RRF1.6 }\\
                \hline
                UCN1.5 & UCN1.5 Creare modello line chart & \parbox[t]{4cm}{ RRF1.2 }\\
                \hline
                UCN1.5.1 & UCN1.5.1 Scegliere opzioni line chart & \parbox[t]{4cm}{ RRF1.5 }\\
                \hline
                UCN1.5.1.1 & UCN1.5.1.1 Scegliere il colore di ciascuna linea & \parbox[t]{4cm}{ RRF1.5.18 }\\
                \hline
                UCN1.5.1.2 & UCN1.5.1.2 Scegliere il formato di stampa dei valori del line chart & \parbox[t]{4cm}{ RRF1.5.7 }\\
                \hline
                UCN1.5.1.2.1 & UCN1.5.1.2.1 Scegliere la dimensione dei punti del line chart & \parbox[t]{4cm}{ RDF1.5.7.6 }\\
                \hline
                UCN1.5.1.2.2 & UCN1.5.1.2.2 Scegliere se la linea del line chart è curva o segmentata & \parbox[t]{4cm}{ RRF1.5.7.7 }\\
                \hline
                UCN1.5.1.3 & UCN1.5.1.3 Scegliere il nome di ciascuna linea & \parbox[t]{4cm}{ RRF1.5.21 }\\
                \hline
                UCN1.5.1.4 & UCN1.5.1.4 Scegliere il massimo numero di punti visualizzati sull'asse dei valori indipendenti & \parbox[t]{4cm}{ RRF1.5.11 }\\
                \hline
                UCN1.5.1.5 & UCN1.5.1.5 Scegliere le opzioni riguardanti la legenda di un line chart & \parbox[t]{4cm}{ RRF1.5.1 }\\
                \hline
                UCN1.5.1.5.1 & UCN1.5.1.5.1 Scegliere se la legenda è visualizzata & \parbox[t]{4cm}{ RRF1.5.1.1 }\\
                \hline
                UCN1.5.1.5.2 & UCN1.5.1.5.2 Scegliere la posizione in cui è visualizzata la legenda & \parbox[t]{4cm}{ RRF1.5.1.2 }\\
                \hline
                UCN1.5.1.6 & UCN1.5.1.6 Scegliere le opzioni riguardanti il piano cartesiano del line chart & \parbox[t]{4cm}{ RRF1.5.5 }\\
                \hline
                UCN1.5.1.6.1 & UCN1.5.1.6.1 Scegliere il nome degli assi & \parbox[t]{4cm}{ RRF1.5.5.1 }\\
                \hline
                UCN1.5.1.6.2 & UCN1.5.1.6.2 Scegliere se le linee della griglia sono visualizzate & \parbox[t]{4cm}{ RRF1.5.5.2 }\\
                \hline
                UCN1.5.2 & UCN1.5.2 Inserire dati line chart & \parbox[t]{4cm}{ RRF1.6 }\\
                \hline
                UCN1.5.2.1 & UCN1.5.2.1 Inserire i valori indipendenti nel line chart & \parbox[t]{4cm}{ RRF1.6 }\\
                \hline
                UCN1.5.2.2 & UCN1.5.2.2 Inserire i valori dipendenti per ciascuna linea & \parbox[t]{4cm}{ RRF1.6 }\\
                \hline
                UCN1.6 & UCN1.6 Creare modello map chart & \parbox[t]{4cm}{ RRF1.3 }\\
                \hline
                UCN1.6.1 & UCN1.6.1 Scegliere opzioni map chart & \parbox[t]{4cm}{ RRF1.5 }\\
                \hline
                UCN1.6.1.1 & UCN1.6.1.1 Scegliere il formato di stampa dei valori del map chart & \parbox[t]{4cm}{ RRF1.5.7 }\\
                \hline
                UCN1.6.1.1.1 & UCN1.6.1.1.1 Scegliere la forma dei marcatori del map chart & \parbox[t]{4cm}{ RDF1.5.7.5 }\\
                \hline
                UCN1.6.1.2 & UCN1.6.1.2 Scegliere le dimensioni dell'area mostrata sulla mappa & \parbox[t]{4cm}{ RRF1.5.9 }\\
                \hline
                UCN1.6.1.3 & UCN1.6.1.3 Scegliere il massimo numero di punti da visualizzare per ogni serie & \parbox[t]{4cm}{ RRF1.5.11 }\\
                \hline
                UCN1.6.1.4 & UCN1.6.1.4 Scegliere le coordinate del punto centrale della mappa & \parbox[t]{4cm}{ RRF1.5.10 }\\
                \hline
                UCN1.6.1.5 & UCN1.6.1.5 Scegliere il colore di ciascuna serie di punti & \parbox[t]{4cm}{ }\\
                \hline
                UCN1.6.1.6 & UCN1.6.1.6 Scegliere il nome di ciascuna serie di punti & \parbox[t]{4cm}{ RRF1.5.23 }\\
                \hline
                UCN1.6.1.7 & UCN1.6.1.7 Scegliere le opzioni riguardanti la legenda di un map chart & \parbox[t]{4cm}{ RRF1.5.3 }\\
                \hline
                UCN1.6.1.7.1 & UCN1.6.1.7.1 Scegliere se la legenda è visualizzata & \parbox[t]{4cm}{ RRF1.5.3.1 }\\
                \hline
                UCN1.6.1.7.2 & UCN1.6.1.7.2 Scegliere la posizione in cui è visualizzata la legenda & \parbox[t]{4cm}{ RRF1.5.3.2 }\\
                \hline
                UCN1.6.2 & UCN1.6.2 Inserire nuova serie di punti & \parbox[t]{4cm}{ RRF1.6 }\\
                \hline
                UCN1.7 & UCN1.7 Creare modello table & \parbox[t]{4cm}{ RRF1.4 }\\
                \hline
                UCN1.7.1 & UCN1.7.1 Scegliere opzioni table & \parbox[t]{4cm}{ RRF1.5 }\\
                \hline
                UCN1.7.1.1 & UCN1.7.1.1 Scegliere il formato di una cella della table & \parbox[t]{4cm}{ RRF1.5.7 }\\
                \hline
                UCN1.7.1.1.1 & UCN1.7.1.1.1 Scegliere il colore del testo contenuto in una cella & \parbox[t]{4cm}{ RRF1.5.7.1 }\\
                \hline
                UCN1.7.1.1.2 & UCN1.7.1.1.2 Scegliere il colore dello sfondo di una cella & \parbox[t]{4cm}{ RRF1.5.7.2 }\\
                \hline
                UCN1.7.1.2 & UCN1.7.1.2 Scegliere l'intestazione di una singola colonna & \parbox[t]{4cm}{ RRF1.5.15 }\\
                \hline
                UCN1.7.1.3 & UCN1.7.1.3 Scegliere il massimo numero di righe da visualizzare & \parbox[t]{4cm}{ RRF1.5.12 }\\
                \hline
                UCN1.7.1.4 & UCN1.7.1.4 Scegliere se è possibile ordinare le righe rispetto ai valori di una colonna & \parbox[t]{4cm}{ RRF1.5.13 }\\
                \hline
                UCN1.7.1.5 & UCN1.7.1.5 Scegliere la posizione in cui vengono aggiunte nuove righe & \parbox[t]{4cm}{ RRF1.5.14 }\\
                \hline
                UCN1.7.1.6 & UCN1.7.1.6 Scegliere se le linee della tabella sono visualizzate & \parbox[t]{4cm}{ RRF1.5.16 }\\
                \hline
                UCN1.7.2 & UCN1.7.2 Inserire nuova riga & \parbox[t]{4cm}{ RRF1.6 }\\
                \hline
                UCN1.8 & UCN1.8 Ottenere middleware & \parbox[t]{4cm}{ RRF4 }\\
                \hline
                UCN1.9 & UCN1.9 Visualizzare errore dati non corretti & \parbox[t]{4cm}{ RRF1.7 \\ RRF1.8 }\\
                \hline
                UCN2 & UCN2 Utilizzo API esterne di Norris & \parbox[t]{4cm}{ RRF5 }\\
                \hline
                UCN2.1 & UCN2.1 Accedere a un istanza di Norris & \parbox[t]{4cm}{ RRF5.3 }\\
                \hline
                UCN2.1.1 & UCN2.1.1 Inserire indirizzo di un istanza di Norris & \parbox[t]{4cm}{ RRF5.3 }\\
                \hline
                UCN2.1.2 & UCN2.1.2 Inserire username & \parbox[t]{4cm}{ RRF5.3 }\\
                \hline
                UCN2.1.3 & UCN2.1.3 Inserire password & \parbox[t]{4cm}{ RRF5.3 }\\
                \hline
                UCN2.2 & UCN2.2 Ottenere lista dei grafici presenti in un'istanza di Norris & \parbox[t]{4cm}{ RRF5.1 }\\
                \hline
                UCN2.2.1 & UCN2.2.1 Ottenere l'ID di ciascun grafico & \parbox[t]{4cm}{ RRF5.1.1 }\\
                \hline
                UCN2.2.2 & UCN2.2.2 Ottenere il titolo di ciascun grafico & \parbox[t]{4cm}{ RRF5.1.1 }\\
                \hline
                UCN2.2.3 & UCN2.2.3 Ottenere il tipo di ciascun grafico & \parbox[t]{4cm}{ RRF5.1.1 }\\
                \hline
                UCN2.2.4 & UCN2.2.4 Ottenere la descrizione di ciascun grafico & \parbox[t]{4cm}{ RRF5.1.1 }\\
                \hline
                UCN2.3 & UCN2.3 Ottenere un grafico presente in un'istanza di Norris & \parbox[t]{4cm}{ RRF5.2 }\\
                \hline
                UCN2.4 & UCN2.4 Scollegarsi da un istanza di Norris & \parbox[t]{4cm}{ RRF5.4 }\\
                \hline
                UCN3 & UCN3 Utilizzo API fornite da Chuck & \parbox[t]{4cm}{ RRF6 }\\
                \hline
                UCN3.1 & UCN3.1 Selezionare il modello di grafico che si vuole rappresentare & \parbox[t]{4cm}{ RRF6.1 }\\
                \hline
                UCN3.2 & UCN3.2 Scegliere il tag HTML nel quale si vuole inserire la rappresentazione del grafico & \parbox[t]{4cm}{ RRF6.1 }\\
                \hline
                UCN3.3 & UCN3.3 Cambiare opzioni di visualizzazione di un grafico & \parbox[t]{4cm}{ RDF6.2 }\\
                \hline
                UCN3.4 & UCN3.4 Cambiare opzioni di visualizzazione di un bar chart & \parbox[t]{4cm}{ RDF6.2 }\\
                \hline
                UCN3.4.1 & UCN3.4.1 Cambiare il colore di un set di barre & \parbox[t]{4cm}{ RDF6.2.10 }\\
                \hline
                UCN3.4.2 & UCN3.4.2 Cambiare le opzioni riguardanti la legenda di un bar chart & \parbox[t]{4cm}{ RDF6.2.5 \\ RDF6.2.2 }\\
                \hline
                UCN3.4.2.1 & UCN3.4.2.1 Cambiare il fatto che la legenda sia visualizzata o meno & \parbox[t]{4cm}{ RDF6.2.2 }\\
                \hline
                UCN3.4.2.2 & UCN3.4.2.2 Cambiare la posizione in cui è visualizzata la legenda & \parbox[t]{4cm}{ RDF6.2.5 }\\
                \hline
                UCN3.4.3 & UCN3.4.3 Cambiare il fatto che la griglia del piano cartesiano sia visualizzata o meno & \parbox[t]{4cm}{ RDF6.2.8 }\\
                \hline
                UCN3.5 & UCN3.5 Cambiare opzioni di visualizzazione di un line chart & \parbox[t]{4cm}{ RDF6.2 }\\
                \hline
                UCN3.5.1 & UCN3.5.1 Cambiare il colore di una linea & \parbox[t]{4cm}{ RDF6.2.9 }\\
                \hline
                UCN3.5.2 & UCN3.5.2 Cambiare le opzioni riguardanti la legenda di un line chart & \parbox[t]{4cm}{ RDF6.2.1 \\ RDF6.2.4 }\\
                \hline
                UCN3.5.2.1 & UCN3.5.2.1 Cambiare il fatto che la legenda sia visualizzata o meno & \parbox[t]{4cm}{ RDF6.2.1 }\\
                \hline
                UCN3.5.2.2 & UCN3.5.2.2 Cambiare la posizione in cui è visualizzata la legenda & \parbox[t]{4cm}{ RDF6.2.4 }\\
                \hline
                UCN3.5.3 & UCN3.5.3 Cambiare il fatto che la griglia del piano cartesiano sia visualizzata o meno & \parbox[t]{4cm}{ RDF6.2.7 }\\
                \hline
                UCN3.6 & UCN3.6 Cambiare opzioni di visualizzazione di un map chart & \parbox[t]{4cm}{ RDF6.2 }\\
                \hline
                UCN3.6.1 & UCN3.6.1 Cambiare il colore di una serie di punti & \parbox[t]{4cm}{ RDF6.2.11 }\\
                \hline
                UCN3.6.2 & UCN3.6.2 Cambiare le opzioni riguardanti la legenda di un map chart & \parbox[t]{4cm}{ RDF6.2.6 \\ RDF6.2.3 }\\
                \hline
                UCN3.6.2.1 & UCN3.6.2.1 Cambiare il fatto che la legenda sia visualizzata o meno & \parbox[t]{4cm}{ RDF6.2.3 }\\
                \hline
                UCN3.6.2.2 & UCN3.6.2.2 Cambiare la posizione in cui è visualizzata la legenda & \parbox[t]{4cm}{ RDF6.2.6 }\\
                \hline
                UCN3.7 & UCN3.7 Cambiare opzioni di visualizzazione di una table & \parbox[t]{4cm}{ RDF6.2 }\\
                \hline
                UCN3.7.1 & UCN3.7.1 Cambiare le opzioni riguardanti il formato di una cella & \parbox[t]{4cm}{ RDF6.2.12 \\ RDF6.2.13 }\\
                \hline
                UCN3.7.1.1 & UCN3.7.1.1 Cambiare il colore del testo contenuto in una cella & \parbox[t]{4cm}{ RDF6.2.13 }\\
                \hline
                UCN3.7.1.2 & UCN3.7.1.2 - Cambiare il colore di sfondo di una cella & \parbox[t]{4cm}{ RDF6.2.12 }\\
                \hline
                UCN3.8 & UCN3.8 Accedere a un istanza di Norris & \parbox[t]{4cm}{ RRF6.3 }\\
                \hline
                UCN3.8.1 & UCN3.8.1 Inserire indirizzo di un istanza di Norris & \parbox[t]{4cm}{ RRF6.3 }\\
                \hline
                UCN3.8.2 & UCN3.8.2 Inserire Username & \parbox[t]{4cm}{ RRF6.3 }\\
                \hline
                UCN3.8.3 & UCN3.8.3 Inserire password & \parbox[t]{4cm}{ RRF6.3 }\\
                \hline
                UCN3.9 & UCN3.9 Scollegarsi da un istanza di Norris & \parbox[t]{4cm}{ RRF6.4 }\\
                \hline
                                \caption{Tracciamento fonti-requisiti}
				\end{longtabu}
				

\subsection{Riepilogo}

				\begin{longtabu} spread 1cm [c]{|X[-1,l]|X[-1,l]|X[-1,l]|X[-1,l]|}
					\hline
					\rowfont{\bf \centering}
					Categoria &
					Obbligatorio &
					Desiderabile &
					Opzionale \\
					\hline
					\endhead
					
					Funzionale & 81 & 18 & 1 \\ \hline Prestazionale & 0 & 0 & 0 \\ \hline Di qualità & 3 & 2 & 0 \\ \hline Di vincolo & 7 & 1 & 0 \\ \hline                 \caption{Riepilogo dei requisiti}
				\end{longtabu}

				\begin{longtabu} spread 1cm [c]{|X[-1,l]|X[-1,l]|}
					\hline
					\rowfont{\bf \centering}
					Codice &
					Dettaglio \\
					\hline
					\endhead
					
					Capitolato &
                Capitolato d'appalto C3\\\hline UCA0 &
                UCA0 Utilizzo dell'applicazione Android\\\hline UCA1 &
                UCA1 Accedere a un istanza di Norris\\\hline UCA1.1 &
                UCA1.1 Inserire indirizzo di un istanza di Norris\\\hline UCA1.2 &
                UCA1.2 Inserire username\\\hline UCA1.3 &
                UCA1.3 Inserire password\\\hline UCA2 &
                UCA2 Visualizzare elenco grafici\\\hline UCA2.1 &
                UCA2.1 Visualizzare ID grafici\\\hline UCA2.2 &
                UCA2.2 Visualizzare titolo grafici\\\hline UCA2.3 &
                UCA2.3 Visualizzare tipo grafici\\\hline UCA2.4 &
                UCA2.4 Visualizzare descrizione grafici\\\hline UCA3 &
                UCA3 Visualizzare singolo grafico\\\hline UCA3.1 &
                UCA3.1 Selezione di un grafico dall'elenco di grafici esistenti\\\hline UCA3.2 &
                UCA3.2 Visualizzare il grafico selezionato\\\hline UCA4 &
                UCA4 Visualizzare errore dati di accesso non validi\\\hline UCA4.1 &
                UCA4.1 Visualizzare errore indirizzo dell istanza Norris non valido\\\hline UCA4.2 &
                UCA4.2 Visualizzare errore credenziali di accesso non valide\\\hline UCD0 &
                UCD0 Utilizzo della dashboard\\\hline UCD1 &
                UCD1 Visualizzazione posizione autobus APS in tempo reale\\\hline UCD2 &
                UCD2 Visualizzazione numero di autobus attivi per ciascuna linea\\\hline UCD2.1 &
                UCD2.1 Selezione della linea della quale si vuole visualizzare il numero di bus\\\hline UCD2.2 &
                UCD2.2 Visualizzare il numero di autobus attivi per una linea\\\hline UCD3 &
                UCD3 Filtrare autobus per linea di appartenenza\\\hline UCD3.1 &
                UCD3.1 Selezionare le linee che si intende visualizzare\\\hline UCD3.2 &
                UCD3.2 Visualizzare i bus appartenenti alle linee scelte\\\hline UCN0 &
                UCN0 Utilizzo del framework Norris\\\hline UCN1 &
                UCN1 Utilizzo API interne di Norris\\\hline UCN1.1 &
                UCN1.1 Creare modello chart\\\hline UCN1.1.1 &
                UCN1.1.1 Inserire il titolo del chart\\\hline UCN1.1.2 &
                UCN1.1.2 Scegliere una descrizione per il chart\\\hline UCN1.10 &
                UCN1.10 Visualizzare errore tipologia di aggiornamento non valida\\\hline UCN1.11 &
                UCN1.11 Visualizzare errore raggiunto limite grafici inseribili in una pagina\\\hline UCN1.12 &
                UCN1.12 Fornire funzioni di autenticazione\\\hline UCN1.12.1 &
                UCN1.12.1 Fornire funzione di login\\\hline UCN1.12.2 &
                UCN1.12.2 Fornire funzione di logout\\\hline UCN1.12.3 &
                UCN1.12.3 Fornire funzione di verifica autenticazione\\\hline UCN1.2 &
                UCN1.2 Aggiornare un modello chart\\\hline UCN1.2.1 &
                UCN1.2.1 Aggiornare un modello bar chart\\\hline UCN1.2.1.1 &
                UCN1.2.1.1 Aggiornare un modello bar chart selezionato con metodo in place\\\hline UCN1.2.2 &
                UCN1.2.2 Aggiornare un modello line chart\\\hline UCN1.2.2.1 &
                UCN1.2.2.1 Aggiornare un modello line chart con metodo in place\\\hline UCN1.2.2.2 &
                UCN1.2.2.2 Aggiornare un modello line chart con metodo stream\\\hline UCN1.2.3 &
                UCN1.2.3 Aggiornare un modello map chart\\\hline UCN1.2.3.1 &
                UCN1.2.3.1 Aggiornare un modello map chart con metodo in place\\\hline UCN1.2.3.2 &
                UCN1.2.3.2 Aggiornare un modello map chart con metodo movie\\\hline UCN1.2.4 &
                UCN1.2.4 Aggiornare un modello table\\\hline UCN1.2.4.1 &
                UCN1.2.4.1 Aggiornare un modello table con metodo in place\\\hline UCN1.2.4.2 &
                UCN1.2.4.2 Aggiornare un modello table con metodo stream\\\hline UCN1.3 &
                UCN1.3 Creare pagina\\\hline UCN1.3.1 &
                UCN1.3.1 Inserire il titolo della pagina\\\hline UCN1.3.2 &
                UCN1.3.2 Scegliere le opzioni di visualizzazione della pagina\\\hline UCN1.3.2.1 &
                UCN1.3.2.1 Scegliere il massimo numero di grafici visualizzabili su una riga\\\hline UCN1.3.2.2 &
                UCN1.3.2.2 Scegliere il massimo numero di grafici visualizzabili su una colonna\\\hline UCN1.3.3 &
                UCN1.3.3 Aggiungere grafico alla pagina\\\hline UCN1.4 &
                UCN1.4 Creare modello bar chart\\\hline UCN1.4.1 &
                UCN1.4.1 Scegliere opzioni bar chart\\\hline UCN1.4.1.1 &
                UCN1.4.1.1 Scegliere il colore di ciascun set di barre\\\hline UCN1.4.1.2 &
                UCN1.4.1.2 Scegliere l'orientamento delle barre\\\hline UCN1.4.1.3 &
                UCN1.4.1.3 Scegliere il formato di stampa dei valori del bar chart\\\hline UCN1.4.1.3.1 &
                UCN1.4.1.3.1 Scegliere la dimensione dello spazio tra due serie del bar chart\\\hline UCN1.4.1.3.2 &
                UCN1.4.1.3.2 Scegliere la dimensione dello spazio tra due valori del bar chart\\\hline UCN1.4.1.4 &
                UCN1.4.1.4 Scegliere il nome di ciascun set di barre\\\hline UCN1.4.1.5 &
                UCN1.4.1.5 Scegliere le opzioni riguardanti la legenda di un bar chart\\\hline UCN1.4.1.5.1 &
                UCN1.4.1.5.1 Scegliere se la legenda è visualizzata\\\hline UCN1.4.1.5.2 &
                UCN1.4.1.5.2 Scegliere la posizione in cui è visualizzata la legenda\\\hline UCN1.4.1.6 &
                UCN1.4.1.6 Scegliere le opzioni riguardanti il piano cartesiano del bar chart\\\hline UCN1.4.1.6.1 &
                UCN1.4.1.6.1 Scegliere il nome degli assi\\\hline UCN1.4.1.6.2 &
                UCN1.4.1.6.2 Scegliere se le linee della griglia sono visualizzate\\\hline UCN1.4.1.7 &
                UCN1.4.1.7 Scegliere il massimo numero di barre da visualizzare per ogni serie\\\hline UCN1.4.2 &
                UCN1.4.2 Inserire dati bar chart\\\hline UCN1.4.2.1 &
                UCN1.4.2.1 Inserire i valori indipendenti nel bar chart\\\hline UCN1.4.2.2 &
                UCN1.4.2.2 Inserire i valori dipendenti per ciascun set di barre\\\hline UCN1.5 &
                UCN1.5 Creare modello line chart\\\hline UCN1.5.1 &
                UCN1.5.1 Scegliere opzioni line chart\\\hline UCN1.5.1.1 &
                UCN1.5.1.1 Scegliere il colore di ciascuna linea\\\hline UCN1.5.1.2 &
                UCN1.5.1.2 Scegliere il formato di stampa dei valori del line chart\\\hline UCN1.5.1.2.1 &
                UCN1.5.1.2.1 Scegliere la dimensione dei punti del line chart\\\hline UCN1.5.1.2.2 &
                UCN1.5.1.2.2 Scegliere se la linea del line chart è curva o segmentata\\\hline UCN1.5.1.3 &
                UCN1.5.1.3 Scegliere il nome di ciascuna linea\\\hline UCN1.5.1.4 &
                UCN1.5.1.4 Scegliere il massimo numero di punti visualizzati sull'asse dei valori indipendenti\\\hline UCN1.5.1.5 &
                UCN1.5.1.5 Scegliere le opzioni riguardanti la legenda di un line chart\\\hline UCN1.5.1.5.1 &
                UCN1.5.1.5.1 Scegliere se la legenda è visualizzata\\\hline UCN1.5.1.5.2 &
                UCN1.5.1.5.2 Scegliere la posizione in cui è visualizzata la legenda\\\hline UCN1.5.1.6 &
                UCN1.5.1.6 Scegliere le opzioni riguardanti il piano cartesiano del line chart\\\hline UCN1.5.1.6.1 &
                UCN1.5.1.6.1 Scegliere il nome degli assi\\\hline UCN1.5.1.6.2 &
                UCN1.5.1.6.2 Scegliere se le linee della griglia sono visualizzate\\\hline UCN1.5.2 &
                UCN1.5.2 Inserire dati line chart\\\hline UCN1.5.2.1 &
                UCN1.5.2.1 Inserire i valori indipendenti nel line chart\\\hline UCN1.5.2.2 &
                UCN1.5.2.2 Inserire i valori dipendenti per ciascuna linea\\\hline UCN1.6 &
                UCN1.6 Creare modello map chart\\\hline UCN1.6.1 &
                UCN1.6.1 Scegliere opzioni map chart\\\hline UCN1.6.1.1 &
                UCN1.6.1.1 Scegliere il formato di stampa dei valori del map chart\\\hline UCN1.6.1.1.1 &
                UCN1.6.1.1.1 Scegliere la forma dei marcatori del map chart\\\hline UCN1.6.1.2 &
                UCN1.6.1.2 Scegliere le dimensioni dell'area mostrata sulla mappa\\\hline UCN1.6.1.3 &
                UCN1.6.1.3 Scegliere il massimo numero di punti da visualizzare per ogni serie\\\hline UCN1.6.1.4 &
                UCN1.6.1.4 Scegliere le coordinate del punto centrale della mappa\\\hline UCN1.6.1.5 &
                UCN1.6.1.5 Scegliere il colore di ciascuna serie di punti\\\hline UCN1.6.1.6 &
                UCN1.6.1.6 Scegliere il nome di ciascuna serie di punti\\\hline UCN1.6.1.7 &
                UCN1.6.1.7 Scegliere le opzioni riguardanti la legenda di un map chart\\\hline UCN1.6.1.7.1 &
                UCN1.6.1.7.1 Scegliere se la legenda è visualizzata\\\hline UCN1.6.1.7.2 &
                UCN1.6.1.7.2 Scegliere la posizione in cui è visualizzata la legenda\\\hline UCN1.6.2 &
                UCN1.6.2 Inserire nuova serie di punti\\\hline UCN1.7 &
                UCN1.7 Creare modello table\\\hline UCN1.7.1 &
                UCN1.7.1 Scegliere opzioni table\\\hline UCN1.7.1.1 &
                UCN1.7.1.1 Scegliere il formato di una cella della table\\\hline UCN1.7.1.1.1 &
                UCN1.7.1.1.1 Scegliere il colore del testo contenuto in una cella\\\hline UCN1.7.1.1.2 &
                UCN1.7.1.1.2 Scegliere il colore dello sfondo di una cella\\\hline UCN1.7.1.2 &
                UCN1.7.1.2 Scegliere l'intestazione di una singola colonna\\\hline UCN1.7.1.3 &
                UCN1.7.1.3 Scegliere il massimo numero di righe da visualizzare\\\hline UCN1.7.1.4 &
                UCN1.7.1.4 Scegliere se è possibile ordinare le righe rispetto ai valori di una colonna\\\hline UCN1.7.1.5 &
                UCN1.7.1.5 Scegliere la posizione in cui vengono aggiunte nuove righe\\\hline UCN1.7.1.6 &
                UCN1.7.1.6 Scegliere se le linee della tabella sono visualizzate\\\hline UCN1.7.2 &
                UCN1.7.2 Inserire nuova riga\\\hline UCN1.8 &
                UCN1.8 Ottenere middleware\\\hline UCN1.9 &
                UCN1.9 Visualizzare errore dati non corretti\\\hline UCN2 &
                UCN2 Utilizzo API esterne di Norris\\\hline UCN2.1 &
                UCN2.1 Accedere a un istanza di Norris\\\hline UCN2.1.1 &
                UCN2.1.1 Inserire indirizzo di un istanza di Norris\\\hline UCN2.1.2 &
                UCN2.1.2 Inserire username\\\hline UCN2.1.3 &
                UCN2.1.3 Inserire password\\\hline UCN2.2 &
                UCN2.2 Ottenere lista dei grafici presenti in un'istanza di Norris\\\hline UCN2.2.1 &
                UCN2.2.1 Ottenere l'ID di ciascun grafico\\\hline UCN2.2.2 &
                UCN2.2.2 Ottenere il titolo di ciascun grafico\\\hline UCN2.2.3 &
                UCN2.2.3 Ottenere il tipo di ciascun grafico\\\hline UCN2.2.4 &
                UCN2.2.4 Ottenere la descrizione di ciascun grafico\\\hline UCN2.3 &
                UCN2.3 Ottenere un grafico presente in un'istanza di Norris\\\hline UCN2.4 &
                UCN2.4 Scollegarsi da un istanza di Norris\\\hline UCN3 &
                UCN3 Utilizzo API fornite da Chuck\\\hline UCN3.1 &
                UCN3.1 Selezionare il modello di grafico che si vuole rappresentare\\\hline UCN3.2 &
                UCN3.2 Scegliere il tag HTML nel quale si vuole inserire la rappresentazione del grafico\\\hline UCN3.3 &
                UCN3.3 Cambiare opzioni di visualizzazione di un grafico\\\hline UCN3.4 &
                UCN3.4 Cambiare opzioni di visualizzazione di un bar chart\\\hline UCN3.4.1 &
                UCN3.4.1 Cambiare il colore di un set di barre\\\hline UCN3.4.2 &
                UCN3.4.2 Cambiare le opzioni riguardanti la legenda di un bar chart\\\hline UCN3.4.2.1 &
                UCN3.4.2.1 Cambiare il fatto che la legenda sia visualizzata o meno\\\hline UCN3.4.2.2 &
                UCN3.4.2.2 Cambiare la posizione in cui è visualizzata la legenda\\\hline UCN3.4.3 &
                UCN3.4.3 Cambiare il fatto che la griglia del piano cartesiano sia visualizzata o meno\\\hline UCN3.5 &
                UCN3.5 Cambiare opzioni di visualizzazione di un line chart\\\hline UCN3.5.1 &
                UCN3.5.1 Cambiare il colore di una linea\\\hline UCN3.5.2 &
                UCN3.5.2 Cambiare le opzioni riguardanti la legenda di un line chart\\\hline UCN3.5.2.1 &
                UCN3.5.2.1 Cambiare il fatto che la legenda sia visualizzata o meno\\\hline UCN3.5.2.2 &
                UCN3.5.2.2 Cambiare la posizione in cui è visualizzata la legenda\\\hline UCN3.5.3 &
                UCN3.5.3 Cambiare il fatto che la griglia del piano cartesiano sia visualizzata o meno\\\hline UCN3.6 &
                UCN3.6 Cambiare opzioni di visualizzazione di un map chart\\\hline UCN3.6.1 &
                UCN3.6.1 Cambiare il colore di una serie di punti\\\hline UCN3.6.2 &
                UCN3.6.2 Cambiare le opzioni riguardanti la legenda di un map chart\\\hline UCN3.6.2.1 &
                UCN3.6.2.1 Cambiare il fatto che la legenda sia visualizzata o meno\\\hline UCN3.6.2.2 &
                UCN3.6.2.2 Cambiare la posizione in cui è visualizzata la legenda\\\hline UCN3.7 &
                UCN3.7 Cambiare opzioni di visualizzazione di una table\\\hline UCN3.7.1 &
                UCN3.7.1 Cambiare le opzioni riguardanti il formato di una cella\\\hline UCN3.7.1.1 &
                UCN3.7.1.1 Cambiare il colore del testo contenuto in una cella\\\hline UCN3.7.1.2 &
                UCN3.7.1.2 - Cambiare il colore di sfondo di una cella\\\hline UCN3.8 &
                UCN3.8 Accedere a un istanza di Norris\\\hline UCN3.8.1 &
                UCN3.8.1 Inserire indirizzo di un istanza di Norris\\\hline UCN3.8.2 &
                UCN3.8.2 Inserire Username\\\hline UCN3.8.3 &
                UCN3.8.3 Inserire password\\\hline UCN3.9 &
                UCN3.9 Scollegarsi da un istanza di Norris\\\hline                 \caption{Fonti}
				\end{longtabu}
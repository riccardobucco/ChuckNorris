
				\begin{longtabu} spread 1cm [c]{|X[l]|X[l]|X[l]|}
					\hline
					\rowfont{\bf \centering}
					Codice &
					Dettaglio &
					Requisiti abbinati \\
					\hline
					\endhead
					
					Capitolato & Capitolato d'appalto C3 & \parbox[t]{4cm}{ RRF1 \\ RRF2 \\ RRF3 \\ RRF4 \\ RRF5 \\ RRF1.1 \\ RRF1.5.17 \\ RRF1.5 \\ RRF1.5.1 \\ RRF1.5.5 \\ RRF1.5.5.1 \\ RRF1.5.5.2 \\ RRF1.5.1.1 \\ RRF1.5.1.2 \\ RRF1.5.18 \\ RRF1.5.7 \\ RRF1.5.8 \\ RRF1.5.21 \\ RRF1.5.9 \\ RRF1.5.10 \\ RRF1.5.12 \\ RRF1.5.13 \\ RRF1.5.14 \\ RRF1.5.16 \\ RRF1.5.15 \\ RRF1.5.7.1 \\ RRF1.5.7.2 \\ RRF1.6 \\ RRF2.1 \\ RRF2.2 \\ RRF2.3 \\ RRF3.1 \\ RRF3.2 \\ RRF3.3 \\ RRF3.2.1 \\ RRF3.2.2 \\ RRF5.1 \\ RRF5.2 \\ RRF7 \\ RRF7.1 \\ RRF7.2 \\ ROF8 \\ RRF8.1 \\ RRF8.2 \\ RRF8.3 \\ RRC22 \\ RRC10 \\ RRC11 \\ RRC12 \\ RRQ13 } \\ \hline
                                        Capitolato & Capitolato d'appalto C3 & \parbox[t]{4cm}{RRQ14 \\ RRC15 \\ RRC16 \\ RRC17 \\ RDC18 \\ RRQ19 \\ RDQ20 \\ RDQ21 \\ RRF1.2 \\ RRF1.3 \\ RRF1.4 \\ RRF2.4 \\ RRF2.5 \\ RRF2.6 \\ RRF2.7 \\ RRF1.5.2 \\ RRF1.5.3 \\ RRF1.5.2.1 \\ RRF1.5.3.1 \\ RRF1.5.2.2 \\ RRF1.5.3.2 \\ RRF1.5.6 \\ RRF1.5.6.1 \\ RRF1.5.6.2 \\ RRF1.5.11 \\ RRF1.5.19 \\ RRF1.5.22 \\ RRF1.5.23 }\\
                \hline
                UCA0 & UCA0 Utilizzo dell'applicazione Android & \parbox[t]{4cm}{ ROF8 }\\
                \hline
                UCA1 & UCA1 Accedere a un istanza di Norris & \parbox[t]{4cm}{ RRF8.1 }\\
                \hline
                UCA1.1 & UCA1.1 Inserire indirizzo di un istanza di Norris & \parbox[t]{4cm}{ RRF8.1 }\\
                \hline
                UCA1.2 & UCA1.2 Inserire username & \parbox[t]{4cm}{ RRF8.1 }\\
                \hline
                UCA1.3 & UCA1.3 Inserire password & \parbox[t]{4cm}{ RRF8.1 }\\
                \hline
                UCA2 & UCA2 Visualizzare elenco grafici & \parbox[t]{4cm}{ RRF8.2 }\\
                \hline
                UCA2.1 & UCA2.1 Visualizzare ID grafici & \parbox[t]{4cm}{ RRF8.2 }\\
                \hline
                UCA2.2 & UCA2.2 Visualizzare titolo grafici & \parbox[t]{4cm}{ RRF8.2 }\\
                \hline
                UCA2.3 & UCA2.3 Visualizzare tipo grafici & \parbox[t]{4cm}{ RRF8.2 }\\
                \hline
                UCA2.4 & UCA2.4 Visualizzare descrizione grafici & \parbox[t]{4cm}{ RRF8.2 }\\
                \hline
                UCA3 & UCA3 Visualizzare singolo grafico & \parbox[t]{4cm}{ RRF8.3 }\\
                \hline
                UCA3.1 & UCA3.1 Selezione di un grafico dall'elenco di grafici esistenti & \parbox[t]{4cm}{ RRF8.3 }\\
                \hline
                UCA3.2 & UCA3.2 Visualizzare il grafico selezionato & \parbox[t]{4cm}{ RRF8.3 }\\
                \hline
                UCA4 & UCA4 Visualizzare errore dati di accesso non validi & \parbox[t]{4cm}{ RRF8.4 \\ RRF8.5 }\\
                \hline
                UCA4.1 & UCA4.1 Visualizzare errore indirizzo dell istanza Norris non valido & \parbox[t]{4cm}{ RRF8.4 }\\
                \hline
                UCA4.2 & UCA4.2 Visualizzare errore credenziali di accesso non valide & \parbox[t]{4cm}{ RRF8.5 }\\
                \hline
                UCD0 & UCD0 Utilizzo della dashboard & \parbox[t]{4cm}{ RRF7 }\\
                \hline
                UCD1 & UCD1 Visualizzazione posizione autobus APS in tempo reale & \parbox[t]{4cm}{ RRF7 }\\
                \hline
                UCD2 & UCD2 Visualizzazione numero di autobus attivi per ciascuna linea & \parbox[t]{4cm}{ RRF7.1 }\\
                \hline
                UCD2.1 & UCD2.1 Selezione della linea della quale si vuole visualizzare il numero di bus & \parbox[t]{4cm}{ RRF7.1 }\\
                \hline
                UCD2.2 & UCD2.2 Visualizzare il numero di autobus attivi per una linea & \parbox[t]{4cm}{ RRF7.1 }\\
                \hline
                UCD3 & UCD3 Filtrare autobus per linea di appartenenza & \parbox[t]{4cm}{ RRF7.2 }\\
                \hline
                UCD3.1 & UCD3.1 Selezionare le linee che si intende visualizzare & \parbox[t]{4cm}{ RRF7.2 }\\
                \hline
                UCD3.2 & UCD3.2 Visualizzare i bus appartenenti alle linee scelte & \parbox[t]{4cm}{ RRF7.2 }\\
                \hline
                UCN0 & UCN0 Utilizzo del framework Norris & \parbox[t]{4cm}{ RRF1 \\ RRF2 \\ RRF3 \\ RRF4 \\ RRF5 \\ RRF6 }\\
                \hline
                UCN1 & UCN1 Utilizzo API interne di Norris & \parbox[t]{4cm}{ RRF1 \\ RRF2 \\ RRF3 \\ RRF4 }\\
                \hline
                UCN1.1 & UCN1.1 Creare modello chart & \parbox[t]{4cm}{ RRF1 }\\
                \hline
                UCN1.1.1 & UCN1.1.1 Inserire il titolo del chart & \parbox[t]{4cm}{ RRF1.5.17 }\\
                \hline
                UCN1.1.2 & UCN1.1.2 Scegliere una descrizione per il chart & \parbox[t]{4cm}{ RDF1.5.4 }\\
                \hline
                UCN1.10 & UCN1.10 Visualizzare errore tipologia di aggiornamento non valida & \parbox[t]{4cm}{ RRF1.10 }\\
                \hline
                UCN1.11 & UCN1.11 Visualizzare errore raggiunto limite grafici inseribili in una pagina & \parbox[t]{4cm}{ RRF1.9 }\\
                \hline
                UCN1.12 & UCN1.12 Fornire funzioni di autenticazione & \parbox[t]{4cm}{ RRF9 }\\
                \hline
                UCN1.12.1 & UCN1.12.1 Fornire funzione di login & \parbox[t]{4cm}{ RRF9 }\\
                \hline
                UCN1.12.2 & UCN1.12.2 Fornire funzione di logout & \parbox[t]{4cm}{ RRF9 }\\
                \hline
                UCN1.12.3 & UCN1.12.3 Fornire funzione di verifica autenticazione & \parbox[t]{4cm}{ RRF9 }\\
                \hline
                UCN1.2 & UCN1.2 Aggiornare un modello chart & \parbox[t]{4cm}{ RRF2 }\\
                \hline
                UCN1.2.1 & UCN1.2.1 Aggiornare un modello bar chart & \parbox[t]{4cm}{ RRF2 }\\
                \hline
                UCN1.2.1.1 & UCN1.2.1.1 Aggiornare un modello bar chart selezionato con metodo in place & \parbox[t]{4cm}{ RRF2.4 }\\
                \hline
                UCN1.2.2 & UCN1.2.2 Aggiornare un modello line chart & \parbox[t]{4cm}{ RRF2 }\\
                \hline
                UCN1.2.2.1 & UCN1.2.2.1 Aggiornare un modello line chart con metodo in place & \parbox[t]{4cm}{ RRF2.5 }\\
                \hline
                UCN1.2.2.2 & UCN1.2.2.2 Aggiornare un modello line chart con metodo stream & \parbox[t]{4cm}{ RRF2.1 }\\
                \hline
                UCN1.2.3 & UCN1.2.3 Aggiornare un modello map chart & \parbox[t]{4cm}{ RRF2 }\\
                \hline
                UCN1.2.3.1 & UCN1.2.3.1 Aggiornare un modello map chart con metodo in place & \parbox[t]{4cm}{ RRF2.6 }\\
                \hline
                UCN1.2.3.2 & UCN1.2.3.2 Aggiornare un modello map chart con metodo movie & \parbox[t]{4cm}{ RRF2.3 }\\
                \hline
                UCN1.2.4 & UCN1.2.4 Aggiornare un modello table & \parbox[t]{4cm}{ RRF2 }\\
                \hline
                UCN1.2.4.1 & UCN1.2.4.1 Aggiornare un modello table con metodo in place & \parbox[t]{4cm}{ RRF2.7 }\\
                \hline
                UCN1.2.4.2 & UCN1.2.4.2 Aggiornare un modello table con metodo stream & \parbox[t]{4cm}{ RRF2.2 }\\
                \hline
                UCN1.3 & UCN1.3 Creare pagina & \parbox[t]{4cm}{ RRF3 }\\
                \hline
                UCN1.3.1 & UCN1.3.1 Inserire il titolo della pagina & \parbox[t]{4cm}{ RRF3.1 }\\
                \hline
                UCN1.3.2 & UCN1.3.2 Scegliere le opzioni di visualizzazione della pagina & \parbox[t]{4cm}{ RRF3.2 }\\
                \hline
                UCN1.3.2.1 & UCN1.3.2.1 Scegliere il massimo numero di grafici visualizzabili su una riga & \parbox[t]{4cm}{ RRF3.2.1 }\\
                \hline
                UCN1.3.2.2 & UCN1.3.2.2 Scegliere il massimo numero di grafici visualizzabili su una colonna & \parbox[t]{4cm}{ RRF3.2.2 }\\
                \hline
                UCN1.3.3 & UCN1.3.3 Aggiungere grafico alla pagina & \parbox[t]{4cm}{ RRF3.3 }\\
                \hline
                UCN1.4 & UCN1.4 Creare modello bar chart & \parbox[t]{4cm}{ RRF1.1 }\\
                \hline
                UCN1.4.1 & UCN1.4.1 Scegliere opzioni bar chart & \parbox[t]{4cm}{ RRF1.5 }\\
                \hline
                UCN1.4.1.1 & UCN1.4.1.1 Scegliere il colore di ciascun set di barre & \parbox[t]{4cm}{ RRF1.5.19 }\\
                \hline
                UCN1.4.1.2 & UCN1.4.1.2 Scegliere l'orientamento delle barre & \parbox[t]{4cm}{ RRF1.5.8 }\\
                \hline
                UCN1.4.1.3 & UCN1.4.1.3 Scegliere il formato di stampa dei valori del bar chart & \parbox[t]{4cm}{ RRF1.5.7 }\\
                \hline
                UCN1.4.1.3.1 & UCN1.4.1.3.1 Scegliere la dimensione dello spazio tra due serie del bar chart & \parbox[t]{4cm}{ RDF1.5.7.3 }\\
                \hline
                UCN1.4.1.3.2 & UCN1.4.1.3.2 Scegliere la dimensione dello spazio tra due valori del bar chart & \parbox[t]{4cm}{ RRF1.5.7.4 }\\
                \hline
                UCN1.4.1.4 & UCN1.4.1.4 Scegliere il nome di ciascun set di barre & \parbox[t]{4cm}{ RRF1.5.22 }\\
                \hline
                UCN1.4.1.5 & UCN1.4.1.5 Scegliere le opzioni riguardanti la legenda di un bar chart & \parbox[t]{4cm}{ RRF1.5.2 }\\
                \hline
                UCN1.4.1.5.1 & UCN1.4.1.5.1 Scegliere se la legenda è visualizzata & \parbox[t]{4cm}{ RRF1.5.2.1 }\\
                \hline
                UCN1.4.1.5.2 & UCN1.4.1.5.2 Scegliere la posizione in cui è visualizzata la legenda & \parbox[t]{4cm}{ RRF1.5.2.2 }\\
                \hline
                UCN1.4.1.6 & UCN1.4.1.6 Scegliere le opzioni riguardanti il piano cartesiano del bar chart & \parbox[t]{4cm}{ RRF1.5.6 }\\
                \hline
                UCN1.4.1.6.1 & UCN1.4.1.6.1 Scegliere il nome degli assi & \parbox[t]{4cm}{ RRF1.5.6.1 }\\
                \hline
                UCN1.4.1.6.2 & UCN1.4.1.6.2 Scegliere se le linee della griglia sono visualizzate & \parbox[t]{4cm}{ RRF1.5.6.2 }\\
                \hline
                UCN1.4.1.7 & UCN1.4.1.7 Scegliere il massimo numero di barre da visualizzare per ogni serie & \parbox[t]{4cm}{ RRF1.5.11 }\\
                \hline
                UCN1.4.2 & UCN1.4.2 Inserire dati bar chart & \parbox[t]{4cm}{ RRF1.6 }\\
                \hline
                UCN1.4.2.1 & UCN1.4.2.1 Inserire i valori indipendenti nel bar chart & \parbox[t]{4cm}{ RRF1.6 }\\
                \hline
                UCN1.4.2.2 & UCN1.4.2.2 Inserire i valori dipendenti per ciascun set di barre & \parbox[t]{4cm}{ RRF1.6 }\\
                \hline
                UCN1.5 & UCN1.5 Creare modello line chart & \parbox[t]{4cm}{ RRF1.2 }\\
                \hline
                UCN1.5.1 & UCN1.5.1 Scegliere opzioni line chart & \parbox[t]{4cm}{ RRF1.5 }\\
                \hline
                UCN1.5.1.1 & UCN1.5.1.1 Scegliere il colore di ciascuna linea & \parbox[t]{4cm}{ RRF1.5.18 }\\
                \hline
                UCN1.5.1.2 & UCN1.5.1.2 Scegliere il formato di stampa dei valori del line chart & \parbox[t]{4cm}{ RRF1.5.7 }\\
                \hline
                UCN1.5.1.2.1 & UCN1.5.1.2.1 Scegliere la dimensione dei punti del line chart & \parbox[t]{4cm}{ RDF1.5.7.6 }\\
                \hline
                UCN1.5.1.2.2 & UCN1.5.1.2.2 Scegliere se la linea del line chart è curva o segmentata & \parbox[t]{4cm}{ RRF1.5.7.7 }\\
                \hline
                UCN1.5.1.3 & UCN1.5.1.3 Scegliere il nome di ciascuna linea & \parbox[t]{4cm}{ RRF1.5.21 }\\
                \hline
                UCN1.5.1.4 & UCN1.5.1.4 Scegliere il massimo numero di punti visualizzati sull'asse dei valori indipendenti & \parbox[t]{4cm}{ RRF1.5.11 }\\
                \hline
                UCN1.5.1.5 & UCN1.5.1.5 Scegliere le opzioni riguardanti la legenda di un line chart & \parbox[t]{4cm}{ RRF1.5.1 }\\
                \hline
                UCN1.5.1.5.1 & UCN1.5.1.5.1 Scegliere se la legenda è visualizzata & \parbox[t]{4cm}{ RRF1.5.1.1 }\\
                \hline
                UCN1.5.1.5.2 & UCN1.5.1.5.2 Scegliere la posizione in cui è visualizzata la legenda & \parbox[t]{4cm}{ RRF1.5.1.2 }\\
                \hline
                UCN1.5.1.6 & UCN1.5.1.6 Scegliere le opzioni riguardanti il piano cartesiano del line chart & \parbox[t]{4cm}{ RRF1.5.5 }\\
                \hline
                UCN1.5.1.6.1 & UCN1.5.1.6.1 Scegliere il nome degli assi & \parbox[t]{4cm}{ RRF1.5.5.1 }\\
                \hline
                UCN1.5.1.6.2 & UCN1.5.1.6.2 Scegliere se le linee della griglia sono visualizzate & \parbox[t]{4cm}{ RRF1.5.5.2 }\\
                \hline
                UCN1.5.2 & UCN1.5.2 Inserire dati line chart & \parbox[t]{4cm}{ RRF1.6 }\\
                \hline
                UCN1.5.2.1 & UCN1.5.2.1 Inserire i valori indipendenti nel line chart & \parbox[t]{4cm}{ RRF1.6 }\\
                \hline
                UCN1.5.2.2 & UCN1.5.2.2 Inserire i valori dipendenti per ciascuna linea & \parbox[t]{4cm}{ RRF1.6 }\\
                \hline
                UCN1.6 & UCN1.6 Creare modello map chart & \parbox[t]{4cm}{ RRF1.3 }\\
                \hline
                UCN1.6.1 & UCN1.6.1 Scegliere opzioni map chart & \parbox[t]{4cm}{ RRF1.5 }\\
                \hline
                UCN1.6.1.1 & UCN1.6.1.1 Scegliere il formato di stampa dei valori del map chart & \parbox[t]{4cm}{ RRF1.5.7 }\\
                \hline
                UCN1.6.1.1.1 & UCN1.6.1.1.1 Scegliere la forma dei marcatori del map chart & \parbox[t]{4cm}{ RDF1.5.7.5 }\\
                \hline
                UCN1.6.1.2 & UCN1.6.1.2 Scegliere le dimensioni dell'area mostrata sulla mappa & \parbox[t]{4cm}{ RRF1.5.9 }\\
                \hline
                UCN1.6.1.3 & UCN1.6.1.3 Scegliere il massimo numero di punti da visualizzare per ogni serie & \parbox[t]{4cm}{ RRF1.5.11 }\\
                \hline
                UCN1.6.1.4 & UCN1.6.1.4 Scegliere le coordinate del punto centrale della mappa & \parbox[t]{4cm}{ RRF1.5.10 }\\
                \hline
                UCN1.6.1.5 & UCN1.6.1.5 Scegliere il colore di ciascuna serie di punti & \parbox[t]{4cm}{ }\\
                \hline
                UCN1.6.1.6 & UCN1.6.1.6 Scegliere il nome di ciascuna serie di punti & \parbox[t]{4cm}{ RRF1.5.23 }\\
                \hline
                UCN1.6.1.7 & UCN1.6.1.7 Scegliere le opzioni riguardanti la legenda di un map chart & \parbox[t]{4cm}{ RRF1.5.3 }\\
                \hline
                UCN1.6.1.7.1 & UCN1.6.1.7.1 Scegliere se la legenda è visualizzata & \parbox[t]{4cm}{ RRF1.5.3.1 }\\
                \hline
                UCN1.6.1.7.2 & UCN1.6.1.7.2 Scegliere la posizione in cui è visualizzata la legenda & \parbox[t]{4cm}{ RRF1.5.3.2 }\\
                \hline
                UCN1.6.2 & UCN1.6.2 Inserire nuova serie di punti & \parbox[t]{4cm}{ RRF1.6 }\\
                \hline
                UCN1.7 & UCN1.7 Creare modello table & \parbox[t]{4cm}{ RRF1.4 }\\
                \hline
                UCN1.7.1 & UCN1.7.1 Scegliere opzioni table & \parbox[t]{4cm}{ RRF1.5 }\\
                \hline
                UCN1.7.1.1 & UCN1.7.1.1 Scegliere il formato di una cella della table & \parbox[t]{4cm}{ RRF1.5.7 }\\
                \hline
                UCN1.7.1.1.1 & UCN1.7.1.1.1 Scegliere il colore del testo contenuto in una cella & \parbox[t]{4cm}{ RRF1.5.7.1 }\\
                \hline
                UCN1.7.1.1.2 & UCN1.7.1.1.2 Scegliere il colore dello sfondo di una cella & \parbox[t]{4cm}{ RRF1.5.7.2 }\\
                \hline
                UCN1.7.1.2 & UCN1.7.1.2 Scegliere l'intestazione di una singola colonna & \parbox[t]{4cm}{ RRF1.5.15 }\\
                \hline
                UCN1.7.1.3 & UCN1.7.1.3 Scegliere il massimo numero di righe da visualizzare & \parbox[t]{4cm}{ RRF1.5.12 }\\
                \hline
                UCN1.7.1.4 & UCN1.7.1.4 Scegliere se è possibile ordinare le righe rispetto ai valori di una colonna & \parbox[t]{4cm}{ RRF1.5.13 }\\
                \hline
                UCN1.7.1.5 & UCN1.7.1.5 Scegliere la posizione in cui vengono aggiunte nuove righe & \parbox[t]{4cm}{ RRF1.5.14 }\\
                \hline
                UCN1.7.1.6 & UCN1.7.1.6 Scegliere se le linee della tabella sono visualizzate & \parbox[t]{4cm}{ RRF1.5.16 }\\
                \hline
                UCN1.7.2 & UCN1.7.2 Inserire nuova riga & \parbox[t]{4cm}{ RRF1.6 }\\
                \hline
                UCN1.8 & UCN1.8 Ottenere middleware & \parbox[t]{4cm}{ RRF4 }\\
                \hline
                UCN1.9 & UCN1.9 Visualizzare errore dati non corretti & \parbox[t]{4cm}{ RRF1.7 \\ RRF1.8 }\\
                \hline
                UCN2 & UCN2 Utilizzo API esterne di Norris & \parbox[t]{4cm}{ RRF5 }\\
                \hline
                UCN2.1 & UCN2.1 Accedere a un istanza di Norris & \parbox[t]{4cm}{ RRF5.3 }\\
                \hline
                UCN2.1.1 & UCN2.1.1 Inserire indirizzo di un istanza di Norris & \parbox[t]{4cm}{ RRF5.3 }\\
                \hline
                UCN2.1.2 & UCN2.1.2 Inserire username & \parbox[t]{4cm}{ RRF5.3 }\\
                \hline
                UCN2.1.3 & UCN2.1.3 Inserire password & \parbox[t]{4cm}{ RRF5.3 }\\
                \hline
                UCN2.2 & UCN2.2 Ottenere lista dei grafici presenti in un'istanza di Norris & \parbox[t]{4cm}{ RRF5.1 }\\
                \hline
                UCN2.2.1 & UCN2.2.1 Ottenere l'ID di ciascun grafico & \parbox[t]{4cm}{ RRF5.1.1 }\\
                \hline
                UCN2.2.2 & UCN2.2.2 Ottenere il titolo di ciascun grafico & \parbox[t]{4cm}{ RRF5.1.1 }\\
                \hline
                UCN2.2.3 & UCN2.2.3 Ottenere il tipo di ciascun grafico & \parbox[t]{4cm}{ RRF5.1.1 }\\
                \hline
                UCN2.2.4 & UCN2.2.4 Ottenere la descrizione di ciascun grafico & \parbox[t]{4cm}{ RRF5.1.1 }\\
                \hline
                UCN2.3 & UCN2.3 Ottenere un grafico presente in un'istanza di Norris & \parbox[t]{4cm}{ RRF5.2 }\\
                \hline
                UCN2.4 & UCN2.4 Scollegarsi da un istanza di Norris & \parbox[t]{4cm}{ RRF5.4 }\\
                \hline
                UCN3 & UCN3 Utilizzo API fornite da Chuck & \parbox[t]{4cm}{ RRF6 }\\
                \hline
                UCN3.1 & UCN3.1 Selezionare il modello di grafico che si vuole rappresentare & \parbox[t]{4cm}{ RRF6.1 }\\
                \hline
                UCN3.2 & UCN3.2 Scegliere il tag HTML nel quale si vuole inserire la rappresentazione del grafico & \parbox[t]{4cm}{ RRF6.1 }\\
                \hline
                UCN3.3 & UCN3.3 Cambiare opzioni di visualizzazione di un grafico & \parbox[t]{4cm}{ RDF6.2 }\\
                \hline
                UCN3.4 & UCN3.4 Cambiare opzioni di visualizzazione di un bar chart & \parbox[t]{4cm}{ RDF6.2 }\\
                \hline
                UCN3.4.1 & UCN3.4.1 Cambiare il colore di un set di barre & \parbox[t]{4cm}{ RDF6.2.10 }\\
                \hline
                UCN3.4.2 & UCN3.4.2 Cambiare le opzioni riguardanti la legenda di un bar chart & \parbox[t]{4cm}{ RDF6.2.5 \\ RDF6.2.2 }\\
                \hline
                UCN3.4.2.1 & UCN3.4.2.1 Cambiare il fatto che la legenda sia visualizzata o meno & \parbox[t]{4cm}{ RDF6.2.2 }\\
                \hline
                UCN3.4.2.2 & UCN3.4.2.2 Cambiare la posizione in cui è visualizzata la legenda & \parbox[t]{4cm}{ RDF6.2.5 }\\
                \hline
                UCN3.4.3 & UCN3.4.3 Cambiare il fatto che la griglia del piano cartesiano sia visualizzata o meno & \parbox[t]{4cm}{ RDF6.2.8 }\\
                \hline
                UCN3.5 & UCN3.5 Cambiare opzioni di visualizzazione di un line chart & \parbox[t]{4cm}{ RDF6.2 }\\
                \hline
                UCN3.5.1 & UCN3.5.1 Cambiare il colore di una linea & \parbox[t]{4cm}{ RDF6.2.9 }\\
                \hline
                UCN3.5.2 & UCN3.5.2 Cambiare le opzioni riguardanti la legenda di un line chart & \parbox[t]{4cm}{ RDF6.2.1 \\ RDF6.2.4 }\\
                \hline
                UCN3.5.2.1 & UCN3.5.2.1 Cambiare il fatto che la legenda sia visualizzata o meno & \parbox[t]{4cm}{ RDF6.2.1 }\\
                \hline
                UCN3.5.2.2 & UCN3.5.2.2 Cambiare la posizione in cui è visualizzata la legenda & \parbox[t]{4cm}{ RDF6.2.4 }\\
                \hline
                UCN3.5.3 & UCN3.5.3 Cambiare il fatto che la griglia del piano cartesiano sia visualizzata o meno & \parbox[t]{4cm}{ RDF6.2.7 }\\
                \hline
                UCN3.6 & UCN3.6 Cambiare opzioni di visualizzazione di un map chart & \parbox[t]{4cm}{ RDF6.2 }\\
                \hline
                UCN3.6.1 & UCN3.6.1 Cambiare il colore di una serie di punti & \parbox[t]{4cm}{ RDF6.2.11 }\\
                \hline
                UCN3.6.2 & UCN3.6.2 Cambiare le opzioni riguardanti la legenda di un map chart & \parbox[t]{4cm}{ RDF6.2.6 \\ RDF6.2.3 }\\
                \hline
                UCN3.6.2.1 & UCN3.6.2.1 Cambiare il fatto che la legenda sia visualizzata o meno & \parbox[t]{4cm}{ RDF6.2.3 }\\
                \hline
                UCN3.6.2.2 & UCN3.6.2.2 Cambiare la posizione in cui è visualizzata la legenda & \parbox[t]{4cm}{ RDF6.2.6 }\\
                \hline
                UCN3.7 & UCN3.7 Cambiare opzioni di visualizzazione di una table & \parbox[t]{4cm}{ RDF6.2 }\\
                \hline
                UCN3.7.1 & UCN3.7.1 Cambiare le opzioni riguardanti il formato di una cella & \parbox[t]{4cm}{ RDF6.2.12 \\ RDF6.2.13 }\\
                \hline
                UCN3.7.1.1 & UCN3.7.1.1 Cambiare il colore del testo contenuto in una cella & \parbox[t]{4cm}{ RDF6.2.13 }\\
                \hline
                UCN3.7.1.2 & UCN3.7.1.2 - Cambiare il colore di sfondo di una cella & \parbox[t]{4cm}{ RDF6.2.12 }\\
                \hline
                UCN3.8 & UCN3.8 Accedere a un istanza di Norris & \parbox[t]{4cm}{ RRF6.3 }\\
                \hline
                UCN3.8.1 & UCN3.8.1 Inserire indirizzo di un istanza di Norris & \parbox[t]{4cm}{ RRF6.3 }\\
                \hline
                UCN3.8.2 & UCN3.8.2 Inserire Username & \parbox[t]{4cm}{ RRF6.3 }\\
                \hline
                UCN3.8.3 & UCN3.8.3 Inserire password & \parbox[t]{4cm}{ RRF6.3 }\\
                \hline
                UCN3.9 & UCN3.9 Scollegarsi da un istanza di Norris & \parbox[t]{4cm}{ RRF6.4 }\\
                \hline
                                \caption{Tracciamento fonti-requisiti}
				\end{longtabu}
				